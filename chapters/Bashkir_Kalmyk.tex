\documentclass[output=paper,draft,draftmode,colorlinks,citecolor=brown]{langscibook}
\ChapterDOI{10.5281/zenodo.7353617}
\author{Vlada V. Baranova\affiliation{National Research University Higher School of Economics (NRU HSE), St Petersburg;  Institute for Linguistic Studies, Russian Academy of Sciences (ILS RAS), St Petersburg} and Daria F. Mishchenko\affiliation{Institute for Linguistic Studies, Russian Academy of Sciences (ILS RAS), St Petersburg; Museum of Anthropology and Ethnography (Kunstkamera), Russian Academy of Sciences (MAE RAS), St Petersburg}
}

\title{Non-verbal negation markers and the Negative Existential Cycle in Bashkir and Kalmyk with some typological parallels}

 

\abstract{The Negative Existential Cycle presupposes involvement of negative existentials in a cyclical process whereby negative markers evolve. The aim of this paper is to show that negative markers used with noun phrases can also change their functions and, in particular, transform into verbal negators. As evidenced by Turkic languages, non-verbal negators can be used with future and some past forms as an alternative to verbal negators or instead of them. In Central Mongolic, a negative marker that was initially used as a negative existential first evolved into a nominal negation marker and then intruded into the verbal system, becoming a standard negation marker. At the same time, in Kalmyk an ascriptive negator competes with it as a verbal negator. It should be noted that similar phenomena are found cross-linguistically in genealogically different languages. At first, the markedness of the non-verbal negators contributes to their emphatic meaning in such uses, but their frequent co-occurrence with certain verbal forms can further result in them replacing a verbal negator and becoming the only negator used with these forms. Secondly, we aim to show that changes in a language system of negative markers do not necessarily close a cycle but sometimes shape a “tree”, where a new element shares functions with older elements of the system, launching a new “branch” of changes, independent of the main line.

\keywords{Bashkir, future tense, Kalmyk, Mongolic languages, Negative Existential Cycle, non-verbal negators, Turkic languages}
}


\IfFileExists{../localcommands.tex}{
   % add all extra packages you need to load to this file

\usepackage{tabularx,multicol}
\usepackage{url}
\urlstyle{same}

\usepackage{listings}
\lstset{basicstyle=\ttfamily,tabsize=2,breaklines=true}

\usepackage{langsci-optional}
\usepackage{langsci-lgr}
\usepackage{langsci-gb4e}

%from elena-----------------------------------
\usepackage{pgfplots,pgfplotstable}
\definecolor{lsDOIGray}{cmyk}{0,0,0,0.45}
\usepackage{xassoccnt}
\newcounter{realpage}
\DeclareAssociatedCounters{page}{realpage}
\AtBeginDocument{%
  \stepcounter{realpage}
}
\usepackage{enumitem}
\usepackage[]{longtable}
\usepackage{comment}

%from Nina K.---------------------------------
\usepackage{csquotes}
%\usepackage{expex}
\pgfplotsset{compat=1.16} % Why?
% \usepackage{xeCJK}
% \setCJKmainfont{HanaMinA}
\usepackage{rotating}
\usepackage{colortbl}
\usepackage{multirow}
\usepackage{dirtree}

%from Niina-----------------------------------
\usepackage{verbatim}
\usetikzlibrary{intersections, backgrounds, shapes, angles, quotes,
decorations.pathmorphing, arrows.meta, decorations.text, tikzmark}
% positioning and calc are loaded by document class
	\tikzset{snake it/.style={decorate, decoration=snake}}
\usepackage{tikz-qtree}

% figures side-by-side with body text:
\usepackage{wrapfig}
\usepackage{subcaption}

%\usepackage{enumerate}
%\usepackage[nonumberlist]{glossaries}


\usepackage[linguistics,edges]{forest}
\usetikzlibrary{positioning}
\usepackage{soul}
\usepackage{langsci-bidi}
\usepackage{langsci-branding}

   \newcommand*{\orcid}{}

%-------------------from Elena------------------------------------------------------------

\newcommand{\appref}[1]{Appendix \ref{#1}}
\newcommand{\fnref}[1]{Footnote \ref{#1}} 

\newenvironment{langscibars}{\begin{axis}[ybar,xtick=data, xticklabels from table={\mydata}{pos}, 
        width  = \textwidth,
	height = .3\textheight,
    	nodes near coords, 
	xtick=data,
	x tick label style={},  
	ymin=0,
	cycle list name=langscicolors
        ]}{\end{axis}}
        
\newcommand{\langscibar}[1]{\addplot+ table [x=i, y=#1] {\mydata};\addlegendentry{#1};}

\newcommand{\langscidata}[1]{\pgfplotstableread{#1}\mydata;}


% for annotations above example lines:
\newcommand{\overnote}[1]{\makebox[0pt][l]{\raisebox{\baselineskip}{\upshape
#1}}}
\newcommand{\moreovernote}[1]{\makebox[0pt][l]{\raisebox{2\baselineskip}{\upshape #1}}}

%--------------------from Niina----------------------------------------------------------

% \makeatletter
% \def\blx@maxline{77}
% \makeatother

% An Arabic font added to `fonts' folder, it's free and open-source
% When run on Windows, LaTeX may need the complete file name of the font:
% Amiri-Regular.ttf
% \newfontfamily\arabfont{Amiri}
% \newcommand\textarab[1]{{\arabfont #1}}%%% \textarab{...}




\newcommand{\todoref}[1]{\todo[color=green!40]{#1}}%%% 
\newcommand{\todofix}[1]{\todo[color=blue!40]{#1}}%%% 

\DeclareBibliographyCategory{sources}% filter some references
\DeclareBibliographyCategory{online}

% draw a thick bar of desired length (used for visual presentation in a
% tabular)
\newcommand{\cocabar}[1]{\color{gray}\rule[1pt]{{#1}mm}{1ex}}

% to preserve original formatting in an appendix:
\newenvironment{unindented}[0]{\setlength{\parindent}{0pt}\setlength{\parskip}{1ex
plus 0.5ex minus 0.2ex}}{}
% for annotations above example lines:
%\newcommand{\overnote}[1]{\makebox[0pt][l]{\raisebox{\baselineskip}{\upshape
%#1}}}
%\newcommand{\moreovernote}[1]{\makebox[0pt][l]{\raisebox{2\baselineskip}{\upshape #1}}}

\definecolor{blech}{rgb}{.78,.78.,.62}
\definecolor{ochre}{cmyk}{0, .42, .83, .20}
\newcommand{\exem}[1]{\textit{\textbf{#1}}}
\newcommand{\glem}[1]{\MakeUppercase{\scriptsize{\textbf{#1}}}} 
\newcommand{\denote}[1]{\mbox{$[\![\mbox{#1}]\!]$}}

\newcommand{\stacktwo}[2]{\makebox[0pt][l]{\hspace{.5pt}#1}#2}

%------------------from Nina K.-------------------------------

\newcommand{\citealtv}[1]{\citealt{#1} [this volume]}


% \renewcommand{\textdblhyphen}{⹀}


% \newcommand{\ꜥ}{\textsf{ꜥ}}
\newcommand{\ꜥ}{ʿ}
\newcommand{\ꜣ}{\kern-.25pt\texttt{ꜣ}\kern-.6pt}


\makeatletter
\let\thetitle\@title
\let\theauthor\@author
\makeatother


\newcommand{\togglepaper}[1][0]{
   \bibliography{../localbibliography}
   \papernote{\scriptsize\normalfont
     \theauthor.
     \titleTemp.
     To appear in:
     Change Volume Editor \& in localcommands.tex
     Change volume title in localcommands.tex
     Berlin: Language Science Press. [preliminary page numbering]
   }
   \pagenumbering{roman}
   \setcounter{chapter}{#1}
   \addtocounter{chapter}{-1}
}


\newfontfamily\arabicfont[Script=Arabic,ItalicFont=*,Scale=1.4]{ScheherazadeRegOT_Jazm.ttf}
% \newcommand{\arabscript}[1]{\RL{\Parsifont #1}}
\newcommand{\textarab}[1]{{\arabicfont #1}}



\DeclareCiteCommand{\textCitetv}
  {\usebibmacro{prenote}}
  {\ifciteindex
     {\indexnames{labelname}}
     {}%
   \printtext[bibhyperref]{\printnames{labelname}\addspace\bibopenparen\printfield{year}}}
  {\multicitedelim}
  {\printtext[bibhyperref]{\usebibmacro{postnote}\addspace[this volume]\bibcloseparen}}


\colorlet{karlgrencol}{pink}
\colorlet{normancol}{green}
\colorlet{pancol}{blue}
\colorlet{ohtacol}{gray}
\colorlet{peyraubecol}{orange}
\colorlet{peyraubecolmod}{yellow}
\colorlet{wangcol}{red}

\DeclareNewSectionCommand
  [
    counterwithin = appendixsubsection,
    beforeskip=-10pt,
    afterskip=1sp,
    indent = 0pt,
    font = \usekomafont{subsubsection},
    level = 3,
    tocindent = 7.0em,
    toclevel = 3,
    tocnumwidth = 4.1em,
    tocstyle = section,
    style = section
  ]
  {appendixsubsubsection}

   %% hyphenation points for line breaks
%% Normally, automatic hyphenation in LaTeX is very good
%% If a word is mis-hyphenated, add it to this file
%%
%% add information to TeX file before \begin{document} with:
%% %% hyphenation points for line breaks
%% Normally, automatic hyphenation in LaTeX is very good
%% If a word is mis-hyphenated, add it to this file
%%
%% add information to TeX file before \begin{document} with:
%% %% hyphenation points for line breaks
%% Normally, automatic hyphenation in LaTeX is very good
%% If a word is mis-hyphenated, add it to this file
%%
%% add information to TeX file before \begin{document} with:
%% \include{localhyphenation}
\hyphenation{
    Af-ri-caans
    Bar-tens
    cen-tu-ry
    comitative-existential
    com-ple-ments
    data-set
    dia-chro-nic
    ex-is-ten-tial
    ex-is-ten-tials
    Fraj-zyn-gier
    Has-pel-math
    Hol-ton
    Jes-per-sen
    lo-ca-ti-ve-ex-is-ten-tial
    Ma-khu-wa
    mar-ker
    Nai-khin
    negat-ed
    part-ti-ci-ple
    Swa-hi-li
    Ve-se-li-no-va
}

\hyphenation{
    Af-ri-caans
    Bar-tens
    cen-tu-ry
    comitative-existential
    com-ple-ments
    data-set
    dia-chro-nic
    ex-is-ten-tial
    ex-is-ten-tials
    Fraj-zyn-gier
    Has-pel-math
    Hol-ton
    Jes-per-sen
    lo-ca-ti-ve-ex-is-ten-tial
    Ma-khu-wa
    mar-ker
    Nai-khin
    negat-ed
    part-ti-ci-ple
    Swa-hi-li
    Ve-se-li-no-va
}

\hyphenation{
    Af-ri-caans
    Bar-tens
    cen-tu-ry
    comitative-existential
    com-ple-ments
    data-set
    dia-chro-nic
    ex-is-ten-tial
    ex-is-ten-tials
    Fraj-zyn-gier
    Has-pel-math
    Hol-ton
    Jes-per-sen
    lo-ca-ti-ve-ex-is-ten-tial
    Ma-khu-wa
    mar-ker
    Nai-khin
    negat-ed
    part-ti-ci-ple
    Swa-hi-li
    Ve-se-li-no-va
}

   \togglepaper[1]%%chapternumber
}{}


\shorttitlerunninghead{Non-verbal negation markers and the NEC in Bashkir and Kalmyk}
\begin{document}
\shorttitlerunninghead{Non-verbal negation markers and the NEC in Bashkir and Kalmyk}
\maketitle

\section{Introduction}\label{sec:BK1} 
\subsection{General notes}\label{sec:BK1.1}
As shown in \citet{Veselinova2016}, the Negative Existential Cycle (NEC) is rarely completed. Most of the sampled languages examined in her paper turned out to be at stages with variation. Even in cases when the system has already restructured completely, the former standard negation (SN) markers remain on the “periphery” of the syntax (“constructualized negation” in \citealt{payne1985a}). In this paper, we argue that the scenario of the changes within the Negative Existential Cycle should include, along with negative existentials, other non-verbal negators. These are constituent negator for nouns, ascriptive negation and possessive negation markers defined further below. As demonstrated by Mongolic and Turkic languages, such markers can participate in the NEC as well, occupying their own functional niche in competition with other negators. The aim of this study is to reveal these functions and to explain what ensures stability of systems where different negators – both verbal and non-verbal – co-exist.

A special type of the use of non-verbal negative markers as SN is the intrusion of nominal negation. \citet[228]{payne1985a} mentioned rare cases when the marker of sentential negation has nominal properties (see also \citealp{miestamo2017a}). For instance, Nadëb (mbj, Makú) has a nominal negator \textit{dooh}, which functions as a SN marker \citep[294--295]{weir1994a}. At the same time, this grammaticalization path includes a category change and extending functions of the negation marker, so that it may acquire new properties.

Therefore, it is reasonable to say that we need to rethink the diachronic change of non-verbal negation based on Croft’s seminal article \citeyearpar{Croft1991} to include in the discussion some other types of inroads of non-verbal negation into the sphere of use of standard negation. Moreover, we will show that there can be two or more processes of incorporation of different types of non-standard negation into the verbal negation system, which can influence each other and do not form separate cycles. The intrusion of non-existential, non-verbal negation into the domain of SN develops in conformity with a previous process of the expansion of existential negation and \textit{vice versa}.

According to Payne’s \citeyearpar{Payne1997} classification, there are six types of nominal predicates: proper inclusion, equation, attribution, location, existence and possession. The literature on negation mostly establishes a reduced distinction between adjectival, nominal and locative predicates only (see \citealp{dryer2007a}). The classification that is most relevant to our purpose is that of Ljuba \citet{Veselinova2015}. According to \citet[550]{Veselinova2015}, negative counterparts for predication without a verbal predicate – i.e. negators different from SN – are used for ascriptive (negation of the sentences with a nominal or adjectival predicate conveying the meanings of class inclusion, quality or a temporary state), locative, existential and possessive negation.

To test our hypothesis on the involvement of different non-existential non-verbal negators in the development of verbal negation, we will draw upon data of several types. Thus, this article consists of two parts: in the first part, we present and analyse in detail the data of two Mongolic and Turkic languages, Bashkir (bak) and Kalmyk (xal), where non-verbal negative markers different from negative existentials intrude into the system of verbal negation. Being acquainted with these languages through long-term fieldwork, we hope to be able to comprehensively depict their systems of negators, as well as the more subtle aspects of their use. In the second part of the article, we discuss cases of some other languages of the world in which a similar phenomenon – intrusion of a non-verbal negative marker different from the existential negator into the system of verbal negation – is observed, and try to find regularities in the use of non-verbal negators as verbal negators. It should be noted that this phenomenon has not been examined before, and there are no systematic typological studies on this issue so far. Based on our fieldwork data and several descriptions of the attested phenomenon from grammars of other languages, we thus provide a first description and tentative analysis of this pattern.

Following the first, introductory section, in \sectref{sec:BK2} the data on the use of non-verbal negative markers in Bashkir and closely related Turkic languages is presented. \sectref{sec:BK3} treats the evolution of the ascriptive and existential negation markers of Kalmyk due to changes in the system of Mongolic negation. \sectref{sec:BK4} offers an analysis of examples of the intrusion of non-verbal negative markers different from negative existentials into the system of verbal negation that we found in grammars of genealogically different languages, namely Bornean languages and Egyptian Arabic. In Section 5, we identify and discuss some cross-linguistic regularities regarding this intrusion.

The main conclusion of the article is that the negators normally used for nominal negation and other non-verbal non-existential negative markers, when used as verbal negators, are still associated with some more specific meanings (such as emphatic negation), which determines their predominant use with certain verbal forms, ensuring the stability of systems where verbal and non-verbal negative markers co-exist and do not replace each other.

\subsection{Data and context}\label{sec:BK1.2}

Bashkir belongs to the northern subgroup of the Kipchak branch of Turkic languages, and genealogically it is closest to Tatar. The Bashkir people were first mentioned in the 10th century. Up to the 19th century they shared the same literary language, Volga Turki (Old Tatar, Old Bashkir), a regional variant of Turki, with the Tatars. Volga Turki used an alphabet that was based on Persian Arabic script. Spoken Volga Turki, however, had regional varieties specific to the different ethnic groups that used it. Starting from the 1920s, a literary Bashkir language formed. Initially, it continued to be written in an alphabet based on the Arabic script; this was revised in 1923. It was replaced with a Latin-based alphabet in 1930, and in 1940 an expanded Cyrillic alphabet was adopted, which has been used till the present day \citep[11–12]{yuldashev1981a}.

Currently Bashkir is the co-official (together with Russian) language of the Republic of Bashkortostan in central Russia, west of the Urals. It is also spoken in neighbouring regions, with a total number of approximately 1,200,000 speakers. There are three major dialects: Southern, Eastern and North-western; the first two are very similar and have served as the basis for the literary language.

Bashkir field data was collected in Rahmetovo and Baimovo, two villages in the Abzelilovsky region of the Republic of Bashkortostan, in 2011–2016. It consists of the texts from the corpus of oral texts recorded, transcribed and glossed during the field trips, and elicited sentences collected by using questionnaires. As an additional source, texts from the Folklore Archive of Bashkir State University (\url{http://lcph.bashedu.ru/editions/efolk.php?go=folk_id.28}) were used. In order to ensure compatibility, only those folklore texts recorded in the Abzelilovsky region were taken into account. In addition, some constructions and forms were retrieved from the Internet.
Kalmyk or Oirat (spoken in the Republic of Kalmykia, Russian Federation) belongs to the Western branch of the Mongolic language family and has a total number of approximately 80,000 speakers. In the 17th century, Kalmyk Oirat split off from other Oirat dialects (as are spoken today in parts of Xinjiang and Inner Mongolia, China, the western part of Mongolia and Issyk Kyl province, Kyrgyzstan) and migrated into the current area of the steppe near the Volga River.

The major dialects of Kalmyk (Dörbet, Torghut and Buzava) are close to each other, except for small lexical variations. The standard language is mostly based on the Dörbet dialect. Kalmyk is an official second language in the Republic of Kalmykia. 

\newpage
Kalmyk field data was collected in the Ketchenerovsky region in the Republic of Kalmykia in 2006–2008 and 2014–2015. Data was collected via questionnaires and in the form of oral narratives, which have been compiled as a small corpus of Spoken Kalmyk in Toolbox (approximately 17,000 words). Other sources for Kalmyk are two online corpora and publications of several Written Oirat / Old Kalmyk texts. The relatively small Kalmyk National Corpus by A. Vankaeva (KNC, \url{http://web-corpora.net/KalmykCorpus}) consists of 800,000 words. The National Corpus of Kalmyk Language (NCKL, \url{http://kalmcorpora.ru}) comprises approximately 8 million words of various text types. Written Oirat is the language of historical documents written in the Todo bichig “Clear Script” in the 17th–19th centuries (see \citealp{rakos2015a}), and the subset of such texts written in Kalmykia can be termed Old Kalmyk. Here we use three source collections originally written in Todo bichig script over the span of one century: letters of Ayuki Khan and his circle (1710–1714; \citealp{suseeva2003a}), letters of Dunduk-Dashi Khan (1741–1761; \citealp{kokshaeva2011a}), and letters from different correspondents to Isaac Jacob Schmidt, a missionary and translator of the Bible into Kalmyk in the early 1800s \citep{krueger2002a}.\footnote{We use transcriptions for our own data from texts and questionnaires, as well as for sentences from the online corpora. Examples from grammars and other publications are given in the author’s or editor’s transcription and with author’s/editor’s translations, while parsing and glosses are ours. We translate sources in Russian into English.}

Both Bashkir and Kalmyk have a complex morphology with a rich system of suffixes for both nouns and verbs. The morphology is agglutinative, using affixes, and there are a lot of periphrastic constructions consisting of auxiliary verbs and various participles and converbs to express TAME meanings. Complex verbs formed by a converb and head verb play an important role in expressing different, mostly aspectual, meanings. Syntactically, they are characterized by SOV word order.

\section{Use of nominal negative markers with verbs in Turkic languages}\label{sec:BK2}
\subsection{General description of negation in Bashkir}\label{sec:BK2.1}
\largerpage[-1]
There are two ways to express negation in Bashkir: morphologically and syntactically. Along with verbal suffixes, negation can be expressed by negative copulas, that is, by auxiliaries performing a supportive function with non-verbal predicates (following \citealp{hengeveld1992a}).\footnote{According to \citeauthor{hengeveld1992a}, strictly speaking, one can distinguish between two subclasses of such auxiliaries, copulas (in a narrower sense) and semi-copulas. The first ones are semantically empty, while the second ones are not, i.e. the difference between the subclasses consists in that “the semi-copula adds an element of meaning to the construction in which it occurs, whereas the copula does not” \citeyearpar[35]{hengeveld1992a}. In this vein Bashkir negative predicators are actually semi-copulas. However, for the sake of brevity, we will call them copulas in a broader sense and gloss respectively, which is not an exceptional situation – even \citeauthor{hengeveld1992a} himself, “in a rather loose way” \citeyearpar[32]{hengeveld1992a}, calls auxiliaries of the both subclasses copulas throughout his book.} The SN marker in Bashkir is \textit{‑ma} (\textit{‑mä}):

\ea Bashkir (questionnaires)\footnote{Unless otherwise mentioned, all examples from Bashkir present the authors’ own data.} \label{ex:BK1}\\
  \ea
	\gll Kärim	kitap	uqə-j.\\
	Karim		book	read-\textsc{ipfv}\\
	\glt `Karim is reading a book.'
  \ex
	\gll Kärim	kitap	uqə-ma-j.\\
	Karim		book	read-\textsc{neg-ipfv}\\
	\glt `Karim is not reading a book.'
\z \z


If forming part of a derivational stem, the suffix \textit{-ma} can be kept in a derived word, such as a deverbal noun (\ref{ex:BK2}):

\ea Bashkir (questionnaires) \label{ex:BK2}\\
	\gll fakt-tar-ðəŋ	döröθ	bul-ma-w-ə\\
	fact-\textsc{pl-gen}	truthful	be-\textsc{neg-nmlz-p.3}\\
	\glt `unreliability of facts'
\z

Along with the verbal suffix \textit{‑ma} (\textit{‑mä}), three are other verbal negative suffixes in Bashkir: a negative suffix of the so-called “potential” future \citep[148]{dmitriev1948a} participle \textit{‑maθ} (\textit{‑mäθ}) and a suffix of the negative converb of attendant circumstances \textit{‑majənsa} (\textit{‑mäjensä}). The suffixes \textit{‑maθ} and \textit{‑majənsa} have developed from a combination of the SN marker \textit{‑ma} with other suffixes and subsequent phonological changes. The former (\textit{‑maθ}) derives from \textit{-ma} and the suffix of the “potential” future \textit{-r}, which later underwent the phonological change \textit{r} $\rightarrow$ \textit{ð} $\rightarrow$ \textit{θ} typical of Turkic languages \citep[149]{dmitriev1948a}. The following pair of sentences illustrates the use of affirmative and negative “potential” future forms:
\newpage
\ea Bashkir (questionnaires) \label{ex:BK3}\\
  \ea
	\gll Min	Mäskäw-gä		kit-er-gä			ula-j-əm.\\
	I		Moscow-\textsc{dat}	leave-\textsc{pot-dat}	think-\textsc{prs-1sg}\\
	\glt `I think I will go to Moscow.'
  \ex
	\gll Min	Mäskäw-gä		kit-mäθ-kä				ula-j-əm.\\
	I		Moscow-\textsc{dat}	leave-\textsc{neg.pot-dat}	think-\textsc{prs-1sg}\\
	\glt `I think I will not go to Moscow.'
\z \z

The etymology of \textit{‑majənsa} is not so clear. Analysing forms in \textit{‑majənsa} as negative equivalents of the converb in \textit{‑ɣansa} \citep[188]{dmitriev1948a} is probably inappropriate, at least for data from Rahmetovo Bashkir. Indeed,\textit{‑majənsa} does not seem to paradigmatically or formally correspond to any affirmative form, and it is perhaps best considered as an independent converb marker expressing the absence of secondary action (as discussed in detail in \citealp{mishchenko2011a}). This is seen in (\ref{ex:BK4}) and (\ref{ex:BK5}):

\ea Bashkir (questionnaires) \label{ex:BK4}\\
	\gll Min	däres-kä		kil-de-m			kitap-tǝ    uqə-majənsa.\\
	I		lesson-\textsc{dat}	come-\textsc{pst-1sg}	book-\textsc{acc}    read-\textsc{neg.cvb.att}\\
	\glt `I came to the lesson without having read the book.'
\z

\ea Bashkir (questionnaires) \label{ex:BK5}\\
	\gll Qəð	beje-mä-j			{/ + beje-mäjensä}		malaj	jərla-j.\\
	girl		dance-\textsc{neg-prs}	dance-\textsc{neg.cvb.att}	boy	sing-\textsc{prs}\\
	\glt `The boy is singing, while the girl is not dancing.'
\z

All this shows that the suffixes \textit{‑maθ} and \textit{‑majənsa} are not negative markers proper but rather cumulative markers expressing certain TAME meanings along with negation. They are strictly limited to some specific contexts and therefore do not bear on the focus of this article.

\subsection{Non-verbal negation}\label{sec:BK2.2}

The group of unbound morphemes expressing negation in Bashkir consists of the negative copulas \textit{juq} and \textit{tügel}.\footnote{In an unpublished report on fieldwork \citep{mishchenko2011a}, it is argued that one should postulate two homonymous units \textit{tügel} at the synchronic level, particle and copula, because of the differences in their syntactic functions. However, it is not the only possible interpretation; \textit{tügel} can be regarded as a single polyfunctional negative marker as well. Here we will not touch upon this question and will regard \textit{tügel} in all types of use as one and the same unit.} The first one is a negative existential predicator. The range of meanings it can express includes existential negation (\ref{ex:BK6}), negative “presentative” possession (in the sense of \citealp{hengeveld1992a}), and “indefinite” possession (in the sense of \citealp{stassen2009a}) (\ref{ex:BK7}), alongside negative non-presentative locative meanings \REF{ex:BK8}. Paradigmatically, it is a negative equivalent of the affirmative existential copula \textit{bar}.

\ea Bashkir (questionnaires) \label{ex:BK6}\\
	\gll Aš-həw-ða			öθtäl		juq.\\
	food-water-\textsc{loc}	table		\textsc{neg.ex.cop}\\
	\glt `There is no table in the kitchen.'
\z

\ea Bashkir (questionnaires) \label{ex:BK7}\\
	\gll Mineŋ		mašina-m	juq.\\
	I.\textsc{gen}		car-\textsc{poss.1sg}	\textsc{neg.ex.cop}\\
	\glt `I haven’t got a car.'
\z

\ea Bashkir (questionnaires) \label{ex:BK8}\\
	\gll Mineŋ		kitab-əm		öθtäl-dä		juq.\\
	I.\textsc{gen}		book-\textsc{poss.1sg}	table-\textsc{loc}	\textsc{neg.ex.cop}\\
	\glt `My book is not on the table.'
\z

As shown in \citet{mishchenko2017a}, in the latter meaning \textit{juq} competes with the second negative copula, \textit{tügel}. The choice of a copula depends on the information structure of the clause. If a location is the topic, then the copula \textit{juq} is used, cf. \REF{ex:BK9}. If, by contrast, the topic is a localized object while the location is a comment, then the copula \textit{tügel} will be chosen instead as shown in (\ref{ex:BK10}):

\ea Bashkir (questionnaires) \label{ex:BK9}\\
	\gll Mineŋ		keθä-m-dä				täŋkä		juq.\\
	I.\textsc{gen}		pocket-\textsc{poss.1sg-loc}	coin		\textsc{neg.ex.cop}\\
	\glt \{– That coin I gave you, is it in your pocket? – No,\} `there is no coin in my pocket.'
\z

\ea Bashkir \citep[138]{mishchenko2017a} \label{ex:BK10}\\
	\gll Juq,			min	Räxmät-tä			tügel.\\
	\textsc{neg.ex.cop}	I		Rahmetovo-\textsc{loc}	\textsc{neg.cop}\\
	\glt \{(Talking on a cell phone:) – Hello, where are you, are you in Rahmetovo?\} `– No, I am not in Rahmetovo.'
\z


One can find \textit{tügel} connected with the expression of contrastive focus: for example, ‘Rahmetovo’ in (\ref{ex:BK10}) can be presented as that contrasting current location of the subject, as opposed to other possible locations.\footnote{As Ljuba Veselinova rightly notes, the same is true for its Turkish cognate \textit{değil}.} Since such contrasts are sometimes irrelevant (especially outside a wider context), the negators can be mutually interchangeable:

\ea Bashkir \citep[138]{mishchenko2017a} \label{ex:BK11}\\
	\gll Min	kisä-ge			šəltəra-t-qən-da			äsäj    eš-tä			tügel			ine		{/ juq}				ine.\\
	I		yesterday-\textsc{adj}	ring-\textsc{caus-ptcp.pst-loc}	mother  work-\textsc{loc}	\textsc{neg.cop}		be.\textsc{pst}	\textsc{neg.ex.cop}		be.\textsc{pst}\\
	\glt `Yesterday, when I phoned, my mom wasn’t at work.'
\z

It is also interesting that the copula \textit{juq} can be used for negating predicates referring to age, in which a regular negator is \textit{tügel} \REF{ex:BK12}. But this usage is only possible in a specific situation: when expressing a scalar meaning of unachieved age (\ref{ex:BK13}a). And even in this case, \textit{tügel} is possible, as seen from (\ref{ex:BK13}b):

\ea Bashkir \citep[138]{mishchenko2017a} \label{ex:BK12}\\
	\gll Morat-qa	ös		jäš		tügel,			ä		biš		jäš		Ø.\\
	Murat-\textsc{dat}	three	year	\textsc{neg.cop}		but	five	year	\textsc{cop}\\
	\glt `Murat isn’t three, he is five years old.'
\z


\ea Bashkir \citep[139]{mishchenko2017a} \label{ex:BK13}\\
  \ea
	\gll Min	Öfö-gä	bar-ɣan-da		un		biš		juq			ine.\\
	I		Ufa-\textsc{dat}	go-\textsc{ptcp.pst-loc}	ten	five	\textsc{neg.ex.cop}	be.\textsc{pst}\\
	\glt `When I went off to Ufa, I wasn’t even 15 years old.' (lit. I wasn’t 15 years old)
  \ex
	\gll Min	Öfö-gä	bar-ɣan-da		un		biš		tügel			ine.\\
	I		Ufa-\textsc{dat}	go-\textsc{ptcp.pst-loc}	ten	five	\textsc{neg.cop}		be.\textsc{pst}\\
	\glt `When I went off to Ufa, I wasn’t even 15 years old.' (lit. I wasn’t 15 years old)
\z \z

The capability of \textit{juq} to be used in such “scalar” contexts is understandable if one interprets reaching a certain age as its subsequent existence. Thus, the non-existence of the unachieved age is expressed by means of \textit{juq}, as in (\ref{ex:BK13}a).

To return to existential contexts, it is important to note that the use of the negative existential depends on the tense and referential status of the NP in subject/object position. While \textit{juq} is the only possibility when referring to the present (cf. (\ref{ex:BK6})), with future time references only a verbal strategy with the verb \textit{bul-} ‘to be’ and the standard negator can be used (\ref{ex:BK14}):

\ea Bashkir (questionnaires) \label{ex:BK14}\\
	\gll Donja-la		bal		qort-o		bötön		bul-ma-jasaq.\\
	world-\textsc{loc}	honey	worm-\textsc{poss.3}	whole	\textsc{be-neg-fut}\\
	\glt `Soon there will be no bees in the world.'
\z

For past time references, both copula-based and verbal strategies are employed, depending on the communicative status of the NP in subject/object position:

\ea Bashkir (questionnaires) \label{ex:BK15}\\
	\gll Mindä		at		juq				ine.\\
	I.\textsc{loc}		horse	\textsc{neg.ex.cop}		be.\textsc{pst}\\
	\glt `I didn’t have a horse.'
\z

\ea Bashkir (questionnaires) \label{ex:BK16}\\
    \gll Min	awǝl-da		jäšä-gän-dä		traktor-ǝm  bul-ma-nǝ.\\
    I		village-\textsc{loc}	live-\textsc{ptcp.pst-loc}	tractor-\textsc{poss.1sg} be-\textsc{neg-pst}\\
	\glt `When I lived down in the country, I didn’t have my tractor.'
\z


In terms of the difference between sentences like (\ref{ex:BK15}) and those like (\ref{ex:BK16}), in sentences of the former type there is a generic NP in a subject position, while in sentences of the latter type the NP expressing a possessee is specific and definite.

Thus, in this respect Bashkir should be classified as a language of A~B type in NEC \citep{Veselinova2016}: the negative existential \textit{juq} is used in the present tense (\ref{ex:BK6}) and the SN marker \textit{-ma-} is used in the future (\ref{ex:BK14}), while in sentences with past reference both negators are possible, and the choice depends on the referential status of the subject, as seen in (\ref{ex:BK15}) and (\ref{ex:BK16}).

The main sphere of use of the negative copula \textit{tügel} comprises sentences with a referential predicate, that is, a predicate based on terms (referring to expressions with a nominal head) and larger referential units (predications, propositions, clauses) \citep[77–91]{hengeveld1992a}, such as (\ref{ex:BK17}):

\ea Bashkir (questionnaires) \label{ex:BK17}\\
	\gll Min	jað-əw-sə			tügel			{/ tügel-men.}\\
	I		write-\textsc{nmlz-ag}	\textsc{neg.cop}		\textsc{neg.cop-1sg}\\
	\glt `I am not a writer.'
\z

Example (\ref{ex:BK17}) also illustrates the fact that, as distinct from \textit{juq}, \textit{tügel} can optionally agree with the subject in person and number. All the interviewed speakers considered forms of \textit{tügel} bearing a suffix of person and number agreement to be grammatical, although they do not generate similar forms themselves.

Other meanings of \textit{tügel} include localization at a particular point on the time axis (for example, on a particular date, day of week, or at specific time of day) (\ref{ex:BK18}), inner states and states of environment (\ref{ex:BK19}), and “non-presentative” possession \citep{hengeveld1992a} or “definite” possessive \citep{stassen2009a} (\ref{ex:BK20}) meanings. The copula \textit{tügel} also competes with \textit{juq} when expressing a non-presentative locative meaning and a meaning of age, as was shown above. In present tense affirmative sentences of this type, there is a zero copula.

\ea Bashkir (questionnaires) \label{ex:BK18}\\
	\gll Jəjələš	kisä-gä				tügel			ine,		ä		joma    kön-dö	bul-də.\\
	meeting	yesterday-\textsc{dat}		\textsc{neg.cop}		be.\textsc{pst}	but	Friday  day-\textsc{acc}	be-\textsc{pst}\\
	\glt `The meeting was not yesterday, but on Friday.'
\z

\ea Bashkir (questionnaires) \label{ex:BK19}\\
	\gll Uram-da		həwəq	tügel.\\
	street-\textsc{loc}	cold		\textsc{neg.cop}\\
	\glt `It is not cold outside.'
\z

\ea Bashkir (questionnaires) \label{ex:BK20}\\
	\gll Bəl		urən	hineke				tügel.\\
	this	place	you.\textsc{poss}	\textsc{neg.cop}\\
	\glt `This place isn’t yours.'
\z


Broadly stated, the distribution of negative copulas in the Bashkir non-verbal sentences can be formulated as follows: \textit{juq} negates existence of a generic entity, while \textit{tügel} negates identity between the object and a referential unit. This property of \textit{tügel} manifests in non-sentential use as well. Apart from copulative use, \textit{tügel} can be employed for constituent negation, as, for example, in (\ref{ex:BK21}):

\ea Bashkir (questionnaires) \label{ex:BK21}\\
	\gll Min	ður	tügel			alma		aš-tə-m.\\
	I		big	\textsc{neg.cop}		apple		eat-\textsc{pst-1sg}\\
	\glt `I ate a small [lit. not big] apple.'
\z

The scope of \textit{tügel} following an NP can include only part of it, for example, an adverb, like \textit{bik} ‘very’ in (\ref{ex:BK22}):

\ea Bashkir (questionnaires) \label{ex:BK22}\\
	\gll bik=ük		aqəllə		tügel			keše\\
	very=same	intelligent	\textsc{neg.cop}		person\\
	\glt `not a very intelligent person'
\z

\subsection{Negative copulas in verbal clauses}\label{sec:BK2.3}

What is especially important for the purposes of this article is the capability of negative copulas to be used in some verbal clauses. In Bashkir, the existential copula \textit{bar} is used with the so-called past participle (which in modern Bashkir can be the head of the main – or only – clause) to express experiential meaning (\ref{ex:BK23}) or some perfect meanings. As a negative equivalent of \textit{bar}, \textit{juq} is used in corresponding negative sentences (\ref{ex:BK24}):

\ea Bashkir (questionnaires) \label{ex:BK23}\\
	\gll Min	unǝ		osra-t-qan-ǝm				bar.\\
	I		that.\textsc{acc}	get-\textsc{caus-ptcp.pst-p.1sg}	there.is\\
	\glt \{– Have you ever met my sister? –\} `I have met her.' (roughly `there is my having met her')
\z

\ea Bashkir (questionnaires) \label{ex:BK24}\\
	\gll Unda		ber	qasan=da		bul-ɣan-ǝm		juq {/\textsuperscript{??} bul-ma-ɣan-ǝm.}\\
	that.\textsc{loc}	one	when=\textsc{emph}	be-\textsc{ptcp.pst-1sg}	\textsc{neg.ex.cop}  be-\textsc{neg-ptcp.pst-1sg}\\
	\glt `I have never been there.'
\z

A similar situation is found in other Turkic languages; see examples from Tatar (tat) (\ref{ex:BK25}) and Uzbek (uzb) (\ref{ex:BK26}):

\ea Tatar \citep[126]{poppe1961a} \label{ex:BK25}\\
	\gll Anǝ		hich		kür-gän-em			jük.\\
	that.\textsc{acc}	never		see-\textsc{ptcp.pst-1sg}	\textsc{neg.ex.cop}\\
	\glt `I have never seen him.'\footnotemark
\z

\footnotetext{Here and throughout this section, examples borrowed from grammars are given with our glosses and original translation; the spelling and punctuation of the original are kept.}

\newpage
\ea Uzbek \citep[123]{sjoberg1963a} \label{ex:BK26}\\
	\gll Men	kor-gan-im			yoq.\\
	I		see-\textsc{ptcp.pst-1sg}	\textsc{neg.ex.cop}\\
	\glt `I didn’t see.'
\z

Therefore, in this respect, Bashkir and the other Turkic languages in which negative existentials can be used in similar past tense constructions should be classified as belonging to the B~C type. Hence, Bashkir is situated simultaneously at stage A~B and B~C, depending on which part of its negation system is concerned.

However, it is not only the negative existential \textit{juq} that intrudes into verbal negation. The non-verbal negator \textit{tügel} participates in forming negative finite verbal clauses as well. Thus, in Bashkir, as an alternative to the regular verbal negation of future forms, nominal negation can be used; for example, see:

\ea \label{ex:BK27}
  \ea Bashkir (questionnaires)\\
	\gll Ul		miŋä		aqsa		bir-mä-jäsäk.\\
	that	I.\textsc{dat}		money	give-\textsc{neg-fut}\\
	\glt `He will not give me the money.'
  \ex Bashkir \citep[349]{say2017a}\\
	\gll Ul		miŋä		aqsa		bir-äsäk		tügel.\\
	that	I.\textsc{dat}		money	give-\textsc{fut}	\textsc{neg.cop}\\
	\glt `He will not give me the money.'
\z \z

It is worth noting that the future form \textit{-asak(-jäsäk)} is etymologically a participle \citep[152]{dmitriev1948a}, even though in modern Bashkir it is used almost exclusively as a finite form and has lost the attributive use. The origin of this form thus explains its compatibility with the non-verbal negator \textit{tügel}.

It seems that in Bashkir there is no semantic difference between the two variants, a future form negated by a SN marker or non-verbal negator. However, the situation is different in Turkish (tur). Here, future verbal forms that are negated by nominal negation are interpreted as expressing a kind of higher degree of confidence. Compare neutral (\ref{ex:BK28}a) with the SN marker and emphatic (\ref{ex:BK28}b) with a nominal negation marker:

\ea Turkish \citep[244]{ketrez2012a} \label{ex:BK28}\\
  \ea
	\gll Size		inan-\textbf{ma}-yacağ-im.\\
	you.\textsc{dat}	believe-\textsc{neg-fut-1sg}\\
	\glt `I will not believe you.'
  \ex
	\gll Size		inan-acak		\textbf{değil-im}.\\
	you.\textsc{dat}	believe-\textsc{fut}		\textsc{neg.cop-1sg}\\
	\glt `Of course I will not believe you.'
\z \z

A similar situation of use of nominal negation markers along with SN markers is found in many other Turkic languages, such as Tatar \citep{poppe1961a}, Karachay-Balkar (krc) \citep{seegmiller1996a}, extinct Chagatay (chg) \citep{bodrogligeti2001a} and others. The grammatical descriptions available to us do not always provide a clear description of the difference between these two strategies of negating future forms, and sometimes it is not obvious whether a difference exists at all. However, if any semantic specifics of sentences with nominal negation are mentioned, it is emphatic prominence or a higher degree of speaker confidence that the situation will not take place. This fits well with the observation on the nature of non-verbal negators by \citet{Horn1989}. He notes that it is common for a non-verbal negation marker to be used in verbal clauses in pragmatically marked contexts, for example, for contrastive or metalinguistic negation \citep[446, 451–452]{Horn1989}.

There are also related languages in which the nominal negation marker is used with past verbal forms. An example of this is Tatar where perfect meanings can be conveyed by combinations of a past participle with both existential (\ref{ex:BK30}) and nominal negation (\ref{ex:BK31}). Compatibility of a non-verbal negator with this form is evidently explained by the fact that even when heading a clause, it is still of a nominal origin. Unfortunately, we have no data on how exactly the two constructions are distributed, though the contrast of both to a verbal form with SN, which is aspectually neutral, is evident (see (\ref{ex:BK29})).

\ea Tatar \citep[69]{poppe1961a} \label{ex:BK29}\\
	\gll Men	ešlä-mä-gän-men.\\
	I	work-\textsc{neg-ptcp.pst-1sg}\\
	\glt `I didn’t work.'
\z

\ea Tatar \citep[126]{poppe1961a} \label{ex:BK30}\\
	\gll Anǝ		hich		kür-gän-em			jük.\\
	that.\textsc{acc}	never		see-\textsc{ptcp.pst-1sg}	\textsc{neg.ex.cop}\\
	\glt `I have never seen him.'
\z

\ea Tatar \citep[126]{poppe1961a} \label{ex:BK31}\\
	\gll Jašlǝk		uŋ-gan			tügel.\\
	crop		grow-\textsc{ptcp.pst}	\textsc{neg.ex.cop}\\
	\glt `The crops haven’t grown.'
\z


\subsection{Concluding remarks about nominal negation in Turkic languages}\label{sec:BK2.4}

To sum up, the data of Turkic languages proves that nominal negation can intrude into verbal systems along with the negative existential, filling its own niche. Initially, its compatibility with certain verbal forms is explained by nominal – namely, participial – origin. At the synchronic level, when used with verbal forms, the nominal negation marker is associated with emphatic negation.

\section{Verbal and non-verbal Kalmyk negation}\label{sec:BK3}

\subsection{General description of negation in Kalmyk}\label{sec:BK3.1}

This section starts with a short overview of negation markers in Kalmyk. Modern Kalmyk has six distinct markers to express negation: namely, the preverbal negative particle \textit{esǝ}, the preverbal prohibitive particle \textit{bičä} and the postverbal negation markers \textit{uga} and \textit{bišǝ}, which also have the respective contracted variants \textit{‑go} and \textit{‑šǝ}. For the emergence of these suffixes and the corresponding grammaticalization process, see \citet{baranova2018a}. There is an asymmetry in the morphosyntactic organization of positive and negated predicates in that most of the negated verb forms are participles and converbs, as opposed to the finite verbal suffixes that dominate positive sentences. 

The section further contains a detailed description of the negative copula-like markers \textit{uga} and \textit{bišǝ} in non-verbal and verbal clauses. The preverbal prohibitive particle \textit{bičä} is used with the different imperative forms of a verb, including all second person imperatives and the jussive form \textit{‑txa}.

\ea Kalmyk (Oral Corpus) \label{ex:BK32}\\
	\gll eeǯǝ,				bičä		jumǝ	ke-tn!\\
	grandmmother	\textsc{neg.imp}	thing	do-\textsc{imp.pl}\\
	\glt `Grandma, don’t do anything.'
\z

The preverbal particle \textit{esǝ} is used mostly in subordinate clauses (see \citealp{baranova2019a}).

\ea Kalmyk (NCKL) \label{ex:BK33}\\
	\gll oda	deer-än			en		šin		oborudovani-gə	esə	av-xla,			xöön-nj	öŋgär	ir-š-go-ʁi-nʲ	med-x				kergtä\\
	now	while-\textsc{poss.refl}	this	new	equipment-\textsc{acc}	\textsc{neg}	take-\textsc{cvb.suc}	after-\textsc{poss.3}	for.free	come-\textsc{ptcp.fut-neg-acc-p.3}	know-\textsc{ptcp.fut}	must\\
	\glt `While as of now [we] haven’t received this new equipment, it is important to know that later on it won’t come free of charge.'
\z


In what follows, we discuss the distribution of only two negators, \textit{uga} and \textit{bišǝ}, along with their contracted variants; the other two negators, being unable to negate declarative clauses, are excluded. Both \textit{uga} and \textit{bišǝ}, on the other hand, can function as SN markers. The negation marker \textit{uga} is used in a rich variety of forms to express negation and fulfil different functions, including that of a nominal, existential and verbal negator. The negation marker \textit{bišǝ} functions as an ascriptive negator (in the sense of \citealp{Veselinova2015}) but also intrudes into verbal negation. 

\subsection{\textit{Uga} and \textit{bišǝ} as non-verbal negators}\label{sec:BK3.2}

To understand the current function of the negation markers discussed in this section, it is useful to start with a historical note. The word \textit{uga} derived from \textit{ügei} ‘(there is) not, none’, while the grammaticalization path of \textit{bišǝ} includes the reanalysis of an element \textit{bisi / bišǝ} ‘other’ > ‘other than’ > ‘not the one’ \citep[250–251]{janhunen2012a}. The negation marker \textit{uga} has some nominal properties in that it may take case marking in contexts in which it means ‘absent’, though such forms are not very frequent in Modern Kalmyk. In \REF{ex:BK34}, the form \textit{uga-ʁar} \textsc{neg.cop-ins} may be translated as ‘with lack (of permission)’.

\ea Kalmyk (Oral Corpus) \label{ex:BK34}\\
	\gll xörǝ-n		tavǝ-n	duuna-d			komendant-in		selvǝg	\textbf{uga-ʁar}			madǝn-dǝ	jov-dǝg		\textbf{alʲ}			\textbf{uga}	bää-sǝn		bol-ǯa-na\\
	20-\textsc{ext}	five-\textsc{ext}	kilometer-\textsc{dat}	commander-\textsc{gen}	advise	\textsc{neg.cop-ins}	\textsc{1pl-dat}		go-\textsc{ptcp.hab}	where	\textsc{neg.cop}	be-\textsc{ptcp.pst}	become-\textsc{prog-prs}\\
	\glt `Without the commander’s permission, we could not get out anywhere (more than) 25 kilometres.'
\z


Both markers, \textit{uga} and \textit{bišǝ}, may function as constituent negators, as in the second part of (\ref{ex:BK34}) where \textit{uga} is postposed to the word \textit{alʲ} ‘where/which’ and expresses the spatial reference ‘nowhere’. The negation marker \textit{bišǝ} mostly occurs with words that express attributive or adverbial meaning. It should be noted that Kalmyk adjectives are morphologically similar to nouns. There are a few roots in Kalmyk that in combination with \textit{bišǝ} can be used attributively, such as \textit{sän} ‘good’ versus \textit{sän bišǝ} ‘bad’.

In negative non-verbal sentences, both negation markers correlate with the copula verb \textit{bää-nä} ‘\textsc{be-prs}’, carrying the verbal markers in existential affirmative clauses and locative predicates. In spoken Kalmyk, the copula verb \textit{bää-nä} ‘\textsc{be-prs}’ is often omitted, as in (\ref{ex:BK35}):

\ea Kalmyk (Oral Corpus) \label{ex:BK35}\\
	\gll madn-də,		un-är		temän	uga,	mörə-n,		xö-n,			bod-malə\\
	1.\textsc{pl-dat}		true-\textsc{ins}	camel	\textsc{neg.ex.cop}	horse-\textsc{ext}	sheep-\textsc{ext}	cattle\\
	\glt `Honestly speaking, there are no camels (on our farms), (only) horses, sheep, cattle.'
\z


The negation marker \textit{uga} occurs in non-verbal existential clauses.

\ea Kalmyk (Oral Corpus) \label{ex:BK36}\\
	\gll sän		jumǝ-n		uga\\
	good	thing-\textsc{ext}	\textsc{neg.ex.cop}\\
	\glt `There is nothing good.'
\z


The negation marker \textit{bišǝ} is functionally more diverse; it occurs in non-verbal sentences that negate a quality/attribute (\ref{ex:BK37}) or identity (\ref{ex:BK38}).

\ea Kalmyk (Oral Corpus) \label{ex:BK37}\\
	\gll donta		biš-i\\
	crazy		\textsc{neg-q}\\
	\glt `Isn’t she crazy?'
\z

\ea Kalmyk (Oral Corpus) \label{ex:BK38}\\
	\gll oda	cag-tə		uvəl		uvel		bišǝ\\
	now	time-\textsc{dat}	winter	winter	\textsc{neg}\\
	\glt `Nowadays the winter is not (a real) winter.'
\z


\textit{Bišǝ} can also be used to negate temporal localization (\ref{ex:BK39}).

\ea Kalmyk (questionnaires) \label{ex:BK39}\\
	\gll asx-na			hotə	dolan		čas-la		bišǝ\\
	evening-\textsc{gen}	meal	seven		hour-\textsc{com}	\textsc{neg}\\
	\glt `The dinner is not at 7 o’clock (it will be at 8 p.m.).'
\z


The negation marker \textit{bišǝ} also occurs in the construction ‘not only … but also’; here it is an expletive or pleonastic negation (that is, a marker of negation without negative meaning, according to \citet[126]{Horn2010}.

\ea Kalmyk (Oral Corpus) \label{ex:BK40}\\
	\gll terǝ	ʁancxǝn		gergǝ-nj		bišǝ			alʲ		ezǝvltǝ	ol-ǯǝ				av-čǝ\\
	that	single		wife-\textsc{poss.3}		\textsc{neg.prs}		or		property	find-\textsc{cvb.ipfv}	take-\textsc{evid}\\
	\glt `He has not only obtained his wife but also a kingdom.'
\z


There is the context of locative predication where both negation markers occur. Example (\ref{ex:BK41}a) may be negated either by the existential negator \textit{uga} or the negation marker \textit{bišǝ}. The example (\ref{ex:BK41}b) with marker \textit{uga} means that whatever is designated by the nominal in object position cannot be found under the table, so with a negative existential the object nominal receives a generic reading. In the variant in (\ref{ex:BK41}c) with the non-verbal negation marker \textit{bišǝ}, ‘the ball’ is interpreted as specific and definite and it denies its particular location. Similar competition between a non-verbal negator and 'an existential negator that depends on focus can also be found in Slavonic languages (see \citealp[197]{Veselinova2010}).

\ea Kalmyk (questionnaires) \citep[14]{baranova2015a} \label{ex:BK41}\\
  \ea
	\gll širä	doorǝ		mečik		bää-nä\\
	table	under		ball		be-\textsc{prs}\\
	\glt `There is a / the ball under the table.’ / ‘The ball is under the table.'
  \ex
	\gll mečik		širä		doorǝ		uga\\
	ball		table		under		\textsc{neg.ex.cop}\\
	\glt `There is no ball under the table.'
  \ex
	\gll mečik		širä	doorǝ		bišǝ\\
	ball		table	under		\textsc{neg.prs}\\
	\glt `The ball is not under the table.'
\z \z

As for the formal properties of the negators \textit{uga} and \textit{bišǝ} in non-verbal clauses, negation markers function as a copula which stands at the end of the clause and may carry personal verbal affixes (i.e. as \textit{bišǝ} in example (\ref{ex:BK42})). It should be noted that regular predicative adjectives or nouns cannot take such suffixes.

\ea Kalmyk (KNC) \label{ex:BK42}\\
	\gll bi				čon	bišə-v\\
	\textsc{1.sg.nom}	wolf	\textsc{neg-1sg}\\
	\glt `I’m not a wolf.'
\z


In a similar way, the negation marker \textit{uga} may receive a personal marker. It also can take an indirect evidential marker \textit{=č} (which clitisized from a former copula).

\newpage
\ea Kalmyk (Oral Corpus) \label{ex:BK43}\\
	\gll xojr	kövü-n		xojr	küükǝ-n	däkčǝ		tedn-ä	ǯil-in			küük-tǝ	uga=č\\
	two	boy-\textsc{ext}		two	girl-\textsc{ext}	again		\textsc{3pl-gen}	year-\textsc{gen}		child-\textsc{pl}	\textsc{neg.cop=evid}\\
	\glt `There are 2 boys and 2 girls and it looks like there is no more children with that year (of birth).'
\z


It should be stressed again that the marker \textit{uga} combines properties of a noun and a copula: when used in nominal negation, it may undergo nominal declination, while when used as a copula-like negation marker it combines with personal verbal affixes or the evidential clitic, which normally could be added to verbs.

Thus, to summarize, \textit{bišǝ} functions as an ascriptive negator in non-verbal predications, while \textit{uga} in non-verbal clauses states the absolute absence of the predicated entities. In the next section, we examine these markers in the function of verbal negators.

\subsection{Negative copulas in verbal clauses}\label{sec:BK3.3}

The main focus of this article is the capability of negation markers which originated as non-verbal negation to be used in some verbal clauses. The existential negator \textit{uga} has extended to verbal clauses with non-finite verbal forms, including the past participle ending in \textit{-sǝn} (it occurs as a counterpart for a form of past tense ending in \textit{-v} in affirmatives), an anterior converb ending in \textit{-ad} (for remote past ending in \textit{-la}) and some other, more rare forms. The combination of the non-finite form with the negative copula \textit{uga} is the neutral and only way to negate past-referring forms.

\ea Kalmyk (Oral Corpus) \label{ex:BK44}\\
	\gll däkčǝ zarʁ-dǝ od-sǝn uga\\
	again court-\textsc{dat} go-\textsc{ptcp.pst} \textsc{neg.cop}\\
	\glt `He hasn’t gone to court again.'
\z

\ea Kalmyk (Oral Corpus) \label{ex:BK45}\\
	\gll a\footnotemark madǝn tüü-n-dǝ tas mu ke-ʁäd uga-vidn\\
	but \textsc{3.sg} that-\textsc{ext-dat} very bad do-\textsc{cvb.ant} \textsc{neg.cop-1pl}\\
	\glt `But we did not do anything awfully bad to him.'
\z

\footnotetext{Russian adversative conjunction.}

There are several ways to negate verbs with non-past time reference: the copula-like \textit{bišǝ} and the negative affixes \textit{-go} and \textit{-šǝ}. \textit{Bišǝ} occurs with two participles: the future participle ending in \textit{-x} and the habitual participle ending in \textit{-dəg}. The etymologically participial forms in Modern Kalmyk can be used predicatively but tend to co-occur with an affirmative affix or clitic \textit{-n/mən} (derived from \textit{mön} 'same'). Even with the future participle, \textit{bišǝ} negates the present states. When \textit{bišǝ} occurs in the verbal predication with a future participle or a habitual participle, it has a modal component of meaning (obligation or permission) or leads to an emphatic expression. It should be noted that the negation marker \textit{bišǝ} with the habitual participle and the future participle predominantly co-occurs with the affirmative affix \textit{-n}, and the modal meaning may come from the combination of \textit{-ǝm bišǝ}.

\ea Kalmyk (KNC) \label{ex:BK46}\\
	\gll tednä		tuskar	mart-x-mǝn			bišǝ\\
	\textsc{3pl.gen}	about		forget-\textsc{ptcp.fut-aff}	\textsc{neg.prs}\\
	\glt `One should not forget about them.'
\z

\ea Kalmyk (questionnaires) \label{ex:BK47}\\
	\gll sään	kövü-d	tii-gd-ǝm				bišǝ!\\
	good	boy-\textsc{pl}	do.so-\textsc{ptcp.hab-aff}	\textsc{neg.prs}\\
	\glt `Good boys do not act like this!'
\z


Another way of expressing SN with a non-past time reference in Kalmyk is the use of contracted suffixes that emerged from shortening the full negation markers during the grammaticalization process. The affixes are mostly used interchangeably with the whole negation markers \textit{uga} and \textit{bišǝ}, but there is a tendency in the distribution.

The affix \textit{‑šǝ} is mostly used within the negated form of the future participle in \textit{–x}, while the affix \textit{‑go} occurs with the habitual participle in \textit{-dǝg}, with the deverbal affix in \textit{-l} or modal converb in \textit{-l}, with the anterior converb in \textit{-ad} and, most frequently, with the affix \textit{-š} \citep[13]{baranova2018a}, which is usually considered as a future participle allomorph with changing \textit{x>š} before the negation marker \textit{-go}. There is another understanding of an element \textit{-š} before the negation marker as the deverbal affix (\citep[181-182]{janhunen2012a}). Interestingly, the future participle ending in \textit{-x} with affix \textit{‑šǝ} expresses the negation of present tense (\ref{ex:BK48}), whereas the affix \textit{‑go} with the same participle, as in (\ref{ex:BK49}), has a more straightforward meaning in that it tends to negate a future event.

\newpage
\ea Kalmyk (Oral Corpus) \label{ex:BK48}\\
	\gll nan-ta			xamdan	\textbf{unt-ča-x-šǝ}\\
	1.\textsc{sg-assoc}	together	sleep-\textsc{prog-ptcp.fut-neg}\\
	\glt `(She) is not (at this moment) sleeping together with me.'
\z

\ea Kalmyk (Oral Corpus) \label{ex:BK49}\\
	\gll meklä-tä		us-ar			xotǝ	ke-ǯǝ	\textbf{bol-š-go}\\
	frog-\textsc{assoc}	water-\textsc{ins}	food	do-\textsc{cvb.ipfv}	become-\textsc{ptcp.fut-neg}\\
	\glt `(She/he) cannot cook with water containing frogs.'
\z

The negation marker \textit{-güi} or other contracted variants from the cognate \textit{ügei} (Kalmyk \textit{uga}) have intruded into verbal negation in all Central Mongolic languages, and it is also attested in some Southern Mongolic languages \citep[70--81]{brosig2015a}, thus suggesting a development that already started in the Middle Mongol period. But the extension of the ascriptive negator \textit{bišǝ} and its contracted form \textit{‑šǝ} into SN is only attested for Kalmyk/Oirat, Southern Mongolic Bonan (peh; \citealp{wu2003a}) and Central/Southern Mongolic Eastern Shira Yughur (yuy; \citealp{nugteren2003a}). In Bonan and Eastern Yughur, the marker \textit{-ši} attaches to the future participle, too, but then negates future events. Only in Kalmyk/Oirat does the combination of the future participle plus \textit{‑šǝ} have present time reference. So, the first steps in the grammaticalization of the existential negator \textit{ügei} took place in several related languages, including all of Central Mongolic, while the second variant (from \textit{bišǝ}) developed only in OiratEastern Yugur and Bonan, which as a Southern Mongolic language should only have participated in this innovation if at some historically indeterminate point in the past it was spoken many hundred kilometres north-west of its current position in the south-eastern Qinghai province of China. Thus, the non-compositional present tense meaning with the future participle developed only in Kalmyk/Oirat.

In Kalmyk, the marker \textit{bišǝ} is less frequent than the negative copula \textit{uga}, as shown in \tabref{tab:BK1}. The frequency of the negative affix \textit{-šǝ}, in turn, is also lower than the affix \textit{-go}.

\begin{table}[!h]
\caption{Frequency of \textit{bišǝ} and \textit{uga} and negation affixes in the written corpora.}
\label{tab:BK1}
\resizebox{\textwidth}{!}{
\begin{tabular}{lrrrr}
\lsptoprule
 & \textit{bišǝ}~ & \textit{-šǝ}~ & \textit{uga}~ & \textit{-go}~ \\
 \midrule
NCKL & 17\,321 (2\,156 ipm) & 18\,288 (2\,286 ipm) & 6\,7649 (8\,456 ipm) & 42\,260 (\,5282 ipm)
\\
KNC & 1\,771 (2\,213 ipm) & 970 (1\,212 ipm) & 6\,390 (7\,987 ipm) & 2\,330 (2\,912 ipm) \\ \lspbottomrule
\end{tabular}
}
\end{table}

Nevertheless, the less frequent negation marker also can be used in some verbal clauses. So, the frequency of use itself is not the sole determining factor for the expansion into the verbal domain. This nominal negator intruded into the verbal system along with the negative existential marker. Analogically to the contracted variant \textit{-go} from \textit{uga}, it developed into an affix which is now used as SN. This requires some comments about SN in Middle Mongolian and its diachronic development.

\subsection{The historical development of Mongolic and Old Kalmyk negation}\label{sec:BK3.4}

Two different types of negatives in Mongolic languages are distinguished according to their position. \citet[3]{yu1991a} called this the principle of “preverbal and postnominal” negativity marking in Mongolic. The original and (in terms of 13th-century synchronic morphology) non-derived verbal negators \textit{ese} and \textit{ülü} were used in Middle Mongolian (xng, 13th–15th centuries) and consistently placed before the predicate. Similarly, the prohibitive particles \textit{bü / bütügei} (> Modern Kalmyk \textit{bičä}) always preceded the imperative form of a verb. On the other hand, the nominal negators \textit{ügei, busu / busi} and \textit{üdü’üi} in Middle Mongolian were placed after the word that they negated. The main transformation of this negative system in Mongolic was the gradual replacement of the preverbal particles by the extension of the use of the existential negator \textit{ügei} to verbal clauses (see esp. \citealp{yu1991a, brosig2015a}). This functional extension of \textit{ügei} to verbal negation occurred with converbal and participial forms and is thus cross-linguistically typical for negative existentials intruding into the SN domain through their use with nominalized verbs, as stated by \citet[155]{Veselinova2016}.

The negation marker \textit{ügei} took over SN. In terms of Croft’s model \citeyearpar{Croft1991}, Central Mongolic languages including Kalmyk belong to type C, while in most other Mongolic languages an existential negation marker makes inroads into verbal negation (stage B~C) (as summarized in \citealp[128]{brosig2015a}. Thus, the SN marker in most modern Central Mongolic languages developed from existential negation. But in addition to this process of expansion of negative existential negation into verbal negation, another grammaticalization process has taken place that involves the simultaneous extension of the ascriptive negator \textit{bišǝ} into SN. That will be examined in the rest of this section.

In “The Secret History of the Mongols” (13th century), there are the words \textit{busu} and \textit{buši} ‘other’, which \citet[134]{yu1991a} states were just phonological variants, with a prevalence of \textit{busu} in early MM texts. Most contemporary Mongolic languages have inherited some variants of the item \textit{buši > biši}.

The grammaticalization of the element \textit{busu} ‘other’ into a negator of nouns starts in late MM (from the 14th century). \citet[127]{yu1991a} mentions that as a lexical word ‘other’ it precedes the NP, while as a negation marker \textit{busu} always follows the NP. In postnominal position, \textit{busu} negates nouns, as in example \REF{ex:bashkir:50}:


\ea\label{ex:bashkir:50} MM: Twelve deeds of the Buddha F40v (\citealt[141]{poppe_1967} glossed by \citealt[105]{brosig2015a}) \label{ex:BK50}\\
	\gll ene				mör	ber	jobalang-i		maγad	γar-γa-n			cida-qu			mör	\textbf{busu}			bu-i\\
	\textsc{dem.prox}	path	\textsc{foc}	suffering-\textsc{acc}	certain	exit-\textsc{caus-cvb}	can-\textsc{npst.ptcp}	path	\textsc{asc.neg}		\textsc{cop-prs}\\
	\glt `This path is not the path that can save from suffering.'
\z

In Old Written Kalmyk (17th–19th centuries), there was a negation marker \textit{biši} and sometimes \textit{bišai}. It was used in two ways: in some texts, \textit{biši} functions as a constituent negator, while in others it is a copula in non-verbal sentences.

In the letters written by Kalmyks in Kalmyk to Isaac Jacob Schmidt, the missionary and Mongolist who lived among that people between 1804 and 1806 \citep{krueger2002a}, in particular, it functions as a constituent negator which always directly follows the negated noun (i.e. has scope only over it). In 19 out of 23 examples in these manuscripts, it occurs as part of the formula \textit{erke biši} capricousness \textsc{neg} ‘inadvertently, at once, immediately, without fail’. So, this usage looks similar to other expressions with a comparable meaning with negation \textit{ügei}, which are more common, including a construction that is very typical for mail: \textit{udal ügei} take.time-\textsc{vrbl neg} ‘without delay’ \citep[57]{krueger2002a}.

Apart from this construction, the negation marker \textit{biši} occurred in non-verbal predication.

\ea Old Written Kalmyk \citep[27, Letter 9, lines 7–8]{krueger2002a} \label{ex:BK51}\\
	\gll ike		sayin		mör-in		bišai		bilei\\
	very	good		horse-\textsc{gen}	\textsc{neg.prs}	be.\textsc{pst}\\
	\glt `This is not [a sign] of a very good horse.'
\z

In a stage of evolution that is probably subsequent to this stage, the negation marker \textit{biši} occurs in verbal predication. In some less known manuscripts, such as the letters of Donduk-Dashi (1741–1761) \citep{kokshaeva2011a}, there are some interesting cases of the use of \textit{biši} with participles and other verbal forms.

\newpage
\ea Old Written Kalmyk \citep[167]{kokshaeva2011a} \label{ex:BK52}\\
	\gll ünen	xudal-iyini	labla-ji			mede-kü			biši\\
	true	lie-\textsc{poss.3}	specify-\textsc{cvb.ipfv}	know-\textsc{ptcp.fut}	\textsc{neg.prs}\\
	\glt `We do not know if it is true or false.'
\z


The later manuscript “Legend about pilgrimage to Tibet Baaza-bakshi from Maloderbet” (from the late 19th or early 20th century) contains an example in which \textit{bišǝ} occurs with a past participle in \textit{-gsan} \REF{ex:BK53}, which more commonly was negated by \textit{ügei} \REF{ex:BK54}.\footnote{In modern Kalmyk, the form of the past participle ending in \textit{-sən} normally occurs with the negation marker \textit{uga} or its contracted version \textit{-go}.} The form \textit{-gsan bišǝ} is semantically rather different from \textit{-gsan ügei}: it has an emphatic meaning with more broad scope and describes the negation of a presupposition, while the form \textit{-gsan ügei} just negates an event designated by a verb; see (\ref{ex:BK53}) and (\ref{ex:BK54}):

\ea Old Written Kalmyk \citep[103]{bembeev2004a} \label{ex:BK53}\\
	\gll sedkil-d-e:n		sana-qu-du		maniyi-gi	küün	küči:r	\textbf{yabu-ulu-gsan}	\textbf{biši}\\
	heart-\textsc{dat-p.refl}	think-\textsc{ptcp.fut-dat}	\textsc{3pl-acc} 	man	by.force	go-\textsc{caus-ptcp.pst}	\textsc{neg}\\
	\glt `(Although I am suffering so much I always) remember in my heart that that it was not so that a person sent us by force.'
\z

\ea Old Written Kalmyk \citep[103]{bembeev2004a} \label{ex:BK54}\\
	\gll beye-ni cu-γa:r šarrqu bol-o:d od-bai ge-be čigi \textbf{ükü-gsen} \textbf{ügei}\\
	body-\textsc{poss.3} all-\textsc{ins} wound become-\textsc{cvb.ant} go-\textsc{pst} say-\textsc{pst} \textsc{conc} die-\textsc{ptcp.pst} \textsc{neg.ex.cop}\\
	\glt `Although (our camels) went there becoming covered with wounds, they did not die.'
\z


As shown in Sections \ref{sec:BK3.2}-\ref{sec:BK3.3}, in Modern Kalmyk, the marker \textit{bišǝ} is used as both a verbal and non-verbal negator. Compared to closely related languages, it looks more frequent than in Khalkha Mongolian, where it is used mostly in nominal sentences \citep[123--125]{yu1991a} or it is relatively seldom used as a verbal negator with participles \citep[251]{janhunen2012a}.

The last point of interest is the emergence of the affixes \textit{-go} and especially \textit{-šə} from \textit{uga} and \textit{bišə}. In the absence of reliable data, it is hard to identify the period when the development of the contracted form and its dissemination in spoken Oirat took place. In the Old Kalmyk texts of the 18th–19th centuries which we analysed, there are no negative affixes. Probably they were used in the spoken Kalmyk of that time. According to \citet[114]{bembeev2004a}, there is a trace of the use of negative affixes in a manuscript on the verge of the 20th century.


\subsection{Concluding remarks about Kalmyk negation}\label{sec:BK3.5}

To sum up, the Kalmyk data shows a co-existence of two negation markers functioning on a synchronic level, both in non-verbal clauses and in verbal predication with non-finite form. Both markers developed a contracted form which is restricted to verbal clauses. The evolution of an existential negation into the verbal negation is typical of the NEC, and the negation marker \textit{ügei} follows the cline. At the same time, Kalmyk is arguably particularly suited for the discussion, because its other non-verbal negator, \textit{bisə}, evolved into a SN marker as well.

The item \textit{bišǝ} developed in Middle Mongolian from an adjective \textit{busu / bisu} ‘other’ into a negator for non-verbal negation. Thus, in Old Written Kalmyk, the negation marker \textit{bišǝ} is used as a constituent negation with nouns and in non-verbal sentences. Then, it occurs with participles, in particular with future participles and occasionally with a past participle. So, the negation marker \textit{bišǝ} extends its function and intrudes into the verbal negation. It is frequent in Kalmyk (possibly more so than in other Central Mongolic languages); unlike in most of those other languages, there it has grammaticalized to the affix of SN \textit{-šǝ} in parallel with the contraction of \textit{ügei / uga} to \textit{‑go}.

\section{Other cases of non-verbal negators developing into SN markers}\label{sec:BK4}

\subsection{General notes}\label{sec:BK4.1}

The penultimate section of the paper deals with other possible types of evolution of nominal negation and its inroads into the domain of SN in some other, unrelated languages. As mentioned in \sectref{sec:BK1}, relatively little attention has been paid to the cross-linguistic description of non-verbal negation markers that have intruded into the verbal clause. The selection of cases was partly determined by the data available to the authors. Due to the limited number of sources, we focus on two aspects of the evolution of originally nominal negation to verbal negators: their use as emphatic negative markers and their compatibility with future verbal predicates.

\sectref{sec:BK4.2} deals with emphatic negation as found in Bornean languages. \sectref{sec:BK4.3} treats the asymmetric use of negators or, more precisely, cases when nominal negators “occupy” future tense negation, like in Bashkir (see \sectref{sec:BK2} above), as exemplified by Egyptian Arabic.

\subsection{Bornean languages}\label{sec:BK4.2}

According to \citet{kroeger2014a}, Bornean languages,\footnote{“Bornean languages” are a group of Austronesian languages clustered according to a geographic principle. These include languages spoken on Borneo (Kalimantan), an island divided between Malaysia, Indonesia and Brunei.} especially Malayic Dayak and languages of the Northeast Borneo subgroup, distinguish rather consistently between nominal and non-nominal negation. That is, different negators are used for verbal and adjectival predicates on the one hand and nominal predicates on the other, as in the following examples from Malay (ind, zsm; \citeauthor{kroeger2014a} gives examples from Standard Malay and from Indonesian, labelling both of them as “Malay”):

\ea Malay (\citealp[195]{sneddon1996a}, cit. by \citealp[1]{kroeger2014a}) \label{ex:BK55}\\
	\gll Mereka	tidak	menolong	kami.\\
	\textsc{3pl}			\textsc{neg}	help			\textsc{1pl.excl}\\
	\glt `They didn’t help us.'
\z

\ea Malay (\citealp[88]{sudaryono1993a}, cit. by \citealp[1]{kroeger2014a}) \label{ex:BK56}\\
	\gll Saya	tidak	lapar.\\
	\textsc{1sg}	\textsc{neg}	hungry\\
	\glt `I am not hungry.'
\z

\ea Malay (\citealp[195]{sneddon1996a}, cit. by \citealp[1]{kroeger2014a}) \label{ex:BK57}\\
	\gll Dia	bukan	/*tidak	guru.\\
	\textsc{3sg}	\textsc{neg}		\textsc{neg}		teacher\\
	\glt `She isn’t a teacher.'
\z


Thus, in Malay \textit{tidak} is used with verbal (\ref{ex:BK55}) and adjectival (\ref{ex:BK56}) predicates, but it cannot be used with nominal predicates (\ref{ex:BK57}) – this function is fulfilled by \textit{bukan}. The latter can also be used with verbs, but only if there is any emphatic meaning in the sentence, like in (\ref{ex:BK58}):

\ea Malay (\citealp[145]{asmahhj1982a}, cit. by \citealp[1]{kroeger2014a}) \label{ex:BK58}\\
	\gll Dia	bukan	tidur	tetapi		ber-baring		sahaja.\\
	\textsc{3sg}	\textsc{neg}		sleep	but		mid-lie.down	only\\
	\glt `He is not sleeping, but only lying down.'
\z

The same is true for most Bornean languages: the nominal negation marker, which normally does not negate verbs and adjectives, can be used in verbal clauses to mark emphatic negation. Specific types of “emphatic” uses may include contrastive negation \REF{ex:BK59b}, contradiction of a proposition that has been asserted or could be assumed \REF{ex:BK60b}, focus marking of an argument (\ref{ex:BK61}b), or just unspecified emphasis (\ref{ex:BK62}b). The (a) cases in each pair of examples illustrate the use of the correspondent negator for negating nominal predicates.


\ea Mualang (mtd) (\citealp[examples 9–102, 110]{tjia2007a}, cit. by \citealt[5]{Kroeger2014}) \label{ex:BK59}\\
  \ea\label{ex:BK59a}
	\gll Ia’		ukay	uma			ku.\\
	that	\textsc{neg}	rice.field	\textsc{1sg}\\
	\glt `That is not my rice field.'
  \ex\label{ex:BK59b}
	\gll Ku		ukay	pulay.		Baru’		ka’	angkat.\\
	\textsc{1sg}	\textsc{neg}	go.home		just		\textsc{fut}	go\\
	\glt `I am not going home; I am just about to leave.'
\z \z

\ea Kimaragang Dusun (kqr) \citep[7--8]{kroeger2014a} \label{ex:BK60}\\
  \ea\label{ex:BK60a}
	\gll Kada		matagur,	okon.ko’		tidi		ku			ika!\\
	don’t		scold			\textsc{neg}			mother	\textsc{1sg.gen}	\textsc{2sg.nom}\\
	\glt `Don’t scold me, you are not my mother!'
  \ex\label{ex:BK60b}
	\gll Okon.ko’	bobogon			dialo		ilot	tanak		yo	dat		maanakaw,		suuon		nogi.\\
	\textsc{neg}			beat.\textsc{ov}			\textsc{3sg}		that	child		\textsc{3sg}	\textsc{rel}	steal.hab		order.\textsc{ov}		\textsc{prt}\\
	\glt `He doesn’t beat that child of his who keeps on stealing, he actually orders / sends him (to steal)!'
\z \z

\ea Timugon Murut (tih) (example a in \citealp[10]{brewis1988a}, cit. by \citealp[8]{kroeger2014a}); (example b in \citealp[612]{brewis2004a}, cit. by \citealp[8]{kroeger2014a}) \label{ex:BK61}\\
  \ea
	\gll Sala’=ka	lalaing	ku				io.\\
	\textsc{neg=prt}	child		\textsc{1sg.gen}		\textsc{3sg.nom}\\
	\glt `He is not my child.'
  \ex
	\gll Sala’=ka	aku			mangansak		ra		kaluu’.\\
	\textsc{neg=prt}	\textsc{1sg.nom}	cook				\textsc{acc}	rice\\
	\glt `\textit{I} didn’t cook rice.'
\z \z

\ea Tatana’ (txx) (\citealp[6, 44]{chan1989a}, cit. by \citealp[11]{kroeger2014a}) \label{ex:BK62}\\
  \ea
	\gll Loin	ko			disio	baloi		dino.\\
	\textsc{neg}	\textsc{prt}	his		house		that\\
	\glt `That house is not his.'
  \ex
	\gll Loin	ko			idagang	ku				anak	ku.\\
	\textsc{neg}	\textsc{prt}	be.sold	\textsc{1sg.gen}		child	\textsc{1sg.gen}\\
	\glt `It’s not like I’m selling my child.’ (said during bride-price negotiations)'
\z \z

Interestingly, Bornean languages also provide an example of the nominal negator being used as a part of double negation (together with SN) conveying a positive meaning. In Begak-Ida’an (dbj), \textit{(a)pon}\footnote{This occurs along with \textit{(n)inga’}, another SN marker. The author discusses the subtle differences between the two forms; however, they seem to be irrelevant for the purposes of the present study.} serves as a SN marker (\ref{ex:BK63}). The nominal negator is \textit{pǝngka}, a contracted form of \textit{(a)pon} with a discourse particle \textit{ka}\footnote{As \citet[304]{goudswaard2005a} states, “The combination \textit{pon ka} is most of the times pronounced as \textit{pǝngka} rather than as \textit{pon ka}: the vowel /o/ of \textit{pon} being reduced to schwa.”} (\ref{ex:BK64}). In sentences with double negation like (\ref{ex:BK65}), both \textit{(a)pon} and \textit{pǝngka} are used, where \textit{pǝngka} serves as the first negative marker with scope over the SN marker \textit{(a)pon} as the second negator.

\ea Begak-Ida’an (\citealp[300]{goudswaard2005a}, cit. by \citealp[15]{kroeger2014a}) \label{ex:BK63}\\
	\gll Siti		apon	mangan	bakas.\\
	Siti		\textsc{neg}	eat		wild.pig\\
	\glt `Siti does not eat pork.'
\z

\ea Begak-Ida’an (\citealp[304]{goudswaard2005a}, cit. by \citealp[15]{kroeger2014a}) \label{ex:BK64}\\
	\gll Ino			pa		asu	matay,	pon.ka\footnotemark	anak	mo.\\
	yonder	\textsc{prt}	dog	dead		\textsc{neg}		child	\textsc{2sg.gen}\\
	\glt `This is a dead dog hey, this is not your child.'\footnotemark
\z

\footnotetext{In the original work \citep{goudswaard2005a}, this unit is written separately as two words, \textit{pon ka}, according to its interpretation as a combination of the marker of sentential negation \textit{(a)pon} with a discourse particle \textit{ka}}

\footnotetext{Kroger’s translation of this sentence is not clear, but it is supposed to convey a general “emphatic” meaning, as in other cases.}

\ea Begak-Ida’an (\citealp[305]{goudswaard2005a}, cit. by \citealp[16]{kroeger2014a}) \label{ex:BK65}\\
	\gll aku	pǝngka	pon	atow		muli,		aku	atow, ...\\
	\textsc{1sg}	\textsc{neg}		\textsc{neg}	know		return	\textsc{1sg}	know\\
	\glt `It is not the case that I do not know how to go home, I do know.' (lit. I do not not know (how) to go home, I do know)
\z


On the one hand, such use of nominal negation markers is just a logical extension of their emphatic use and/or their compatibility with focus markers. But at the same time, they demonstrate that nominal negators are considered by the speaker as an additional opportunity to express verbal negation when another means has already been employed. It seems that this opens the door for subsequent expansion of nominal negative markers into the verbal system.

To sum up, according to the current information, in almost all the Bornean languages nominal negators can be used to negate verbal clauses.\footnote{Exceptions seem to be very few; \citet{kroeger2014a} mentions Tombonuwo.} However, their use with verbal predicates is limited to pragmatically marked contexts. There is some parallelism with the Turkic data discussed in \sectref{sec:BK2}, where nominal markers can be used with some specific verbal forms along with SN markers, bringing emphatic meaning, and more broadly with typological observations on the development of negative markers, such as that by \citet{Horn1989}, which testifies to the tendency for non-verbal negators to be used in verbal clauses for contrastive and narrow focus negation.

\subsection{Egyptian Arabic}\label{sec:BK4.3}

Egyptian Arabic (arz) and particularly its Cairene dialect differ from Standard Arabic in several domains of grammar, including negation. In Egyptian Arabic, there is a negative particle \textit{muš}, which negates the nouns and adjectives which it precedes. This item is a result of grammaticalization of a negative particle plus a word meaning ‘thing’. \citet{wilmsen-a} shows the broader context of negation \textit{muš} in Arabic languages. It covers the meanings of negation of identification and attribution.

\ea Egyptian Arabic (Ramazan Mamedshakhov, pers. commun.) \label{ex:BK66}\\
  \ea
	\gll huwwa	muš	tˤaalib	huwwa	farraaf\\
	\textsc{3.sg}		\textsc{neg}	student	\textsc{3.sg}		messenger\\
	\glt `He is not a student, he is a messenger.'
  \ex
	\gll ʔil-beet=da			muš	kibir\\
	\textsc{def}-house=\textsc{dem}	\textsc{neg}	big\\
	\glt `This house is not big.'
\z \z

\textit{Muš} as used in (\ref{ex:BK66}b) is not an existential negator, for which Egyptian Arabic employs a circumflex or doubled negative markers on the locative word \textit{fii} ‘in’, as in (\ref{ex:BK67}a). It differs from Standard Arabic, which uses a particle \textit{laa} ‘no’, as in (\ref{ex:BK67}b) for existential negation, which is also used as a SN for present tense events \citep[232]{gadalla2000a}.

\ea Egyptian and Standard Arabic \citep[234]{gadalla2000a} \label{ex:BK67}\\
  \ea
	\gll ma-fii-š		riggaala		fi-l-madiin-a\\
	\textsc{neg}-in-\textsc{neg}	men			in.the-city\\
	\glt `There are no men in the city.'
  \ex
	\gll laa	rijaal-a		fi-l-madiin-at-(i)\\
	no		men-\textsc{acc}	in.the.city-(\textsc{gen})\\
	\glt `There are no men in the city.'
\z \z

SN in Egyptian Arabic is expressed by the combination of the proclitic \textit{ma-} and affix \textit{-š} (it also can be treated as a circumflex / discontinuous morpheme; see \citealp[234]{gadalla2000a}), as in (\ref{ex:BK68}a). This combination of negative markers occurs in the past and present, but the only way to express verbal negation in the future is via the nominal negator \textit{muš}, which co-occurs with the finite form of the future tense,\footnote{It should be noted that future tense in Egyptian Arabic is made up of two elements: the preterite of the grammaticalized verb \textit{raaħ} ‘to go’ together with a verb in the present tense: e.g. \textit{raaħ jiktib} ‘(he) will write’.} as in (\ref{ex:BK68}b):

\ea Egyptian (Cairine) Arabic (Ramazan Mamedshakhov, pers. commun.) \label{ex:BK68}\\
  \ea
	\gll ʔil-raagil=da	ma=rga<a>ˤ-ø-š\\
	\textsc{def}-man=\textsc{dem}	\textsc{neg}=return<\textsc{pfv}>return-\textsc{3sg.m-neg}\\
	\glt `This man did not return.'
  \ex
	\gll ʔil-raagil=da	muš	ha-ji-rga<a>ˤ-Ø\\
	\textsc{def}-man=\textsc{dem}	\textsc{neg}	\textsc{fut-3sg.m}-return<\textsc{ipfv}>return-\textsc{3sg.m}\\
	\glt `This man will not return.'
\z \z

Other examples of the negator \textit{miš / muš} as verbal negation in Egyptian Arabic can be found in \citet[93--94]{wilmsen-a}. He states that \textit{miš / muš} with a verb instead of \textit{mā- … -š} has pragmatic meanings, such as rhetorical or metalinguistic negation \citet[94]{wilmsen-a}. Meanwhile, our data – as well as examples from Wilmsen’s paper itself – show that in some cases it is a quite neutral way of negating future events.

\subsection{Discussion of the typological context}\label{sec:BK4.4}
\largerpage
The use of nominal negation markers with verbal predicates can be conditioned pragmatically, as can be seen from the data of the Bornean languages, or grammatically, as in Egyptian Arabic. The reasons for the first type of use seem to be more understandable: if a language possesses several negative markers, one can expect that the use of a marker originally supposed to be used as a non-verbal negation marker in a new context will be pragmatically marked – and, indeed, there are examples of this among the world’s languages. In contrast, the possibility of grammatical motivation towards the use of nominal negation markers requires explanations. We suggest that one such explanation may be in the nature of certain grammatical meanings (and, consequently, forms) that makes them “predisposed” to become negators other than standard verbal negators. Cross-linguistically, the future tense is an outstanding category, often being weaker and grammaticalized later than past and present tense \citep[771]{lindstedt2001a}, with its semantics close to modality \citep[280]{BybeePerkins1994}. This possibly makes future forms the best candidates to combine with new negative markers which intrude from non-verbal forms to standard negation, but further research is necessary in order to prove or disprove this speculation.

It should be noted that the situation is different in Egyptian Arabic and Bashkir. In the latter, future verbal forms are the only grammatical context where nominal negators can be used along with verbal SN markers, and no semantic difference is observed. In the former, the nominal negator has established itself as the only grammatical means to form negative future forms. In both languages a nominal negator does not add an emphatic semantic component. Alongside the languages described in this paper, there are other examples of use of the non-verbal negation marker in the function of verbal negation with future tense. For instance, “the attributive negator \textit{gə̀nyi} has been grammaticalized as the SN negator for verbs with future and near-past time reference” in Kanuri \citep[172]{Veselinova2016}. The traces of this pattern are also attested in a number of languages. However, a detailed description of the relation between non-verbal negators, their uses as focus markers and their subsequent uses as negators for the future tense still remains to be made.

One can expect that the markedness of the nominal negation marker will gradually wear out, it will expand its functional scope, and a new emphatic negative marker will arise (a well-known example of a similar semantic weakening of a negative marker is French \textit{pas} in \textit{ne … pas}, which has come a long way from the emphatic to a neutral strategy of negation). However, no evidence of such a process is observed in any of the languages examined here, though a more thorough investigation of the diachronic sources is necessary to definitively conclude this.

\section{Conclusion}\label{sec:BK5}

In this paper we have discussed several issues related to Croft’s Negative Existential Cycle. First, do the processes of change really occur cyclically, and if so, is there only one cycle? Second, what elements can be involved in it? Our initial data from Kalmyk and Bashkir bore evidence that changes taking place in a language system of negative markers do not necessarily close a cycle. The development of negation systems includes different new markers, where a newly introduced element shares functions with older elements of the system. Moreover, not only negative existentials but also other types of non-verbal negation can participate in this process.

One of the interesting points is a place taken by a new negation marker in the system of negation, as well as the relationship between nominal negators intruding into the verbal negation and negative existentials, which typically develop the function of SN in languages of the world. Observed cases show that there are different possibilities. For Kalmyk, we postulate an analogical evolution. The marker \textit{bišə} develops from a marker in the non-verbal identity/ascriptive predication to assume the function of negation of verbal predication with participles and converbs in parallel with the development of the marker \textit{ügei / uga}, which started this development earlier and underwent it in more Mongolic varieties. As for the Bornean languages and Egyptian Arabic, the intrusion of negative existentials into the domain of SN is not attested.

According to typological data, involvement of the non-verbal negative markers into the system of verbal negation can first exploit their markedness to express emphatic negation. It may be supposed that at the first step, they can be just another way to negate a verbal predicate, with an additional emphatic meaning, as compared to a neutral verbal negator. But their frequent co-occurrence with certain forms or high compatibility of the emphatic meaning with certain grammatical semantics can allow them to replace a verbal marker initially used with certain verbal forms and become the only way to negate them.

In particular, we have considered examples of the use of nominal negative markers with future forms, which seems to be a relatively frequent situation. Our sampled languages show different statuses of forms of nominal negation with future forms. In Bashkir and some other Turkic languages, the nominal negation competes with SN in the future tense. Egyptian Arabic displays a common way to negate an event in the future with the nominal negator \textit{muš}. Thus, Egyptian Arabic should be classified as situated at a more advanced stage of the nominal negation’s intrusion into the verbal system, as compared to Bashkir and other Turkic languages, where a similar phenomenon is found. We suggest that such an association of the future with noun negation is explained by specific properties of the future tense as a grammatical category.
 
\section*{Abbreviations}
\begin{tabularx}{.45\textwidth}{lQ}
    \textsc{1, 2, 3} & 1, 2, 3 person\\
    \textsc{acc} & accusative\\
    \textsc{aff} & affirmative clitic\\
    \textsc{ag} & agentive nominal\\
     \textsc{ant} & anterior\\
    \textsc{asc} & ascriptive (negation)\\
    \textsc{assoc} & associative\\
    \textsc{att} & attendant circumstance\\
    \textsc{caus} & causative\\
    \textsc{com} & comitative case\\
    \textsc{conc} & concessive\\
    \textsc{cop} & copula\\
    \textsc{cvb} & converb\\
    %\textsc{cvb.mdl} & modal converb\\
    \textsc{dat} & dative\\
    \textsc{def} & definite\\
    \textsc{dem} & demonstrative\\
     \textsc{emph} & emphatic\\
    \textsc{evid} & evidential\\
     \textsc{ex} & existential\\
    \textsc{excl} & exclusive\\
    \textsc{ext} & extension (unstable consonant -n in nominative of some nouns which disappears in oblique cases)\\
    \textsc{fut} & future\\
    \textsc{gen} & genitive\\
    \textsc{hab} & habitualis\\
    \textsc{ins} & instrumental case\\
    \textsc{ipfv} & imperfective form\\
    \textsc{ipm} & instances per million words\\
    \textsc{knc} & Kalmyk National Corpus\\
\end{tabularx}
\begin{tabularx}{.45\textwidth}{lQ}
    \textsc{loc} & locative\\
    \textsc{m} & masculine\\
    \textsc{mid} & middle voice\\
    \textsc{mm} & Middle Mongolian\\
    \textsc{nckl} & National Corpus of Kalmyk Language\\
    \textsc{nec} & Negative Existential Cycle\\
    \textsc{neg} & negation\\
     \textsc{neg.cop} & negative copula\\
    \textsc{neg.ex} & negative existential\\
    \textsc{nmlz} & nominalizaton\\
    \textsc{nom} & nominative\\
    \textsc{npst} & nonpast\\
    \textsc{ov} & object voice\\
    \textsc{pfv} & perfective\\
    \textsc{pl} & plural\\
   \textsc{poss} & possessive\\
    \textsc{pot} & potential\\
    \textsc{prog} & progressive aspect\\
    \textsc{prs} & present\\
   \textsc{prt} & particle\\
    \textsc{pst} & past\\
    \textsc{ptcp} & participle\\
    \textsc{q} & question marker\\
     \textsc{refl} & reflexive\\
    \textsc{rel} & relative\\
    \textsc{rem} & remote past\\
    \textsc{sg} & singular\\
     \textsc{sn} & standard negation\\
    \textsc{sov} & subject-object-verb word order\\
    \textsc{tam} & tense, aspect, modality\\
\end{tabularx}


\section*{Acknowledgements}
We would like to thank the editors of this volume, Benjamin Brosig, and an anonymous reviewer for their valuable comments on earlier versions of this chapter, Ramazan Mamedshakhov and Sergey Klimenko for their help in interpreting the data of Arabic and Bornean, and our Bashkir and Kalmyk informants. This study was supported by the Russian Foundation for Basic Research, grant 16--34--01015 ‘Negation in Bashkir, Kalmyk and Nanai and its interaction with tense and aspect’.



{\sloppy\printbibliography[heading=subbibliography,notkeyword=this]}

\end{document}
