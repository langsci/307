\documentclass[output=paper]{langsci/langscibook} 
\author{Jacobus A. Naudé\affiliation{University of the Free State}\and
Cynthia L. Miller-Naudé\affiliation{University of the Free State}\lastand
Daniel J. Wilson\affiliation{University of the Free State}
}
\title{The negative existential cycle in Ancient Hebrew}
%\chapterDOI{}

\abstract{Evidence of diachronic change as opposed to synchronic variation
in Ancient (Pre-Modern) Hebrew is currently disputed, as is the
relationship of Biblical Hebrew to later varieties of Hebrew as found in
Qumran Hebrew and Mishnaic Hebrew. Recent work in historical linguistics,
particularly the study of cyclical change in individual constructions, has
provided a means to use stages of synchronic variation within a diachronic
trajectory for analyzing how languages change. Such a diachronic trajectory
includes synchronic variation, transitional stages and overlapping
constructions. One cycle which manifests synchronic variation within a
diachronic trajectory is the Negative Existential Cycle as introduced by
\textcite{Croft1991}. This cycle is evident in the ancient varieties of Hebrew
and adds evidence to the claim that diachronic change is discernible in
Ancient Hebrew. One additional change that is observed is a shift in
subject agreement from more synthetic to more analytic in certain
constructions, which is consistent with the Subject Agreement Cycle in 
\citet{Gelderen2011}.  

\textbf{Keywords:} \todo{keywords?}}

\maketitle
\begin{document}

\section{Introduction}\label{sec:Hebrew-1}

Current research in the historical linguistics of Ancient Hebrew\il{%
Hebrew|(} is engaged in a controversy concerning the evidence of diachronic
change as opposed to synchronic variation in Biblical Hebrew and in the
relationship of the language of the Bible to later varieties of Hebrew as
found in Qumran %
%\il{Hebrew!Qumran Hebrew} % näin saa alakohdat hakemistoon
Hebrew and Mishnaic Hebrew.%
%
\footnote{Biblical Hebrew
refers to the Hebrew as found in the Hebrew Bible, which is based on the
medieval manuscript tradition of the Masoretes but ``reflects to a large
extent varieties of Hebrew spoken in Israel from the beginning of the Iron
Age (about 1200 BCE) to the Hellenistic era (about 165 BCE)'' (%
\cite[1]{MerweNaude2017}, see also pp. 2--6 for an overview of the
development of Ancient [Pre-Modern] Hebrew). Qumran Hebrew reflects the
Hebrew of the texts found in the eleven caves around Khirbet Qumran (ca.
200 BCE to 70 CE)
\parencites(see)(){Naude2003}{NaudeMiller2016b}%
.
Mishnaic Hebrew reflects the Hebrew of the sages, the Tannaim and Amoraim,
in Palestine and Babylonia. Literature written in Mishnaic Hebrew covers
the period of 70 CE to 500 CE, although Mishnaic Hebrew as a living
language was spoken in Palestine only until about 200 CE \parencites%
[116]{BarAsher1999}[see also][5]{MerweNaude2017}{NaudeMiller2016b}.}%
\todo{missing data: Naudé 2003 (in footnote)}{} %
%
Traditionally, the Hebrew of the Bible was understood to exhibit at least
two, if not three, diachronic stages -- archaic Biblical Hebrew, classical
Biblical Hebrew (of the monarchic period) and late Biblical Hebrew (of the
post-exilic period).%
%
\footnote{See \textcite[1--6]{MerweNaude2017} for an overview of the
development of ancient Hebrew. Archaic Biblical Hebrew (ca. 1200--1000 BCE)
reflects the oldest stratum of Hebrew in the Bible, as found especially in
the ancient poems. Classical Biblical Hebrew of the monarchical period (ca.
1000--586\slash 7 BCE) includes both the Israelian dialect and the standard
Judean scribal dialect and is the language of the pre-exilic prose sections
of the Hebrew Bible. Late Biblical Hebrew (ca.  539--165 BCE) is the
language of the post-exilic sections of the Hebrew Bible.} Recently,
however, the diachronic model has been challenged, by inter alia,
\textcite{YoungRezetko2008} and \textcite{RezetkoYoung2014} who claim that
Biblical Hebrew exhibits only synchronic variation and no clear trajectory
can be made between Biblical Hebrew and the much later varieties of
Pre-Modern Hebrew.\footnote{The arguments are summarized in
\cites{NaudeMiller2016a}{NaudeMiller2016b}.} The issue is complicated by
the oral-written context within which the Bible was written, by scribal
redaction of the text, and by scribal transmission over more than a
millennium. In this paper, we present an overview of the evidence for the
negative existential cycle in Ancient Hebrew, including Biblical Hebrew and
epigraphic Hebrew, and describe how this cycle demonstrates a trajectory
from Biblical Hebrew to Qumran Hebrew and Mishnaic Hebrew.

After we introduce the syntactic indications of negative scope in Hebrew,
we will describe the negative existential constructions. Then we will
demonstrate the stages of \citegen{Croft1991} Negative Existential Cycle
which are present in Biblical Hebrew and how they persist or change in
later varieties of Hebrew. Finally, we will demonstrate some syntactic
changes in one specific construction which diffuse into post-biblical
Hebrew, providing further evidence for a diachronic trajectory. 

As preliminary to the following discussion, we describe the syntactic
features of negative scope, which relates to standard verbal negation (as
illustrated in \ref{ex:heb-no-prophets} and \ref{ex:heb-not-you-but-God})
as well as the negative existential (as illustrated in
\ref{ex:heb-vegetation} and \ref{ex:heb-stink}).\todo{omitted brackets
around reference numbers (they are not required)}{} There are two kinds of
negative scope in Biblical Hebrew 
\parencites{SnymanNaude2003}{Snyman2004}{NaudeRendsburg2013}.\todo{assuming
Snyman \& Naudé 2003 for Naudé \& Snyman 2003 (missing)}{} Sentential negation occurs when the negative
marker immediately precedes the verb, which is regularly in initial
position in the sentence \REF{ex:heb-no-prophets}. %
%
By contrast, constituent negation occurs when the negative marker precedes
a non-verbal constituent \REF{ex:heb-not-you-but-God}:\todo{sublists of
examples must be continuous}{}
%
\begin{exe}
\ex\label{ex:heb-send-prophets-God}\begin{xlist}
\ex \label{ex:heb-no-prophets}
\gll lōʾ=šālaḥtî ʾeṯ=han-nəḇīʾîm\\
   \textsc{neg}=send.\textsc{pfv.1cs}  \textsc{obj}=\textsc{art}-prophets\\
   \glt `I did not send the prophets.' \hspace{\stretch{1}}\mbox{[BHS
   Jeremiah 23:21]}
\ex \label{ex:heb-not-you-but-God}
\gll wə-ʿattâ lōʾ=ʾattem šəlaḥtem ʾōṯî hēnnâ kî hā-ʾĕlōhîm\\
  and-now \textsc{neg}=\textsc{2mp} send.\textsc{pfv.2mp} \textsc{obj.1cs}
  here   \textsc{comp} \textsc{art}-God\\
\glt `And not you sent me here, but rather God.'
\hspace{\stretch{1}}\mbox{[BHS Genesis 45:8]}
  \end{xlist}
  \end{exe}
% % %
The scope of the negative modifies the semantic interpretation of the
sentence. In \REF{ex:heb-no-prophets} above, the sentence indicates that
God did not send the prophets; whereas in \REF{ex:heb-not-you-but-God}, the sentence does not deny that Joseph was sent, but only that it is not his addresees, his brothers, who effected the sending.

Negative existential constructions usually involve sentential negation, as
illustrated in \REF{ex:heb-vegetation}. %
%
Less frequently negative existential constructions may involve constituent
negation.\footnote{See also the examples discussed in \textcite[803,
§2.5]{NaudeRendsburg2013} as closely related to constituent negation.} In
\REF{ex:heb-stink} the negative existential negates a bare noun and the negative existential followed by the noun are the object of the preposition.
%
\begin{exe}\ex\label{ex:heb-vegetation-stink}
\begin{xlist}
\ex\label{ex:heb-vegetation}
\gll ʾên  ʿēśeḇ\\
    \textsc{neg.ex} vegetation\\
\glt `There is no vegetation.' \hspace{\stretch{1}}\mbox{[BHS Jeremiah
14:6]}
\ex\label{ex:heb-stink}
\gll tiḇʾaš dəgāt-ām mē-ʾên mayim\\
stink.\textsc{ipfv3fs} fish-\textsc{3mp}  from-\textsc{neg.ex} water\\
\glt `Their fish stink from no water.' \hspace{\stretch{1}}\mbox{[BHS
Isaiah 50:2]}
\end{xlist}
\end{exe}
%

Possession is regularly expressed in Hebrew using a existential
construction with a prepositional phrase headed by the preposition
\textit{l-} (`to'). With this construction, sentential and constituent
negation occurs with the negative existential marker, as illustrated in
\REF{ex:heb-no-king} and \REF{ex:heb-no-son}, respectively in present
time:%
%
\footnote{The positive possessive construction uses the positive
existential marker \textit{ÿēš} for present time reference and a form of
the copular verb \textit{hyh} for past time, future time or non-indicative
modality.}
%
\begin{exe}\ex
\begin{xlist}\ex\label{ex:heb-no-king}
\gll ʾên meleḵ lā-nû\\ 
  \textsc{neg.ex}  king  to-1\textsc{p}\\ 
  \glt `We have no king (lit. there is no king for us).'
  \hspace{\stretch{1}}\mbox{[BHS Hosea 10:3]}
\ex\label{ex:heb-no-son}
\gll ʾên lî bēn\\
\textsc{neg.ex}  to-1\textsc{s}  son\\ 
  \glt `Not I have a son (lit. there is not to me a son).'
  \hspace{\stretch{1}}\mbox{[BHS 2 Samuel 18:18]}
\end{xlist}\end{exe}
Possession can also be expressed in Hebrew using the copula with the same
prepositional phrase to indicate past time, future time or non-indicative
modality. The possessive construction  may involve sentential negation, as
in \REF{ex:heb-not-for-you}, or constituent negation, as in
\REF{ex:heb-no-offspring}:
%
\begin{exe}\ex\begin{xlist}
\ex\label{ex:heb-not-for-you}
\gll lōʾ yihyeh lāḵem \\ 
  \textsc{neg} \textsc{cop.3ms.ipfv} to.\textsc{3mp}\\ 
  \glt `It will not belong to you (lit. it will not be to you).'\\
  \hspace{\stretch{1}}\mbox{[BHS Jeremiah 35:7]}
\ex\label{ex:heb-no-offspring}
\gll lōʾ lô yihyeh haz-zāraʿ\\
\textsc{neg} to.\textsc{3ms} \textsc{cop.3ms.ipfv} \textsc{art}-seed\\
\glt `Not belonging to him would be the offspring (lit. not to him will be
the offspring).' \hspace{\stretch{1}}\mbox{[BHS Genesis 38:9]}
\end{xlist}\end{exe}

\section{Constructions with the negative existential
marker}\label{sec:Hebrew-2}

In Biblical Hebrew, the marker of standard negation in finite, indicative
verbal sentences is \textit{lōʾ}, as illustrated in
\REF{ex:heb-no-prophets} above \parencite[see also][143--172]{Sjors2018}.
There is also a negative existential marker, \textit{ʾayin} (usually
vocalized as the ``construct form'' \textit{ʾên})%
%
    \footnote{The historical origin of the negative existential marker has
    been connected to the interrogative adverb and homonym \textit{ʾayin}
    `where' \parencite[see, e.g.][569]{JouonMuraoka2009}. The two
    vocalizations of the negative existential relate to the syntactic
    contexts in which they occur; see
    \cites{NaudeMiller2018}{NaudeMiller2019}.} 
%    
and a positive existential marker, \textit{yēš}
\REF{ex:heb-just-holy-bread}:
\begin{exe}\ex\label{ex:heb-just-holy-bread}
    \gll ên=leḥem ḥōl ʾel=taḥaṯ yāḏî kî= ʾim=leḥem qōḏeš yēš\\
\textsc{neg.ex}=bread common to=under   hand.1\textsc{s} \textsc{comp}=
if=bread holy \textsc{ex} \\
    \glt `There is no common bread on hand, but holy bread there is.'\\
\hspace{\stretch{1}}\mbox{[BHS 1 Samuel 21:5]}
\end{exe}
%
The two existential markers do not index tense or aspect; they default for
present time. As a result, the verbal copula \textit{hyh} is used for
existential sentences that specify perfective aspect
\REF{ex:heb-land-without-rain}, or imperfective aspect with a future sense
\REF{ex:heb-no-flood-again}. The verbal copula is never used to express existence that is linked to the moment of speaking.
%
\begin{exe}\ex\begin{xlist}
\ex\label{ex:heb-land-without-rain}
\gll lōʾ=hāyâ g̱ešem bā-ʾāreṣ\\
\textsc{neg}=\textsc{cop.pfv.3ms} rain in.\textsc{art}-land  \\
\glt `There was no rain in the land.' \hspace{\stretch{1}}\mbox{[BHS 1
Kings 17:7]}
%
\ex\label{ex:heb-no-flood-again}
\gll wǝ-lōʾ=yihyeh ʿôḏ mabbûl lǝ-šaḥēṯ hā-ʾāreṣ\\
and-\textsc{neg}=\textsc{cop.ipfv.3ms} again flood to-destroy.\textsc{inf}
\textsc{art}-land\\
\glt `And there will not again be a flood to destroy the land.'\\
\hspace{\stretch{1}}\mbox{[BHS Genesis 9:11]}
\end{xlist}\end{exe}
%
The copula is also used for existential sentences which express
non-indicative modality. In \REF{ex:heb-strength}, the positive
construction is illustrated and in \REF{ex:heb-dispute}, the negative construction illustrates the fact that the negative marker \textit{ʾal}  is used with non-indicative finite verbs rather than the indicative negative marker \textit{lōʾ}:
%
\begin{exe}\ex\begin{xlist}
\ex\label{ex:heb-strength}
\gll wîhî ḇə-ḵā kōaḥ kî ṯēlēḵ bad-dāreḵ\\
and.\textsc{cop.jus.3ms} in-\textsc{2ms} strength \textsc{comp}
go.\textsc{ipfv.2ms} in.\textsc{art}{}-way\\
\glt `So that there may be strength in you when you go on your
way.' \\\hspace{\stretch{1}}\mbox{[BHS 1 Samuel 28:22]}
%
\ex\label{ex:heb-dispute}
\gll ʾal=nāʾ ṯəhî mərîḇâ bên-î û-ḇênê-ḵā \\
\textsc{neg.non-ind}=please  \textsc{cop.jus.2ms} dispute between-\textsc{1s}
and-between-\textsc{2ms}\\
\glt `Please may there not be a dispute between me and you.'\\
\hspace{\stretch{1}}\mbox{[BHS Genesis 13:18]}
\end{xlist}\end{exe}
%
This picture of the distribution of the standard negator and the negative
existential marker becomes more complex, because the negative existential
marker is also used to negate some verbal predications, most prominently
participial predications \parencite[see][]{MillerNaude2015}, as illustrated
in \REF{ex:heb-king-not-saved}:
%
\begin{exe}\ex\label{ex:heb-king-not-saved}
    \gll ʾên=ham-meleḵ nôšāʿ bə-roḇ=ḥāyil\\
  \textsc{neg.ex}=\textsc{art}-king save.\textsc{pass.ptcp}\footnotemark{} %
 in-abundance.\textsc{gen}=army \\
  \footnotetext{The verbal form is in the Niphal stem, which is used for
  passive or reflexive meanings; see \cite[78--79]{MerweNaude2017}.} 
    \glt  `The king is not saved by the greatness of his army.'
    \hspace{\stretch{1}}\mbox{[BHS Psalm 33:16]}
    \end{exe}
As is indicated below, the negative existential marker expands its use so
that it is used to negate verbal sentences. At the same time, the
participle expands its use as the main predication in a sentence.

In this section we listed the various constructions in Biblical Hebrew which utilize the negative existential markers. In the following section we examine aspects of the negative existential cycle in ancient Hebrew. 

\section{The negative existential cycle in Ancient
Hebrew}\label{sec:Hebrew-3}

Before demonstrating the stages of the negative existential cycle in
Ancient Hebrew, a word is necessary on what constitutes a ``stage'' in historical linguistics. This matter is important and is connected to the debated issue of synchrony vs. diachrony, which has received increasing scrutiny in recent decades. \cite[12]{Lass1997} poses the following question:
%
\begin{quote}
How much of what looks like (synchronic) structure really is, and how much
is rather detritus left behind by historical processes, that even if they
leave notable residues have no particular present relevance? \ldots{} In this sense a language-state as an object of academic scrutiny is no different in principle from a kidney, a mountain range, or an art style. 
\end{quote}
%
Certain formulations such as A > B are commonly used to represent stages of
linguistic change. Another, more appropriate formulation A > A
{\textasciitilde} B > B has been used by \cite{Croft1991} in his seminal
work on the negative existential cycle. \citeauthor{BrintonTraugott2005} critique
this formulation saying, ``Even this is misleading, since often, especially
in domains that involve meaning, earlier patterns only become restricted or
fossilized, not entirely lost'' \parencite[6]{BrintonTraugott2005}. They propose an alternate formulation:
%
\begin{equation*}  
A > %
\left\{\begin{aligned} A\\B \end{aligned}\right\} > (B)
\end{equation*}
%
This formulation states that the emergence of B as a distinct stage may or
may not occur. Any theory of a stage in historical linguistics must,
therefore, acknowledge the mixture of older and newer forms existing
contemporaneously while also acknowledging that some stages will not evolve
\parencites(see also)()[22--25]{Croft1991}{Veselinova2016}. Additionally, newer forms may emerge yet not diffuse throughout the language, but be subsumed by other forms.

In his theory of language change and diffusion, \citet{Naude2012} lays out four dimensions that are relevant for the analysis of ancient texts in historical linguistics. The first dimension is the idiolect that develops when the grammar of an individual differs from the input source (e.g. child and his parents). This is called the individual dimension and is the source of language change. 

The second dimension is the sociological dimension. This relates to the diffusion of the change throughout the language community. \citet[214]{RingeEska2013} describe this process as follows:
%
\begin{quote}
Yet, should it be the case that a syntactic parameter changes its setting
from one generation to the next via imperfect learning in the acquisition
process, we have to ask why we find that change takes place only gradually
in the documentary record. This seeming paradox has been solved by Kroch
1989, who points out that a parameter for which only a small amount of data
is present in the primary linguistic data heard during the process of
acquisition can lead two learners to acquire two different grammars. This
has given rise to Kroch's Grammars in Competition Hypothesis, in which parameter settings, not entire grammars, compete; it is manifested in the variation found in the documentary record as the reflex of an innovative parametric setting competes with and eventually supplants the reflex of the older parametric setting.
\end{quote}
%
\citeauthor{Naude2012} adds that this sociological diffusion occurs in the shape of an S-shaped curve with the new option beginning slowly, accelerating, and finally leveling off once the competition is resolved.

The third dimension is the chronological dimension. In this dimension,
newer forms exist and change side-by-side with older forms called
``stylistic fossils''. \citeauthor{Naude2012} says, ``These stylistic
fossils are in competition -- at certain stages they are dominant and at
other stages they are dominated -- and they may be present in the speech
community for centuries'' \parencite[73]{Naude2012}. As older forms erode
and become limited in their use, newer forms pick up the slack and
represent a renewal, a ``diachronic cycle''. This cycle is not a reversal
of directionality, but a termination of one unidirectional process and the
restarting of another in the same general direction \parencite[73]{Naude2012}.  

Naudé's fourth dimension acknowledges that analyzing ancient texts involves
working with written language. All historical linguistic studies that span
more than a few generations must interact with written text.%
%
\footnote{Historical linguistics involving languages which have only a very
recent written tradition (or no written tradition) can only be accomplished
by comparative historical analysis of related languages or dialects for
which a written tradition exists.} 
%
This point is
important for our definition of a stage. The written dimension preserves a
picture of the status of diffusion at the time of writing. If written texts
comprise our data, then each text which gives evidence that a change has
diffused represents a stage. This stage may only reflect change in a single
construction, but it still should be considered a stage. For this reason,
syntactic structures in different corpora reflect different stages insofar
as they have qualitative differences. A stage in diachronic syntax, then,
is construction-specific and is discerned by observing the degree of
diffusion between written texts. These stages are part of a cycle which is
constantly being renewed. 

In terms of Croft's diachronic typology of the negative existential cycle,
Biblical Hebrew exhibits a variable stage A {\textasciitilde} B in that
there is a dedicated negative existential form (\textit{ʾên}) but it is
used in specific contexts in which past or future time do not need to be
specified but can be inferred from the surrounding context. For those
contexts which specify past or future time, the \textit{lōʾ} + verbal
copula construction is used. While it might be possible to posit that
\textit{lōʾ} + verbal copula represents a pre-biblical stage A in which the
standard verbal negator is used to negate existential sentences, this stage
cannot be clearly discerned in the Hebrew Bible.%
%
    \footnote{An anonymous reviewer suggested that Job 38:26 provides an
    example of the standard verbal negator \textit{lōʾ} without a copula:
    \textit{ʾereṣ lōʾ ʾîš} `a land [which] no man (is).' We argue instead
    that if \textit{lōʾ ʾîš} (lit. not man) was a verbless sentence with a
    null (implicit) copula, it would have a pronominal clitic for
    disambiguation  \parencite[see][]{Naude1996}.  We understand the phrase
    in Job 38:26 as a noun (\textit{land}) modified by a noun phrase with
    constituent negation (\textit{no} \textit{man}); the phrase means `an
    uninhabited (lit.  no human) land.'} %
%
Furthermore, the use of \textit{ʾên} is far more
prominent than \textit{lōʾ}  + verbal copula in expressing negative
existential sentences in Biblical Hebrew.%
%
    \footnote{We have identified only
    61 examples of the negative existential \textit{lōʾ} + verbal copula in
    Biblical Hebrew as compared to 383 examples of the dedicated negative
    existential \textit{ʾên.}} %
%
In post-biblical Hebrew, the \textit{lōʾ}  +
copula form of the negative existential decreases in use in Qumran, though
it does still occur:
%
\begin{exe}\ex \label{ex:heb-survivor}
    \gll w-plṭh lʾ thyh \\
and-survivor    \textsc{neg}    \textsc{cop.ipfv.3fs}  \\
    \glt
  `There will be no survivor.' \hspace{\stretch{1}}\mbox{[DSSR 1QM 1:6]}
    \end{exe} 
%
In Mishnaic Hebrew there are very few examples of a genuine negative existential with the construction \textit{lōʾ}  + copula:
%
\begin{exe}\ex \label{ex:heb-Hashshoebah}
    \gll lōʾ hāyətâ ḥāṣēr bîrû\u{s}ālayim \u{s}e-ʾên-āh məʾîrâ mê-ʾôr  bêt ha\u{s}\u{s}ôʾêbâ\\
 \textsc{neg} \textsc{cop.pfv.3fs} courtyard in.Jerusalem  \textsc{rel}-\textsc{neg.ex}-\textsc{3fs}   illumined.\textsc{ptcp}   from-light.\textsc{gen} house.\textsc{gen}  hashshoebah \\
\glt `And there was not a courtyard in Jerusalem that was not illuminated
by the light of the house of Hashshoebah.' 
\hspace{\stretch{1}}\mbox{[M Sukkah 5:3]}
    \end{exe}
%
An alternative strategy for Stage A is attested in Biblical Hebrew in a single example:
%
\begin{exe}\ex \label{ex:heb-arbiter} \gll lôʾ yēš=bênênû môḵîaḥ \\
\textsc{neg} \textsc{ex}=between.\textsc{1cp}  arbiter.\textsc{ptcp} \\
\glt `There does not exist between us an arbiter.'
\hspace{\stretch{1}}\mbox{[BHS Job 9:33]} \end{exe}
%
This construction uses the standard verbal negator before the positive
existential marker (\textit{yēš}). This strategy does not seem to be
attested in later stages of Hebrew.%
%
    \footnote{In the Mishnah, there is a single example of \textit{lōʾ}
    preceding \textit{yēš} but this construction is unique because of its
    connection with the interrogative marker, meaning `is it not the case
    that there exist' (with the pragmatic sense `it is certainly the case
    that there are') \textit{wahălōʾ yeš šeʾênān
    mōsǝqîn [zîṯêhn] ʾellāʾ ləʾāḥar
    rəḇêʿâ šənîyyâ.} `But is it not the case that there
    are not those who pick the olives only after the second rain [falls]?'
    (M Peʾah 8.1).} %
%
However, this strategy occurs in Biblical Aramaic, a related Northwest
Semitic language, in which the positive existential particle \textit{ʾîṯay}
is negated by the standard verbal negative \textit{lāʾ} (e.g. Daniel 2:10,
3:29, 4:32). In later Targumic Aramaic, the standard verbal negative and
the positive existential became fused into a new existential marker
\textit{layiṯ}, an illustration of Stage B. The sole example in Biblical
Hebrew may thus be an Aramaism \parencite[see][xlvi-xlvii]{DriverGray1921} or
it may reflect a change which did not diffuse or develop in Hebrew as it
did in Aramaic.

In addition to the stage A{\textasciitilde}B in which both the dedicated
negative existential \textit{ʾên} and \textit{lōʾ}  + copula occur, it is
also clear that the dedicated negative existential marker \textit{ʾên}  is
expanding its domain of use from existential sentences to verbal sentences
-- Croft's variable stage B {\textasciitilde} C -- and this is the most
dominant pattern in the Hebrew Bible.%
%
    \footnote{For cross-linguistic data illustrating that a contextually
    restricted negator expands into the domain of verbal negation, see also
    \textcite{Veselinova2014} for data from Hawai'ian (Polynesian) and
    \textcite{Veselinova2015} for data from Zyryan Komi (Uralic).}
%
The extension of the negative existential marker \textit{ʾên} to verbal
sentences occurs only when the verbal predicate is a non-finite verb and
especially a participle. The fact that the participle has both nominal and
verbal characteristics
\parencites{AndersenForbes2007}[33--35]{AndersenForbes2012}
undoubtedly facilitates the expansion of the negative existential from
purely nominal predicates to participial predicates.
\textcite[157]{Veselinova2016} has found that non-finite verbal forms
cross-linguistically are often the first to allow negation with a negative
existential marker. The stage B {\textasciitilde} C which is observed in
Biblical Hebrew can be seen in Epigraphic Hebrew, the non-biblical
materials from the time of the Bible, which can be dated paleographically.
In a few examples, \textit{ʾên} negates both verbless existential sentences
\REF{ex:heb-silver} and verbal predicates with participles
\REF{ex:heb-sending}. 
%
\begin{exe}\ex \label{ex:heb-silver}
    \gll ʾyn   {\ob}p{\cb}h ksp\\
\textsc{neg.ex}  here    silver \\
    \glt `There is no silver here.' \hspace{\stretch{1}}\mbox{[HAE Jer(7):2
    line 1]\footnotemark}
\footnotetext{See also the example in HAE Lak(6):1.4 line 5.} 
\ex \label{ex:heb-sending}
    \gll ʾyn{\ob}n{\cb}y šlḥ \\
\textsc{neg.ex.1cs}      send.\textsc{ptcp.ms} \\
    \glt `I am not sending.' \hspace{\stretch{1}}\mbox{[HAE Lak(6):1.4
    lines 7--8]\footnotemark}
\footnotetext{See also HAE Arad(8):40 lines 13--14.} 
    \end{exe}
%
For an example of how these constructions expand their use, in Qumran Hebrew it is possible for \textit{ʾên} to negate an infinitival clause:
%
\begin{exe}\ex \label{ex:heb-arrows}
    \gll b-htʿwpp kwl ḥṣy sḥt l-ʾyn hšb  \\
  when-fly.\textsc{inf}     all.\textsc{gen}   arrows.\textsc{gen}   pit to-\textsc{neg.ex}   return.\textsc{inf} \\
    \glt `when the arrows of the pit fly off without returning'
    \hspace{\stretch{1}}\mbox{[DSSR 1QH\textsuperscript{a} 11.28]}
    \end{exe}
%
In Mishnaic texts the plural participle may be used with \textit{ʾên} to express an impersonal and permanent prohibition. 
%
\begin{exe}\ex \label{ex:heb-women}
    \gll nāšîm wa-ʿăḇāḏîm û-kǝtannîm ʾên məzammənîm ʿălê-hem \\
  women   and-slaves   and-minors \textsc{neg.ex}   invite.\textsc{ptcp.pl}
  on-\textsc{3mp}\\
    \glt `Women, slaves, or minors may not invite others.'
    \hspace{\stretch{1}}\mbox{[M Berakot 7.2]}
    \end{exe}
%
The use of \textit{yēš} with the plural participle similarly expresses a
general, impersonal, positive statement \parencite[134]{Perez1997}:
%
\begin{exe}\ex \label{ex:heb-firstfruits}
    \gll  yēš məḇîʾîm bîkkurîm\\
\textsc{ex}       bring.\textsc{ptcp.mp}     firstfruits \\
    \glt `There are those who bring the firstfruits.'
    \hspace{\stretch{1}}\mbox{[M Bikkurim 1.1]}
    \end{exe}
%
Examples (\ref{ex:heb-arrows}--\ref{ex:heb-firstfruits}) thus demonstrate that in post-Biblical
Hebrew, there is further expansion of the use of the negative existential
marker for the negation of verbal constructions as part of the B
{\textasciitilde} C cycle. Although this change began in post-exilic
Biblical Hebrew with a few examples, it becomes very common in Qumran
Hebrew and Mishnaic Hebrew \parencite[36--39]{Hurvitz2014}.

An additional environment where \textit{ʾên} functions similarly to a
simple negator is in a verbless locative sentence, as in example
\REF{ex:heb-Joseph}.\footnote{See also BHS Exodus 17:7, 1 Samuel 9:11, 14:39.}
%
\begin{exe}\ex \label{ex:heb-Joseph}
    \gll  wə-hinnê ʾên=yôsēp̱ bab-bôr\\
and-behold \textsc{neg.ex}=Joseph in.\textsc{art}-pit \\
    \glt `Behold, Joseph was not in the pit.'
    \hspace{\stretch{1}}\mbox{[BHS Genesis 37:29]} \end{exe}
%
In example \REF{ex:heb-Joseph}, \textit{Joseph} cannot serve as the pivot of an
existential because it is a proper name. Proper names cannot function as
the pivots of existentials due to the \textit{definiteness effect}
(or \textit{definiteness restriction}) \parencites[195]{Milsark1974}%
{Leonetti2008}. The definiteness effect is a cross-linguistic
phenomenon of existentials whereby definite NPs are prohibited from serving
as the pivot, as in the English example \REF{ex:heb-eng-dog}.
%
\begin{exe}\ex[*]{% 
     There is \underline{it\slash the dog\slash that dog\slash Fido\slash}
     \hspace{\stretch{1}}\citep[132]{Leonetti2008}}\label{ex:heb-eng-dog}  \end{exe}
% 
The proper noun \textit{Joseph} in \REF{ex:heb-Joseph}, instead, is the
subject of a simple predication. Example \REF{ex:heb-Joseph} demonstrates
that \textit{ʾên} may be used in locative predication. This is especially
significant in light of the semantics of existential sentences.
Existentials have been compared to locatives–the main difference between
them being the reorientation of the figure-ground relationship. Creissels
compares existentials to locatives saying,
%
\begin{quote}
  What distinguishes existential clauses from plain locational   clauses is
  a different perspectivization of figure-ground   relationships whose most
  obvious manifestation is that, contrary   to plain locational clauses,
  existential clauses are not adequate   answers to questions about the
  location of an entity, but can be   used to identify an entity present at
  a certain location \citep[2]{Creissels2014}. 
\end{quote}
%
\textcite{ParteeBorschev2002}\todo{missing reference: Partee \& Borschev 2002} introduce the notion of Perspective Centre
to compare existentials to locatives. In a locative sentence, the THING is
chosen as the perspectival centre while the LOCATION is chosen in an
existential sentence. The difference is represented in
\REF{ex:heb-eng-glass},
with the Perspectival Centre underlined:
%
\begin{exe}\ex \label{ex:heb-eng-glass}
 \begin{xlist}
 \ex Existential    \\
 ``There is a glass \underline{on the table}.'' 
   \ex Locative \\
  ``\underline{The glass} is on the table.''
   \end{xlist}\end{exe}\todo{boldface is preferred for highlighting, ok to change here and in the following examples?}
%
It is significant that in Biblical Hebrew, the negative existential
particle may be used in both existential and locative sentences. Other
languages which allow this have been identified in \textcite{Veselinova2013}.

There is a second variable stage B {\textasciitilde} C in which the
\textit{lōʾ} + copula construction negates a participle. Just as the
negative existential particle \textit{ʾên} enters the verbal domain by
negating the participle, \textit{lōʾ} +copula does as well. Biblical Hebrew
has only 5 examples of this construction \REF{ex:heb-fastened}.%
%
    \footnote{See also the Biblical Hebrew examples in BHS Exodus 23:26; Isaiah
    10:14; Jeremiah~50:3; Ezekiel 41:6; Daniel 8:7.} 
%
\begin{exe}\ex \label{ex:heb-fastened}
    \gll wə-lōʾ=yihyû ʾăḥûzîm bə-qîr hab-bāyiṯ \\
  and-\textsc{neg}=\textsc{cop.ipfv.3mp} fasten.\textsc{ptcp.mp}   in-wall.\textsc{gen}   \textsc{art}-house \\
    \glt `They were not fastened to the wall of the house.'
    \hspace{\stretch{1}}\mbox{[BHS Ezekiel 41:6]}
    \end{exe}
%
In Qumran Hebrew and Mishnaic Hebrew, the use of construction \textit{lōʾ}
+ copula + participle is used in additional contexts in which Biblical
Hebrew generally use the negator \textit{lōʾ} plus a perfective or
imperfective verb. The Qumran Hebrew example in \REF{ex:heb-enter-temple} uses a
\textit{lōʾ} + copula + participle in contrast with the Biblical Hebrew example
in \REF{ex:heb-enter-house} which uses \textit{lō}ʾ + imperfective verb.
%
\begin{exe}\ex\begin{xlist}
\ex\label{ex:heb-enter-house}
\gll lōʾ yāḇôʾ bêṯ YHWH\\
\textsc{neg} enter.\textsc{ipfv.3ms} house.\textsc{gen} YHWH \\
\glt `It will not enter the house of YHWH.' \hspace{\stretch{1}}\mbox{[BHS
Hosea 9:4]}
%
\ex\label{ex:heb-enter-temple}
\gll w-lwʾ yhyw bʾym blʿ ʾl twk mqdšy \\
and-\textsc{neg} \textsc{cop.ipfv.3mp} enter.\textsc{ptcp.mp} suddenly 
into midst.\textsc{gen} temple.\textsc{1cs} \\
\glt `So that they will not enter suddenly into the midst of my temple.' \\
\hspace{\stretch{1}}\mbox{[DSSR 11Q19 46:10--11]}
\end{xlist}\end{exe}
%
Example \REF{ex:heb-found-proof} illustrates that the \textit{lōʾ} + copula
+ participle construction persists in Mishnaic Hebrew where Biblical Hebrew
would use a \textit{lōʾ} + finite verb \REF{ex:heb-father-and-mother}.
%
\begin{exe}\ex\begin{xlist}
\ex\label{ex:heb-father-and-mother}
\gll wə-ʾāḇîw wə-ʾimmô lōʾ yāḏəʿû kî mē-YHWH hîʾ \\ 
and-father.\textsc{3ms}  and-mother.\textsc{3ms} \textsc{neg}
know.\textsc{pfv.3mp}  \textsc{comp}   from-YHWH \textsc{3fs} \\
\glt `His father and his mother did not know that it was from YHWH.'\\
\hspace{\stretch{1}}\mbox{[BHS Judges 14:4]}
%
\ex\label{ex:heb-found-proof}
 \gll lōʾ hāyâ yôdēaʿ šey-yeš lô rəʾāyâ û-māṣāʾ rəʾāyâ \\
\textsc{neg} \textsc{cop.pfv.3ms} know.\textsc{ptcp.ms} \textsc{rel-ex}
to.\textsc{3ms}   proof and-find.\textsc{pfv.3ms}   proof   \\
\glt `He did not know that he had proof but he found proof.'\\
\hspace{\stretch{1}}\mbox{[M Sanhedrin 3:8]}
\end{xlist}\end{exe}
%
This construction provides yet another example of the expansion of forms into post-biblical Hebrew.

There may also be evidence for the variable stage C {\textasciitilde} A in which the negative existential is used not only for verbal predications, but also to negate the affirmative existential. There is only one example in Biblical Hebrew which may possibly point to this stage:
%
\begin{exe}\ex \label{ex:heb-ears}
    \gll ʾoznayim lā-hem wə-lōʾ yaʾăzînû ʾa\={p} ʾên=yeš=rûaḥ bə-p̱îhem \\
ears    to-\textsc{3mp}  and-\textsc{neg}  hear.\textsc{ipfv.3mp}   indeed 
  \textsc{neg.ex=ex}=breath  in-mouth.\textsc{3mp} \\
    \glt
`They have ears, but they cannot hear; nor is there breath in their mouth
(lit. there does not exist the existence of breath in their mouth).''\\
\hspace{\stretch{1}}\mbox{[BHS Psalm 135:17]}
    \end{exe}
%
In \REF{ex:heb-ears}, a sentence is predicated with the positive
existential \textit{yēš} but preceded by the negative existential
\textit{ʾên.}%
%
    \footnote{In BHS 1 Samuel 21:9, a positive existential
    sentence is preceded by \textit{ʾîn}, whose identification is uncertain. It
    might be an alternate spelling of the negative existential marker (the
    reading of some manuscripts) or it might be a mistaken vocalization of the
    interrogative marker \textit{ʾên} ``where?''} %
%
It is possible that this
example reflects poetic license or that the sentence reflects a change in
the language resembling the last stage of the negative existential cycle
which was not successfully diffused through the language \parencite[see][]%
{Naude2012}. With only one example, we cannot be certain about the
status of  a variable C {\textasciitilde} A stage.

We have seen extensive evidence for stages A {\textasciitilde} B and B {\textasciitilde} C of the negative existential cycle, including further expansions of the use of the negative existential marker to negate various kinds of verbal constructions in post-biblical Hebrew. 

The negative existential cycle provides a means to use stages of synchronic variation within a diachronic trajectory for analyzing these Ancient Hebrew constructions. Such a diachronic trajectory acknowledges synchronic variation, transitional stages and overlapping constructions, all of which reflect the ways in which languages change over time.

\section{The shift from synthetic to analytic pronominal subjects of
negative existentials}\label{sec:Hebrew-4}

In this section we revisit the question of diachronic change exhibited in
the negation of the participle with special reference to constructions
involving left dislocation \parencite[see][]{NaudeMiller2016}\todo{Naudé
\& Miller-Naudé 2016a or 2016b?}. In
this section we provide additional evidence that seemingly small changes
involving left dislocation constructions reflect change in syntactic
structures. Furthermore, some constructions which seem to be synchronic
variants in terms of the negative cycle can be shown to be diachronically
rather than synchronically related on the basis of syntax.

This section demonstrates that in addition to the expanding domains of
various existential forms, the forms themselves are subject to change based
on other factors. \Textcite[7]{Gelderen2016} reviews the treatment of
analytic and synthetic languages and demonstrates how macro-cycles can be
discerned in addition to cycles such as the ones demonstrated in
\sectref{sec:Hebrew-3}. In macro-cycles, languages can move from being more
analytic, in which they are closer to having a one-to-one relationship
between word and morpheme, to more synthetic in which isolated forms move
to become more agglutinative and separate words are reanalysed
morphologically as part of another word (e.g. English \textit{going
to} > \textit{gonna})  \parencite[see][6--8 for a
description of the development of this notion]{Gelderen2016}. As the cycle continues,
eventually the synthetic forms move toward being more analytic and
reproduce isolated forms again. Analytic and synthetic stages can occur
simultaneously in different systems of a language. A language can be in one
stage for agreement and in another for negation 
\parencite[7]{Gelderen2016}. In this section, we demonstrate that the
pronominal subject of participial predicates negated with the negative
existential marker is manifesting a shift in agreement from a synthetic
inflectional stage where the subject is a pronominal suffix into an
analytic isolating stage where the subject is an independent personal
pronoun. 

There are three types of constructions in which the participle is negated
with \textit{ʾên} in Biblical Hebrew  (see \citeauthor{MillerNaude2015}
\citeyear{MillerNaude2015}
for additional
details). In the first type, a pronominal suffix is
joined to the negative existential marker which is followed by a participle
with its object and adjuncts:
%
\begin{exe}\ex \label{ex:heb-you-die}
    \gll ʾim=ʾênḵā mēšîḇ daʿ kî=môṯ tāmûṯ ʾattâ wə-ḵol=ʾăšer=lāḵ \\
    if=\textsc{neg.ex.2ms} return.\textsc{ptcp} know.\textsc{imp.ms}
  that=die\textsc{.inf.ab}   die.\textsc{ipfv.2ms} \textsc{2ms}
  and-all=\textsc{rel}=to.\textsc{2ms}  \\
    \glt `If you do not return, know that you shall surely die, you and all who
are yours.' \hspace{\stretch{1}}[BHS Genesis 20:7]
    \end{exe}
%
This construction can be modified through left dislocation, in which a constituent appears outside the initial boundary of the sentence and is resumed within the sentence as a pronominal suffix on the negative existential marker:
%
\begin{exe}\ex \label{ex:heb-dead}
    \gll kî ha-ḥayyîm  yôḏǝʿîm    šey-yāmūṯû wə-ham-mēṯîm ʾên-ām yôḏǝʿîm
    məʾûmâ       \\ 
    for \textsc{art}-living.\textsc{pl} know.\textsc{ptcp.pl}
    rel-die.\textsc{ipfv.3mp} and-\textsc{art}-die.\textsc{ptcp}
    \textsc{neg.ex}-\textsc{3mp} know.\textsc{ptcp.pl} anything \\
    \glt
`For the living know that they will die, but the dead, they do not   know
anything.' \hspace{\stretch{1}}[BHS Qohelet 9:5]
    \end{exe}
%
The subject constituent (\textit{the dead}) is left dislocated, and a
resumptive subject pronoun is suffixed to the negative existential. (For
the syntactic and semantic features of topicalization and left dislocation
in Biblical Hebrew, see \cites{Naude1990}{Holmstedt2014}{NaudeMiller2017}). The same construction occurs in
Qumran Hebrew:
%
\begin{exe}\ex \label{ex:heb-my-mother}
    \gll {\ob}wə{\cb}-ʾp    ʾmy    ʾynnh    mʾmnt ʾšr trʾn{\ob}y{\cb} ʿwd\\
    [and]-even   mother.\textsc{1s}    \textsc{neg.ex.3fs}
    believe.\textsc{ptcp.fs} \textsc{rel}   see.\textsc{ipfv.3fs.1s}
    again\\
    \glt `Even my mother, she does not believe that she will see me
    again.'\\
             \hspace{\stretch{1}}[DSSR 4Q200 f4:4]
    \end{exe}
%
The construction is also found in Mishnaic Hebrew:
%
\begin{exe}\ex \label{ex:heb-garlic}
    \gll haš-šum wə-hab-bəṣālîm ʾên-ān    miṣṯārp̱în \\
  \textsc{art}-garlic   and-\textsc{art}-onion.\textsc{pl} \textsc{neg.ex-3mp}  
join.\textsc{ptcp.mp} \\
    \glt `Garlic and onions, they do not join together.'
    \hspace{\stretch{1}}[M Peah 6:9]
    \end{exe}
%
In Qumran Hebrew an independent personal pronoun can be used instead for the subject instead of a pronominal suffix on the negative existential marker:
%
\begin{exe}\ex \label{ex:heb-qualified}
    \gll w-ʾm  ʾyn  hwʾ  bḥwn        b-kl ʾlh \\
    and-if    \textsc{neg.ex} \textsc{3ms} distinguish.\textsc{ptcp.pass.ms}
    in-all.\textsc{gen}  these\\
    \glt `if he is not qualified in these (rules)' \hspace{\stretch{1}}[DSSR CD
13:3]
    \end{exe}
%
This innovation has diffused and is also found in Mishnaic Hebrew:
%
\begin{exe}\ex \label{ex:heb-unable-to-help}
    \gll {\ob}wə{\cb}-ʾên ʾat yāḵol lə-panəśô \\
[and]-\textsc{neg.ex} \textsc{2ms} be.able.\textsc{ptcp}
to-help.\textsc{inf.3ms} \\
    \glt `And you are not able to help him.' \hspace{\stretch{1}}[M Nedarim
9:4]
    \end{exe}
%
In Qumran Hebrew, a left dislocated pronoun may be resumed with an
independent personal pronoun following the negative existential, rather
than with a pronominal suffix (contrast example
\ref{ex:heb-dead}):\todo{an odd opening bracket in \REF{ex:heb-dressed}?}{}
%
\begin{exe}\ex \label{ex:heb-dressed}
    \gll w-hwʾ ʾyn hwʾ lbwš b-g{\ob}dy h-qwdš     
    \\ and-\textsc{3ms} \textsc{neg.ex} \textsc{3ms}
    dressed.\textsc{pass.ptcp}  in-garments.\textsc{gen}
    \textsc{art}-holiness\\
    \glt `and he, he is not dressed with the sacred vestments'\\
          \hspace{\stretch{1}}[DSSR 11Q19 35:6 (= 11QT)]
    \end{exe}

What is important is that the constructions found in Biblical Hebrew in
which pronominal subjects of the negative existential marker are realized
as pronominal suffixes (\ref{ex:heb-you-die}, \ref{ex:heb-dead}) all continue in Qumran Hebrew
\REF{ex:heb-my-mother} and in Mishnaic Hebrew \REF{ex:heb-garlic}. However, new
constructions in which the pronominal subject is realized as an independent
pronoun are appearing in Qumran Hebrew
(\ref{ex:heb-qualified}, \ref{ex:heb-dressed}) alongside those
inherited from Biblical Hebrew and those changes are diffusing and
persisting into Mishnaic Hebrew \REF{ex:heb-unable-to-help}. The shift from synthetic
(inflectional) to analytic (isolating) pronouns is apparent in the new
constructions that have developed after Biblical Hebrew. There is, however,
one similar example with independent subject pronouns in post-exilic
Biblical Hebrew:
%
\begin{exe}\ex \label{ex:heb-weapon}
    \gll wə-ʾên ʾănî wə-ʾaḥay û-nəʿāray     wə-ʾanšê ham-mišmār ʾăšer
    ʾaḥăray    ʾên=ʾănaḥnû p̱ōšəṭîm bə\={g}āḏênû  ʾîš šilḥ-ô ham-māyim
    \\
  and-\textsc{neg.ex} \textsc{1cs} and-brothers.\textsc{1cs}
  and-servants.\textsc{1cs} and-men.\textsc{gen}  \textsc{art}-guard
  \textsc{rel}  after.\textsc{1cs}   \textsc{neg.ex}=\textsc{1cp}
  put.off.\textsc{ptcp}  clothing.\textsc{1cp}   man    
  weapon-\textsc{3ms}   \textsc{art}-water \\
    \glt `So not \underline{I nor my brothers nor my servants nor the men of the guard
    who} \underline{followed me – we} did not take off our clothes; each (kept) his
    weapon (even) at the water.' \hspace{\stretch{1}}[BHS Nehemiah 4:17]
    \end{exe}\todo{boldface is generally preferred for highlighting}%
%
The example in \REF{ex:heb-weapon} is striking because it involves both
constituent negation of the subject with the first person plural
independent pronoun (as well as conjoined noun phrases) and left
dislocation with the subject resumed in the sentence proper 
\parencite[see][]{MillerNaude2015}. It is also the only example in the Bible
which uses an independent subject pronoun for both the dislocated element
and the resumed element. In this respect, the example exhibits an early
change which was diffused and persisted in Qumran Hebrew and into Mishnaic
Hebrew. The left-dislocation construction is a plausible environment for
the birth of an idiolect which was subsequently embraced and diffused
throughout the linguistic community. 

Example \REF{ex:heb-weapon} is also striking for another reason. According to
the Subject Agreement Cycle as described in \citet[41]{Gelderen2011},
the cycle of change in subject agreement often begins with the
1\textsuperscript{st} and 2\textsuperscript{nd} person rather than
3\textsuperscript{rd} person. Van Gelderen describes three stages of the
Subject Agreement Cycle. In stage (A), a full pronoun is used for the
subject. In stage (B), a pronominal suffix is used for the subject. In
stage (C), a new nominal element is needed alongside the pronominal suffix,
usually a noun phrase functioning as the topic 
\parencite[41]{Gelderen2011}. The developments in Ancient Hebrew subject
agreement in constructions with the negative existential marker display a
trajectory of language change. Biblical Hebrew exhibits the stage (B) – the
subject is indicated as a pronominal suffix on the negative existential –
as illustrated in examples \REF{ex:heb-you-die} and \REF{ex:heb-dead}. Stage (B)
persists in Qumran Hebrew \REF{ex:heb-my-mother} and Mishnaic Hebrew
\REF{ex:heb-garlic}.
Stage (C) is attested in Biblical Hebrew in only one late, post-exilic case
\REF{ex:heb-weapon}, but it becomes more frequent in Qumran Hebrew
(\ref{ex:heb-qualified}, \ref{ex:heb-dressed}) and Mishnaic Hebrew
\REF{ex:heb-unable-to-help}. 

In the second construction involving negation of the participle with the
negative existential marker in Biblical Hebrew, the negative existential is
followed by an explicit noun phrase subject and the participle with its
objects and\slash or adjuncts:\todo{there was no ex. no. 32, following
numbers have been updated}{}
%
\begin{exe}\ex \label{ex:heb-righteous}
    \gll haṣ-ṣaddîq ʾāḇāḏ wə-ʾên ʾîš śām      ʿal=lēḇ \\
\textsc{art}-righteous perish.\textsc{pfv.3ms}   and-\textsc{neg.ex}   man
put.\textsc{ptcp} on=heart \\
    \glt `The righteous person perishes and no one considers (lit. puts it
    on the heart).' \hspace{\stretch{1}}[BHS Isaiah 57:1]
    \end{exe}
%
The construction is found in Qumran Hebrew:\todo{an odd closing bracket in
\REF{ex:heb-hands}?}{}
%
\begin{exe}\ex \label{ex:heb-hands}
    \gll w-ʾ{\cb}yn yd{\ob}yw{\cb} šṭ{\ob}w{\cb}pwt b-mym \\
  and-\textsc{neg.ex}   hands.\textsc{3ms}   wash.\textsc{pass.ptcp} in-water \\
    \glt `\ldots{} and his hands are not washed with water.'
\hspace{\stretch{1}}[DSSR 4Q277 f1ii:11]
    \end{exe}
%
The construction is also found in Mishnaic Hebrew:
%
 \begin{exe}\ex \label{ex:heb-donkey}
\gll ʾên ḥămôr yŏṣēʾ bə-mardaʿaṯ bi-zman še-ʾênāh qəšûrâ lô \\
\textsc{neg.ex} donkey go.out.\textsc{ptcp}   with-saddle      at-time    
\textsc{rel}-\textsc{neg.ex.3fs} tied.\textsc{ptcp.fs}   to.\textsc{3ms} \\
\glt `A donkey does not go out with its saddle cloth when it is not tied
to   him.'\\ \hspace{\stretch{1}}[M Šabbat 5:4]
\end{exe}
%
With a dislocated subject constituent, constructions of this type take the
shape of \REF{ex:heb-unable-to-help} above and do not manifest the shift from suffixes to independent pronouns.

In the third construction of the negative existential marker with the
participle, the negative existential occurs in a sentence in which a
participle does not have an explicit subject:
%
\begin{exe}\ex \label{ex:heb-deliver}
    \gll wə-ʾim=ʾên môšîaʿ ʾōṯānû wə-yāṣāʾnû ʾēlêḵā          
    \\ and-if=\textsc{neg.ex}   deliver.\textsc{ptcp} \textsc{obj.1cp}
    and-go.out.\textsc{ipfv.1cp}  to.\textsc{2ms}  \\
    \glt `\ldots{} if no one delivers us, then we will go out to you'
    \hspace{\stretch{1}}[BHS 1 Samuel 11:3]
    \end{exe}
%
This use of the negative existential marker is also found in Qumran Hebrew,
as illustrated in \REF{ex:heb-bury}:
%
\begin{exe}\ex \label{ex:heb-bury}
    \gll wʾn qbr   \\
  and-\textsc{neg.ex}    bury-\textsc{ptcp.ms} \\
    \glt `and no one buries' \hspace{\stretch{1}}[DSSR 4Q176:Frgs. 1--2, col.
    1:4]
    \end{exe}

In contrast to example \REF{ex:heb-donkey} in which the scope of the
negative existential is the sentence, in \REF{ex:heb-deliver} and
\REF{ex:heb-bury}, the negative existential marker syntactically modifies a
null (or, implicit) subject -- the scope of the negative existential
particle is the null subject constituent and not the entire predication. In
effect, the negative existential marker is functioning as a quantifier.
Three arguments have been advanced for this claim \parencites(see)()%
{MillerNaude2015}{NaudeMiller2016a}.
First, the participle is always in the unmarked masculine singular form, as
illustrated in \REF{ex:heb-battle}:
%
\begin{exe}\ex \label{ex:heb-battle}
    \gll tāqǝʿû bat-tāqôaʿ wǝ-hāḵîn hak-kol  wǝ-ʾên hōlēḵ lam-milḥāmâ \\
sound.\textsc{pfv.3mp} on.\textsc{art}-horn   and-prepare.\textsc{inf}
\textsc{art}-all  and-\textsc{neg.ex}   go.\textsc{ptcp.ms}
to.\textsc{art}-battle \\
    \glt `They have sounded the horn and everything is prepared, but no one
    goes to battle.' \hspace{\stretch{1}}[BHS Ezekiel 7:14]
    \end{exe}
%
The finite verb (\textit{they} \textit{have} \textit{sounded}) has a plural subject; the participle (\textit{goes}) is masculine singular and has an indefinite, non-referential subject.

Second, the negative existential may be followed by a prepositional phrase
which modifies the null subject and not the participle:
%
\begin{exe}\ex \label{ex:heb-none-but-me}
    \gll wə-ʾên mib-balʿāḏay môšîaʿ \\
and-\textsc{neg.ex}   from-beside.\textsc{1s}   save.\textsc{ptcp.ms} \\
    \glt `and no one beside me saves' \hspace{\stretch{1}}[BHS Isaiah
43:11]
    \end{exe}
%
Third, the negative existential as a quantifier may serve as the subject of
more than one participle, as in \REF{ex:heb-concerned}:
%
\begin{exe}\ex \label{ex:heb-concerned}
    \gll wǝ-ʾên=ḥōleh mikkem ʿālay  wǝ-gōleh ʾeṯ=ʾoznî \\
and-\textsc{neg.ex}=sick.\textsc{ptcp.ms}   from.\textsc{2mp}
over.\textsc{1s}  and-uncover.\textsc{ptcp.ms}
\textsc{acc}=ear.\textsc{1s} \\
    \glt
`and no one of you is concerned over me and informs me' (lit., and no one
of you is sick for me and uncovers my ear)\\
\hspace{\stretch{1}}[BHS 1 Samuel 22:8]
    \end{exe}
%
The negative existential
marker in \REF{ex:heb-concerned} cannot be understood as negating the two
predications expressed by the participles because 
negation of a predication in ancient Hebrew, either by the negative
existential marker or by the marker of standard negation, regularly
requires that the negative marker be overtly expressed with each
 predication  \parencite[for the ways in which negation in poetry may differ from
 prose in this regard, see][]{Miller2005}. Instead, the negative
existential marker in \REF{ex:heb-concerned} functions as a negative quantifier `no one'.


\section{Conclusion}\label{sec:Hebrew-5}
%
In this paper we have provided evidence that Ancient Hebrew manifests
diachronic change which corresponds to Croft's negative existential cycle.
While stages A (at the beginning of the cycle) and C {\textasciitilde} A
(at the end of the cycle) are only rarely attested in Biblical Hebrew,
variable stages A {\textasciitilde} B and B {\textasciitilde} C are well
attested. In Qumran Hebrew, variable stage A {\textasciitilde} B continues,
but the use of the negative existential marker expands its range of
constructions while the alternate construction for expressive negative
existentials (the standard negator \textit{lōʾ} + copula)
decrease. The same is true in Mishnaic Hebrew. Variable stage B
{\textasciitilde} C presents a similar kind of expansion involving the
negative existential marker in Qumran Hebrew and Mishnaic Hebrew and a
concomitant decrease in the alternative construction involving the copula.
The stages of the negative existential cycle thus reveal both complex
toleration of multiple constructions in single stages, as well as clear
diachronic trajectories of change from Biblical Hebrew into Qumran Hebrew
and Mishnaic Hebrew.

We have also demonstrated a trajectory of change from Biblical Hebrew to
Qumran Hebrew and Mishnaic Hebrew in the form of the subject of a
participial predicate negated with the negative existential marker. In
Biblical Hebrew, the subject in this construction is a pronominal suffix
affixed to the negative existential marker (synthetic morphology). In
Qumran Hebrew and Mishnaic Hebrew, the subject may continue to be a
pronominal suffix (synthetic morphology) or it may be an independent
subject pronoun (analytic morphology). The trajectory from a pronominal
suffix to an independent subject pronoun is in accord with the Subject
Agreement Cycle.

The syntactic data on negative existentials support a complexity approach
to language change and diffusion in ancient Hebrew in which there is a
diachronic trajectory from Biblical Hebrew to both Qumran Hebrew and
Mishnaic Hebrew. This is in direct contradiction to the claims of those who
deny any diachronic trajectories in ancient Hebrew\il{Hebrew|)}.

\section*{Sources}\todo{``Sources'' is the preferred name for this
section}{}
\begin{tabularx}{\textwidth}{@{} l X @{}}
BHS     &\textit{Biblia Hebraica Stuttgartensia}. Edited by K. Elliger \&
W. Rudolph. Stuttgart: Bibelgesellschaft, 1997.\\
DSSR    &\textit{The Dead Sea Scrolls Reader}. 2 vols. 2nd edn. Edited by
Donald W. Parry \& Emanuel Tov. Leiden: Brill, 2014.\\
HAE     &\textit{Handbuch der Althebraischen Epigraphik}. 3 vols. Edited by
Johannes Renz \& Wolfgang Röllig. Darmstadt: Wissenschaftliche
Buchgesellschaft, 1995.\\
M       &\textit{Six Divisions of the Mishnah (Shisha Sidrei Mishnah)}.
Eshkol edition. Accordance software, 2000.
\end{tabularx}

\section*{Acknowledgements}

We thank the editors and anonymous reviewers for their comments on an
earlier version of this paper. This work is based on research supported in
part by the National Research Foundation of South Africa (Jacobus A. Naudé
UID 85902 and Cynthia L. Miller-Naudé UID 95926). The grantholders
acknowledge that opinions, findings and conclusions or recommendations
expressed in any publication generated by the NRF supported research are
those of the authors, and that the NRF accepts no liability whatsoever in
this regard.

\section*{Abbreviations and symbols}

In addition to the Leipzig Glossing Rules, the following abbreviations are
used in the examples: \textsc{ab} = absolute form; \textsc{ex} =
existential; \textsc{jus} = jussive; \textsc{non-ind} = non-indicative. The
stem (\textit{binyān}) of verbal forms is not indicated; footnotes clarify
the glosses where necessary.

The equals sign represents the orthographic symbol \textit{maqqef} in the
Hebrew text, which indicates cliticization of one word on another. Square
brackets are used to indicate reconstructed letter(s) in places where the
text is fragmentary.

In the epigraphic texts cited from HAE, texts are identified by location
(e.g. Jer = Jerusalem, Lak = Lakish, Arad = Arad) and century (e.g. 6 = 6th
century BCE).

\href{https://en.wikipedia.org/wiki/ISO_639-3}{{ISO 639-3}}~code~\href{http://www-01.sil.org/iso639-3/documentation.asp?id=hbo}{{hbo}}

\todo{there were several unused references, omitted from this list}{}%
%{\sloppy
%\printbibliography[heading=subbibliography,notkeyword=this,notcategory=sources]}
{\sloppy\printbibliography[heading=subbibliography,notkeyword=this]}

\end{document} 
