\documentclass[output=paper]{langsci/langscibook} 
\author{Arja Hamari\affiliation{University of Helsinki}}  
\title{%From a negative existential into a negative auxiliary: 
The negative existential cycle in Moksha: From a negative existential into a negative auxiliary}
\shorttitlerunninghead{The negative existential cycle in Moksha}
% \chapterDOI{} %will be filled in at production


\abstract{Moksha (Mordvin, Uralic) has a complex negation system with
several negative markers. I examine two of these markers: the negative
existential \textit{ɑš} ({\textasciitilde} \textit{ɑjɑš}) and the
past tense negative auxiliary \textit{ɑš}-. This auxiliary is generally
assumed to have developed when the negative existential \textit{ɑš}
acquired the additional function of an auxiliary. This study demonstrates
that the negative existential cycle provides a framework to understand the
development of the negative existential into a negative auxiliary; the
negative existential entered the verbal domain as the short answer `no' and
was fused with the older negative auxiliary, \textit{iź}-, which
continues to be used as an alternative to \textit{ɑš}-. This study is
based on the analysis of corpus data: First, to clarify their relationship
in the contemporary language, the different functions of the negative
existential are introduced. Second, the competing paradigms of the two
negative auxiliaries \textit{ɑš}- and \textit{iź}- are investigated.

\textbf{Keywords:} Uralic, Moksha, negative existential, negative auxiliary, past tense}
\maketitle

\begin{document}

\section{Introduction}

Moksha\il{Moksha|(} and its closest sister language, Erzya, form the Mordvin\il{Mordvin|(} branch of the Uralic\il{Uralic|(} language family. The Proto-Mordvin period began in approximately 1500 BCE and this period was preceded by a protolanguage that was common to the Mordvin, \ili{Finnic} and \ili{Saami} languages. The division of Proto-Mordvin into Moksha and \ili{Erzya} probably began around the eighth century CE (\citealt[13--15]{Bartens1999}; \citealt[13–14]{Keresztes2011}).

  The Mordvin languages are spoken in Russia, with less than half of Mordvins living in the autonomous Republic of Mordovia, which is situated in the middle course of the Volga. The remaining Mordvins reside in the surrounding provinces of the Russian Federation as well as in the neighboring republics of Chuvashia and Tatarstan. According to the latest census of the Russian Federation in 2010, there are approximately 806,000\footnote{According to some sources, the number is 744,237 (for example, see \citealt{HamariAasmae2015}). However, this smaller figure does not include the persons who declared themselves either as Erzyas or Mokshas rather than Mordvins (cf. \href{http://www.gks.ru/free\_doc/new\_site/perepis2010/croc/Documents/Vol4/pub-04-01.pdf}{http://www.gks.ru/free\_doc/new\_site/perepis2010/croc/Documents/Vol4/pub-04-01.pdf}).} ethnic Mordvins. Of these, 431,600 were reported to have mastered either \ili{Erzya} or Moksha, but no reliable data is available on their native languages. However, it is estimated that around a third of the speakers speak Moksha, while two-thirds speak Erzya. The number of speakers of both languages are declining, as Russian is replacing them, especially among the younger generations.

  The negation system of both Mordvin languages is known to be complex. Both languages have different types of negative markers and their distribution is determined by factors such as the type of clause, type of predicate, tense, and mood. The system can be explained predominantly by innovations that occurred during the Proto-Mordvin period and are therefore shared by both sister languages (\citealt[140--144]{Bartens1999}; \citealt{Hamari2007,Hamari2011,Hamari2013,HamariAasmae2015}). However, the situation is somewhat different for the negative existentials. The Moksha and \ili{Erzya} languages have a special negator for existential and possessive clauses (Moksha \textit{ɑš}\footnote{For purposes of this study, the negative existential is referred to as \textit{aš}, while the negative auxiliary with the same stem is marked with a hyphen, that is, \textit{aš}-. This is because the negative existential can appear without further inflection, whereas the negative auxiliary always has a personal ending following the stem.} \textasciitilde{} \textit{ɑjɑš} and \ili{Erzya} \textit{ɑrɑś}), but the origin of both negators is uncertain and no common source can be reconstructed for these markers. In addition, only Moksha \textit{ɑš} has further developed the function of a past tense negative auxiliary. Moksha also has an older past tense negative auxiliary \textit{iź}- (dialectally \textit{əź}-), which has the etymological cognate \textit{eź}- in \ili{Erzya}.

  The purpose of this paper is to examine the development of the Moksha existential marker \textit{ɑš} into a past tense negative auxiliary \textit{ɑš}- of verbal clauses from the viewpoint of the negative existential cycle originally described by \citet{Croft1991}. I will begin the examination in \sectref{sec:2:3} by providing an overview of the negative constructions reconstructed for Proto-Uralic and of what is known about the development of negative existentials in Uralic languages. In \sectref{sec:2:4} I will introduce the negation system of Moksha to clarify the functions of the different negative markers. In \sectref{sec:2:5} I examine the several functions of the Moksha negative existential \textit{ɑš}. In \sectref{sec:2:6}, in order to determine the current situation for the negative existential cycle in the language, I will analyze the contemporary uses of the negative auxiliary \textit{ɑš}- and its relationship to the auxiliary \textit{iź}-. The development of the negative existential \textit{ɑš} and the auxiliary \textit{ɑš}- is discussed in \sectref{sec:2:7} and conclusions are presented in \sectref{sec:2:8}.

I previously studied the functions of the Moksha negative markers --
including the existential \textit{ɑš} and auxliliary \textit{ɑš}- in \textcites{Hamari2007}{Hamari2013}. However, this paper examines their mutual relationship in the contemporary language in the light of new data, and discusses the historical development of the functions in closer detail than in the earlier studies.

  The data for the present study were gathered from an electronic corpus referred to as MokshEr. The corpus is administered by The Research Unit for Volgaic Languages at the University of Turku. This corpus includes literary texts, such as journals and newspapers from the years 2002--2005, as well as works of fiction. The size of the corpus is approximately 485,000 words. The references to MokshEr indicate the locations of the data within the corpus itself.

\section{Transcription}\label{sec:2:2}

The Uralic Phonetic Alphabet (also known as the Finno-Ugric transcription system) was adopted to transcribe Moksha (see \citealt{SovijarviPeltola1977}). It is important to mention that Moksha has a reduced vowel /ə/ that usually occurs in an unstressed position, predominantly in non-initial syllables but also in unstressed initial syllables. This reduced vowel has both velar and palatal allophones. The Uralic Phonetic Alphabet represents the palatal allophone as /ə/ and the velar as /ǝ̑/. In the transcriptions of the present study, the Moksha reduced vowel is indicated by a /ə/ but its velar and palatal allophones are not differentiated.

\section{Negative markers in the Uralic languages}\label{sec:2:3}

Two negative markers are reconstructed in the Uralic protolanguage: the negative auxiliary *\textit{e}- in standard negation, and the imperative auxiliary *\textit{elV}-, which may have been some type of extension or a supplementary form of *\textit{e}- \citep[37]{Janhunen1982}. According to Janhunen, the negative auxiliaries probably carried the marking of the subject person, tense, and mood, while the lexical verb had a fixed form with a suffix in *-\textit{k}\footnote{In Uralic linguistics, the form of the lexical verb that occurs in a negative auxiliary is often referred to as a connegative.}. This suffix was most likely an original nominalizer of verbs, and it was also used to mark the imperative of the second person singular in the affirmative. \citet[241--242]{Honti1997} argues that as the form of the lexical verb in standard negation was originally based on a nominalized verb form, the negative constructions could have originated as copula clauses with a negative copula verb, as in *\textit{e-m mene-k} `I am not a goer' (*-\textit{m} `1\textsc{sg}', *\textit{mene}- `go') > `I don't go, I am not going'. Most, but not all, contemporary Uralic languages have retained at least some traces of the original negative markers *\textit{e}- and/or *\textit{elV}- in their negation of verbal clauses; in many languages, these are still negative auxiliaries with conjugational properties but in others, some form of the auxiliary may have developed into a generalized negative particle (for example, see \citealt{Comrie1981}).

  In addition to the negators of verbal clauses, many Uralic languages have separate negative markers for non-verbal clauses. \citet{Veselinova2015} examines the special negators in the Uralic languages that negate stative predications, that is, predications without a verbal predicate. Veselinova concludes that this language family has three types of special negators: (i) \textit{Negative existentials} are most typically used to negate existence, location, and possession; (ii) \textit{Ascriptive negators} are used to negate predications in identity, class inclusion, and property attribution; and (iii) \textit{General stative negators} negate all stative predications. The special negators that most widely occur in the Uralic languages are the negative existentials and while ascriptive negators are also rather common, general stative negators only occur in \ili{Udmurt} and in (now extinct) \ili{Kamas}. None of these negators, however, descend from Proto-Uralic but instead must be regarded as more recent innovations (see \citealt[567--568, 570--571, 572]{Veselinova2015} for references to diachronic information). As a consequence, if special negators of stative predications existed in Proto-Uralic, no evidence can be found in the daughter languages to suggest this.

  The development of negative existentials that occur in contemporary Uralic languages differ. \citet[566--567]{Veselinova2015} concludes that a negative existential can be (a) a fusion of a negative marker and a (nominalized) form of a copula or copula-like verb, (b) a specified function of a particular form of the original negative verb, (c) a reanalysis of a word with an inherently negative content, or (d) a borrowing. Furthermore, as Bartens (\citeyear{Bartens1996}, in passim) observes, both the affirmative and negative existentials of Uralic languages typically have nominal properties. As I later demonstrate in \sectref{sec:2:7.1}, mechanisms that are usually suggested as explanations for the development of the Moksha negative existential are the fusion of a negative marker and a copula-like verb or reanalysis.

\section{Negation in Moksha}\label{sec:2:4}

The complexity of the Mordvin negation systems and their development have been addressed in several studies (for example, see \citealt[140--144]{Bartens1999}; \citealt[87--87]{Keresztes2011}; \citealt{Hamari2007,Hamari2011,Hamari2013}), and I will therefore not provide a full account of negation in Moksha. The formation of verbal negation in Moksha is summarized in \tabref{tab:2:1}. Moksha is a pro-drop language and the person and number of the subject are expressed in the verbal suffixes. This means that the examples of both negative and affirmative constructions in \tabref{tab:2:1} can be considered full clauses.

\begin{table}\begin{small}
\newlength{\colmood}\settowidth{\colmood}{\textbf{Desiderative}}
%\newlength{\colindic}\settowidth{\colindic}{\textbf{Indicative}}
\newlength{\coltense}\settowidth{\coltense}{\textbf{Second}}
\newlength{\colnegator}\settowidth{\colnegator}{auxiliary}
\newlength{\colneggloss}\settowidth{\colneggloss}{\textsc{neg.imp-opt-2sg} \hspace{.75em} sing.\textsc{cng}}
\newlength{\colaffgloss}\settowidth{\colaffgloss}{\small `I sing / I am singing}
\newlength{\colnegaffgloss}\settowidth{\colnegaffgloss}{auxiliary\hspace{.75em}\textsc{neg.imp-opt-2sg}\hspace{.75em} sing.\textsc{cng}}
\begin{tabularx}{\textwidth}%
%{@{} L{\colmood} L{\coltense} L{\colnegator} L{\colneggloss} L{\colaffgloss} @{}}%
{@{} l l l l @{}}%
\lsptoprule
\textbf{Mood}		&\textbf{Tense}		&{\begin{tabularx}{\colnegaffgloss}{@{} l l @{}}
										\textbf{Negator}			&{\begin{tabularx}{\colneggloss}{@{} l @{}}
																\textbf{Example of a}\\
																\textbf{negative clause}
																\end{tabularx}}		
										\end{tabularx}}																	&{\begin{tabularx}{\colaffgloss}{@{} l @{}}
																														\textbf{Corresponding}\\
																														\textbf{affirmative clause}
																														\end{tabularx}}							\\ \midrule
					&{\begin{tabularx}{\coltense}{@{} l @{}}
					Present\\tense\end{tabularx}}
										&{\begin{tabularx}{\colnegaffgloss}{@{} l l @{}}
										{\begin{tabularx}{\colnegator}{@{} l @{}}
										particle \\
										\textit{ɑf}
										\end{tabularx}}			&{\begin{tabularx}{\colneggloss}{@{} l @{\hspace{1ex}} l @{}}
																\textit{ɑf}&\textit{mor-ɑn}\\%
																\textsc{neg}&sing-\textsc{prs.1sg}\\%
																\multicolumn{2}{@{} l @{}}{
																`I do not sing / I am not}\\
																\multicolumn{2}{@{} l @{}}{
																singing / I will not sing'
																}\end{tabularx}}											
										\end{tabularx}}																	&{\begin{tabularx}{\colaffgloss}{@{} l @{}}
																														\textit{mor-ɑn}\\
																														sing-\textsc{prs.1sg}\\
																														`I sing / I am singing\\
																														 / I will sing'
																														\end{tabularx}}							\\ \tablevspace%[2em]
																																								%\cline{2-4} \\[-.5em]																									
\textbf{Indicative}	&{\begin{tabularx}{\coltense}{@{} l @{}}
					First\\past\\tense\end{tabularx}}		
										&%\multicolumn{2}{@{} l @{}}{
										{\begin{tabularx}{\colnegaffgloss}{@{} l l @{}}
										{\begin{tabularx}{\colnegator}{@{} l @{}}
										a) \\
										auxiliary \\
										\textit{iź-}
										\end{tabularx}}	
																&{\begin{tabularx}{\colneggloss}{@{} l @{\hspace{1ex}} l @{}}
																\textit{iź-əń}&\textit{morɑ}\\%
																\textsc{neg.pst-pst.1sg}&sing.\textsc{cng}\\%
																\multicolumn{2}{@{} l @{}}{%
																`I did not sing'
																}\end{tabularx}}																	\\ \tablevspace %\cline{3-4} \tablevspace
																
										{\begin{tabularx}{\colnegator}{@{} l }
										b) \\
										auxiliary \\
										\textit{ɑš}-
										\end{tabularx}}
																&{\begin{tabularx}{\colneggloss}{@{} l @{\hspace{1ex}} l @{}}
																\textit{ɑ{š}-əń}&\textit{morɑ}\\%
																\textsc{neg.pst-pst.1sg}&sing.\textsc{cng}\\%
																\multicolumn{2}{@{} l @{}}{%
																`I did not sing'
																}\end{tabularx}}			
										\end{tabularx}}																	&{\begin{tabularx}{\colaffgloss}{@{} l @{}}
																														\textit{morɑ-ń}\\
																														sing-\textsc{pst.1sg}\\
																														`I sang'
																														\end{tabularx}}								\\ \tablevspace %\cline{2-4} \tablevspace
					&{\begin{tabularx}{\coltense}{@{} l @{}}
					Second\\past\\tense\end{tabularx}}
										&{\begin{tabularx}{\colnegaffgloss}{@{} l l @{}}
										{\begin{tabularx}{\colnegator}{@{} l @{}}
										particle \\
										\textit{ɑf}
										\end{tabularx}}			&{\begin{tabularx}{\colneggloss}{@{} l @{\hspace{1ex}} l @{}}
																\textit{ɑf}&\textit{morɑ-ľəń}\\%
																\textsc{neg}&sing-\textsc{pst.1sg}\\%
																\multicolumn{2}{@{} l @{}}{%
																`I didn't use to sing'
																}\end{tabularx}}
										\end{tabularx}}																	&{\begin{tabularx}{\colaffgloss}{@{} l @{}}
																														\textit{morɑ-ľəń}\\
																														sing-\textsc{pst.1sg}\\
																														`I used to sing'
																														\end{tabularx}}							\\ \midrule
\multicolumn{2}{@{} l}{\textbf{Imperative}}&{\begin{tabularx}{\colnegaffgloss}{@{} l l @{}}
										{\begin{tabularx}{\colnegator}{@{} l @{}}
										auxiliary \\
										\textit{ťɑ-}
										\end{tabularx}}			&{\begin{tabularx}{\colneggloss}{@{} l @{\hspace{1ex}} l @{}}
																\textit{ťɑ-t}		&\textit{morɑ}\\%
																\textsc{neg.imp-2sg}&sing.\textsc{cng}\\%
																\multicolumn{2}{@{} l @{}}{%
																`do not sing'
																}\end{tabularx}}
										\end{tabularx}}																	&{\begin{tabularx}{\colaffgloss}{@{} l @{}}
																														\textit{morɑ-k}\\
																														sing-\textsc{imp.2sg}\\
																														`sing'
																														\end{tabularx}}							\\ \midrule
\multicolumn{2}{@{} l}{\textbf{Optative}}	&{\begin{tabularx}{\colnegaffgloss}{@{} l l @{}}
										{\begin{tabularx}{\colnegator}{@{} l @{}}
										auxiliary \\
										\textit{ťɑ}
										\end{tabularx}} 			&{\begin{tabularx}{\colneggloss}{@{} l @{\hspace{1ex}} l @{}}
																\textit{ťɑ-z-ɑt}	&\textit{morɑ}\\%
																\textsc{neg.imp-opt-2sg}&sing.\textsc{cng}\\%
																\multicolumn{2}{@{} l @{}}{%
																`may you not sing'
																}\end{tabularx}}
										\end{tabularx}}																	&{\begin{tabularx}{\colaffgloss}{@{} l @{}}
																														\textit{morɑ-z-ɑt}\\
																														sing-\textsc{opt-2sg}\\
																														`may you sing'
																														\end{tabularx}}							\\ \midrule
\multicolumn{2}{@{} l}{\textbf{Desiderative}}	&{\begin{tabularx}{\colnegaffgloss}{@{} l l @{}}
										{\begin{tabularx}{\colnegator}{@{} l @{}}
										auxiliary \\
										\textit{ɑfəľ}-
										\end{tabularx}}			&{\begin{tabularx}{\colneggloss}{@{} l @{\hspace{1ex}} l @{}}
																\textit{ɑfəľə-ksəľəń}&\textit{morɑ}\\%
																\textsc{neg-des.1sg}&sing.\textsc{cng}\\%
																\multicolumn{2}{@{} l @{}}{%
																`I didn't intend to sing'
																}\end{tabularx}}
										\end{tabularx}}																	&{\begin{tabularx}{\colaffgloss}{@{} l @{}}
																														\textit{morɑ-ľəksəľəń}\\
																														sing-\textsc{des.1sg}\\
																														`I intended to sing'
																														\end{tabularx}}							\\ \midrule
\multicolumn{2}{@{} l}{\textbf{Conjunctive}}	&{\begin{tabularx}{\colnegaffgloss}{@{} l l @{}}
										{\begin{tabularx}{\colnegator}{@{} l @{}}
										auxiliary \\
										\textit{ɑfəľ}-
										\end{tabularx}}			&{\begin{tabularx}{\colneggloss}{@{} l @{\hspace{1ex}} l @{}}
																\textit{ɑfəľəń}&\textit{morɑ}\\%
																\textsc{neg.conj.1sg}&sing.\textsc{cng}\\%
																\multicolumn{2}{@{} l @{}}{%
																`if I did not sing'
																}\end{tabularx}}
										\end{tabularx}}																	&{\begin{tabularx}{\colaffgloss}{@{} l @{}}
																														\textit{morɑ-ľəń}\\
																														sing-\textsc{conj.1sg}\\
																														`if I sang'
																														\end{tabularx}}							\\ \midrule
\multicolumn{2}{@{} l}{\textbf{Conditional}}	&{\begin{tabularx}{\colnegaffgloss}{@{} l l @{}}
										{\begin{tabularx}{\colnegator}{@{} l @{}}
										particle \\
										\textit{ɑf}
										\end{tabularx}}			&{\begin{tabularx}{\colneggloss}{@{} l @{\hspace{1ex}} l @{}}
																\textit{ɑf}&\textit{morɑ-ńďäŕɑ-n}\\%
																\textsc{neg}&sing-\textsc{cond-1sg}\\%
																\multicolumn{2}{@{} l @{}}{%
																`if I do not sing'
																}\end{tabularx}}
										\end{tabularx}}																	&{\begin{tabularx}{\colaffgloss}{@{} l @{}}
																														\textit{morɑ-ńďäŕɑ-n}\\
																														sing-\textsc{cond-1sg}\\
																														`if I sing'
																														\end{tabularx}}							\\ \midrule
\multicolumn{2}{@{} l}{%
\begin{tabularx}{\colmood}{@{} l}
\textbf{Conditional-}\\
\textbf{conjunctive}\end{tabularx}}			&{\begin{tabularx}{\colnegaffgloss}{@{} l l @{}}
										{\begin{tabularx}{\colnegator}{@{} l @{}}
										particle \\
										\textit{ɑf}
										\end{tabularx}}			&{\begin{tabularx}{\colneggloss}{@{} l @{\hspace{1ex}} l @{}}
																\textit{ɑf}&\textit{morɑ-ńďäŕɑľəń}\\%
																\textsc{neg}&sing-\textsc{cond.conj.1sg}\\%
																\multicolumn{2}{@{} l @{}}{%
																`if I hadn't sung'
																}\end{tabularx}}
										\end{tabularx}}																	&{\begin{tabularx}{\colaffgloss}{@{} l @{}}
																														\textit{morɑ-ńďäŕɑľəń}\\
																														sing-\textsc{cond.conj.1sg}\\
																														`if I had sung'
																														\end{tabularx}}							\\ \lspbottomrule %\end{comment}%
\end{tabularx}
\end{small}
\caption{The negation of verbal clauses in Moksha}%
\label{tab:2:1}\end{table}

 As can be seen in \tabref{tab:2:1}, the negation patterns of the present and second past tense indicative as well as the conditional and conditional-conjunctive moods are symmetric: The only difference between the affirmative and negative verb forms is the existence of the negative particle before the inflected predicate verb in the negative construction. The negation of all other verb forms is asymmetric: These forms are negated by negative auxiliaries followed by an invariant connegative form of the lexical verb, which means that the marking of finiteness appears in the negative marker instead of the lexical verb. (See \citealt{Miestamo2005} for a detailed study on symmetric and asymmetric negation.) However, the negative auxiliary \textit{ɑfəľ}- of the desiderative and the conjunctive moods has most likely developed from a fusion of the particle \textit{af} and the inflected form of the verb \textit{uľə}- `be' (for example, see \citealt[142]{Bartens1999}).

  In addition to the negative markers presented in \tabref{tab:2:1}, there are two negative suffixes: -\textit{fťäŕɑ}-/-\textit{fťäŕä}- of the conditional and -\textit{fťäŕɑľə}-/-\textit{fťäŕäľə}- of the conditional-conjunctive mood (\citealt[392--393]{Klemm1934}; \citealt[012]{Paasonen1953}; \citealt[221]{Pall1957}; \citealt[141]{Bartens1999}). These suffixes are fusions of the negative particle \textit{af} and the following auxiliary verbs: *\textit{ťäŕɑ}- `try' for the conditional and both *\textit{ťäŕɑ}- `try' and \textit{uľə}- `be' for the conditional-conjunctive \citep[129--137]{Bartens1999}. The auxiliary constructions followed the connegatives of the lexical verbs and finally agglutinated with them. However, the suffixes are extremely rare in contemporary Moksha, which prefers constructions formed on the particle \textit{af} and the affirmative form of the lexical verb.

  According to the classification proposed by \citet{Veselinova2015}, Moksha belongs to the group of the Uralic languages that have a negative existential but no other special negators for non-verbal (or stative) predications. The negative particle \textit{af} that occurs in verbal negation is used to negate ascriptive clauses, while the negative existential \textit{ɑš} (with a longer variant \textit{ɑjɑš}) occurs in existential and possessive clauses. In locative clauses, both are possible but with certain semantic differences (see \sectref{sec:2:5.3}). \tabref{tab:2:2} illustrates the functions of the different negators of the non-verbal predications in the present tense and \tabref{tab:2:3} lists the functions of the negators in the past tense.

\begin{table}
\begin{small}
\newlength{\colclausfunc}\settowidth{\colclausfunc}{\textbf{Existential}}
\settowidth{\colnegator}{\textbf{Negator}}
\settowidth{\colneggloss}{\textsc{1sg.gen}\hspace{1ex}\textit{ajaš}\hspace{1ex}con-\textsc{poss.1sg.sg}}
\newlength{\colnegex}\settowidth{\colnegex}{\textbf{Negator}\hspace{1ex}\textsc{1sg.gen}\hspace{1ex}\textit{ajaš}\hspace{1ex}son-\textsc{poss.1sg.sg}\hspace{1ex}}
\settowidth{\colaffgloss}{\textsc{1sg.gen neg} be-\textsc{prs.3sg} son-\textsc{poss.1sg.sg}}
\begin{tabularx}{\textwidth}%
{@{} l l l @{}}
\lsptoprule
{\begin{tabularx}{\colclausfunc}{@{} l @{}}
\textbf{Non-}%\\
\textbf{verbal}\\
\textbf{clause}\end{tabularx}}
						&{\begin{tabularx}{\colnegex}{@{} l l @{}}
						{\begin{tabularx}{\colnegator}{@{} l @{}}
						\textbf{Negator}
						\end{tabularx}}		
											&{\begin{tabularx}{\colneggloss}{@{} l @{}}
											\textbf{Example of a}\\
											\textbf{negative clause}\end{tabularx}}
						\end{tabularx}}																&{\begin{tabularx}{\colaffgloss}{@{} l @{}}
																									\textbf{Corresponding}\\
																									\textbf{affirmative clause}\end{tabularx}}\\ \midrule
\textbf{Ascriptive}		&{\begin{tabularx}{\colnegex}{@{} l l @{}}
						{\begin{tabularx}{\colnegator}{@{} l @{}}
						\textit{ɑf}
						\end{tabularx}}		&{\begin{tabularx}{\colneggloss}{@{}l @{\hspace{1ex}} l @{\hspace{1ex}} l@{}}
											(\textit{mon})	&\textit{ɑf}	&\textit{od-ɑn}\\
											\textsc{1sg}		&\textsc{neg}&young-\textsc{prs.1sg}\\
											\multicolumn{3}{@{}l@{}}{
											`I am not young'}\end{tabularx}}	
						\end{tabularx}}																&{\begin{tabularx}{\colaffgloss}{@{} l @{\hspace{1ex}} l @{}}
																									(\textit{mon})	&\textit{od-ɑn}\\
																									\textsc{1sg}		&young-\textsc{prs.1sg}\\
																									\multicolumn{2}{@{}l@{}}{
																									`I am young'
																									}\end{tabularx}}								\\ \midrule
\textbf{Existential}		&{\begin{tabularx}{\colnegex}{@{} l l @{}}
						{\begin{tabularx}{\colnegator}{@{} l @{}}
						a) \textit{ɑš}
						\end{tabularx}}		&{\begin{tabularx}{\colneggloss}{@{}l @{\hspace{1ex}} l @{\hspace{1ex}} l@{}}
											\textit{pɑkśɑ-sɑ}	&\textit{ɑš}	&\textit{trɑktər}\\
											field-\textsc{iness}		&\textsc{neg}	&tractor\\
											\multicolumn{3}{@{}l@{}}{
											`there is no tractor}\\
											\multicolumn{3}{@{}l@{}}{
											in the field'}\end{tabularx}}																			\\ \tablevspace %\cline{2-3} \tablevspace
						{\begin{tabularx}{\colnegator}{@{} l @{}}
						b) \textit{ɑjɑš}
						\end{tabularx}}
											&{\begin{tabularx}{\colneggloss}{@{}l @{\hspace{1ex}} l @{\hspace{1ex}} l@{}}
											\textit{pɑkśɑ-sɑ}	&\textit{ɑjɑš}	&\textit{trɑktər}\\
											field-\textsc{iness}		&\textsc{neg}	&tractor\\
											\multicolumn{3}{@{}l@{}}{
											`there is no tractor}\\
											\multicolumn{3}{@{}l@{}}{
											in the field'}\end{tabularx}}
						\end{tabularx}}																&{\begin{tabularx}{\colaffgloss}{@{}l @{\hspace{1ex}} l @{\hspace{1ex}} l@{}}
																									\textit{pɑkśɑ-sɑ}	&\textit{uľ-i}	&\textit{trɑktər}\\
																									field-\textsc{iness}		&be-\textsc{prs.3sg}	&tractor\\ 
																									\multicolumn{3}{@{}l@{}}{
																									`there is a tractor in the field'
																									}\end{tabularx}}											\\ \midrule
\textbf{Possessive}		&{\begin{tabularx}{\colnegex}{@{} l l @{}}
						{\begin{tabularx}{\colnegator}{@{} l @{}}
						a) \textit{ɑš}
						\end{tabularx}}		
											&{\begin{tabularx}{\colneggloss}{@{}l @{\hspace{1ex}} l @{}}
											\textit{moń}	&\textit{ɑš}	\\
											\textsc{1sg.gen}	&\textsc{neg}	\\
											\multicolumn{2}{@{}l@{}}{
											\textit{ćora-źä}}\\
											\multicolumn{2}{@{}l@{}}{
											son-\textsc{poss.1sg.sg}}\\
											\multicolumn{2}{@{}l@{}}{
											`I don't have a son'
											}\end{tabularx}}											\\ \tablevspace %\cline{2-3} \tablevspace
						{\begin{tabularx}{\colnegator}{@{} l @{}}
						b) \textit{ɑjɑš}
						\end{tabularx}}	 	&{\begin{tabularx}{\colneggloss}{@{}l @{\hspace{1ex}} l @{\hspace{1ex}} l@{}}
											\textit{moń}	&\textit{ɑjɑš}	&\textit{ćora-źä}\\
											\textsc{1sg.gen}	&\textsc{neg}	&son-\textsc{poss.1sg.sg}\\
											\multicolumn{3}{@{}l@{}}{
											`I don't have a son'
											}\end{tabularx}}		
						\end{tabularx}}																&{\begin{tabularx}{\colaffgloss}{@{}l @{\hspace{1ex}} l @{}}
																									\textit{moń}	&\textit{uľ-i}\\
																									\textsc{1sg.gen}	&be-\textsc{prs.3sg}\\
																									\multicolumn{2}{@{}l@{}}{
																									\textit{ćora-źä}}\\
																									\multicolumn{2}{@{}l@{}}{
																									son-\textsc{poss.1sg.sg}}\\
																									\multicolumn{2}{@{}l@{}}{
																									`I have a son'}\end{tabularx}}							\\ \midrule
\textbf{Locative}			&{\begin{tabularx}{\colnegex}{@{} l l @{}}
						{\begin{tabularx}{\colnegator}{@{} l @{}}
						a) \textit{ɑf}
						\end{tabularx}}		&{\begin{tabularx}{\colneggloss}{@{}l @{\hspace{1ex}} l @{\hspace{1ex}} l@{}}
											(\textit{mon})&\textit{ɑf}	&\textit{pɑkśɑ-s-ɑn}\\
											\textsc{1sg}	&\textsc{neg}& field-\textsc{iness-prs.1sg}\\
											\multicolumn{3}{@{}l@{}}{
											`I am not in the field'
											}\end{tabularx}}											\\ \tablevspace %\cline{2-3} \tablevspace
						{\begin{tabularx}{\colnegator}{@{} l @{}}
						b) \textit{ɑš}
						\end{tabularx}}		&{\begin{tabularx}{\colneggloss}{@{}l @{\hspace{1ex}} l @{\hspace{1ex}} l@{}}
											(\textit{mon})&\textit{ɑš-ɑn}	&\textit{pɑkśɑ-sɑ}\\
											\textsc{1sg}	&\textsc{neg-prs.1sg}	&field-\textsc{iness}\\
											\multicolumn{3}{@{}l@{}}{
											`I am not in the field'
											}\end{tabularx}}
						\end{tabularx}}																&{\begin{tabularx}{\colaffgloss}{@{}l @{\hspace{1ex}} l @{}}
																									(\textit{mon})	&\textit{pɑkśɑ-s-ɑn}\\
																									\textsc{1sg}		&field-\textsc{iness-prs.1sg}\\
																									\multicolumn{2}{@{}l@{}}{
																									`I am in the field'
																									}\end{tabularx}}											\\ \lspbottomrule
\end{tabularx}\caption{The negation of non-verbal clauses in the present tense in Moksha.}
\label{tab:2:2}
\end{small}
\end{table}

\begin{table}
\begin{small}
\settowidth{\colclausfunc}{\textbf{Existential}}
\settowidth{\colnegator}{\textbf{Negator}}
\settowidth{\colneggloss}{(\textit{mon})\hspace{1ex}\textsc{neg}\hspace{1ex}field-\textsc{ine-pst.1sg}}
\settowidth{\colnegex}{\textbf{Negator}\hspace{1ex}(\textit{mon})\hspace{1ex}\textsc{neg}\hspace{1ex}field-\textsc{ine-pst.1sg}\hspace{1em}}
\settowidth{\colaffgloss}{\textbf{Corresponding affirmative clause}}
\begin{tabularx}{\textwidth}%
{@{} l l l @{}}
\lsptoprule
{\begin{tabularx}{\colclausfunc}{@{} l @{}}
\textbf{Non-}%\\
\textbf{verbal}\\
\textbf{clause}
\end{tabularx}}		&{\begin{tabularx}{\colnegex}{@{} l l @{}}
					{\begin{tabularx}{\colnegator}{@{} l @{}}
					\textbf{Negator}
					\end{tabularx}}	&{\begin{tabularx}{\colneggloss}{@{} l @{}}
									\textbf{Example of a}\\
									\textbf{negative clause}\end{tabularx}}
					\end{tabularx}}															&{\begin{tabularx}{\colaffgloss}{@{} l @{}}
																							\textbf{Corresponding}\\
																							\textbf{affirmative clause}\end{tabularx}}		\\ \midrule
\textbf{Ascriptive}	&{\begin{tabularx}{\colnegex}{@{} l l @{}}
					{\begin{tabularx}{\colnegator}{@{} l @{}}
					\textit{ɑf}
					\end{tabularx}}	&{\begin{tabularx}{\colneggloss}{@{} l @{\hspace{1ex}} l @{\hspace{1ex}} l @{}}
									(\textit{mon})	&\textit{ɑf}	&\textit{odə-ľəń}\\
									\textsc{1sg}		&\textsc{neg}&young-\textsc{pst.1sg}\\
									\multicolumn{3}{@{}l@{}}{
									`I was not young'
									}\end{tabularx}}
					\end{tabularx}}															&{\begin{tabularx}{\colaffgloss}{@{} l @{\hspace{1ex}} l @{}}
																							(\textit{mon})	&\textit{odə-ľəń}\\
																							\textsc{1sg}		&young-\textsc{pst.1sg}\\
																							\multicolumn{2}{@{}l@{}}{
																							`I was young'
																							}\end{tabularx}}								\\ \midrule
\textbf{Existential}	&{\begin{tabularx}{\colnegex}{@{} l l @{}}
					{\begin{tabularx}{\colnegator}{@{} l @{}}
					\textit{ɑš}
					\end{tabularx}}	&{\begin{tabularx}{\colneggloss}{@{} l @{\hspace{1ex}}  l @{}}
									\textit{pɑkśɑ-sɑ}	&\textit{ɑšə-ľ}	\\
									field-\textsc{iness}		&\textsc{neg-pst.3sg}\\
									\multicolumn{2}{@{}l@{}}{
									\textit{trɑktər}}\\
									\multicolumn{2}{@{}l@{}}{
									tractor}\\
									\multicolumn{2}{@{}l@{}}{
									`there was no tractor}\\
									\multicolumn{2}{@{}l@{}}{
									in the field'}\end{tabularx}}
					\end{tabularx}}															&{\begin{tabularx}{\colaffgloss}{@{} l @{\hspace{1ex}} l @{}}
																							\textit{pɑkśɑ-sɑ}	&\textit{uľ-ś}	\\
																							field-\textsc{iness}	&be-\textsc{pst.3sg}\\
																							\multicolumn{2}{@{}l@{}}{
																							\textit{trɑktər}}\\
																							\multicolumn{2}{@{}l@{}}{
																							tractor}\\
																							\multicolumn{2}{@{}l@{}}{
																							`there was a tractor}\\
																							\multicolumn{2}{@{}l@{}}{
																							in the field'
																							}\end{tabularx}}								\\ \midrule
\textbf{Possessive}	&{\begin{tabularx}{\colnegex}{@{} l l @{}}
					{\begin{tabularx}{\colnegator}{@{} l @{}}
					\textit{ɑš}
					\end{tabularx}} 	&{\begin{tabularx}{\colneggloss}{@{} l @{\hspace{1ex}} l @{}}
									\textit{moń}	&\textit{ɑšə-ľ}\\
									\textsc{1sg.gen}	&\textsc{neg-pst.3sg}\\
									\multicolumn{2}{@{}l@{}}{
									\textit{ćora-źä}}\\
									\multicolumn{2}{@{}l@{}}{
									son-\textsc{poss.1sg.sg}}\\
									\multicolumn{2}{@{}l@{}}{
									`I didn't have a son'
									}\end{tabularx}}
					\end{tabularx}}															&{\begin{tabularx}{\colneggloss}{@{} l @{\hspace{1ex}} l @{}}
																							\textit{moń}	&\textit{uľ-ś}\\
																							\textsc{1sg.gen}	&be-\textsc{pst.3sg}\\
																							\multicolumn{2}{@{}l@{}}{
																							\textit{ćora-źä}}\\
																							\multicolumn{2}{@{}l@{}}{
																							son-\textsc{poss.1sg.sg}}\\
																							\multicolumn{2}{@{}l@{}}{
																							`I had a son'
																							}\end{tabularx}}								\\ \midrule
\textbf{Locative}		&{\begin{tabularx}{\colnegex}{@{} l l @{}}
					{\begin{tabularx}{\colnegator}{@{} l @{}}
					a) \textit{ɑf}
					\end{tabularx}}	&{\begin{tabularx}{\colneggloss}{@{} l @{\hspace{1ex}} l @{\hspace{1ex}} l @{}}
									(\textit{mon})	&\textit{ɑf}	&\textit{pɑkśɑ-sə-ľəń}\\
									\textsc{1sg}		&\textsc{neg}&field-\textsc{iness-pst.1sg}\\
									\multicolumn{3}{@{}l@{}}{
									`I was not in the field'
									}\end{tabularx}}																						\\ \tablevspace %\cline{2-3} \tablevspace
					{\begin{tabularx}{\colnegator}{@{} l @{}}
					b) \textit{ɑš}
					\end{tabularx}}	&{\begin{tabularx}{\colneggloss}{@{} l @{\hspace{1ex}} l @{\hspace{1ex}} l @{}}
									(\textit{mon})	&\textit{ɑšə-ľəń}	&\textit{pɑkśɑ-sɑ}\\
									\textsc{1sg}    	&\textsc{neg-pst.1sg}  	&field-\textsc{iness}\\
									\multicolumn{3}{@{}l@{}}{
									`I was not in the field'
									}\end{tabularx}}
					\end{tabularx}}															&{\begin{tabularx}{\colneggloss}{@{} l @{\hspace{1ex}} l @{}}
																							(\textit{mon})	&\textit{pɑkśɑ-sə-ľəń}\\
																							\textsc{1sg}		&field-\textsc{ine-pst.1sg}\\
																							\multicolumn{2}{@{}l@{}}{
																							`I was in the field'
																							}\end{tabularx}}								\\ \lspbottomrule
\end{tabularx}\caption{The negation of non-verbal clauses in the past tense in Moksha.}
\label{tab:2:3}
\end{small}
\end{table}

  Finally, the negative particle \textit{apak} is used to negate participles and converbs. As both affirmative and negative participles can occur in the predicate position, \textit{apak} could also be regarded as a negator of non-verbal clauses. In this analysis, however, I will exclude these clauses because they are not prototypical stative expressions as the predicates have a verbal basis.

  Before moving on to the functions of \textit{ɑš} and \textit{ɑjɑš}, it is necessary to clarify how predication is expressed in Moksha. As is presented in \tabref{tab:2:2} and \tabref{tab:2:3}, the non-verbal predicates of ascriptive and locative clauses take the verbal personal suffixes and agree with the subject person and number. In the present tense, these suffixes are the normal personal endings, except for the third persons. The third person singular of non-verbal predication has no personal ending, whereas the third person plural takes the plural suffix -\textit{t}/-\textit{ť} of nouns instead of that of verbs.\footnote{However, historically, the verbal suffixes of the third person forms can be traced to participial forms with the participle ending in -\textit{i}. So the verb forms are, in fact, original nominal predicates with no person marking in the singular, and the plural ending in -\textit{t}/-\textit{ť} in the plural.}

  The situation for the past tense is slightly more complicated. In verbal predication, there are two past tense categories: the first past tense which is unmarked (for instance, \textit{morɑ-ń} `I sang') and the second past tense which has a habitual or progressive reading (such as \textit{morɑ-ľəń} `I used to sing, I was singing'). However, non-verbal predication only takes the second past tense, and in this case, it is unmarked, that is, it is not habitual or progressive but a neutral past tense (for example, \textit{odə-ľəń} `I was young').\footnote{Historically, the second past tense endings are personal forms of the verb \textit{uľə}- `be' that were attached to the predicate; this \textit{uľə}- `be' was conjugated in the first past tense.} \tabref{tab:2:4} presents the tense suffixes that are possible for non-verbal as compared to verbal predicates of Moksha.

\begin{table}\begin{small}
\newlength{\colnonverb}\settowidth{\colnonverb}{\textbf{Non-verbal predication of}}
\newlength{\colverb}\settowidth{\colverb}{\textbf{Verbal predication of}}
\newlength{\colfirst}\settowidth{\colfirst}{\textbf{First past}}
\newlength{\colsecond}\settowidth{\colsecond}{\textbf{Second past}}
\caption{Non-verbal predication versus verbal predication in Moksha.}
\label{tab:2:4}
\begin{tabularx}{\textwidth}{ l l l l l l }
\lsptoprule
			&\multicolumn{2}{l}{\begin{tabularx}{\colnonverb}{@{} l @{}}
			\textbf{Non-verbal predication of}\\
			\textbf{\textit{od} `young'}\end{tabularx}}
																					&\multicolumn{3}{l}{\begin{tabularx}{\colverb}{@{} l@{}}
																					\textbf{Verbal predication of}\\
																					\textbf{\textit{morɑ}- `sing'}\end{tabularx}}								\\ \midrule
			&\textbf{Present tense}		&{\begin{tabularx}{\colsecond}{@{} l @{}}
										\textbf{Second past}\\
										\textbf{tense}\end{tabularx}}					&\textbf{Present tense}	&{\begin{tabularx}{\colfirst}{@{} l @{}}
																											\textbf{First past}\\
																											\textbf{tense}\end{tabularx}}
																																	&{\begin{tabularx}{\colsecond}{@{} l @{}}
																																	\textbf{Second past}		\\
																																	\textbf{tense}\end{tabularx}}\\ \midrule
\textbf{1sg}	&\textit{od-ɑn} 				&\textit{odə-ľəń}							&\textit{mor-ɑn}			&\textit{morɑ-ń}		&\textit{morɑ-ľəń}\\
\textbf{2sg}	&\textit{od-ɑt}				&\textit{odə-ľəť}								&\textit{mor-ɑt}			&\textit{morɑ-ť}			&\textit{morɑ-ľəť}\\
\textbf{3sg}	&\textit{od}					&\textit{odə-ľ}								&\textit{mora-j}			&\textit{morɑ-ś}		&\textit{morɑ-ľ}\\
\textbf{1pl}	&\textit{od-tamɑ}			&\textit{odə-ľəmä}							&\textit{morɑ-tamɑ}		&\textit{morɑ-mä}		&\textit{morɑ-ľəmä}\\
\textbf{2pl} 	&\textit{od-tɑdɑ}				&\textit{odə-ľəďä}							&\textit{morɑ-tadɑ}		&\textit{morɑ-ďä}		&\textit{morɑ-ľəďä}\\
\textbf{3pl}	&\textit{od-t}				&\textit{odə-\textsc{ľ}t}						&\textit{mora-\textsc{j}ť}	&\textit{morɑ-śť}		&\textit{morɑ-\textsc{ľ}t}\\
\lspbottomrule
\end{tabularx}
\end{small}
\end{table}

\section{The negative existential \textit{aš}}\label{sec:2:5}

\subsection{The general properties of \textit{aš}}\label{sec:2:5.1}

If we adopt the definition suggested by \citet[118--139]{Veselinova2013}, we can state that Moksha \textit{ɑš} behaves similarly to a prototypical negative existential. First, as will be demonstrated, it is difficult to pinpoint a specific word class that \textit{ɑš} belongs to because it has different inflectional properties in different functions. Second, \textit{ɑš} is used to negate existence, possession and location, which are the most common contexts for negative existentials cross-linguistically. Third, \textit{ɑš} appears as a pro-sentence and a short word for `no', which are also frequent uses of negative existentials. Fourth, in existential and possessive clauses, \textit{ɑš} replaces the affirmative existential rather than negates it.

  In the following, the functional and semantic properties of \textit{ɑš} are considered in the order that reflects the order of frequency of functions found in the negative existentials cross-linguistically \citep[118--119]{Veselinova2013}. In \sectref{sec:2:5.2}, existential and possessive clauses are examined together, as their prototypical negative constructions resemble each other, while locative clauses are analyzed separately in \sectref{sec:2:5.3} due to their different predicational properties. \sectref{sec:2:5.4} presents the use of \textit{ɑš} as a negative pro-sentence and a negative interjection. Finally, the occurrences of \textit{ɑš} as a noun are considered in \sectref{sec:2:5.5}

\subsection{Existential and possessive clauses}\label{sec:2:5.2}

As the constructions of possessive clauses are rather similar to
existential clauses, both clause types will be addressed in this section. I
will first consider the affirmative constructions of these clause types and then focus on the negative forms.

  In Moksha existential clauses, the subject of the sentence is in the indefinite nominative form and, being indefinite, it is necessarily in the third person (see \tabref{tab:2:2}). While the existential sentence may express the plain existence of the referent without further specifications of a location, a locative phrase can be present, as in \REF{ex:moksha-stove}. In the affirmative, the existential predicate is the third person form of the verb `be', that is, (sg.) \textit{uľi}, (pl.) \textit{uľi\textsc{j}ť}. (For more details, see \citealt[47--52]{Hamari2007}.)

\ea\label{ex:moksha-stove}
\gll Kuχńa-sa pľita uľ-i, no son ušńə-ma penga-sa.\\
kitchen-\textsc{iness} stove be-\textsc{prs.3sg} but \textsc{3sg} warn-\textsc{inf} firewood-\textsc{iness}\\
\glt `There is a stove in the kitchen, but it must be warmed with firewood.' [MokshEr-V.3/V.1/Moksha/Moksha/2002/16.txt]
\z

	In the possessive clauses, the possessor is often referred to by a noun or a pronoun in the genitive case.\footnote{However, the plural personal pronouns are ambiguous as to the nominative and the genitive case (for example, \textit{śiń} is both \textsc{3pl.nom} and 3\textsc{pl.gen}).} Furthermore, the subject has a possessive suffix that refers to the possessor; as the possessor can be concluded from the possessive suffix, the noun or pronoun can be dropped, as in \REF{ex:moksha-cattle}. In the affirmative, the forms of the verb `be' are used as predicates. (See, for example, \citealt[52--57]{Hamari2007}.)

\ea\label{ex:moksha-cattle}
\gll Kudo-ńkä uľ-i, žuvata-ńkä uľ-i\textsc{j}ť.\\
house-\textsc{poss.1pl} be-\textsc{prs.3sg} cattle-\textsc{poss.1pl} be-\textsc{prs.3pl}\\
\glt `We have a house, we have cattle.' [MokshEr-V.3/V.1/Moksha/Moksha/2005/22.txt]
\z

  Both \textit{ɑš} and \textit{ɑjɑš} can occur in the negation of existential and possessive clauses, although \textit{ɑš} is far more common than \textit{ɑjɑš}. Moreover, according to Nadezhda Kabaeva (p.c.), \textit{ɑjɑš} is regarded as a colloquial form, whereas \textit{ɑš} is in general use both in the spoken and in the literary language. However, as both variants appeared in the written data of the present study, I will consider both of them.

  In the present tense, \textit{ɑš} and \textit{ɑjɑš} can be described as invariant negative predicates because neither of them agrees with a plural subject -- unlike the affirmative predicate based on the verb `be' that occurs in \REF{ex:moksha-stove} and \REF{ex:moksha-cattle}. \tabref{tab:2:2} shows the present tense existential and possessive clauses with a singular subject, whereas in \REF{ex:moksha-wheel} and \REF{ex:moksha-no-sons}, a plural subject occurs.

\ea\label{ex:moksha-wheel}
\gll \textbf{Ajɑš} trɑktə\textsc{r}-t, šɑri\textsc{j}ť aš.\\
\textsc{neg} tractor-\textsc{pl} wheel.\textsc{pl} \textsc{neg}\\
\glt `There are no tractors, there are no wheels.' [MokshEr-V.3/V.1/Moksha/Varia/C/1.txt]
\z

\ea\label{ex:moksha-no-sons}
\ea\label{ex:moksha-no-sons-1}
\gll Moń \textbf{ɑš} ćora-ńä.\\
\textsc{1sg.gen}    \textsc{neg}    son-\textsc{poss.1sg.pl}\\
\glt `I don't have sons.' [Nadezhda Kabaeva, p.c.]
\ex\label{ex:moksha-no-sons-2}
\gll Moń \textbf{ɑjɑš} ćora-ńä.\\
\textsc{1sg.gen}    \textsc{neg}    son-\textsc{poss.1sg.pl}\\
\glt `I don't have sons.' [Nadezhda Kabaeva, p.c.]
\z\z

  The plurality of the subject in existential clauses (such as \ref{ex:moksha-wheel}) is expressed by the plural suffix -\textit{t} / -\textit{ť}, whereas plurality in possessive clauses (such as \ref{ex:moksha-no-sons}) is indicated by the possessive suffix, which is attached to the possessee. This only applies when the possessor is one of the singular persons, as these persons have separate possessive suffixes for a singular and a plural possessee (such as \textit{ćora-źä} son-\textsc{poss.1sg.sg} `my son'; \textit{ćora-ńä} son-\textsc{poss.1sg.pl} `my sons'), as in \REF{ex:moksha-no-sons-1} and \REF{ex:moksha-no-sons-2}, respectively. When the possessor is in the plural, the number of the subject is not explicitly marked in the possessive suffixes (such as \textit{ćora-ńkä} son-\textsc{poss.1.pl.sg/pl} `our son; our sons'). \REF{ex:moksha-tail} and \REF{ex:moksha-eye} exemplify the latter instances of possessive clauses; in both clauses, the possessive suffix is ambiguous with regards to the number of the possessee and that number must be deduced from the context. 

\ea\label{ex:moksha-tail}
\gll Da śiń pulə-snə-vək \textbf{aš}.\\
and  3\textsc{pl.gen}    tail-\textsc{poss.3pl.sg/pl-clt}  \textsc{neg}\\
\glt `And they do not have tails either.' [MokshEr-V.3/V.1/Moksha/Moksha/2002/9.txt]
\z

\ea\label{ex:moksha-eye}
\gll Lomaťť, kaľi \textbf{ajaš} ťiń-gä śeľmə-ńťä?\\
person.\textsc{pl}  \textsc{q}      \textsc{neg}  \textsc{2pl.gen-clt}  eye-\textsc{poss.2pl.sg/pl}\\
\glt `People, don't either of you have eyes?' [MokshEr-V.3/V.1/Moksha/Mokshen\_pravda/2004-9/28.txt]
\z

  The main difference between the variants \textit{ɑš} and \textit{ɑjɑš} is that only \textit{ɑš} can be used in the past tense. Moreover, unlike the present tense in which \textit{ɑš} is invariant, it is inflected in the past tense; \textit{ɑš} acquires the suffix of the second past tense as well as the agreement marker of the plural subject (sg. \textit{ɑšə-ľ} (\textsc{neg-pst.3sg}); pl. \textit{ɑšə-\textsc{ľ}ť} (\textsc{neg-pst.3pl)}). Examples of past tense existential and possessive clauses with a plural subject are presented in \REF{ex:moksha-tractor} and \REF{ex:moksha-no-sons-pst}.

\ea\label{ex:moksha-tractor}
\gll pɑkśɑ-sɑ \textbf{ɑšə-\textsc{ľ}ť} trɑktə\textsc{r}-t\\
field-\textsc{iness}    \textsc{neg-pst.3pl}    tractor-\textsc{pl}\\
\glt `there were no tractors in the field' [Nadezhda Kabaeva, p.c.]
\z

\ea\label{ex:moksha-no-sons-pst}
\gll moń \textbf{ɑšə-\textsc{ľ}ť} ćora-ńä\\
\textsc{1sg.gen}    \textsc{neg-pst.3pl}    son-\textsc{poss.1sg.pl}\\
\glt `I didn't have sons' [Nadezhda Kabaeva, p.c.]
\z

  It should be noted that the negative existentials \textit{ɑš} and \textit{ɑjɑš} are remarkably different from their affirmative equivalent in terms of their inflectional properties. As was illustrated in Tables \ref{tab:2:2} and \ref{tab:2:3}, as well as in examples \REF{ex:moksha-stove} and \REF{ex:moksha-cattle}, existential and possessive clauses in Moksha have an affirmative existential \textit{uľi} that is actually the third person singular form of the verb \textit{uľə}- `be'. The affirmative existential agrees with the number of the subject. In the present tense, the regular verbal third person plural form \textit{uľi\textsc{j}ť} is used with a plural subject, whereas in the past tense, the first past tense forms (sg.) \textit{uľś}, and (pl.) \textit{uľśť} are used -- a tense form that is neither possible with a non-verbal predicate nor with the negative existential \textit{ɑš}.

  Finally, a special type of modal construction occurs where \textit{ɑš} appears before an interrogative pronoun. These constructions denote the impossibility to perform certain actions. These actions are expressed by verbs that usually take the infinitive, as in \REF{ex:moksha-book}.\footnote{It is important to emphasize that these are not cases of negative indefinite pronouns. The indefinite pronouns in Moksha are formed by attaching the suffix -\textit{vək}/-\textit{gək} {\textasciitilde} -\textit{ga}/-\textit{gä} {\textasciitilde} -\textit{ka}/-\textit{kä} to an interrogative pronoun. The resulting indefinite pronouns can be used either in an affirmative or a negative context (for example, \textit{kosa} `where': \textit{kosəvək} `somewhere, (not) anywhere, nowhere').} 
\ea\label{ex:moksha-book}
\gll \textbf{Aš} kosɑ, \textbf{ɑš} məźɑrdɑ kńigɑ-ńä luvə-ms,
\ob\ldots\cb\\
\textsc{neg}  where.\textsc{iness}  \textsc{neg}  when      book-\textsc{dim}    read-\textsc{inf}\\
\glt `There is no place and time to read a book' (Lit. ``There is no where, there is no when to read a book'') [MokshEr-V.3/V.1/Moksha/Moksha/2002/16.txt]
\z
No detailed analysis has thus far been published on these constructions, but it seems that they should be regarded as a type of a functional extension of existential clauses. This is because they have affirmative equivalents that are formed with the regular existential predicate, the verb `be', as in \REF{ex:moksha-loss}.

\ea\label{ex:moksha-loss}
\gll T’äńi uľ-i koda azə-ms: jumafksə-ńkä  oćuftə-\textsc{ľ}ť.\\
now be-\textsc{prs.3sg} how say-\textsc{inf} loss-\textsc{poss.1pl} big.\textsc{pl-pst.3pl}\\
\glt `Now it is possible to say: Our losses were great.' (Lit. ``Now there is how to say [\ldots]'') [MokshEr-V.3/V.1/Moksha/Moksha/2003-9-10/11.txt]
\z

\subsection{Locative clauses}\label{sec:2:5.3}

Moksha locative clauses require a locative phrase, as it is the predicate of the clause. The locative phrase acquires the endings of non-verbal predication, that is, the present or the second past tense as well as the marking of the subject person. The clause does not contain a copula or any other predicate item besides the locative phrase. Another feature that differentiates existential and locative expressions is that the subject of the locative clause is definite. This subject is either a personal pronoun or a noun with the definite nominative case suffix; nonetheless, the subject can be omitted because it is expressed in the personal ending of the locative predicate, as in \REF{ex:moksha-Moscow}:

\ea\label{ex:moksha-Moscow}
\gll Nona ši-ťńəń Mosku-sə-ľəń.\\
previous day-\textsc{gen.def.pl} Moscow-\textsc{iness-pst.1sg}\\
\glt `During the previous days I was in Moscow.' [V.1/Moksha/Moksha/2002/19.txt]
\z

  Negative clauses take either the particle \textit{ɑf} or the existential \textit{ɑš}. When the particle \textit{af} occurs, the predication is the same as in the affirmative equivalent: The conjugated locative phrase is the predicate of the clause, as in \REF{ex:moksha-nicely}.

\ea\label{ex:moksha-nicely}
\gll Mäľaft-k, ćora-j, af kud-s-at / \'Cebäŕńasta pačkəťť!\\
remember-\textsc{imp.2sg>3sg} boy-\textsc{voc} \textsc{neg} house-\textsc{iness-prs.2sg} {} nicely arrive.\textsc{imp.2sg}\\
\glt `Remember, my son, you are not at home / Go [to the house] nicely!' [V.1/Moksha/Varia/B/9.txt]
\z

  Tables \ref{tab:2:2} and \ref{tab:2:3} present locative clauses that may also be negated by \textit{ɑš}. The semantic difference between the functions of the ascriptive negator \textit{ɑf} and the existential negator \textit{ɑš} is identical to what appears between the functions of ascriptive and existential negators in \ili{Erzya} (see \citealt[91]{Hamari2007}); the ascriptive \textit{af} implies that the referent is not at the location expressed in the clause but somewhere else, whereas \textit{ɑš} negates the existence of the referent in the location without the assumption that the referent might be somewhere else (Nadezhda Kabaeva, p.c.). The difference can be captured by comparing examples \REF{ex:moksha-nicely} and \REF{ex:moksha-Vanja}. In \REF{ex:moksha-nicely} (retrieved from a poem), the listener is asked to approach a house respectfully because he is not at his own home but at a house that belongs to someone else. As a consequence, \textit{af} is used in negation. By contrast, the listener in example \REF{ex:moksha-Vanja} is asked whether or not he is home; the listener responds with a negative answer without implying further as to his location. For this reason, \textit{ɑš} appears.

\ea\label{ex:moksha-Vanja}
\gll -- Vɑńɑ, ton kud-s-ɑt? -- \textbf{Aš-ɑn} kud-sɑ, -- atvečɑ-ś śä.\\
{} Vańa \textsc{2sg} house-\textsc{iness-prs.2sg} {} \textsc{neg-prs.1sg} house-\textsc{iness} {} answer-\textsc{pst.3sg} it\\
\glt `--Vańa, are you at home? -- I am not at home, -- he answered.' [MokshEr-V.3/V.1/Moksha/Moksha/2002/32.txt]
\z

  In addition to the semantic difference, locative clauses that are negated with \textit{ɑf} and \textit{ɑš} also display a morphosyntactic difference. With the ascriptive negator \textit{ɑf,} the locative phrase remains the non-verbal predicate, as the negative marker is an invariant particle. Nonetheless, the existential negator \textit{ɑš} acquires the personal endings of the non-verbal conjugation in locative clauses -- a property that this negator does not exhibit in existential clauses. As a consequence, \textit{ɑš} could be regarded as the negative copula of the locative clause. The conjugations of \textit{ɑš} in the present tense and in the second past tense are presented in \tabref{tab:2:5} and I provide examples of the uses of the forms in \REF{ex:moksha-Vanja} in the present tense and in \REF{ex:moksha-never-Moscow} in the past tense, respectively.

\begin{table}
\caption{The non-verbal conjugation of \textit{ɑš} in locative clauses.}
\label{tab:2:5}
\begin{tabularx}{.5\textwidth}{l Q Q}
\lsptoprule
			&\textbf{Present tense}	&\textbf{Past tense}\\
\midrule
\textbf{1sg}	&\textit{ɑšɑn}			&\textit{ɑšəľəń}\\
\textbf{2sg}	&\textit{ɑšɑt}			&\textit{ɑšəľəť}\\
\textbf{3sg}	&\textit{ɑš}			&\textit{ɑšəľ}\\
\textbf{1pl}	&\textit{ɑšətɑmɑ}		&\textit{ɑšəľəmä}\\
\textbf{2pl}	&\textit{ɑšətɑdɑ}		&\textit{ɑšəľəďä}\\
\textbf{3pl}	&\textit{ɑšət}			&\textit{ɑšə\textsc{ľ}ť}\\
\lspbottomrule
\end{tabularx}
\end{table}

\ea\label{ex:moksha-never-Moscow}
\gll Mon Mosku-sa vesťə-vək \textbf{ɑšə-ľəń}.\\
\textsc{1sg} Moscow-\textsc{iness} once-\textsc{clt} \textsc{neg-pst.1sg}\\
\glt `I have never been to Moscow.' [MokshEr-V.3/V.1/Moksha/Moksha/2003-11-12/10.txt]
\z

It is important to note that the conjugational properties of \textit{ɑš}
in locative clauses are non-verbal rather than verbal: In the present
tense, the third person singular has no personal ending, whereas in the
third person plural, the plural suffix of nouns (instead of verbs)
occurs. Resembling non-verbal predicates, the second past tense is an
unmarked tense in negative locative expressions, which means that it does
not have the habitual or progressive meaning that it conveys in verbal clauses.

  The longer variant \textit{ɑjɑš} is not used in locative clauses that occur in written texts and consequently, it does not have non-verbal conjugation. When used colloquially, however, \textit{ɑjɑš}, sometimes acquires the same personal suffixes as \textit{ɑš} (Nadezhda Kabaeva, p.c.).

\subsection{Negative pro-sentences}\label{sec:2:5.4}

It is typologically common for negative existentials to become negative pro-sen\-ten\-ces and to be used as general words for `no' \citep[127]{Veselinova2013}. In this respect, the Moksha \textit{ɑš} is no exception. However, as I established in \citet[270--271]{Hamari2007}, the invariant Moksha negative markers, \textit{af} and \textit{ɑš} (\textasciitilde{} \textit{ɑjɑš}), are in complementary distribution as negative pro-sentences or one-word answers. The particle \textit{af} is selected for present tense verbal clauses or ascriptive clauses, whereas \textit{ɑš} (\textasciitilde{} \textit{ɑjɑš}) is normally used in contexts related to the existential, possessive and locative clauses. In addition, the variant \textit{ɑš} is used in verbal clauses of the first past tense.

  To illustrate the distribution of \textit{ɑf} and \textit{ɑš}, \REF{ex:moksha-Korochkov} and \REF{ex:moksha-watching} are cited as examples of the \textit{ɑf} used as a one-word answer to questions or commands involving a verbal clause in the present tense. In \REF{ex:moksha-residents}, on the other hand, \textit{af} is used in a context of an ascriptive clause. 
\ea\label{ex:moksha-Korochkov}
\gll -- Suva-k, požalsta, päľə-n! Aďä! -- \textbf{Af}, \textbf{af}, suv-śə-ms aš məźarda, -- atkaza-ś Koročkov.\\
{} enter-\textsc{imp.2sg} please side-\textsc{poss.1sg} come.on {} \textsc{neg} \textsc{neg} enter-\textsc{freq-inf}  \textsc{neg} when {} refuse-\textsc{pst.3sg} Koročkov\\
\glt `-- Please, enter my place! Come on! -- No, no, there is no time to enter, -- Koročkov refused.' [MokshEr-V.3/V.1/Moksha/Moksha/2002/3.txt]
\z 
\ea\label{ex:moksha-watching}
\gll -- S’kamə-t van-at? -- \textbf{Af}, tosa taga uľ-i\textsc{j}ť śťiŕ-ńa-t, śiń-gä van-i\textsc{j}ť\ldots\\
{} alone-\textsc{2sg} watch-\textsc{prs.2sg} {} \textsc{neg} there yet be-\textsc{prs.3pl} girl-\textsc{dim-pl} \textsc{3pl-clt} watch-\textsc{prs.3pl}\\
\glt `-- Are you watching [the calves] alone? -- No, there are other girls; they are also watching.' [MokshEr-V.3/V.1/Moksha/Varia/A/9.txt]
\z
\ea\label{ex:moksha-residents}
\gll -- Toń ťäďä-ćä-vək, ɑľä-ćä-vək Käšɑlə-ńńə-t?\\
{} \textsc{2sg.gen} mother-\textsc{poss.2sg-clt} father-\textsc{poss.2sg-clt} Käšal-\textsc{gen.ext-pl}\\
\sn
\gll -- \textbf{Af}.\\
{} \textsc{neg}\\
\glt `--Are your mother and father residents of Käšal? -- No.' [MokshEr-V.3/V.1/Moksha/Moksha/2003-5-6/23.txt]
\z
The existential \textit{aš} (\textasciitilde{} \textit{ɑjɑš}) is in turn found in existential, possessive and locative contexts \citep[270]{Hamari2007}, as shown in \REF{ex:moksha-uncle}, \REF{ex:moksha-right} and \REF{ex:moksha-Maksim}, respectively. 
\ea\label{ex:moksha-uncle}
\gll -- Aš mezevək, što-li? -- \textbf{Ajaš}, ot’sä-j!\\
{} \textsc{neg} anything \textsc{q-q} {} \textsc{neg} uncle-\textsc{voc}\\
\glt `Isn't there anything? -- No, my uncle!' [\citealt[270]{Hamari2007} < \citealt[888]{PaasonenRavila1947}]
\z
\ea\label{ex:moksha-right}
\gll Uľ-i\textsc{j}ť ľi ťiń kodaməvək prava-ńťä? Śembə-ń ťiń inksənt aťveč-an: \textbf{aš}.\\
be-\textsc{prs.3pl} \textsc{q} 2\textsc{pl.gen} any.kind.of right-\textsc{poss.2pl} all-\textsc{gen} 2\textsc{pl.gen} for.\textsc{poss.2pl} answer-\textsc{prs.1sg} \textsc{neg}\\
\glt `Do you have rights of any kind? I shall answer for all of you: no.' [\citealt[270]{Hamari2007} < Mokša 1/1998: 126]
\z
\ea\label{ex:moksha-Maksim}
\gll ``-- maksim·-tsä kut-sa?" -- ``\textbf{ajaš}, \textbf{ajaš},
\ob\ldots\cb''\\
{} Maksim-\textsc{poss.2sg} house-\textsc{iness} {} \textsc{neg} \textsc{neg}\\
\glt `-- Is your Maksim at home? -- No, no, [\ldots]' [\citealt[270]{Hamari2007} < \citealt[894]{PaasonenRavila1947}]
\z
Finally, \REF{ex:moksha-intelligence} and \REF{ex:moksha-opinion} are examples of the invariant \textit{ɑš} when it is used as a negative one-word answer to questions in the first past tense.
\ea\label{ex:moksha-intelligence}
\gll -- Estəńbeŕä Pŕɑvijə-ń ćentrɑ-ś lotkɑ-ś vɑno-mdɑ Moda-ť meľgä? \ob\ldots\cb\\
{} since.then intelligent-\textsc{gen} centre-\textsc{nom.def.sg} stop-\textsc{pst.3sg} look-\textsc{inf} Earth-\textsc{gen.def.sg} after {}\\
\sn
\gll -- \textbf{Aš}, ɑšəź lotka.\\
{} \textsc{neg} \textsc{neg.pst.3sg} stop.\textsc{cng}\\
\glt `-- Since then, the Centre of intelligence stopped watching over the Earth? -- No, it did not stop.' [MokshEr-V.3/V.1/Moksha/Moksha/2002/26.txt]
\z
\begin{exe}
\ex\label{ex:moksha-opinion}
\gll -- A toń koj-sə-t, meźəvək iź ľiśə?\\
{} but \textsc{2sg.gen} way-\textsc{iness-poss.2sg} nothing \textsc{neg.pst.3sg} go.\textsc{cng} \\
\sn
\gll -- Moń koj-sə-n, \textbf{ɑš}, \ob\ldots\cb\\
{} \textsc{1sg.gen} way-\textsc{iness-poss.1sg} \textsc{neg}\\
\glt `-- But in your opinion, nothing happened? -- In my opinion, no, [\ldots]' [MokshEr-V.3/V.1/Moksha/Moksha/2003-5-6/23.txt]
\end{exe}

  It is perfectly logical to use \textit{ɑš} in answers to questions in which existential, possessive or locative clauses appear (\ref{ex:moksha-uncle}--\ref{ex:moksha-Maksim}), because \textit{ɑš} is the regular negator of these clause types. In contrast the usage of \textit{ɑš} in verbal clauses in the past tense, such as (\ref{ex:moksha-intelligence}--\ref{ex:moksha-opinion}), is not as logical because in verbal clauses, \textit{ɑš} is always conjugated according to the subject (and possibly object) person of the clause. A possible explanation for this, could be that the use of \textit{ɑš} as a one-word answer \textit{preceded} the development of this marker into a negative auxiliary \citep[272--275]{Hamari2007}. The invariant existential \textit{ɑš} may have developed analogically to how the one-word negation \textit{af} was used as a short word for `no'. After all, the use of the invariant \textit{af} was restricted to the present and the second past tense as well as to certain moods. This may explain why speakers began to use the invariant \textit{ɑš} elsewhere -- including the first past tense where other option would have been a conjugated form of the negative auxiliary \textit{iź}-. As argued in \sectref{sec:2:7}, the use of the invariant \textit{ɑš} in one-word negations might be the key in understanding how the negative auxiliary \textit{ɑš}- developed from the negative existential.

\subsection{\textit{Aš} as a noun}\label{sec:2:5.5}

The variant \textit{ɑš} (but not \textit{ɑjɑš}) can be used as a lexical noun in its basic form or with further derivation \citep[268--270]{Hamari2007}. Without derivation, \textit{ɑš} has meanings such as `nothingness' or `poverty'. It can also be inflected in different cases: In \REF{ex:moksha-poverty}, \textit{ɑš} is followed by the inessive case suffix. However, a more extensive study would be needed to clarify the extent of its inflectional potential. 

\ea\label{ex:moksha-poverty}
\gll \ob\ldots\cb{} kodɑmɑ \textbf{ɑš-sɑ} eŕä-i\textsc{j}ť lomɑttńä!\\
{} what.kind.of poverty-\textsc{iness} live-\textsc{prs.3pl} person.\textsc{nom.def.pl}\\
\glt `what poverty people live in!' [MokshEr-V.3/V.1/Moksha/Moksha/2002/29.txt]
\z

  The derivational suffixes that can be attached to \textit{ɑš} are -\textit{ši}, which is used for abstract nouns (\textit{ɑšši} `extreme poverty, need') and the diminutive suffix -\textit{ńä} (\textit{ɑšəńä} `non-existence, smallness') (for example, \citealt[73]{MWb}). Furthermore, as observed by \citet[79]{Bartens1996}, \textit{ɑšu} `poor' is derived from \textit{ɑš} with the derivational suffix -\textit{u} of adjectives \REF{ex:moksha-poor}. This adjective is used as a base for further derivations, such as \textit{ɑšuši} `poverty' and \textit{ɑšustɑ} `poorly; in a poor way'.
\ea\label{ex:moksha-poor}
\gll \ob\ldots{\cb} \textbf{ɑšu} mokšə-ń śemjä-stɑ ćorɑ-ńä-ś
ɑrɑ-ś sodɑ-f pisɑťeľ-ks, \ob\ldots{\cb}\\
{} poor Moksha-\textsc{gen} family-\textsc{elat} boy-\textsc{dim-nom.def.sg} become-\textsc{pst.3sg} know-\textsc{pst.ptcp} author-\textsc{tra}\\
\glt `the little boy from a poor Moksha family became a well-know author' [MokshEr-V.3/V.1/Moksha/Moksha/2003-8/8.txt]
\z

  Finally, the form \textit{ɑšɑjkɑ} `not a thing, nothing' \citep[51]{MRV} is also derived from \textit{ɑš} \REF{ex:moksha-find}; It has another diminutive suffix, -\textit{(aj)ka}.
\ea\label{ex:moksha-find}
\gll Čɑst-škɑ-dɑ meľä mo\textsc{ľ}-ť tozɑ i muj-ɑt -- \textbf{ɑšɑjkɑ}.\\
hour-\textsc{cmpr-abl} after go-\textsc{pst.2sg} there.\textsc{ill} and find-\textsc{prs.2sg} {} not.a.thing\\
\glt `After about an hour you went there and find -- not a thing.' [MokshEr-V.3/V.1/Moksha/Moksha/2003-11-12/3.txt]
\z

  According to \citet[in passim]{Bartens1996}, it is rather common for existentials to have nominal uses in the Uralic languages; Negative existentials often convey meanings such as `nothingness', `smallness', and `poverty', while affirmative existentials denote `wealth', `riches', and `property'.

\section{The past tense auxiliary \textit{aš}-}\label{sec:2:6}

\subsection{Comparison of the existential \textit{aš} and auxiliary \textit{aš}-}\label{sec:2:6.1}

The Moksha auxiliary \textit{ɑš}- is synchronically separate from the negative existential \textit{ɑš}, as they have different functions, semantics and conjugational properties. The auxiliary \textit{ɑš}- is used for the negation in verbal clauses in the first past tense. When \textit{ɑš}- occurs in intransitive clauses as well as in transitive clauses with an indefinite object, it acquires the personal endings of the subjective conjugation. In other words, \textit{ɑš-} agrees with the subject person and number as presented in \tabref{tab:2:6}. It can also take a personal ending of the objective conjugation in which case it additionally agrees with the definite object person and number (see the paradigms in Tables \ref{tab:2:9} and \ref{tab:2:10} of \sectref{sec:2:6.2}). The lexical verb of the negative construction in the past tense takes the connegative form. In Moksha, the connegative form is the stem of the verb.\footnote{There is some alternation in the stem vowel of the connegative. If the stem ends in -\textit{ɑ} or the palatal allophone -\textit{ə} of the reduced vowel the stem vowel is usually preserved (for example, \textit{pala}- `kiss': \textit{ɑšəń} \textit{pala} `I did not kiss'; \textit{peľə}- `be afraid': \textit{ɑšəń peľə} `I was not afraid'). However, the stem-final -\textit{ə} is sometimes omitted and the stem ends in a consonant (as in \textit{ɑšəń peľ} `I was not afraid'). On the other hand, when the stem vowel is the velar allophone -\textit{ǝ̑} of the reduced vowel, it becomes -\textit{a} (as in \textit{udǝ̑}- `sleep': \textit{ɑšəń uda} `I did not sleep'). Finally, if the stem ends in the passive-reflexive derivational suffix -\textit{və}-, the vowel is omitted (for example, \textit{ɑtkɑzɑvə}- refuse, decline': \textit{ɑšəń ɑtkɑzɑv} `I did not refuse').}

\begin{table}
\caption{The subjective conjugation of the first past tense negative forms of \textit{mora}- `sing'}
\label{tab:2:6}
\begin{tabularx}{.33\textwidth}{XXX}
\lsptoprule
\textbf{\textsc{1sg}} & \textit{ɑšəń}	&\textit{mora}    \\
\textbf{\textsc{2sg}} & \textit{ɑšəť}		&\textit{mora}\\
\textbf{\textsc{3sg}} & \textit{ɑšəź}		&\textit{mora}\\
\textbf{\textsc{1pl}} & \textit{ɑšəmä}	&\textit{mora}  \\
\textbf{\textsc{2pl}} & \textit{ɑšəďä}		&\textit{mora}  \\
\textbf{\textsc{3pl}} & \textit{ɑšəśť}	&\textit{mora}\\
\lspbottomrule
\end{tabularx}
\end{table}

  Examples \REF{ex:moksha-road} and \REF{ex:moksha-leave} illustrate the use of the negative auxiliary \textit{ɑš}- in the first past tense. In \REF{ex:moksha-road}, the negative auxiliary takes the personal ending of the subjective conjugation, whereas \REF{ex:moksha-leave} has the objective conjugation.

\ea\label{ex:moksha-road}
\gll Mes \textbf{ɑšəť} kočkɑ ľijä ki?\\
why \textsc{neg.pst.2sg} choose.\textsc{cng} another road\\
\glt `Why didn't you choose another road?' [MokshEr-V.3/V.1/Moksha/Moksha/2003-5-6/23.txt]
\z
\ea\label{ex:moksha-leave}
\gll Son \textbf{ɑšəďäź} \textbf{kɑdɑ}!\\
\textsc{3sg} \textsc{neg.pst.3sg>2pl} leave.\textsc{cng}\\
\glt `He did not leave you!' [MokshEr-V.3/V.1/Moksha/Moksha/2005/33.txt]
\z

  To summarize, the auxiliary \textit{ɑš}- and existential \textit{ɑš} exhibit the following clausal differences in negation: (1) The auxiliary \textit{ɑš}- only appears in verbal clauses in connection with the connegative form of the lexical verb, whereas the use of the existential \textit{ɑš} is restricted to existential, possessive and locative non-verbal clauses in clausal negation; (2) The auxiliary \textit{ɑš}- is only used in the past tense, whereas the existential \textit{ɑš} expresses tense through its conjugation; (3) The auxiliary \textit{ɑš}- is conjugated according to the past tense of either the subjective or the objective conjugation of verbs, whereas the existential \textit{ɑš} is invariant in existential and possessive clauses in the present tense but acquires the suffixes of non-verbal predicates in the past tense as well as both present and past tenses of the locative clauses with definite subjects. \tabref{tab:2:7} presents the functions and inflectional properties of the auxiliary and the existential in more detail.

\begin{table}
\settowidth{\coltense}{\textbf{Present}}
\settowidth{\colclausfunc}{\textbf{3. Subject}}
\newlength{\colconj}\settowidth{\colconj}{\textbf{conjugation}}
\newlength{\colexist}\settowidth{\colexist}{encoding of the}
\newlength{\colloc}\settowidth{\colloc}{verbal suffixes of}
\newlength{\colaux}\settowidth{\colaux}{verbal suffixes of}
\caption{A comparison of the Moksha negative existential \textit{aš} and the auxiliary \textit{aš}-.}
\label{tab:2:7}
\begin{tabularx}{\textwidth}{@{} p{\colclausfunc} p{\coltense} p{\colexist} p{\colloc} p{\colaux} @{}}
															%Q Q Q @{}}
\lsptoprule
					&					&\textbf{\textit{Aš} in existential and possessive clauses}&\textbf{\textit{Aš} in locative clauses} 	&\textbf{Auxiliary \textit{aš}-}			\\
\tablevspace
\multicolumn{2}{@{} l @{}}{
\textbf{1. Clausal function}}			&Negative predicate of existential and possessive clauses.	&Negative copula of locative clauses.			&Negative auxiliary of a verbal clause.\\
\tablevspace
\textbf{2. Tense}	&\textbf{Present tense}
										& Invariant: no overt tense marking.							&Present tense personal suffixes of verbs, except in third person forms.
																																								&Not used in the present tense.		\\
					&\textbf{Past tense}
										&Second past tense only.										&Second past tense only.						&First past tense only.				\\
\tablevspace
\textbf{3. Subject encoding}
					&\textbf{Present tense}
										& Invariant: no encoding of the subject person or number. 	&Subject person and number encoded by verbal suffixes, except in the third person forms.
																																								&Not used in present tense.\\
					&\textbf{Past tense}
										&Encoding of a plural subject.									&Subject person and number encoded by verbal suffixes of the second past tense.
																																								& Subject person and number encoded by verbal suffixes of the first past tense.\\
\tablevspace
\multicolumn{2}{@{} p{\colconj} @{}}{
\textbf{4. Objective conjugation}}
										&No objective conjugation.										&No objective conjugation.						&Both subjective and objective conjugation (object person and number encoded by verbal suffixes).\\
\lspbottomrule
\end{tabularx}
\end{table}

\subsection{A comparison of the auxiliaries \textit{iź}- and \textit{aš}-}\label{sec:2:6.2}

Before discussing how the negative element \textit{aš} received its new function, it is necessary to examine the relationship between the negative auxiliary \textit{aš}- and its functional synonym, \textit{iź}-. As was noted previously in this analysis, the auxiliary \textit{iź}- must have had this function before \textit{ɑš}-. Grammatical descriptions of Moksha generally consider these two auxiliaries synonymous and completely interchangeable. Nevertheless, to my knowledge, their relationship has not been examined in detail. If \textit{aš}- were to compete or even gradually substitute \textit{iź}- in past tense negative clauses, this would be reflected in their contemporary uses. In this section, I explore their relationship by analyzing the frequency of their occurrence in the MokshEr corpus.

  The conjugational properties of the auxiliary \textit{iź}- are identical to those of \textit{ɑš}-. Similar to \textit{ɑš}-, the auxiliary \textit{iź}- is conjugated according to both subjective and objective conjugation and the personal endings are the same (see Tables \ref{tab:2:8}--\ref{tab:2:10}). The connegative form of the lexical verb is also the same for both auxiliaries. 

Both \textit{ɑš}- and \textit{iź}- appear in written contemporary Moksha in all personal forms of the subjective and objective conjugations. In addition, there are no differences as to the types of verbs they can occur with. Their interchangeability is further evidenced by the fact that both auxiliaries can be used within a single text and even within a single sentence, as in \REF{ex:moksha-evolution} and \REF{ex:moksha-vessel}. In fact, as these types of sentences are rather frequent, this suggests that the alternation of the auxiliaries is at least partly determined by stylistics factors. In other words, the purpose of this alternation is to avoid repetition when several negative constructions occur.

\ea\label{ex:moksha-evolution}
\gll Da, viďə-nc azə-ms, käľə-ń šačə-ma-kasə-ma-sa tufta\textsc{l}ńä, məźar-s kodamə-vək učonajə-ńďi lac-ŕäc \textbf{ašəśť} \textbf{sodav}, \textbf{iśť} \textbf{muv}.\\
yes truth-\textsc{gen.poss.3sg.sg} tell-\textsc{inf} language-\textsc{gen} be.born-\textsc{nmlz}-grow-\textsc{nmlz-iness} reason.\textsc{nom.def.pl}  how.many-\textsc{ill}    what.kind.of-\textsc{clt} scholar-\textsc{dat} well-in.order \textsc{neg.pst.3pl} be.known.\textsc{cng} \textsc{neg.pst.3pl} be.found.\textsc{cng}\\
\glt `Yes, to tell the truth, the reasons of the evolution of language have so far not been well known, not been discovered by any scholar.' [V.3/V.1/Moksha/Moksha/2003-x/1.txt]
\z
\ea\label{ex:moksha-vessel}
\gll No kɑrɑbəľ-ś \textbf{ɑšəź} \textbf{ku\textsc{l}cəndɑ}, \textbf{iź} \textbf{šɑrkśńə} ɑf śej, ɑf tov.\\
but vessel-\textsc{nom.def.sg} \textsc{neg.pst.3sg} obey.\textsc{cng} 
\textsc{neg.pst.3sg} turn.\textsc{cng} \textsc{neg} here.\textsc{lat} \textsc{neg}    there.\textsc{lat}\\
\glt `But the vessel didn't obey, didn't turn this way or that way.' [V.3/V.1/Moksha/Moksha/2002/26.txt]
\z

  The frequency of usage can also clarify the relationship of \textit{ɑš}- and \textit{iź}-. The fundamental assumption of the negative existential cycle is that in the intermediate stage B > C, the younger verbal negator that has developed from a negative existential gradually substitutes the older verbal negator. If we assume that there is competition between \textit{ɑš}- and \textit{iź}-, it should be possible to capture the current state of that competition by determining whether one is more common than the other in contemporary language. 

  \tabref{tab:2:8} summarizes the frequency of occurrence of \textit{ɑš}- and \textit{iź}- in the subjective conjugation in the MokshEr corpus. The table provides the number as well as the percentage of occurrence of each personal form. As can be seen, the auxiliary \textit{ɑš-} most commonly occurs with first and second person subjects, whereas \textit{iź}- is slightly more common with third person subjects. As a tentative hypothesis, it could therefore be proposed that the younger auxiliary \textit{ɑš}- has substituted the original \textit{iź}- in non-third person forms faster than in third person forms. A possible explanation for this is that the third person forms are more frequent than the others and may have resisted the change more persistently. After all, past tense auxiliaries occur most often when the subject is in the third person, with the singular being more frequent than the plural. The number of occurrences of \textit{ɑš}- is not far behind \textit{iź}- even in the third person forms, as \textit{ɑš}- appears in almost half of all the constructions.

\begin{table}\begin{small}
\caption{The subjective conjugation of \textit{aš}- and \textit{iź}- in the MokshEr corpus.}
\label{tab:2:8}
\begin{tabularx}{\textwidth}{lXXXXXXX}
\lsptoprule
							&\textbf{\textit{ɑš}-}		&			&			&\textbf{\textit{iź}-}	&			&			& \textbf{Total}\\ \midrule
\textbf{\textsc{1sg}}	&\textit{ɑšəń}			&175		& 82\% 	&\textit{iźəń}			&39		&18\% 	&214\\
\textbf{\textsc{2sg}}	&\textit{ɑšəť}				&43		&88\%		&\textit{iźəť}			&6			&12\%		&49\\
\textbf{\textsc{3sg}}	&\textit{ɑšəź}			&734		&44\%		&\textit{iź}				&921		&56\%		&1655\\
\textbf{\textsc{1pl}}		&\textit{ɑšəmä}			&29		&76\% 	&\textit{iźəmä}			&9			&24\%		&38\\
\textbf{\textsc{2pl}}		&\textit{ɑšəďä}			&10		&100\% 	&\textit{iźəďä}			&0 			&0\% 		&10\\
\textbf{\textsc{3pl}}		&\textit{ɑšəśť}			&202		&45\%		&\textit{iśť}				&244		&55\% 	&446\\ \midrule
							&								&			&			&							&			&			& 2412\\
\lspbottomrule
\end{tabularx}\end{small}
\end{table}
  Indeed it can be argued that the number of the non-third person forms in this corpus is rather small and interpretations must therefore be made cautiously. This need for caution is even more essential when analyzing the relationship of \textit{ɑš}- and \textit{iź}- from the perspective of the objective conjugation. \tabref{tab:2:9} presents the data of the past tense auxiliaries that occur with an object in the singular and \tabref{tab:2:10} for those with a plural object. As the data are extremely scarce, the frequency of occurrences is displayed in terms of the number, not in percentages. As can be seen, the frequency of all forms of the objective conjugation is extremely low, except for the forms of the singular third person objects. Even so, there is a clear tendency for \textit{ɑš}- to be more common than \textit{iź}- throughout the paradigm. The only exception is the form with a third person plural subject and a first person plural object. This ratio, nonetheless, can be regarded as being unreliable, as only one example of \textit{iź}- and no examples of \textit{ɑš}- were discovered in this category.\footnote{As there are 6 person-number categories in Moksha, the combinations of subject and object person in the objective conjugation would theoretically render 36 different endings in each table, but in practice, the number is smaller. This is because some of the forms are impossible and because there is rather extensive syncretism in the forms. Note also that in the case of \textit{ɑš}-, there are two possible endings for \textsc{3sg>3sg} -- one that is in accordance with the same form of \textit{iź}- and another in which the final vowel has been dropped, making the form identical to that of \textsc{3pl>3sg} and \textsc{3pl>3pl}. (See \citealt{Trosterud1994} and \citealt{Keresztes1999} for more details on the objective conjugation in the Mordvin languages.)}

\begin{table}\begin{small}
\caption{Objective conjugations of \textit{ɑš}- vs. \textit{iź}- in the MokshEr corpus (singular object).}
\label{tab:2:9}
\begin{tabularx}{\textwidth}{ l @{\hspace{2ex}} l @{\hspace{2ex}} l @{\hspace{2ex}} l @{\hspace{2ex}} l @{\hspace{2ex}} l @{\hspace{2ex}} l @{\hspace{2ex}} l @{\hspace{2ex}} l @{\hspace{2ex}} l @{\hspace{2ex}} l @{\hspace{2ex}} l @{\hspace{2ex}} l }
\lsptoprule
\textbf{O →}			&\multicolumn{4}{@{}X}{\textbf{\textsc{1sg}}}				&\multicolumn{4}{@{}X}{\textbf{\textsc{2sg}}}						&\multicolumn{4}{@{}X}{\textbf{\textsc{3sg}}}\\
\textbf{S ↓}			&\textbf{\textit{ɑš}-}	&		&\textbf{\textit{iź}-}&	&\textbf{\textit{ɑš}-}		&		&\textbf{\textit{iź}-}		&	&\textbf{\textit{ɑš}-}	&	&\textbf{\textit{iź}-}&\\
\textbf{\textsc{1sg}}	&-- 						&		&-- 					&	&\textit{ɑši\textsc{j}ťəń}	&1		&\textit{iźi\textsc{j}ťəń}&0 &\textit{ɑšińä}			&74	&\textit{iźińä}	&5\\
\textbf{\textsc{2sg}}	&\textit{ɑšəmɑjť}		&3		&\textit{iźəmɑjť}	&0	&-- 							&		&-- 						&	&\textit{ɑšiť}			&35	&\textit{iźiť}		&0\\
\textbf{\textsc{3sg}}	&\textit{ɑšəmɑń}		&8		&\textit{iźəmɑń}	&1 &\textit{ɑšəńźä}			&0 		&\textit{iźəńźä}		&0	&\textit{ɑšəźä}	&242	&\textit{iźəźä}	&11\\
					&						&		&					&	&							&		&						&	&\textasciitilde{} \textit{ɑšəź}	&4\\
\textbf{\textsc{1pl}}	&-- 						&		&-- 					&	&\textit{ɑšəďäź}			&1		&\textit{iźəďäź}		&0	&\textit{ɑš əśk}	&25	&\textit{iźəśk}	&3\\
\textbf{\textsc{2pl}}	&\textit{ɑšəmaśť}		&1		&\textit{iźəmaśť}	&0	&-- 							&		&-- 						&	&\textit{ɑšəśť}	&5		&\textit{iźəśť}	&0\\
\textbf{\textsc{3pl}}	&\textit{ɑšəmaź}		&6		&\textit{iźəmaź}	&1	&\textit{ɑšəďä\'ź}			&1		&\textit{iźəďäź}		&0	&\textit{ɑšəź}	&66	&\textit{iźəź}	&4\\
\lspbottomrule
\end{tabularx}\end{small}
\end{table}

\begin{table}\begin{small}
\caption{Objective conjugations of \textit{ɑš}- vs. \textit{iź}- in the MokshEr corpus (plural object).}
\label{tab:2:10}
\begin{tabularx}{\textwidth}{ l @{\hspace{2ex}} l @{\hspace{2ex}} l @{\hspace{2ex}} l @{\hspace{2ex}} l @{\hspace{2ex}} l @{\hspace{2ex}} l @{\hspace{2ex}} l @{\hspace{2ex}} l @{\hspace{2ex}} l @{\hspace{2ex}} l @{\hspace{2ex}} l @{\hspace{2ex}} l }
\lsptoprule
\textbf{O →}			&\multicolumn{4}{@{}X}{\textbf{\textsc{1pl}}}					&\multicolumn{4}{@{}X}{\textbf{\textsc{2pl}}}							&\multicolumn{4}{@{}X}{\textbf{\textsc{3pl}}}\\
\textbf{S ↓}			&\textbf{\textit{ɑš}-}	&		&\textbf{\textit{iź}-}	&	&\textbf{\textit{ɑš}-}		&		&\textbf{\textit{iź}-}		&	&\textbf{\textit{ɑš}-}	&	&\textbf{\textit{iź}-}&\\
\textbf{\textsc{1sg}} 	&-- 						&		&-- 					&	&\textit{ɑšəďäź}			&0		&\textit{iźəďäź}		&0	&\textit{ɑšińä}		&8	&\textit{iźińä}	&1\\
\textbf{\textsc{2sg}}	&\textit{ɑšəmaśť}		&1		&\textit{iźəmaśť}	&0	&-- 							&		&-- 						&	&\textit{ɑšiť}			&0	&\textit{iźiť}	&0\\
\textbf{\textsc{3sg}}	&\textit{ɑšəmaź}		&1		&\textit{iźəmaź}	&1	&\textit{ɑšəďäź}			&1		&\textit{iźəďäź}		&0	&\textit{ɑšəźəń}		&33	&\textit{iźəźəń}&2\\
\textbf{\textsc{1pl}}	&-- 						&		&-- 					&	&\textit{ɑšəďäź}			&0		&\textit{iźəďäź}		&0	&\textit{ɑšəśk}		&2	&\textit{iźəśk}	&0\\
\textbf{\textsc{2pl}}	&\textit{ɑšəmaśť}		&0		&\textit{iźəmaśť}	&0	&-- 							&		&-- 						& 	&\textit{ɑšəśť}    		&4	&\textit{iźəśť}	&0\\
\textbf{\textsc{3pl}}	&\textit{ɑšəmaź}			&0		&\textit{iźəmaź}	&1	&\textit{ɑšəďäź}			&1		&\textit{iźəďäź}		&0	&\textit{ɑšəź}			&40	&\textit{iźəź}	&0\\
\lspbottomrule
\end{tabularx}\end{small}
\end{table}
  The frequencies of occurrence suggests there is in fact competition between the negative auxiliaries \textit{ɑš}- and \textit{iź}-. Even though the auxiliaries can be used interchangeably in the same contexts, \textit{ɑš}- seems to be selected more often than \textit{iź}-. This could indicate that a gradual substitution of the older auxiliary by the newer one -- based on the negative existential -- is in progress.

\section{The development of Moksha \textit{ɑš} and \textit{ɑš}-}\label{sec:2:7}

\subsection{The negative existential \textit{ɑš}}\label{sec:2:7.1}

I mentioned in the introduction that the origin of the Moksha \textit{ɑ\v
s} is uncertain. The same applies to the negative existential
\textit{ɑrɑś} that occurs in the closest sister language, \ili{Erzya}.
The etymology of these two negative markers has been discussed in detail in
earlier literature \parencites[see][107--113]{Hamari2007}[477--479]{Hamari2013}, which is why I provide only a short summary on the development of the Moksha \textit{ɑš} {\textasciitilde} \textit{ɑjɑš}. As I observed previously, the Moksha \textit{ɑš} {\textasciitilde} \textit{ɑjɑš} and the Erzya \textit{ɑrɑś} most likely do not share an etymological connection. This means that both existentials have probably developed after the split of Proto-Mordvin. There is also no positive evidence of an earlier negative existential in the protolanguage. Regarding the origin of the Moksha \textit{ɑš} {\textasciitilde} \textit{ɑjɑš}, two hypotheses have been proposed:

\begin{exe}
\exi{1)} {According to \citet[388]{Klemm1934}, \textit{ɑš} could have
    originated from a combination of the negative particle in *\textit{ɑ} and the Moksha verb \textit{ɑšč:ə}- {\textasciitilde} \textit{ɑš:ə}- `be, be situated' ({\textasciitilde} Erzya \textit{ɑšťe}- `id.') that originally would have taken the connegative suffix in *-\textit{k} (*\textit{ɑ-ɑšč:ə-k} > *\textit{ɑšk} > \textit{ɑš}). The longer variant \textit{ɑjɑš} developed from a form in which /\textit{j}/ was introduced to prevent hiatus (*\textit{ɑ-j-ɑš} > \textit{ɑjɑš}).}
\exi{2)} {According to \citet[79]{Bartens1996}, \textit{ɑš} could have originally been a noun with meanings such as `non-existence' or `poverty'. Bartens elaborates that the variant \textit{ɑjɑš} could have an emphasizing prefix \textit{ɑj}-, which is also sometimes added to the negative particle \textit{ɑf} (> \textit{ɑjɑf}).}
\end{exe}

Klemm's assumption would fit the outline of the negative existential cycle proposed by \citet{Croft1991}, but its etymological explanation is questionable. The negative particle \textit{ɑ} and its variant \textit{ɑj} that occurs in the front of a verb with a word-initial vowel are only attested in \ili{Erzya}, while the Moksha equivalent of the Erzya \textit{ɑ} is \textit{ɑf}. In other words, I am more inclined to agree with Bartens, although the noun \textit{ɑš} `non-existence; poverty' also lacks etymology. Nonetheless, considering the non-verbal conjugation and the use of \textit{ɑš} as a noun, it is justified to assume that \textit{ɑš} was not originally a verb but a nominal item. 

\subsection{The negative auxiliary \textit{ɑš}-}\label{sec:2:7.2}

Theories on the origin of the Moksha past tense auxiliary \textit{ɑš}- have always taken into account its relationship to the negative existential \textit{ɑš}. The traditional view (originally presented by \citealt[148]{Szinnyei1884}) is that the auxiliary developed when verbal conjugation was introduced to the negative existential marker. However, there are both semantic and functional problems with the assumption that a negative marker used primarily with the existential present tense would begin to be used as a past tense negative auxiliary in verbal predicates without acquiring other functions as a negator of verbal clauses. In Hamari (\citeyear[275]{Hamari2007}; \citeyear[480]{Hamari2013}), I argue that the auxiliary \textit{aš}- may actually have originated as a combination of the existential \textit{aš} and the older negative auxiliary \textit{iź}-. Thus, the development of the auxiliary \textit{aš}- would represent an instance of the negative existential cycle where the negative existential begins to be used in verbal negation as a reinforcement for the regular verbal negator (cf. \citealt[10--11]{Croft1991}).

It has been suggested that the negative existential marker \textit{ɑš}
could have originally been attached to the older negative auxiliary
\textit{iź}- for emphasis (as in \textsc{1sg} *\textit{ɑš} +
\textit{iźəń} > *\textit{ɑšəźəń}\footnote{In Moksha, vowels
other than \textit{ɑ} and \textit{ä} are generally reduced in non-initial
syllables and this means that the change \textit{i} > \textit{ə} in the
construction is fully plausible.}, \textsc{2sg} *\textit{ɑš} +
\textit{iźəť} > *\textit{ɑšəźəť} etc.) \parencites[273--275]{Hamari2007}[479--480]{Hamari2013}. As the construction lost its emphatic force, it was analogically adapted to the conjugation of the auxiliary \textit{iź}- by dropping the element -\textit{əź}- (for example, \textsc{1sg} *\textit{ɑšəźəń} > *\textit{ɑšəń}, \textsc{2sg} *\textit{ɑšəźəť} > \textit{ɑšəť}). The element -\textit{əź}- has nonetheless been preserved in the first and second person forms of the subjective conjugation in the dialect of the Kovylkino district; \citet[178]{Schemerova1972} observes that this dialect has forms containing the element -\textit{əź}- which is absent from other dialects (Kovylkino: 1\textsc{sg} \textit{ɑšəźəń} `I did not', 2\textsc{sg} \textit{ɑšəźəť} `you did not'). These dialectal forms could be regarded as transparent relics of the fusion of the existential and the original past tense negative auxiliary.

  Another important point is that although the first and second person forms as well as the third person plural form of the auxiliary \textit{ɑš}- are the regular first past tense forms of lexical verbs (such as cf. 1\textsc{sg} \textsc{pst} \textit{pala}-\textbf{\textit{ń}} `I kissed' {\textasciitilde} \textit{ɑšə-}\textbf{\textit{ń}} `I did not'), the third person singular is irregular (cf. 3\textsc{sg} \textit{pala-}\textbf{\textit{ś}} `(s)he kissed' {\textasciitilde} \textit{ɑšə-}\textbf{\textit{ź}} `(s)he did not'). As the only verb that has a voiced palatalized sibilant as a third person singular marker is the negative auxiliary \textit{iź}- (3\textsc{sg} \textit{iź} `(s)he did not'), it can be argued that the fusion of the original existential and the auxiliary continues to be visible in the singular third- person form of the subjective conjugation of the auxiliary \textit{ɑš}- (3\textsc{sg} \textit{ɑšəź} < * \textit{ɑš} + \textit{iź}).

  One question still remains: what were the circumstances that led to the agglutination of existential \textit{ɑš} and the past tense negative auxiliary \textit{iź}-? There are no traces of \textit{ɑš} functioning to emphasize negation which, in my opinion, means that we could search for answers in the context of \textit{ɑš} in the past tense meaning, that is, as a pro-sentence.

As discussed in \sectref{sec:2:5.4}, the Moksha invariant \textit{ɑš} is
used as a pro-sentence, as a one-word answer to a question when it contains
a clause that is existential, possessive or locative, but also when the
question has a verbal predicate in the first past tense. This
clause-initial position \REF{ex:moksha-sing-a} could have offered a
possibility for an agglutination of the invariant \textit{ɑš} and a
following past tense auxiliary \REF{ex:moksha-sing-b}, accompanied by the
reduction of the vowel \textit{i} in an unstressed position. The next step
in this development would have been the agglutinated auxiliary form
\mbox{*\textit{ɑš-əź}} adjusting to the conjugation of the older auxiliary \textit{iź}-, and the disappearance of the element -\textit{əź}- \REF{ex:moksha-sing-c}. A final observation is that the invariant \textit{ɑš} could also appear as a one-word answer with the new negative auxiliary \textit{ɑš}- \REF{ex:moksha-sing-d}. It is important to note that steps \REF{ex:moksha-sing-a}, \REF{ex:moksha-sing-c} and \REF{ex:moksha-sing-d} are still possible in contemporary Moksha and even \REF{ex:moksha-sing-b} is possible in the Kovylkino dialect.
\begin{exe}
\ex\label{ex:moksha-sing}
Question:
\sn[--]{%
\gll morɑ-ť ɑľi ɑš?\\
sing-\textsc{pst.2sg} or \textsc{neg}\\
\glt `Did you sing or not?'}
Answer:
\sn{%
\begin{xlista}%
\ex[--]{\label{ex:moksha-sing-a}%
\gll ɑš, iźəń morɑ.\\
\textsc{neg}  \textsc{neg.pst.1sg}  sing.\textsc{cng}\\
\glt `No, I did not sing.'\\
>}
\ex[--]{\label{ex:moksha-sing-b}%
\gll ɑš-əźəń mora\\
\textsc{neg-neg.pst.1sg} sing.\textsc{cng}\\
\glt `I did not sing.'\\
>}
\ex[--]{\label{ex:moksha-sing-c}%
\gll ɑšəń mora.\\
\textsc{neg.pst.1sg} sing.\textsc{cng}\\
\glt `I did not sing.'\\
>}
\ex[--]{\label{ex:moksha-sing-d}%
\gll \textit{ɑš}, ɑšəń mora.\\
\textsc{neg} \textsc{neg.pst.1sg} sing.\textsc{cng}\\
\glt `No, I did not sing.'}
\end{xlista}}
\end{exe}

\citet[127--133]{Veselinova2013} observes that the process of negative existentials developing into a pro-sentence and then into a standard negator has also been observed in other languages.

\subsection{Discussion}\label{sec:2:7.3}

Moksha could be regarded as a language in Stage B of the negative existential cycle proposed by in \citet{Croft1991} because it has separate negative markers for verbal negation and for the negation of existential clauses. According to Croft, the negative existential marker may be found to function in ways that previously were characteristic of some other negative markers. Thus,  the negative existential begins to be used even in the negation of verbal clauses, and in time, this existential displaces the original negative marker of verbal clauses. When this type of displacement occurs, the negative existential becomes the only negative marker for verbal and existential clauses, and thus the cycle reaches the third stage, stage C. Croft's hypothesis is that the transition from stage B to C can proceed in the following ways:
\begin{exe}[h]
\exi{1)} ``the negative existential may compete with the ordinary verbal negator, sometimes being used instead of it''
\exi{2)} ``the negative existential can reinforce the (presumably older) regular verbal negator''
\exi{3)} ``(there is) only gradual substitution of the negative existential for the verbal negator in only part of the verbal grammatical system''
\sn{\citep[9--11]{Croft1991}}
\end{exe}

   The Moksha negative existential is not necessarily the result of an older negative marker of verbal clauses fusing with a positive existential, but the negative existential could also originally be a noun. Yet the further development of the negative existential displays characteristics of the cyclic development described by Croft, as the existential has penetrated the sphere of verbal negation by becoming a negative auxiliary of past tense verbal predicates. All in all, there are evident traces of all three means that Croft described. Thus, due to the use of the Moksha negative existential as a pro-sentence, the existential became used to possibly reinforce the older verbal negator \REF{ex:moksha-cattle} and subsequently a new verbal negator arose. The new negator that was based on the existential began competing with the ordinary negator and became the most frequently chosen option \REF{ex:moksha-stove}. Finally, the new negator has not supplanted the entire negation system, but this form is only used in one part of the verbal grammatical system \REF{ex:moksha-wheel}: to negate past tense verbs. 

  Etymological evidence suggests that the time span of the evolution of the Moksha negative existential and its development into a negative auxiliary spans approximately one thousand years. As there is no cognate for the Moksha negative existential in the closest sister language, \ili{Erzya} or in any other Uralic language, this negative existential cannot be dated beyond the division of Proto-Mordvin that began around the eighth century.

\section{Conclusions}\label{sec:2:8}

I have demonstrated that \textit{ɑš} occurs in many negative constructions of Moksha and has varied inflectional properties in different clausal functions. It is invariant as a negative pro-sentence, and it is likewise invariant in present tense existential and possessive clauses, with the exception of taking the second past tense marker when necessary. In locative clauses \textit{ɑš} is conjugated in the present and the second past tenses. As a noun, \textit{ɑš} is subject to derivation and case inflection and finally, as a past tense negative auxiliary, \textit{ɑš}- is exclusively conjugated in the first past tense and used only in verbal clauses. 

  There are many possible reasons for the negative existential \textit{ɑš} developing different functions. It may have originally been a noun meaning `non-existence; poverty' and acquired personal suffixes in the predicate position of non-verbal clauses, such as in existential and possessive expressions. It also became an invariant negative pro-sentence in contexts where the other one-word negator \textit{af} could not appear. In this clause-initial position, \textit{ɑš} may have agglutinated into the earlier negative auxiliary \textit{iź}- and created its own past tense personal conjugation. Consequently, both \textit{ɑš}- and \textit{iź}- are used as past tense negative auxiliaries in contemporary Moksha. The higher frequency of \textit{ɑš}- may indicate that it is gradually replacing the original auxiliary.

  Finally, \textit{ɑš} has a longer invariant form \textit{ɑjɑš} which is only found in present tense existential and possessive clauses as well as a negative word for `no'. Similarly, the particle \textit{af} has the variant \textit{ajaf}, which suggests the prefix \textit{aj}- was originally most probably an emphasizing element.\il{Mordvin|)}\il{Moksha|)}\il{Uralic|)}

\nocite{Moksha1998-1,PaasonenRavila1947}
\addtocategory{sources}{Moksha1998-1,PaasonenRavila1947}% jätetään nämä pois lähdeluettelosta
{\sloppy
\printbibliography[heading=subbibliography,notkeyword=this,category=sources,title={Sources}]
}

\section*{Acknowledgements}
I would like to thank my native Moksha consultant, Nadezhda Kabaeva for her help in eliciting the example clauses in Tables 2 and 3 as well as some of the other examples. This article was prepared as a part of the project \textit{The} \textit{Descriptive} \textit{Mordvin} \textit{Grammar,} funded by Kone Foundation, and I would also like to express my gratitude to the foundation.

\section*{ISO codes}
Erzya: myv\\
Moksha: mdf

\section*{Abbreviations}

\begin{minipage}{\textwidth}
\newlength{\colabbrmord}\settowidth{\colabbrmord}{\textsc{cond.conj}}
\newlength{\colglossmord}\settowidth{\colglossmord}{Dependent progressive}
\begin{tabularx}{.5\textwidth}[t]{@{} p{\colabbrmord} p{\colglossmord} }
1&				first person\\
2&				secont person\\
3&				third person\\ 
\textsc{abl}&		ablative\\
\textsc{clt}&		clitic\\
\textsc{cng}&	connegative\\
\textsc{cmpr}&	comparative\\
\textsc{cond}&	conditional\\
\textsc{conj}&conjuctive\\
\textsc{def}&		definite\\
\textsc{des}&	desiderative\\
\textsc{dim}&	diminutive\\
\textsc{elat}&		elative\\
\textsc{ext}&	extension\\
\textsc{freq}&	frequentative\\
\textsc{gen}&	genitive\\
\textsc{ill}&		illative case\\
\textsc{imp}&	imperative mood\\
\end{tabularx}
\settowidth{\colabbrmord}{\textsc{neg.ex}}
\settowidth{\colglossmord}{Non-human/locative pronoun}
\begin{tabularx}{.5\textwidth}[t]{ p{\colabbrmord} p{\colglossmord} @{}}
\textsc{iness}&		inessive case\\
\textsc{inf}&	infinitive\\
\textsc{lat}&		lative\\
\textsc{neg}&	negative\\
\textsc{nmlz}&	nominalizer\\
\textsc{nom}&	nominative\\
\textsc{opt}&	optative\\
\textsc{pl}&		plural\\
\textsc{poss}&	possessive\\
\textsc{prs}&		present\\
\textsc{pst}&	past\\
\textsc{ptcp}&	participle\\
\textsc{Q}&		question marker\\
\textsc{sg}&		singular\\
\textsc{tra}&		translative\\
\textsc{voc}&	vocative\
\end{tabularx}
\end{minipage}

{\sloppy
%\printbiblist[heading=subbibliography,notkeyword=this,title={Abbreviations}]{Sorthands}
\printbibliography[heading=subbibliography,notkeyword=this,notcategory=sources]
}
\end{document}
