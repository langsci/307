\documentclass[output=paper]{langsci/langscibook} 
\author{Antoine Guillaume\affiliation{Laboratoire Dynamique du Langage (CNRS \& Université de Lyon)}}
\title{Negation in Tacana (Amazonian Bolivia): synchronic description and diachronic reconstruction}

\renewcommand{\lsCollectionPaperFooterTitle}{Negation in Tacana (Amazonian Bolivia): synchronic description and diachronic reconstruction}


\abstract{The goal of this paper is to provide, for the first time, a synchronic description and diachronic reconstruction of negation in Tacana, a critically endangered language of the small Takanan family in the Amazonian lowlands of Bolivia and Peru. One significant contribution of the paper is the reconstruction, for a standard negation marker, of an etymology (stand-alone negation word ‘no’) and type of Jespersen Cycle (from the right of the verb to the left of the verb) that are not commonly reported in general studies on negation. The proposed reconstruction also contributes to current studies on the interactions between standard negation and the Negative Existential Cycle (the general theme of the volume) in arguing that the Tacana stand-alone negation word ‘no’ originated in a negative existential predicate. In so doing, the paper adds to the diachronic literature on languages where a negative existential breaks into the verbal domain through a stand-alone negation stage.

\textbf{Keywords:} Takanan languages, negation, Jesperson Cycle, Negative Existential Cycle

}

\shorttitlerunninghead{Negation in Tacana (Amazonian Bolivia)}
\begin{document}
\shorttitlerunninghead{Negation in Tacana (Amazonian Bolivia)}
\maketitle

\section{Introduction}

Tacana %\il{Tacana|(} 
is one of the five extant languages of the small Takanan family from
the Amazonian lowlands of Bolivia and Peru (together with Araona, Cavineña, Ese Ejja, and Reyesano). The language is critically endangered, being
only spoken by a few dozens of elderly people, and basically undescribed,
except in the form of a tagmemic grammar \parencites{Ottaviano1965}{Ottaviano1967}.

This paper is the first study of negation in this language. It is mostly
based on a corpus of firsthand data (texts and elicitations) from the Tumupasa
dialect that I collected during four months of fieldwork conducted on
four field trips between 2009 and 2013. The data are complemented by second
hand materials published by the Summer Institute of Linguistics, consisting
of a number of texts \citep{Ottaviano1980} and sentences that illustrate
dictionary entries \parencites{Ottaviano1989}. Ultimately,
some data from the mid-XIX\textsuperscript{th} century are also presented
in the section on diachrony \parencites{LafoneQuevedo1902}. Note that no
controlled elicitation with native speakers was conducted specifically on
the topic of negation, which means that the study relies exclusively on my
own interpretation of the available corpus.

The goals of this paper are twofold: (1) to provide a detailed
description of a wide range of negation (SN) constructions in Tacana and
(2) to attempt a historical reconstruction of some of the negation
markers and patterns. One significant contribution to the field of
diachronic typologies of negation is the reconstruction for one of the
Standard Negation (hereafter SN) markers of an etymology (stand-alone
negation word `no') and type of Jespersen Cycle (from the right of the verb
to the left of the verb), which are not commonly reported in general
studies on negation.  The proposed reconstruction also contributes to the
current studies on the interactions between SN and the Negative Existential
Cycle (the general theme of the volume) in arguing that the Tacana
stand-alone negation word `no' originated in a negative existential
predicate. In doing so, the paper adds to the diachronic literature on
languages where a negative existential breaks into the verbal domain
through a stand-alone negation stage
\parencites[10, 13--14]{Croft1991}[127ff]{Veselinova2013}[155--156]{Veselinova2016}.

In the literature, the grammaticalization of stand-alone negation words
(also known as ``absolute negators,'' ``pro-sentence words no!,''
``negative replies to polarity questions,'' ``short answers no!,''
``negative interjections,'' etc.) as SN markers through a
reinforcement\slash strengthening process (Jespersen Cycle) is well
documented. The phenomenon is attested,
for instance, in languages such as Brazilian Portuguese, Caribbean Spanish
dialects, northern Italian dialects, Dutch, Africaans, Swedish and
Sino-Russian (see, among others,
\cites{Schwegler1988}[49]{Auwera2009}[127ff]{Veselinova2013}[155--156]{Veselinova2016}).
As illustrated, for instance, in Brazilian Portuguese%\il{Portuguese}
\REF{ex:tacana-portuguese}, a word meaning `no,' originally used outside of a negative
clause (stage 2), ends up replacing the original SN marker inside of the
clause (stage 5).
%
\begin{exe}\ex\label{ex:tacana-portuguese} Brazilian
Portuguese%\il{Portuguese}
 (evolutionary path based on discussion in
\citealt[149--150]{Creissels2006})\\
\begin{tabularx}{.8\textwidth}{l l}
\tablevspace
stage 1 & \textit{\textbf{Não} sei.}    \\
stage 2 & \textit{\textbf{Não} sei,} (\textit{\textbf{\uline{não}}!})  \\
stage 3 & \textit{\textbf{Não} sei \textbf{\uline{não}}.} \\
stage 4 & (\textit{\textbf{Não}}) \textit{{sei} \textbf{\uline{não}}.}   \\
stage 5 & \textit{Sei \textbf{\uline{não}}.}
\end{tabularx}\end{exe}

% \todo{underline is usually not allowed, but should be ok here}

In all these languages, however, the cycle operates from the left of the
verb to the right of the verb, in other words from a preverbal to
postverbal SN marker. In Tacana, as I argue here, the same type of
etymology and grammaticalization pathway holds, but the direction of the
cycle is the opposite. Here, a postverbal SN marker is in the process of
being replaced by a negative stand-alone word in preverbal position. This
results in the phenomenon called a Jespersen Cycle ``in reverse'' by 
\citet{AuweraVossen2016} and \citet{Vossen2016}; according to
these authors, it appears to be common in South American languages.

The paper is organized in two main parts. The first part is descriptive,
with a short introduction on Tacana clause structure and verbal predication
(\sectref{sec:tacana-2}) and a presentation of six negation constructions:
(1) Standard Negation (SN), which applies, by definition, to
% \todo{boldface is not allowed in running text, only for highlighting in examples}
declarative main clauses with a finite verb predicate
\parencite{Miestamo2005}
(\sectref{sec:tacana-3}) but also, in Tacana, to non-verbal clauses
expressing equation, proper inclusion,
attribution relations (\sectref{sec:tacana-4}) and (in rare cases)
existence or location (\sectref{sec:tacana-5}); 
(2) negation of existential\slash loca\-ti\-ve adjective
predicates;
(3) negation of declarative clauses with a non-finite
predicate with two subtypes (\sectref{sec:tacana-6});
(4) non-clausal stand-alone negation
(\sectref{sec:tacana-7}); 
(5) non-clausal constituent negation with two subtypes (\sectref{sec:tacana-7}); and
(6) negation of command (hortative and imperative) clauses with
two subtypes (\sectref{sec:tacana-8}).\footnote{Essentally, two further
types of negation are not discussed in the paper: negation of indefinites
and quantifiers and negation in dependent clauses.} 

A summary table of all the constructions is provided in section
\sectref{sec:tacana-9}. The second part
of the paper is diachronic, engaging in a reconstruction of the
declarative clause negation markers and patterns: SN and negation of
declarative clauses with a non-finite predicate (\sectref{sec:tacana-10}).
A summary and a conclusion are provided in \sectref{sec:tacana-11}.

\section{Basic facts on Tacana clause structure and verbal predication}%
\label{sec:tacana-2}

Main clauses in Tacana consist of an obligatory predicate\footnote{Note that the term ``predicate'' used here does not make
    reference to any participant of the clause.} 
and optional arguments and\slash or obliques
and\slash or adjuncts. When overtly expressed, the arguments, whether NPs
or pronouns, display a (split)\footnote{The case system is conditioned by the type of referent:
    strictly ergative with 1SG\slash 2SG pronouns, optionally ergative
    with 3SG pronouns and NPs, and neutral with all non-singular
    pronouns \parencites{Guillaume2016b}{Guillaume-forthcoming}.} 
ergative case-marking system, as illustrated in \REF{ex:tacana-monkey-sun}, with an ergative
marked A NP and absolutive (unmarked) O NP, and
\REF{ex:tacana-monkey-branch}, with an
absolutive (unmarked) S NP.\footnote{\label{fn:tacana-phonemes}The Tacana consonant phonemes are
\textit{p}, \textit{b}, \textit{t}, \textit{d} [ḍ], \textit{ď}
[d̪͡t̪/\textsuperscript{ə}d̪͡t̪], \textit{dh}
[ð], \textit{ts} [ṭṣ], \textit{ch} [tɕ],
\textit{s} [ṣ], \textit{sh} [ɕ], \textit{k}, \textit{kw}
[kʷ], \textit{r} [ɾ], \textit{m}, \textit{n},
\textit{j} [h], \textit{w} [w/ß̞] and \textit{y} [j]. The vowel
phonemes are \textit{a}, \textit{e,} \textit{i} [i/j] and \textit{u}
[ʊ/w]. Stress falls on the 3\textsuperscript{rd} mora (i.e., vowel
or semi-vowel [j] or [w]) counting from the left). Note that the
illustrative examples include the free translation in local Bolivian
Spanish that was given by the native speakers who helped me transcribe and
translate the texts. The codes that follow the translation lines (e.g.,
so007) correspond to the place of the example in my (Toolbox) database.
Unless explicitely specified, the examples come from texts.}

\begin{exe}\ex\label{ex:tacana-monkey-sun-branch} transitive and intransitive declarative main clauses
    \begin{xlist}
    \ex\label{ex:tacana-monkey-sun}
    \gll Jiawe	=da	\overnote{O}id'eti	\overnote{A}biwa=\textbf{ja}	\overnote{V}y-abu-ta-(a)ni.\\
    now	=\textsc{ptc}	sun	spider\_monkey=\textsc{erg}	\textsc{ipfv}-carry-3A-\textsc{ipfv.sit}\\
    \glt `Ahora lo está cargando el marimono.'\\
    `Now the spider monkey is carrying the sun.’ so007
    \ex\label{ex:tacana-monkey-branch}
    \gll \overnote{S}Biwa	=da	kema	{\ob}tsakwa	echa=su{\cb}	\overnote{V}bade-ti-a.\\
    spider\_monkey	=\textsc{ptc}	\textsc{1sg.dat}	mapajo(tree)	branch=\textsc{loc}	hang-go-\textsc{pst}\\
    \glt ‘Mi marimono se colgó en gajo de mapajo.’ (elicited)\\
    ‘The spider monkey (that I shot) went to hang on the branch of a mapajo (Ceiba pentandra) tree.’
    \end{xlist}\end{exe}

The arguments, especially when expressed by NPs, do not have strict ordering restrictions and can appear in any position in the clause depending on their discourse-pragmatic status. Pronominal arguments, on the other hand, tend to occur either in first position in the clause (when contrastive) or second position (when refering to continuting topics).

Verbal main clauses can be headed by two types of predicates without
semantic differences: a finite verb predicate,
where the verb takes the inflectional morphology, as illustrated in
\REF{ex:tacana-monkey-sun-branch}, and a non-finite verb
predicate, where the verb does not take the inflectional morphology, which
is optionally carried by a generic auxiliary (light verb). Both types are
illustrated in \REF{ex:tacana-spill-the-fishes} with the same verb root
\textit{tutua} `spill,' used in the finite verb construction in the first
clause and in the non-finite verb construction in the second.

\begin{exe}\ex
\label{ex:tacana-spill-the-fishes}
\gll Dapia  =da  etse  beu  se  \textbf{\uline{e}-tutua}.  E-jemi-tsua
=da  etse beu  \textbf{tutua}  =da  etse  \textbf{\uline{y}-a}.\\
    there  =\textsc{ptc}  \textsc{1du}  \textsc{ptc}  fish
    \textsc{fut}-spill  \textsc{fut}-take.out-go.up  =\textsc{ptc}
    1\textsc{du} \textsc{ptc}  spill  =\textsc{ptc}  1\textsc{du}
    \textsc{fut-}do\\ 
\glt `There we are going to spill the fishes (on the ground). We are going
to lift (the fish trap) up and then spill them (on the ground).' em044--045
\end{exe}

There is no difference in (at least propositional) meaning between finite
verb constructions and non-finite verb constructions. The reasons
motivating the use of one construction or the other is not fully understood
yet, although it might have to do with the discourse status of the event or
some specific discourse genres.\footnote{I noticed, for instance, that non-finite verb constructions are often used when an event is repeated, as is the case in two consecutive sentences from the same text in \REF{ex:tacana-going-to-let-go}. I also found that non-finite verb constructions are used more often in informal style and hardly ever in elicitated material.}

Finally, both types of predicates do not have any impact on the way the
arguments are expressed. When negated, however, they require distinct
constructions. The construction used to negate finite verbal main clauses
is also used to negate a number of non-verbal clauses. It will be called
Standard Negation (SN) and discussed first, as it applies to
verbal clauses (\sectref{sec:tacana-3}) and non-verbal clauses (\sectref{sec:tacana-4} and \sectref{sec:tacana-5}). The construction used to negate non-finite verbal main clauses will be discussed in a separate section (\sectref{sec:tacana-6}).

\section{Negation of declarative clauses with a finite verb predicate (Standard Negation)}\label{sec:tacana-3}

In clauses with a finite verb predicate, the lexical verb stem,
with or without derivational morphology, directly and obligatorily bears
the inflectional morphology (basically TAM and 3\textsuperscript{rd} person
indexation), as illustrated in (\ref{ex:tacana-monkey-sun-branch}a,b)
above, in the first clause of \REF{ex:tacana-spill-the-fishes} and
(\ref{ex:tacana-going-to-let-go}a,b), and in \tabref{tab:tacana-finite-verbs} below, which shows the morphological structure of the predicate. In the examples and the Table, the inflectional affixes are in bold and underlined.

\begin{exe}\ex\label{ex:tacana-going-to-let-go}  transitive and intransitive main clauses with a finite verbal predicate
\begin{xlist}
\ex\label{ex:tacana-going-to-spill}
 \gll Dapia  =da  \overnote{A}etse  beu \overnote{O}se \overnote{V}\textbf{\uline{e}}-tutua.\\
there  =\textsc{ptc}  1\textsc{du}  \textsc{ptc}  fish
\textsc{fut}-spill\\
\glt `Ahí ya lo vamos a vaciar los peces.' em044\\
`There we are going spill the fish (on the ground).'
\ex\label{ex:tacana-let-go}
 \gll Beu  =pa \overnote{V}ja-mesia-ti-\textbf{\uline{idha}}
 jida \overnote{S}deja, mawi echa=jenetia.\\
    \textsc{ptc}  =\textsc{rep}
    \textsc{mid}-let\_go\_of-\textsc{mid-rem.pst}  that  man almendrillo
    branch=\textsc{abl}\\
\glt `Ya dice se largó del gajo del alamendrillo.' ch083\\
`He let go of himself from the almendrillo branch.'
\end{xlist}\end{exe}

\begin{table}
\begin{tabularx}{.6\textwidth}{@{\hspace{1ex}} L{.5in} @{} l @{}}
\lsptoprule
\textbf{-3} & \textbf{\uline{TAM}}\\
-2 & valency change\\
-1 & compounded\slash incorporated noun\\\midrule
\multicolumn{2}{|@{\hspace{1ex}} Q @{}|}{%
    %\framebox[.6\textwidth][l]{%
        \makebox[.5in][l]{%
        \textbf{0}}%
        \textbf{verb root}}\\\midrule
+1 & compounded\slash incorporated verb\\
+2 & valency change\\
+3 & ``adverbial-like''\\
\textbf{+4} & \textbf{\uline{3\textsuperscript{rd} person indexation}}\\
\textbf{+5} & \textbf{\uline{temporal distance -\textit{iti}-}}\\
\textbf{+6} & \textbf{\uline{``back'' -\textit{iba}-}}\\
+7 & ``adverbial-like''\\
+8 & \textbf{\uline{TAM}}\\
+9 & reiterative -\textit{yu}\\
\lspbottomrule
\end{tabularx}
\caption{Diagram of finite verb construction}\label{tab:tacana-finite-verbs}
\end{table}

Negation of declarative (and interrogative) main clauses containing a
finite verb construction is realized through a discontinuous embracing
construction involving two particles: the preverbal independent
\textbf{\textit{aimue}} [ajmue] {\textasciitilde} [ajmwe]\footnote{As will be seen later, \textit{aimue} is a contracted variant
    of \textit{aimawe} [ajmaße] which shows up in some examples of the SN
    construction when applied to non-verbal clauses.} 
and the postverbal enclitic
\textbf{=\textit{mawe}} [maße] (with a variant =\textit{mue} [mue]
{\textasciitilde} [mwe]). The construction is illustrated in
\REF{ex:tacana-jungles} with a transitive clause, and in
\REF{ex:tacana-comeback} with an
intransitive clause. 

\begin{exe}\ex
\label{ex:tacana-jungles-comeback}      
\begin{xlist}
\ex\label{ex:tacana-jungles}  
\gll \textbf{Aimue} \overnote{O}ejije=kwana \overnote{A}yama \overnote{V}\textbf{\uline{e}}-shanapa-eni-\textbf{\uline{inia}}=\textbf{mawe}.\\
    \textsc{neg}  jungle=\textsc{pl}  \textsc{1sg.erg}
    \textsc{ipfv}-know-\textsc{well-ipfv.sit.}1/2=\textsc{neg}\\
\glt `Ya no conozco estos montes bien.' ch132\\
`I don't know these jungles well anymore.'
\ex\label{ex:tacana-comeback}
\gll  \textbf{Aimue}  =da \overnote{S}ema \overnote{V}\textbf{\uline{e}}-siapati-yu=\textbf{mue}.\\
    \textsc{neg}  \textsc{=ptc}  \textsc{1sg}
    \textsc{fut}-come\_back-\textsc{iter=neg}\\
\glt `Ya no voy a regresar.' na191\\
`I'm not going to come back again anymore.'
\end{xlist}\end{exe}

The preverbal particle \textit{aimue} is phonologically stressed and
syntactically free, occurring anywhere before the verb. It is often
the first word of the clause, as in (\ref{ex:tacana-jungles-comeback}a,b); see also (\ref{ex:tacana-wind-deer}a,b) further below. But this is not an
absolute requirement, as in (\ref{ex:tacana-damned-inlaws}a,b), where it is
preceded by several clausal constituents. The exact motivations for placing
\textit{aimue} in different positions before the verb remain to be
investigated.

\begin{exe}\ex\label{ex:tacana-damned-inlaws}      
\begin{xlist}
\ex\label{ex:tacana-damned}
\gll  Jade, \overnote{A}{\ob}ye=base=ja{\cb}  =mu  \textbf{aimue},  sai-da
 \overnote{V}\textbf{\uline{y}}-a-\textbf{\uline{ta-ani}}=\textbf{mawe}.\\
    let's\_see  this=\textsc{depr}=\textsc{erg}  \textsc{=contr}
    \textsc{neg}  well-\textsc{asf}
    \textsc{ipfv-}affect-3A-\textsc{ipfv.sit=neg}\\
\glt `A ver, este no lo hace bien.' bu092\\
`Let's see, this damned one doesn't do it well.'
\ex\label{ex:tacana-inlaws}
\gll \overnote{S}{\ob}Piada deja{\cb} =pa,
{\ob}mesa,  d'aki=neje{\cb},  \textbf{aimue} sai-da
\overnote{V}jadusuti-\textbf{\uline{ina}}=\textbf{mawe}.\\
one  man  =\textsc{rep}  \textsc{3sg.gen}
    brother\_in\_law=\textsc{assc}  \textsc{neg}  well-\textsc{asf}
    get\_along-\textsc{hab.pst=neg}\\
\glt `Un hombre, dice, que con su cuñado no se llevaba bien.' ch003\\
`There were a man and his brother-in-law who did not get along well.'
\end{xlist}\end{exe}

The postverbal particle =\textit{mawe}, by contrast, is a phonologically
unstressed enclitic with a rigid position. It can only attach to the verb.
If the verb of the negated clause is followed by one or more clausal
constituents, =\textit{mawe} necessarily remains on the verb, as in
(\ref{ex:tacana-wind-deer}a,b).

\begin{exe}\ex
\label{ex:tacana-wind-deer}      
\begin{xlist}
\ex\label{ex:tacana-wind}
\gll \textbf{Aimue} \overnote{V}\textbf{\uline{e}}-juseute-\textbf{\uline{ta}=}\textbf{mawe}, \overnote{A}beni=ja.\\
    \textsc{neg}  \textsc{fut}-fell-3A=\textsc{neg}  wind=\textsc{erg}\\
\glt `No los va tumbar el viento.' bu072\\
`The wind will not fell (the trees).'
\ex\label{ex:tacana-deer}          
\gll  Bute-ke!  \textbf{Aimue}  =da \overnote{V}\textbf{\uline{e}}-kwinana\textbf{=mawe}, \overnote{S}dukei.\\
    go\_down-\textsc{imp}  \textsc{neg}  \textsc{=ptc}
    \textsc{fut}-go\_out=\textsc{neg}  deer\\
\glt `Bájate! No va salir el venado.' du018\\
`Go down! The deer will not go out.'
\end{xlist}\end{exe}

In my corpus, very few examples of negation of interrogative clauses can be
found. The ones that are available, such as \REF{ex:tacana-caiman}, suggest
nevertheless that they are negated by means of the same pattern as in declarative clauses.

\begin{exe}\ex
\label{ex:tacana-caiman}  
\gll Jukwajasu  =da  \textbf{aimue}  dasu
\textbf{\uline{e-}}nubi\textbf{\uline{-ani}}=\textbf{mawe} {\ob}yawi
tipa=su{\cb}?\\
 why  =\textsc{ptc}  \textsc{neg}  then
 \textsc{ipfv}-enter-\textsc{ipfv.sit}=\textsc{neg}  water
 bottom=\textsc{loc}\\
\glt `¿Porque pues no entra debajo de agua?'\\
`Why does it (the caiman) not go underneath the water?' bo080
\end{exe}

In the available corpus, \textit{aimue} is never omitted. As for
=\textit{mawe,} I found a couple of examples where it is left out, as in
\REF{ex:tacana-caiman-second} and \REF{ex:tacana-arrive}, which suggests
that =\textit{mawe} might not be obligatory. It is reminded that no
controlled elicitation with native speakers was conducted on negation
constructions.

\begin{exe}\ex\label{ex:tacana-caiman-second}
\gll {\ob}Ena  dume=su{\cb}  \textbf{aimue}
\textbf{\uline{e}}-nubi-ti-\textbf{\uline{ani}}.\\
    stream  inside=\textsc{loc}  \textsc{neg}
    \textsc{ipfv}-enter-\textsc{go-ipfv.sit}\\
\glt `No entra dentro del agua.' bo081\\
`(The caiman) does not enter into the water.'
\ex\label{ex:tacana-arrive} 
\gll \textbf{Aimue}  \textbf{\uline{e}}-kwina-yu.\\
 \textsc{neg}  \textsc{pst}-arrive-\textsc{iter}\\
\glt `No llegó.' ch037\\ 
`He didn't arrive.'
\end{exe}

From the perspective of Miestamo's \parencites*{Miestamo2005}{Miestamo2007} typology of negative constructions, the Tacana negative construction under discussion is symmetric. Apart from the addition of the negative markers, there do not appear to be any obvious morphosyntactic differences, with the same argument-coding system (split ergative case-marking and constituent order flexibility) and same morphological possibilities on the verbal predicate (derivational and inflectional). 

\section{Negation of non-verbal clauses (1): equation, proper inclusion, attribution}
\label{sec:tacana-4}

The SN construction is also used for negating non-verbal clauses. The
negation of equation, proper inclusion and attribution clauses is discussed
in this section. The negation of existential and locative predication is discussed in the next.

In affirmative equation, proper inclusion and attribution clauses, the
predicate consists of an NP or an adjective optionally followed by
the inflected copula verb \textit{pu} `be,' as illustrated in
(\ref{ex:tacana-bearman-mother}a-d). The S NP is expressed like the S NP of any other intransitive verbal clauses, being optional and, when expressed, not subject to any ordering restrictions.

\begin{exe}\ex\label{ex:tacana-bearman-mother}
\begin{xlist}
\ex\label{ex:tacana-bearman} 
proper inclusion (with copula)\\
\gll \overnote{S}{\ob}Tueda edeje{\cb} \overnote{NP}{\ob}a'una deja{\cb} \overnote{COP}pu-ina.\\
    that  youngster  bear  man  be-\textsc{hab.pst}\\
\glt`Ese joven era joven (lit. hombre) oso.' au004a\\
`That youngster was a bear-man.'
\ex\label{ex:tacana-mother} 
 equation (without copula)\\
\gll \overnote{S}{\ob}Mike ebakepuna{\cb} \overnote{NP}{\ob}kema kwara{\cb}.\\
 \textsc{2sg.gen}  daughter  \textsc{1sg.gen}  mother\\
\glt `Tu hija es mi madre.' au155\\
`Your daughter is my mother.'
\ex\label{ex:tacana-veryhot}
 attribution (with copula)\\
\gll \overnote{S}Id'eti  =di \overnote{ADJ}tuche-da \overnote{COP}e-pu-eti.\\
    sun  =\textsc{ptc}  strong-\textsc{asf}
    \textsc{ipfv}-be-\textsc{ipfv.stand}\\
\glt `El sol también estaba fuerte.' lp078\\
`The sun was very hot (lit. strong).' 
\ex\label{ex:tacana-garden}
 attribution (without copula)\\
\gll \overnote{S}Te  =mu \overnote{ADJ}ai-da  beju…\\
    garden  =\textsc{contr}  grande-\textsc{asf}  \textsc{ptc}\\
\glt `Ahora el chaco es grande...' gu054\\
`The garden is big…' 
\end{xlist}\end{exe}

When negated, these non-verbal clauses require the SN construction through
the discontinuous embracing construction with the predicate-preposed
independent \textbf{\textit{aimue}} and the predicate-postposed enclitic
\textbf{=\textit{mawe}}, as illustrated in \REF{ex:tacana-inhuman-husband}.
Both have the same properties as when applied to clauses with a finite
verb: ordering flexibility for \textit{aimue}, as long as it appears before
the predicated NP or predicative adjective; strict position for
=\textit{mawe}, directly attached to the copula verb (if present) or to the
predicated NP or predicative adjective (if the copula is absent). As can be
seen in \REF{ex:tacana-inhuman}, the predicate-preposed negation marker can show up in a longer (more conservative) form \textbf{\textit{aimawe}} [ajmaße]. As for =\textit{mawe}, I have too few examples of SN applied to non-verbal clauses to know if it displays the shorter variant =\textit{mue} found in SN applied to verbal clauses.

\begin{exe}\ex
\label{ex:tacana-inhuman-husband} \begin{xlist}
\ex\label{ex:tacana-inhuman}
proper inclusion (with copula)\\
\gll \overnote{S}{\ob}Tueda  edeje{\cb} \textbf{aimawe}  \overnote{NP}{\ob}kristianu eni{\cb} \overnote{COP}pu-ina=\textbf{mawe}.\\
 that  child  \textsc{neg}  person  real  be-\textsc{hab.pst=neg}\\
\glt `Ese joven no era humano.' au003\\
`That youngster was not a real person.'
\ex\label{ex:tacana-husband}
equation (without copula)\\
\gll  \textbf{Aimue}  =jia  maida \overnote{S}{\ob}ye  deja{\cb} \overnote{NP}{\ob}kema y-awe{\cb}=\textbf{mawe}?\\
 \textsc{neg}  =\textsc{dubit}  \textsc{ptc}  this  man  \textsc{1sg.gen}
 \textsc{npf}-husband=\textsc{neg}\\
\glt `¿No es mi marido este hombre?'
\parencite[4]{Ottaviano1989}\\
`Is this man not my husband?' 
\ex\label{ex:tacana-nice}  attribution (with copula)\\
\gll \overnote{S}{\ob}Kea tata{\cb} =mu \textbf{aimue} \overnote{ADJ}sai-da \overnote{COP}pu-ina=\textbf{mawe}.\\
    \textsc{1sg.gen}  father  =\textsc{contr}  \textsc{neg}
    well\textsc{-asf}  be-\textsc{hab.pst=neg}\\
\glt `Mi papá no era tan bueno (renegaba \slash{}  pegaba).' ps058\\
`My father wasn't nice.' 
 \ex\label{ex:tacana-child}  attribution (without copula)\\
\gll 
    \textbf{Aimue} \overnote{ADJ}sai-da=\textbf{mawe} \overnote{S}{\ob}mike e-bakwa{\cb}.\\
    \textsc{neg}  well-\textsc{asf=neg}  \textsc{2sg.gen}
    \textsc{npf}-child\\
\glt `No es bueno tu hijo.' au222\\
`Your child is not nice.' 
\end{xlist}\end{exe}

Note finally that I do not have examples in the whole corpus where
=\textbf{\textit{mawe}} (or \textit{aimawe} {\textasciitilde}
\textit{aimue} for that matter) are omitted. More investigations are
however needed to confirm whether this is also a possibility, as in clauses
with finite verb, especially since I have very few examples of negation of
non-verbal clauses in the current dataset.

\section{Negation of non-verbal clauses (2): existential\slash locative predication}\label{sec:tacana-5}

There is no clear formal distinction between existential and locative
clauses, whether affirmative or negative. In affirmative existential and
locative clauses, the predicate is normally one of four posture verbs,
either \textit{ani} `sit' (\ref{ex:tacana-posture-sit}a-e), \textit{neti}
`stand' \REF{ex:tacana-posture-stand}, \textit{sa} `lie'
\REF{ex:tacana-posture-lie} or \textit{bade} `hang'
\REF{ex:tacana-posture-hang}; less commonly, the predicate can also be the
(copula) verb \textit{pu} `be' (see below). In this function, the posture
verbs are inflected but with severe restrictions, with only a few possible
TAM affixes: a prefix \textit{e-/y-} `\textsc{exist\slash loc},' the habitual
past \textit{-ina} `\textsc{hab.pst}' and the complex past tense
-\textit{iti-a} `\textsc{tdm-pst'}. A locative oblique (or, for that
matter, another type of oblique) can be present with no rigid position in
the clause. With a dative oblique, the same construction yields the
semantics of possessive predication, as seen in
\REF{ex:tacana-posture-sit-child}, with the possessor encoded by the 3SG dative pronoun \textit{mesa}.

\begin{exe}\ex\label{ex:tacana-posture-sit} positive: posture verb `sit' 
\begin{xlist}
\ex\label{ex:tacana-posture-sit-capybaras}
\gll \ldots{} \overnote{LOC}dapia \overnote{V}\textbf{y-ani} \overnote{S}dhududu.\\
    {}  there  \textsc{exist/loc}-sit  capybara\\
\glt `(En los lagos grandes,) ahí hay capihuara.' mc004\\
`There (in the big lakes,) there are capybaras (lit. sitting).'
\ex\label{ex:tacana-posture-sit-mother}
\gll \overnote{LOC}Chue \overnote{LOC}{\ob}enabaki maje=su{\cb} \overnote{V}\textbf{y-ani}.\\
    there  stream  border=\textsc{loc}  \textsc{exist/loc}-sit\\
\glt `Allá está (mi mamá) en la banda del arroyo.'\\
`(My mother) is there (lit. sitting) on the other side of the river.' au164
\ex\label{ex:tacana-posture-sit-families}
\gll \overnote{S}{\ob}Beinte familia{\cb} \overnote{V}\textbf{ani-ina} \overnote{LOC}Napashi=su {\ob}da
mara=su{\cb}.\\
    twenty  family  sit-\textsc{hab.pst}
    Napashi=\textsc{loc}  that  time=\textsc{loc}\\
\glt `Veinte familia había en Napashi en ese año.' na003\\
`There were twenty families (lit. sitting) in Napashi at that time.'
\ex\label{ex:tacana-posture-sit-anoppi}
 \gll \overnote{S}{\ob}Piada deja{\cb} \overnote{V}\textbf{ani-ina}
    \overnote{COMIT}{\ob}mesa ewane=sa kwara=neje{\cb}.\\
    one  man  sit-\textsc{hab.pst}  \textsc{3sg.gen}
    wife=\textsc{gen}  mother=\textsc{assc}\\
\glt `Había un hombre que vivía junto a su suegra.' gu003\\
`There was a man who was living with his mother-in-law.'
\ex\label{ex:tacana-posture-sit-child}
 \gll \overnote{S}Ebakwa=chidi \overnote{DAT}mesa
    \overnote{V}\textbf{y-ani}.\\
 child=\textsc{dim}  \textsc{3sg.dat}  \textsc{exist/loc}-sit\\
\glt `Tenía dice su hijito.' ye020\\
`He had a small child.' [lit. a small child was sitting to him]
\end{xlist}

\ex\label{ex:tacana-posture-stand}  positive: posture verb `stand'\\
\gll \overnote{S}Dukei=base \overnote{V}\textbf{e-neti}
\overnote{LOC}\textbf{ena=su} e-id'i-ti-neti.\\
    deer=\textsc{depr}  \textsc{exist/loc}-stand  stream=\textsc{loc}
    \textsc{ipfv}-drink-\textsc{mid-ipfv.stand}\\
\glt `Ahí está el venado dentro del agua está tomando.' hv027\\
`There is a deer \slash  the dear is standing in the water and driking (standing).'
\ex\label{ex:tacana-posture-lie}  positive: posture verb `lie'\\
\gll
    \overnote{LOC}Ue \overnote{S}dukei \overnote{V}\textbf{e-sa}.\\
    here  deer  \textsc{exist/loc}-lie\\
\glt `Aquí hay un venado echado.' du051\\
`Here, there is a deer (lying).' (lit. here a deer lies)
\ex\label{ex:tacana-posture-hang}  positive: posture verb `hang'\\
\gll
 {\ob}Piada semana{\cb} =pa beu \overnote{V}\textbf{e-bade}
 \overnote{LOC}{\ob}rara dume=su{\cb}.\\
 one  week  =\textsc{rep}  \textsc{ptc}  \textsc{exist/loc}-hang  hole
 inside\textsc{=loc}\\
\glt `Así dice (el tigre) estuvo una semana dentro de la cueva.' bu056\\
`(The jaguar) was (hanging) inside of the hole during a whole week.'
\end{exe}

Instead of a posture verb, the predicate of an existential and locative
clause can apparently also be the verb \textit{pu} `be,' discussed in its
copula use in the preceding section. This is suggested by a couple of
examples found in the corpus, such as those in
\REF{ex:tacana-be-judgement} (existential) and \REF{ex:tacana-be-school} (locative). 

\begin{exe}\ex
\label{ex:tacana-be-judgement-school}  positive: verb `be' 
\begin{xlist}
\ex\label{ex:tacana-be-judgement}
\gll \overnote{S}Juishu  beju \overnote{V}\textbf{pu-iti-a}.\\
    judgment  \textsc{ptc}  be-\textsc{tdm-pst}\\
\glt `Había juicio.' (in064\_ott; \citealt[62]{Ottaviano1980})\\
`There was a judgment.'
\ex\label{ex:tacana-be-school}
\gll  \overnote{S}Tueda \overnote{LOC}escuela=su \overnote{V}\textbf{pu-ina}.\\
    \textsc{3sg}  school=\textsc{loc}  be-\textsc{hab.pst}\\
\glt `El estaba en la escuela.' na205\\
`He was in the school.'
\end{xlist}\end{exe}

When negated, existential and locative clauses display two possible
patterns. The first, illustrated in \REF{ex:tacana-oldtimes} and very
scarcely attested in the corpus, is through the same discontinuous
embracing SN construction with the preverbal independent
\textbf{\textit{aimue}} and the postverbal enclitic
\textbf{\textit{=mawe}}. This first pattern is illustrated with the posture
verb \textit{ani} `sit' in \REF{ex:tacana-oldtimes-noshoes} and
\REF{ex:tacana-oldtimes-Rurrenabaque} and with the copula verb \textit{pu}
`be' in \REF{ex:tacana-oldtimes-village}. Note that due to a
lack of sufficient examples, I am unable to say if \textit{aimue} and\slash
or \textit{=mawe} display the variants they have in SN when applied to
other clause types (\textit{aimawe} [ajmaße], =\textit{mue} [mue] 
{\textasciitilde} [mwe]).

\begin{exe}\ex\label{ex:tacana-oldtimes}\begin{xlist}
\ex\label{ex:tacana-oldtimes-noshoes}
\gll {\ob}Da tiempo{\cb} \textbf{aimue} \overnote{S}sapato
\overnote{V}ani-ina\textbf{=mawe}.\\
    that  time  \textsc{neg}  shoe  sit-\textsc{hab.pst=neg}\\
\glt `En ese tiempo no había zapato.' ci024\\
`At that time, there were no shoes.'
\ex\label{ex:tacana-oldtimes-Rurrenabaque}
\gll \textbf{Aimue} \overnote{V}ani-iti-a\textbf{=mawe} \overnote{S}Rurrenabaque.\\
    \textsc{neg}  sit-\textsc{tdm-pst=neg}  Rurrenabaque\\
\glt `(En los tiempos antiguos), no había Rurrenabaque.' tu002\_ott\\
`(In the old days) Rurrenabaque did not exist.'
\ex\label{ex:tacana-oldtimes-village}
\gll \overnote{LOC}Upia  =mu  =da \textbf{aimue} \overnote{S}ejude ekene
\overnote{V}pu-iti-a=\textbf{mawe}.\\
    here  =\textsc{contr}  \textsc{=ptc}  \textsc{neg}  village  first
    be-\textsc{tdm}-\textsc{pst=neg}\\
\glt `Aquí no había nada/pueblo \slash{}  no era pueblo todavía.' hi020\\
`Here at first there was no village.'
\end{xlist}\end{exe}

The second pattern, illustrated in (\ref{ex:tacana-villages-firewood}a-e)
and found in many examples in the corpus, consists of
\textbf{\textit{aimawe}} [ajmaße] or variant \textbf{\textit{aimue}}
[ajmue] {\textasciitilde} [ajmwe] used alone with a predicative function,
with or without the inflected copula verb \textit{pu} `be.' In this use, I
analyze \textit{aimawe\slash aimue} as a lexical negative existential\slash
locative adjective in an attributive clause construction, as described in
\sectref{sec:tacana-4}. In the examples provided below, one can see
negation of existence in \REF{ex:tacana-villages}, \REF{ex:tacana-firewood} and \REF{ex:tacana-nofish}, negation of location in
\REF{ex:tacana-mybed} and negation of possession in \REF{ex:tacana-no-meat} (possessor encoded by the 3SG dative pronoun \textit{mesa}).

\begin{exe}\ex\label{ex:tacana-villages-firewood}\begin{xlist}
\ex\label{ex:tacana-villages}
\gll {\ob}Biawa tiempo{\cb} =mu =da \overnote{ADJ}\textbf{aimue}
\overnote{COP}pu-iti-a \overnote{S}ejude=kwana.\\
    old  time  =\textsc{contr}  \textsc{=ptc}  nonexistent
    be-\textsc{tdm-pst}  village=\textsc{pl}\\
\glt `En tiempos antiguos no habían pueblos.' tu001\_ott\\
`In the old days, there were no villages.' [lit. villages were nonexistent]
\ex\label{ex:tacana-firewood}
\gll \overnote{S}Kwati  =mu \overnote{ADJ}\textbf{aimue}  =tsu'u.\\
    firewood  =\textsc{contr}  nonexistent  =\textsc{still}\\
\glt `La leña todavía no hay.' ci104\\
`There is no firewood yet.' [lit. firewood was nonexistent]
\ex\label{ex:tacana-nofish}
\gll \overnote{ADJ}\textbf{Aimue}  beu \overnote{S}se. \overnote{ADJ}\textbf{Aimue}  beu.\\
    nonexistent  \textsc{ptc}  fish  nonexistent  \textsc{ptc}\\
\glt `Ya no hay pescado, ya no hay.' em075\\
`There no fish. There were no (fish).' [lit. fish were nonexistent]
\ex\label{ex:tacana-mybed}
\gll \overnote{S}Ema  =mu \overnote{ADJ}\textbf{aimawe}  beu
 \overnote{LOC}{\ob}kema tawi-jude=su{\cb}.\\
    \textsc{1sg}  \textsc{=contr}  nonexistent  \textsc{ptc}
    \textsc{1sg.gen}  sleep-\textsc{place=loc}\\
\glt `(Me buscaron ya) yo no estaba en mi cama.' du101\\
`(They searched for me but) I wasn't in my bed.' [lit. I was nonexistent in my bed]
\ex\label{ex:tacana-no-meat}
\gll \overnote{ADJ}\textbf{Aimue} \overnote{DAT}mesa \overnote{S}aicha…\\
    nonexistent  \textsc{3sg.dat}  meat\\
\glt `No tenía carne…' ye006\\
`He didn't have meat.' [lit. the meat was nonexistent to him]
\end{xlist}\end{exe}

\section{Negation of declarative non-finite verbal main clauses}\label{sec:tacana-6}

We now turn to the negation of main clauses with a non-finite verbal
predicate. As noted in \sectref{sec:tacana-3}, main clauses with a
non-finite verbal predicate express the same propositional content as those
with a finite verbal predicate, but here the predicate has a different
structure. The lexical verb stem, with or without derivational morphology,
does not directly bear the inflectional morphology. The inflectional
affixes (the same ones used in finite verb constructions and listed in
\tabref{tab:tacana-finite-verbs} (i.e., TAM and 3\textsuperscript{rd}
person indexation)) are
either carried by a generic auxiliary (light verb), which in this
construction is specifically used for this (inflection-carrying) purpose
or, more commonly, altogether absent.

The examples in (\ref{ex:tacana-nonfinite-trap-letgo}a,b), based on the
same transitive and intransitive verb stems \textit{tutua} `spill' and
\textit{ja-mesia-ti} `let go of oneself' used in finite verb constructions
in (\ref{ex:tacana-going-to-let-go}a,b), illustrate non-finite verb
constructions with inflections carried by a generic auxiliary.

\begin{exe}\ex \label{ex:tacana-nonfinite-trap-letgo}  transitive and
intransitive declarative main clauses with a non-finite verb and an overt
auxiliary 
\begin{xlist}
\ex\label{ex:tacana-nonfinite-trap}
\gll   E-jemi-tsua   =da  etse,  beu
\moreovernote{V}\overnote{{\ob}tu\textbf{ʔ}twa{\cb}}\textbf{tutua} =da
\moreovernote{A}etse \moreovernote{AUX}\textbf{\uline{y}-a.}\\
    \textsc{fut}-remove-go\_up  \textsc{=ptc}  \textsc{1du} 
    \textsc{ptc}  spill  =\textsc{ptc}  \textsc{1du}  \textsc{fut-}do\\
\glt `Lo vamos a alzar (la trampa para peces) entre los dos y
lo vamos a vaciar.' em045\\
`We are going to lift (the fish trap) up and spill them (the fishes)
(on the ground).'
\ex\label{ex:tacana-nonfinite-letgo}
\gll  Beu,
\moreovernote{V}\overnote{{\ob}ha\textbf{ʔ}mesjati{\cb}}\textbf{ja-mesia-ti}
\moreovernote{S}ema \moreovernote{AUX}\textbf{pu-\uline{ana}}.\\
    \textsc{ptc}  \textsc{mid-}let\_go\_of-\textsc{mid}  \textsc{1sg}
    be\textsc{-rec.pst}\\
\glt `Ya me largué.' lp033\\
`Then I let go of myself.'
\end{xlist}\end{exe}

As one can see, there are two auxiliaries, the use of which depends on the
transitivity of the predicate: \textit{a} `do' when the predicate is
transitive \REF{ex:tacana-nonfinite-trap} and \textit{pu} `be' when the
predicate is intransitive \REF{ex:tacana-nonfinite-letgo}. The auxiliaries
are etymologically related, respectively, to the independent transitive
lexical verbs \textit{a} `affect, make, do (tr.), say (tr.),' illustrated
in \REF{ex:tacana-auxetymologies-build}, and the intransitive verb
\textit{pu} `be\slash exist, be located, do (itr.), say (itr.),' which can
serve, among other things, as the copula predicate in equation, proper
inclusion and attribution clauses (\sectref{sec:tacana-4}) and (less
commonly) the predicate of existential\slash locative clauses
(\sectref{sec:tacana-5}), as illustrated in
\REF{ex:tacana-auxetymologies-school} (repeated).

\begin{exe}\ex
\label{ex:tacana-auxetymologies-build-school}  
\begin{xlist}
\ex\label{ex:tacana-auxetymologies-build}
\gll Upia  mike  ete,  ekwanaju  y-\textbf{a}.\\
    here  \textsc{2sg.gen}  house  \textsc{1pl.excl}  \textsc{fut}-do\\
\glt `Aquí te lo vamos a hacer tu casa.' au313\\
`Here we are going to build your house.'
\ex\label{ex:tacana-auxetymologies-school}
 \gll   Tueda  escuela=su  \textbf{pu}{-ina}.\\
    3\textsc{sg}  school=\textsc{loc}  be-\textsc{hab.pst}\\
\glt `Él estaba en la escuela.' na205\\
`He was in the school.'
\end{xlist}\end{exe}


In the non-finite verb constructions, the auxiliary must follow the lexical
verb, whether contiguously or not; in
(\ref{ex:tacana-nonfinite-trap-letgo}a,b), for instance, the auxiliary is
not contiguous with the lexical verb, being separated from it by a pronoun
in both examples, and also by a particle in \REF{ex:tacana-nonfinite-trap}.
As for the lexical verb in this construction, whether the auxiliary is
present or not, it receives a specific intonation contour, with a
non-phonological prosodic glottal stop [ʔ] in 1\textsuperscript{st}
syllable coda position (see phonological inventory in footnote
\ref{fn:tacana-phonemes}) and apparently a different stress
pattern.\footnote{See footnote \ref{fn:tacana-phonemes} for on the stress system in Tacana.}

As stated above, the inflection-carrying auxiliary is not compulsory, and
in fact it is left out most of the time; in this situation, the TAM and
identity of 3\textsuperscript{rd} person subjects have to be recovered from
the context. This is illustrated with the transitive and intransitive verb
stems \textit{nubi-ame} `make enter' in \REF{ex:tacana-infinite-gringo} and
\textit{pue-yu} `come again\slash back' in
\REF{ex:tacana-infinite-owner}.

\begin{exe}\ex \label{ex:tacana-infinite-gringo-owner}  transitive and intransitive declarative main clauses with a non-finite verb and no overt auxiliary
\begin{xlist}
\ex\label{ex:tacana-infinite-gringo}
\gll  Beu  \moreovernote{O}etseju
\moreovernote{A}gringo=ja  beu,  cuarto=su
\moreovernote{V}\overnote{{\ob}nu\textbf{ʔ}bjame{\cb}}\textbf{nubi-ame}.
\\
    \textsc{ptc} \textsc{1du.excl} gringo=\textsc{erg} \textsc{ptc}
    room=\textsc{loc}  enter-\textsc{caus}\\
\glt `Ya a nosotros el gringo al cuarto nos metió e hizo entrar.' tm057\\
`The gringo made us enter into the room.'
\ex\label{ex:tacana-infinite-owner}
\gll Pero, \moreovernote{S}{\ob}mesa
emetse{\cb} =mu, ekene
\moreovernote{V}\overnote{{\ob}pwe\textbf{ʔ}ju{\cb}}\textbf{pue-yu}.\\
    but  \textsc{3sg.gen}  owner  =\textsc{contr}  first
    come-\textsc{iter}\\
\glt `Pero su dueño primero se vino.' ha017\\
`But his owner came first.'
\end{xlist}\end{exe}

At the level of the clause, predicates with a non-finite verb and predicates with a finite verb do not require different clausal constructions: the argument-coding system remains the same, with an identical split ergative case-marking system and the same constituent order flexibility.

However, when it comes to negation, clauses with a non-finite verbal
predicate are negated differently from clauses with a finite verbal
predicate; the SN construction is not used for their negation. Here, two
additional negation constructions are available, both of which only involve
a single negation marker, which occurs before the verb. The form of the
marker is what distinguishes the constructions, with all other properties
being identical. In the first construction, the negation marker is the
independent morpheme \textbf{\textit{aimue}} (as in SN) while in the
second it is the proclitic \textbf{\textit{mué=}} (segmentally identical
to one of the variants of \textit{=mawe} in SN). Unlike in SN, here the
lexical verb (or the inflection-carrying auxiliary, if expressed) is never
followed or accompanied by a second negation marker.

The negative construction with \textbf{\textit{aimue}} is illustrated in
(\ref{ex:tacana-aimue-jaguar-leave}a,b) with an overt auxiliary and
(\ref{ex:tacana-aimue-brother-fast}a--c) with no overt auxiliary. In both cases, examples of both transitive and intransitive clauses are provided.

\begin{exe}\ex
\label{ex:tacana-aimue-jaguar-leave}  negative clauses with \textbf{\textit{aimue}} and auxiliary
\begin{xlist}
\ex\label{ex:tacana-aimue-jaguar}
\gll  Biame  \textbf{aimue}  =da \overnote{V}\textbf{dia}
\overnote{AUX}\textbf{a-ta-ina}.\\
    on\_the\_contrary  \textsc{neg}  \textsc{=ptc}  eat
    do\textsc{-3A-hab.past}\\
\glt `Pero no lo comió.' qu004\\
`But (the jaguar) would not eat it.'
\ex\label{ex:tacana-aimue-leave}
\gll  \textbf{Aimue}  beu, \overnote{V}\textbf{kwinana-yu} \overnote{AUX}\textbf{pu-idha.}\\
    \textsc{neg}  \textsc{ptc}  emerge-\textsc{iter}  be\textsc{-rem.pst}\\
\glt `Ya no salió más.' qu033\\
`He didn't leave again.'
\end{xlist}
\ex
\label{ex:tacana-aimue-brother-fast}  Negative clauses with \textbf{\textit{aimue}} without auxiliary
\begin{xlist}
\ex\label{ex:tacana-aimue-brother}
\gll  \textbf{Aimue} \overnote{A}yama \overnote{O}d'aki \overnote{V}\textbf{ba}.\\
    \textsc{neg}  \textsc{1sg.erg}  brother\_in\_law  see\\
\glt `No le he visto al cuñado.' ch028\\
`I didn't see my brother-in-law.'
\ex\label{ex:tacana-aimue-fast}
\gll  \textbf{Aimue}, \overnote{S}ema \overnote{V}\textbf{kwinana-puda}.\\
    \textsc{neg}  \textsc{1sg}  go\_out-\textsc{fast}\\
\glt `No he salido rápido.' ch152\\
`I didn't go out fast.'
\ex\label{ex:tacana-aimue-liana}
\gll  \textit{Enekita}  \textit{=pa}  \textbf{\textit{aimue}}
\textit{beu} \overnote{V}\textbf{\textit{ja-tibi-ti}}.\\
really  =\textsc{rep}  \textsc{neg}  \textsc{ptc}
\textsc{mid-}detach-\textsc{mid}\\
\glt `En verdad dice que no se soltó.' ch082\\
`Really, (the vine) didn't break.'
\end{xlist}\end{exe}

The behavior of \textit{aimue} here is the same as that of \textit{aimue} in SN, being phonologically stressed and free to occur in any position as long as it is before the predicate.

The negative construction with \textbf{\textit{mué=}} is illustrated in
\REF{ex:tacana-mueaux} with an overt auxiliary and
(\ref{ex:tacana-mue-realize-answer}a--c), with no overt auxiliary.

\begin{exe}\ex \label{ex:tacana-mueaux}  negative clause with
\textbf{\textit{mué=}} and auxiliary\\
\gll  \textbf{Mué=}pa \overnote{V}\textbf{teje-ti-yu} \overnote{AUX}\textbf{a-ta-idha}
\makebox[0pt][l]{\raisebox{\baselineskip}{O}}{\ob}jida mesa ewane{\cb} beu.\\
 \textsc{neg=rep}  find-\textsc{go-iter}  do\textsc{-3A-rem.pst}  that
 \textsc{3sg.gen}  wife  \textsc{ptc}\\
\glt `Dice que no lo ha ido hallar ese su mujer.' os043\\
`He didn't find his wife.'
\ex \label{ex:tacana-mue-realize-answer} negative clauses with
\textbf{\textit{mué=}} without auxiliary
\begin{xlist}
\ex\label{}
\gll  \moreovernote{A}Yama  =mu
\moreovernote{V}\overnote{{\ob}mwéshánapa{\cb}}\textbf{mué=shanapa}
\ob\ldots{\cb} {\ob}mesa ebakwa{\cb} manu-iti-a.\\
\textsc{1sg.erg}  \textsc{=contr}  \textsc{neg}=know {} 3\textsc{sg.gen}
child die-\textsc{tdm-pst}\\
\glt `Yo no me he dado cuenta que ha muerto su hijo.' su130\\
`I didn't realize that his child had died.'
\ex\label{ex:tacana-mue-answer}
\gll \overnote{{\ob}mwéemahéutsu{\cb}}\textbf{Mué}=\moreovernote{O}ema
\moreovernote{V}\textbf{jeutsu}.\\
    \textsc{neg=1sg} respond\\
\glt `No me contestó.' ch033\\
`He did not answer me.'
\ex\label{ex:tacana-mue-doctor}
\gll  \ldots{} rusu-ta-idha doctor=ja.
\moreovernote{V}\overnote{{\ob}mwé:manu{\cb}}\textbf{Mué=manu}.\\
{}    sew-3A\textsc{-rem.pst}  doctor=\textsc{erg}  \textsc{neg}=die\\
\glt `Se lo costuró el doctor (el cuero de su cabeza). No ha muerto.'
ti041--42\\
`The doctor stitched (the scalp of his head). He hasn't died.'
\end{xlist}\end{exe}

As one can see from these examples, \textit{mué=}, like \textit{aimue},
does not have a specific position as long as it occurs before the
predicate. It can attach to the predicate, as in
(\ref{ex:tacana-mue-realize-answer}a,c) or to any preverbal host, as in
\REF{ex:tacana-mueaux} and \REF{ex:tacana-mue-answer}; not that in
\REF{ex:tacana-mueaux}, \textit{mué}= attaches to a second position clitic,
the reportative \textit{=pa}. The two markers \textit{mué=} and
\textit{aimue} only differ in their prosodic status, \textit{mué=} being
prosodically dependent while \textit{aimue} has prosodic independence. Note
that phonological words formed by \textit{mué}= and its following host have
a peculiar stress pattern where stress falls on \textit{mué=}, as can be
seen in the phonetic transcriptions in
(\ref{ex:tacana-mue-realize-answer}b--c).\footnote{As already mentioned in footnote \ref{fn:tacana-phonemes}, the rule in Tacanan is that
stress falls on the 3\textsuperscript{rd} mora (i.e., vowel or semi-vowel
[j] or [w]) counting from the left. Evidence that the normal rule does not
apply here can be seen in \REF{ex:tacana-mue-doctor}, where one should obtain [mwem\textbf{á}nu] and not [mw\textbf{é:}manu].} 

From a functional perspective, it is not fully clear what motivates the use
of \textit{aimue} versus \textit{mué=}, although it is likely that they
differ in encoding different degrees of emphasis; if so, the longer form, \textit{aimue}, is the more emphatic of the two.

From the perspective of Miestamo's \parencites*{Miestamo2005}{Miestamo2007}
typology of negative constructions, negation of clauses with non-finite
verbs, like the SN, is symmetrical, as it does not result in any obvious
morphosyntactic differences. The argument-coding system and the
morphological possibilities on the verbal predicate remain the same. The
only difference that was noted is, in relation to the lexical verb, the absence of the prosodic glottal stop [ʔ] in 1\textsuperscript{st} syllable coda position, which is otherwise characteristic of the lexical verb in affirmative non-finite verb constructions.

\section{Non-clausal negation}\label{sec:tacana-7}

In this section, I describe two types of non-clausal negation: stand-alone
negation and constituent negation. Stand-alone negation is
realized by way of \textit{aimawe} or \textit{mawe}, whether negation
consists in answering a polar question, as in
\REF{ex:tacana-polar-response}, or rectifying a false statement, as in
\REF{ex:tacana-rectify} and \REF{ex:tacana-rectify-more}.

\begin{exe}\ex\label{ex:tacana-polar-response} response to a polar question
\judgewidth{Sub-prefect:}
\sn[Authority:\hspace{\stretch{1}}]{%
\gll Corregidor=ja  =mi,  e-kisaba-me-ta-ani  apa  =mi
 acompaña a-kwa,  misha,  Semana\_Santa  misha=su, awa  \textbf{mawe}?\\
    judge\textsc{=erg}  \textsc{=2sg}
    \textsc{ipfv}-ask-\textsc{caus}-3A\textsc{-ipfv.sit}  if  =\textsc{2sg}
    accompany do\textsc{-pot}  church\_service  Holy\_Week
    church\_service=\textsc{loc} \textsc{q}  no\\
\glt `El corregidor te hace preguntar si puedes acompañar les a la misa de
Semana Santa o no?' su026\\
 `The corregidor asks whether or not you could accompany them to the Holy
 Week Mass.'}
\sn[Sub-prefect:]{%
\gll \textbf{Mawe}!  Aimue  =da  ema  e-puti=mawe.\\
no \textsc{neg} \textsc{=ptc} \textsc{1sg} \textsc{fut}-go=\textsc{neg}\\
\glt   `No, no voy a ir.' su028\\
 (`Do you want to go to Mass with us?') `\uline{No}! I
 wo\uline{n't} go!'}

\ex\label{ex:tacana-rectify} rectification of a false statement
\judgewidth{Mother:}
\sn[Mother:]{%
\gll Manuame-pe-ta-kwa  tse  ekwana.\\
    kill\textsc{-compl}-3A-\textsc{pot}  \textsc{maybe}  \textsc{1pl}\\
\glt  `¡(Tu padre) nos puede matar a toditos!' au064\\
 `(Your father) can kill us all!'}
\sn[Son:\hspace{\stretch{1}}]{%
\gll \textbf{Aimawe}!  Ema  ebiasu  tuche-da.\\
    no  1\textsc{sg}  a\_lot  strong-\textsc{asf}\\
\glt  `No, yo tengo más fuerza que él.' au066\\
 `\uline{No} (he can't kill us)! (Because) I'm stronger (than him).'}

\ex\label{ex:tacana-rectify-more} rectification of a false statement
\judgewidth{Jaguar:}
\sn[Jaguar:]{%
\gll Jiawe  mida  yama  e-dia.\\
    now \textsc{2sg} \textsc{1sg.erg} \textsc{fut-}eat\\
\glt  `Ahora te voy a comer.' bu028\\
 `Now I'm going to eat you.'}
\sn[Fox:\hspace{\stretch{1}}]{%
\gll \textbf{Mawe} tiyu! Be =tsu ema dia-ji!\\
    no uncle \textsc{imp.neg} =\textsc{yet} 1\textsc{sg}
    eat-\textsc{imp.neg}\\
\glt  `No tío, no me comes todavía!' bu029\\
 `No, Uncle! Do\uline{n't} eat me yet!'}
\end{exe}

Constituent negation is realized by way of the enclitic
\textit{=mawe} or its variant \textit{=mue}, which is attached to the
constituent to be negated. It is attested as a derivation process with
nouns (privative negation), as in \REF{ex:tacana-privative}, and adjectives
(adjectival antonym negation), as in \REF{ex:tacana-antonym}.

\begin{exe}\ex\label{ex:tacana-privative} privative negation
\begin{xlist}
\ex
\gll Pero pisa\textbf{=mue} =da ema.\\
    but gun=\textsc{priv} \textsc{=ptc} \textsc{1sg}\\
\glt `No tengo arma.' co046\\
`I don't have a gun (lit. I am \uline{without} a gun \slash{}
gun-\uline{less}).'
\ex
\gll Dapia lugar=su kristianu=kwana escuela=\textbf{mawe}.\\
there place=\textsc{loc} person=\textsc{pl} school=\textsc{priv}\\
\glt `En este lugar, no tiene escuela la gente.' na073\\
`There, in that place, the people don't have schools.' (Lit. are
\uline{without} a school \slash{} school-\uline{less}).
\end{xlist}
\ex\label{ex:tacana-antonym} adjectival antonym negation\\
\gll \overnote{S}Tueda \overnote{ADJ}sai-da\textbf{=mawe}, \overnote{S}ema =mu \overnote{ADJ}sai-da=kita.\\
3\textsc{sg} nice-\textsc{asf=neg} \textsc{1sg} \textsc{=contr}
nice-\textsc{asf=intens}\\
\glt `He is bad (lit. \uline{not} nice) and I am nice.'\\
\end{exe}

\section{Negation of hortative and imperatives clauses}\label{sec:tacana-8}

Finally, to close the synchronic description of negation strategies in
Tacana, we here provide a brief description of negation in commands.
The first type is hortative clauses (1\textsuperscript{st} and
3\textsuperscript{rd} person imperative), which in the affirmative polarity
are headed by a finite verb marked by a prefix \textit{pa-} instead of TAM
inflectional affixes, as illustrated in \REF{ex:tacana-hortative-positive}.
When negated, hortative clauses require a preverbal independent
particle \textit{be}, which is simply added to the positive construction
without further morphosyntactic modifications (same 3\textsuperscript{rd}
person indexation, same hortative prefix, same argument-marking and constituent order flexibility).

\begin{exe}\ex\label{ex:tacana-hortative}
\begin{xlist}
\ex\label{ex:tacana-hortative-positive} positive\\
\gll \overnote{O}Tueda \overnote{V}\textbf{pa}-dia-ta \overnote{A}señora=ja,
\overnote{V}\textbf{pa}-id'i-ta.\\
that \textsc{hort}-eat-3A wife\textsc{=erg} \textsc{hort-}drink-3A\\
\glt `¡Ese (caldo de gallina) que coma la señora! ¡Que tome!' pa044\\
‘Let the woman eat this (chicken soup)! Let her drink it!
\ex\label{ex:tacana-hortative-negative} negative\\
\gll \overnote{A}Tueda =mu \textbf{be} \overnote{V}\textbf{pa}-dia-ta \overnote{O}{\ob}jida aicha{\cb}!\\
3\textsc{sg} =\textsc{contr} \textsc{neg} \textsc{hort-}eat-3A that meat\\
\glt `¡Que el no coma esa carne!' n2.0138 (elicited)\\
`Don't let him eat that meat!'
\end{xlist}\end{exe}

The second type of command clauses is 2\textsuperscript{nd} person
imperative clauses, which in affirmative polarity are head by a finite
verb marked by the suffix \textit{-ke}, as in
\REF{ex:tacana-imperative-positive}. When negated, the
same preverbal independent particle \textit{be} must be used.
However, the head verb cannot carry \textit{-ke} anymore, which is now
replaced by a suffix \textit{-ji}, as in
\REF{ex:tacana-imperative-negative}. With regards to the other morphosyntactic properties of the clause, they are the same as in the affirmative.

\begin{exe}\ex\label{ex:tacana-imperative} 
\begin{xlist}
\ex\label{ex:tacana-imperative-positive} positive\\
\gll ``Dia-\textbf{ke}  =tsu  empanada,  kupari!''  ema  a-ta-idha.\\
    eat\textbf{-\textsc{imp}}  =\textsc{yet}  empanada  compadre
    1\textsc{sg}  do-3A\textsc{-rem.pst}\\
\glt `{}``¡Come todavía empanada, compadre!'' me dijo (mi comadre).' su057\\
`{}``\uline{Eat} some more `empanada,' compadre!,'' (my comadre) said to me.'
\ex\label{ex:tacana-imperative-negative} negative\\
\gll
Mawe  tiyu.  \textbf{Be}  =tsu  ema  dia-\textbf{ji}!\\
\textsc{neg}  uncle  \textsc{imp.neg}  =\textsc{yet}  1\textsc{sg}
eat-\textsc{imp.neg}\\
\glt `No tío, ¡\uline{No} me comes todavía!' bu029\\
`No, uncle, \uline{don't eat} me yet!'
\end{xlist}\end{exe}

\section{Summary of negation constructions}\label{sec:tacana-9}

The following table summarizes all the negation constructions described
above. In the schematized constructions, for practical reasons, the most
commonly attested variants are given (\textit{aimue} and \textit{=mawe} in
most cases).


\begin{table}
\begin{small}
\caption{Summary of negation constructions in Tacana}
\label{tab:tacana-summary}
\begin{tabularx}{\textwidth}[p]{@{} p{32mm} L{59mm} p{7mm} Q @{}}
\lsptoprule
Type & Construction & Sym\-metri\-cal & Construc- tion 
    type\\\midrule
\uline{Clausal} &  &  & \\
declarative finite\newline verbal main clauses & [\ldots{} \textbf{\textit{aimue}}
\ldots{} \hspace{\stretch{2}} V-\textsc{infl}(=\textbf{\textit{mawe}})
\ldots{} \hspace{\stretch{1}}] & yes & 1\\
\tablevspace
non-verbal clauses:\\

\tablevspace
\hspace{2ex}equation, inclusion & [… \textbf{\textit{aimue}} … NP…
\hspace{\stretch{1}} (be-\textsc{infl})=\textbf{\textit{mawe}} …  ] & yes &\\

\tablevspace
\hspace{2ex}attribution & 
    [… \textbf{\textit{aimue}} … ADJ…
    \hspace{\stretch{1}}
    (be-\textsc{infl})=\textbf{\textit{mawe}} …  ] 
& yes
& \\

\tablevspace
 \hspace{2ex}existential\slash locative & [… \textbf{\textit{aimue}} …
 \hspace{\stretch{1}}V\textsc{post-infl}=\textbf{\textit{mawe}} …
 \hspace{\stretch{1}}    ] & yes\\
& [… \textbf{\textit{aimue}} … \hspace{\stretch{2}}
be-\textsc{infl}=\textbf{\textit{mawe}} … \hspace{\stretch{1}} ] & yes &
\\\midrule
\uline{Clausal} &  &  & \\
non-verbal clauses:\newline existential\slash locative &
[\hspace{\stretch{2}}\textit{…} \textbf{\textit{aimue}}\textit{…}
(be-\textsc{infl})\hspace{\stretch{3}}] & no & 2\\

\midrule
\uline{Clausal} &  &  & \\
declarative non-finite & [\textit{…}
\textbf{\textit{aimue}} …  \hspace{\stretch{1}}   V …  \hspace{\stretch{1}}
(be/do-\textsc{infl}) …  \hspace{\stretch{2}}  ] & (yes) & 3\\
verbal main clauses & [… \textbf{\textit{mué=}} … \hspace{\stretch{1}}    V … \hspace{\stretch{1}}  (be/do-\textsc{infl}) … \hspace{\stretch{1}} ] &
(yes) \\\midrule
\uline{Non-clausal} &  &  & \\
stand-alone & [ \hspace{\stretch{1}} \textbf{\textit{aimawe}}
\hspace{\stretch{2}} ] & --- & 4 \\
& [\hspace{\stretch{1}} \textbf{\textit{mawe}} \hspace{\stretch{2}} ] & ---
&\\
\midrule
\uline{Non-clausal} &  &  & \\
constituent negation:\\\tablevspace
\hspace{2ex}privative & \rule{0pt}{1ex} \hspace{\stretch{2}}
[N=\textbf{\textit{mawe}}] \hspace{\stretch{1}} \rule{0pt}{1ex} & yes & 5\\

\tablevspace
\hspace{2ex} adjectival antonym & \rule{0pt}{1ex} \hspace{\stretch{2}}
[ADJ=\textbf{\textit{mawe}}] \hspace{\stretch{1}} \rule{0pt}{1ex} & yes &
\\

\midrule
\uline{Clausal} &  &  & \\
hortative & [\textit{…} \textbf{\textit{be}} \textit{…}
\hspace{\stretch{2}}  \textsc{hort}-V … \hspace{\stretch{1}} ] & yes &
6\\\tablevspace
prohibitive & [\textit{…} \textbf{\textit{be}} … \hspace{\stretch{2}}
V-\textbf{\textit{ji} }\textit{…} \hspace{\stretch{1}} ] & no & \\
\lspbottomrule

\end{tabularx}\end{small}\end{table}

\section{Reconstructing the origin of negation markers and constructions}\label{sec:tacana-10}

The goal of this section is to identify, on the basis of internal
reconstruction, possible etymologies and evolutionary pathways for the rise
of the different negative makers involved in the negation of
declarative (interrogative) clausal constructions: the SN construction
(§\ref{sec:tacana-3}--\ref{sec:tacana-5}) and the construction (with its
two subtypes) used to negate clauses with a non-finite verbal predicate
(\sectref{sec:tacana-6}), repeated in \tabref{tab:tacana-3}.

\begin{table}\begin{small}
\caption{Summary of negation constructions for verbal main clauses in Tacana}\label{tab:tacana-3}
\begin{tabularx}{\textwidth}{Qll}
\lsptoprule
Type of negated constituent & Construction & Symmetrical\\\midrule
clauses with finite verbal\newline \rule{0pt}{1ex}\hspace{2ex}predicate or with
non-verbal\newline \rule{0pt}{1ex}\hspace{2ex}predicates & … \textbf{\textit{aimue}} … predicate(=\textbf{\textit{mawe}}) … & yes\\
clauses with non-finite & \textit{…}
\textbf{\textit{aimue}} … V … (be/do-\textsc{infl}) … & yes \\
\hspace{2ex}verbal predicate & … \textbf{\textit{mué=}} … V … (be/do-\textsc{infl}) … & yes\\
\lspbottomrule
\end{tabularx}\end{small}\end{table}

The negative markers in these constructions are all formally very similar
and therefore likely to be historically related; such is not the case with
the negative markers involved in negation of command clauses (\textit{be},
\textit{-ji}), the reconstruction of which will not be attempted in this
paper.%
%
    \footnote{One might speculate that \textit{be} is related to
    \textit{mawe}, which manifests reduced variants such as the clitics
    \textit{=mue} and \textit{mué=} that come closer to the form of
    \textit{be.} I will leave this issue for further investigation.} 
%
The markers all consist of either \textbf{\textit{mawe \textasciitilde{}
mue}} used on its own or in combination with a preposed element
\textbf{\textit{ai}}, forming \textbf{\textit{aimawe {\textasciitilde}
aimue}}. Note that \textit{ai} can be used independently of \textit{mawe}
\textit{{\textasciitilde} mue}, as an indefinite noun ``person\slash thing,
someone\slash something'' \REF{ex:tacana-prepared}, suggesting that
\textit{aimawe {\textasciitilde} aimue} may be an erswhile univerbation of
this indefinite pronoun and \textit{mawe} (as in English `nothing,'
for example).
%
\\\protectedex{\begin{exe}\ex\label{ex:tacana-prepared} 
\gll Enekita  beu  =pa  \textbf{ai}=kwana  ja-ba-ti-ana.\\
    really  \textsc{ptc}  \textsc{=rep}  thing=\textsc{pl}
    \textsc{mid-}see-\textsc{mid-rec.pst}\\
\glt `En verdad dice se alistó sus cosas (para el viaje).' co080\\
`Really, he prepared his things (for the trip).'
\end{exe}} 
Depending on the construction, the negation markers \textit{mawe} or
\textit{aimue} have different degrees of grammatical or phonological
freedom (e.g., \textit{mawe} can be a clitic); they can occupy different
positions in the clause (e.g., preposed or postposed to the negated constituent); they can be used alone or in combination with each other (forming an embracing negation construction); and possibly, in the case of the embracing construction, one marker can be optional.

If one looks for possible internal cognates, it is notable that
\textit{mawe} and\slash or \textit{aimue} are also used in many other
negative constructions described earlier in this chapter, such as the
second existential\slash locative negation construction
(\sectref{sec:tacana-5}), stand-alone negation (\sectref{sec:tacana-7}) and
constituent negation (\sectref{sec:tacana-7}).

On the basis of these preliminary observations, we will now proceed to
reconstruct at least parts of the history of the two negative
constructions. We start with the SN construction in
\sectref{sec:tacana-10.1} and then move on to the reconstruction of
negation of clauses with a non-finite verb in \sectref{sec:tacana-10.2}.

\subsection{Evolutionary pathway: SN construction}\label{sec:tacana-10.1}

The embracing preposed marker \textbf{\textit{aimue}} and postposed marker
\textbf{\textit{=mawe}} that are used in SN have quite distinct grammatical
and phonological properties, which suggest that \textbf{\textit{=mawe}} is historically older than \textbf{\textit{aimue}}. Evidence for the likely older status of \textit{=mawe} is to be found in its shorter form (\textit{mawe} {\textasciitilde} \textit{mue}), phonological dependence (clitic status) and rigid position. These properties are all diagnotics of an advanced grammaticalization stage and they can be contrasted with the distinct properties of \textit{aimue}, with its longer form (\textit{aimawe {\textasciitilde} aimue}), phonological independence and free position (before the predicate).

One can therefore hypothesize that originally SN was expressed by a
single marker, the postposed marker \textit{=mawe}, and that the preposed
marker \textit{aimue} was introduced later for reinforcement. If we search
for a likely etymology for this newly introduced marker \textit{aimue}, the
negative stand-alone word \textit{aimawe} `no!'
\REF{ex:tacana-polar-response}, \REF{ex:tacana-rectify} and
\REF{ex:tacana-rectify-more} -- \REF{ex:tacana-rectify} is repeated in
\REF{ex:tacana-standalone-negative} below -- and the negative
existential\slash locative adjective \textit{aimue} `nonexistent'
\REF{ex:tacana-villages-firewood} – \REF{ex:tacana-villages} is repeated in
\REF{ex:tacana-negexadj} below – immediately come to mind, and there is little doubt that the three negation forms (new SN marker, stand-alone negation word and negative existential\slash locative adjective) are all historically related.

\begin{exe}\ex
\label{ex:tacana-standalone-negative}  Stand-alone negative
\textbf{\textit{aimawe}}
\judgewidth{Mother:}
\sn[Mother:]{%
\gll Manuame-pe-ta-kwa  tse  ekwana.\\
    kill\textsc{-compl}-3A-\textsc{pot}  \textsc{maybe}  \textsc{1pl}\\
\glt  `¡(Tu padre) nos puede matar a toditos!' au064\\
 `(Your father) can kill us all!'}
\sn[Son:\hspace{\stretch{1}}]{%
\gll \textbf{Aimawe}!  Ema  ebiasu  tuche-da.\\
    no  1\textsc{sg}  a\_lot  strong-\textsc{asf}\\
\glt  `No, yo tengo más fuerza que él.' au066\\
 `\uline{No} (he can't kill us)! (Because) I'm stronger (than him).'}
%
\ex\label{ex:tacana-negexadj} Negative existential\slash locative adjective
in an attributive construction\\
\gll {\ob}Biawa tiempo{\cb} =mu =da \textbf{aimue} pu-iti-a ejude=kwana.\\
    old  time  =\textsc{contr}  \textsc{=ptc}  nonexistent
    be-\textsc{tdm-pst}  village=\textsc{pl}\\
\glt `En tiempos antiguos no habían pueblos.' tu001\_ott\\
`In the old days, there were no villages.' [lit. villages were nonexistent]
\end{exe}

With regards to the evolution of their use, the hypothesis pursued
here is that the negative existential\slash locative adjective is older,
that it later extended its use as a stand-alone negation word, and that
this use made it possible to develop a new SN marker. In other words, the
immediate etymology of the SN maker \textit{aimue} is a stand-alone
negation word, \textit{aimue}, which itself can be traced back to a
negative existential\slash locative predicative adjective \textit{aimue}.
According to this scenario, which is schematized in
\tabref{tab:tacana-aimue-pathway} with the verb `go' as an illustration in
English, the evolutionary trajectory followed by the Tacana stand-alone
\textit{aimue} would be similar to that of the Brazilian
Portuguese\il{Portuguese}
stand-alone negator \textit{não} illustrated in \REF{ex:tacana-portuguese}.
A clause-external stand-alone negator, originally used to reinforce a
clause-internal negator (stage 2), is reanalyzed as a second
clause-internal negator, forming an embracing negation construction (stage
3). Over time, the original clause-internal negator becomes optional (stage
4) and ends up disappearing altogether (stage 5), with the result that it
is replaced by the new reinforcing (external stand-alone)
negator.

\begin{table}
\caption{Evolutionary pathway of Tacana stand-alone negation \emph{aimue} into the marking of SN}\label{tab:tacana-aimue-pathway}
\small
\begin{tabularx}{\textwidth}{l l Q l}
\lsptoprule
stage 1 & predicate\textit{=\textbf{mawe}} & `I will \textbf{not} go' & hypothesized\\
stage 2 & (\textbf{\textit{aimue}}), predicate\textit{=\textbf{mawe}} & `(\textbf{No},) I will \textbf{not} go' & hypothesized\\
\rowcolor[HTML]{C0C0C0}stage 3 & \textbf{\textit{aimue}}(,)predicate\textit{=\textbf{mawe}} & `\textbf{No}(,) I will \textbf{not} go' & synchronic use\\
\rowcolor[HTML]{C0C0C0}stage 4 & \textbf{\textit{aimue}} predicate(\textbf{\textit{=mawe}}) & `\textbf{No} I will (\textbf{not}) go' & (synchronic use)\\
stage 5 & \textbf{\textit{aimue}} predicate & `\textbf{No} I will
go'\newline (= I will \textbf{not} go') & hypothesized\\
\lspbottomrule
\end{tabularx} \end{table}


An alternative hypothesis would be that the immediate etymology for the new
SN marker is not the stand-alone use of the negator \textit{aimue} but its
use as a negative existential\slash locative adjective. In the context of
SN, this hypothesis is much less plausible, due to the lack of a
conceivable source construction and evolutionary scenario. Had the direct
etymology been the negative existential\slash locative adjective, the only
source construction available in Tacana that I can think of is where
\textit{aimue} negates the existence of a nominal referent, as illustrated
in \REF{ex:tacana-negexadj} (`there were no villages' \slash{} `villages
did not exist').\footnote{According to \textcite[157]{Veselinova2016}, ``the use of
negative existentials with nominalized verb forms is cross-linguistically
the most widespread pathway whereby they can be shown to expand into the
domain of verbal negation.''} 
%
However, negating an event by way of this construction (e.g. `there
is no going for me' \slash{}  `my going is not') would require important
structural changes in the verb form (for instance, the lack of finite
morphology) and argument structure (for instance, a different case frame
for the core arguments) which are absent in the negation of finite verb
constructions. Moreover, the verb to be negated by way of a negative
existential\slash locative adjective should display affirmative polarity;
this is not the case, since the verb is marked by the enclitic negator
\textit{=mawe}.

As proposed above, it is of course very likely that the SN negator
\textit{aimue} and the negative existential\slash locative adjective
\textit{aimue} are historically related, but the link is probably an
indirect one, involving an intermediary stand-alone negation stage;
cross-linguistically, the move from negated existential predicate to
stand-alone negation is a well-attested pathway 
\parencites[10,
13--14]{Croft1991}[127ff]{Veselinova2013}[155--156]{Veselinova2016},
and when there is synchronic polysemy between the two, the evidence
generally points to the negated existential predicate being the source, not
the other way around \citep[8]{Croft1991}.

Reconstructing the diachronic development of \textbf{\textit{=mawe}} in the
SN construction (stage 1) is a more complex task, for which it will be
necessary to resort to comparative data from other Takanan languages. That
is beyond the scope of the current paper, and so here I will restrict
myself to the observation that the most likely internal cognate is the
negative
enclitic \textit{=mawe} used alone in constituent negation (privative
derivation and adjectival antonym negation; \sectref{sec:tacana-7}).

\subsection{Evolutionary pathway: negation of clauses with a non-finite
verb}\label{sec:tacana-10.2}

We now move to the discussion of the possible diachrony of the negation construction of clauses with a non-finite verb. As a reminder, here negation is realized by way of a single negation marker which is preposed to the lexical verb and which can be one of two morphemes, \textit{aimue} or \textit{mué=}, giving the following two constructions: [\textit{…} \textbf{\textit{aimue}} … V … (be/do-\textsc{infl}) …] and [… \textbf{\textit{mué=}} … V … (be/do-\textsc{inflections}) …].

Following the same line of reasoning as above (i.e., taking into account
the respective grammatical and phonological properties of \textit{aimue}
and \textit{mué=}), it is reasonable to believe that \textbf{\textit{mué=}}
is older than \textbf{\textit{aimue}}. This assumption is grounded in the
observation that \textit{mué=} displays a shorter form and less
phonological independence (being a clitic) than \textit{aimue}; note that
in terms of their syntactic distribution, both are free to occur anywhere
before the lexical verb. The hypothesis is also corroborated by
philological evidence found in a Christian catechism in Tacana from
the mid-19\textsuperscript{th} century, which only displays a negation construction that
corresponds to that with \textit{mué=}. In the material available, which
goes back about 150 years, we see that all the instances of negation
of verbal main clauses\footnote{\label{fn:tacana-catechism} Note that in the affirmative polarity, the catechism shows
    examples of clauses with both finite and non-finite verbs.} 
%
are realized by way of a preverbal marker \textit{mawe} (spelled
\textit{mave}) that precedes a non-finite verb, as in the three examples in
\REF{ex:tacana-old-catecism}, and which looks basically similar to our synchronic construction [… \textbf{\textit{mué=}} … V … (be/do-\textsc{infl}) …].

\begin{exe}\ex\label{ex:tacana-old-catecism} Old Tacana
(mid-19\textsuperscript{th} century)
\begin{xlist}
\ex
\gll Quejutcua mara \overnote{S}mi \overnote{NEG}\textbf{mave}
\overnote{V}\textbf{confesa} \overnote{AUX}\textbf{pu}?\\
    how\_many year \textsc{2sg} \textsc{neg} confess be\\
\glt `How many years have you not confessed?' \parencite[297]{LafoneQuevedo1902}
\ex
\gll Jucuajasu ni \overnote{A}mi \overnote{NEG}\textbf{mave}
\overnote{V}\textbf{ichegua}.\\
    why  \textsc{maybe}  \textsc{2sg}  \textsc{neg}  kill\\
\glt `¿Y por qué no lo mataste?' \parencite[310]{LafoneQuevedo1902}\\
`Why didn't you kill it?'
\ex
\gll \overnote{NEG}\textbf{Mave} \overnote{V}\textbf{chanapa} cuaja
 miada ema e-ba-nia.\\
    \textsc{neg}  know  why  \textsc{2sg.erg}  \textsc{1sg}
    \textsc{ipfv}-see-\textsc{ipfv.sit1/2}\\
\glt `No sé porqué me miráis.' \parencite[310]{LafoneQuevedo1902}\\
`I don't know why you are looking at me.'
\end{xlist}\end{exe}

On the basis of these synchronic observations and the historical data, it is possible to suggest that the negation pattern with \textit{mué=} corresponds to the original construction and that the negation pattern with \textit{aimue} is a more recent development.

Turning to the reconstruction of the development paths, since both patterns
only differ in the formal and prosodic properties of their negation marker,
it can be suggested that they arose in a similar way but at different
times in the past. In terms of likely etymologies for \textit{aimue} and
\textit{mué=}, the same candidates are available as those for the
\textit{aimue} and \textit{=mawe} negators in SN: the negative stand-alone
word \textit{aimue} `no!' \REF{ex:tacana-standalone-negative} and negative
existential\slash locative predicative adjective \textit{aimue}
\REF{ex:tacana-negexadj}, to which we can add the second negative
stand-alone word \textit{mawe} `no!,' illustrated in
\REF{ex:tacana-not-to-mass} (repeated from \ref{ex:tacana-polar-response}).
%
\begin{exe}\ex\label{ex:tacana-not-to-mass}
\gll \textbf{Mawe}!  Aimue  =da  ema  e-puti=mawe.\\
    no  \textsc{neg}  \textsc{=ptc}  \textsc{1sg}
    \textsc{fut}-go=\textsc{neg}\\
\glt `No, no voy a ir.' su028\\
(`Do you want to go to Mass with us?') `\uline{No}! I wo\uline{n't} go!'
\end{exe}
%
Although \textit{mawe}, unlike \textit{aimue}, is not attested as a negative existential\slash locative predicative adjective in present-day Tacana, it is plausible that it could have been used in such a way in the past, and that this function fell into disuse.

Although a scenario similar to that proposed for the reconstruction of
SN -- a Jespersen Cycle reinforcement process by way of a stand-alone
negator replacing a former negator in a clause with a non-finite verb -- is
not completely inconceivable, here there is no evidence available which
would support it. That is, there is no possibility of having the two
negative markers \textit{aimue} and \textit{mué=} co-ocurring in the same
construction.

An alternative scenario that seems more probable would be one
which possibly involved as the source construction for both patterns (that
with \textit{aimue} and that with \textit{mué=}) not a clause with a
non-finite verb, but a negative existential\slash locative
predicative adjective of the clause type illustrated in
\REF{ex:tacana-negexadj} (`there were no villages' / `villlages did not
exist'). As commented in footnote \ref{fn:tacana-catechism}, this evolutionary
pathway is cross-linguistically very common. From this perspective, one
could imagine that the transitive and intransitive SN constructions (e.g.,
in \REF{ex:tacana-aimue-brother} `I didn't see my brother-in-law' and
\REF{ex:tacana-aimue-fast} `I didn't go out fast') come from the reanalysis
of clauses with a nominalized verb as the S argument of a negative
existential\slash locative predicate. These are translatable literally as `there is
no seeing of my brother-in-law by me' for \REF{ex:tacana-aimue-brother} and
`there is no fast going out for me' for \REF{ex:tacana-aimue-fast},
respectively. An argument in favor of this hypothesis is that here, unlike
in the SN construction with finite verbs, the verb does show some
similarities with nominal referents in negative existential predicate
constructions, in particular by being obligatorily non-finite and in an
affirmative form. Moreover, the verb to be negated by way of a negative
existential\slash locative adjective displays affirmative polarity.

Yet, there are several unresolved issues with this hypothesis; in
particular, there are many divergent properties between the hypothetical
negative existential predicate source construction and the target negative
construction with a non-finite verb, which would remain to be explained.
One such property is argument coding.  In the negation construction with a
non-finite verb, the argument coding is identical to that of basic
declarative affirmative clauses (the same split ergative case-marking
system). If the negation construction with a
non-finite verb had originated in an existential predicate, one would
expect a different coding pattern, one which should reflect how the
arguments can be coded in nominal predicate construction. Notably, one
would expect the S and the A of the SN construction to be marked like an
experiencer argument in a nominal predicate construction, with
dative(+purpose) case marking, as in (\ref{ex:tacana-meat-night}a,b);
note that \REF{ex:tacana-still-no-meat} is repeated from \REF{ex:tacana-no-meat}.
%
\begin{exe}\ex\label{ex:tacana-meat-night}
\begin{xlist}
\ex\label{ex:tacana-still-no-meat}
\gll \overnote{ADJ}\textbf{Aimue} \overnote{DAT}mesa \overnote{S}aicha…\\
    nonexistent  \textsc{3sg.dat}  meat\\
\glt `No tenía carne…' ye006\\
`He didn't have meat.' [lit. the meat was nonexistent to him]
\ex\label{ex:tacana-no-night}
\gll  \textbf{Aimue}  beju  dhidha  \textbf{kema=puji}.\\
    nonexistent  \textsc{ptc}  night  1\textsc{sg.dat=purp}\\
\glt `Ya no había noche para mi.' du023\\
`The was no night for me (because I could see at night as well as during the day).' du023
\end{xlist}\end{exe}

Another problematic property concerns the form and morphosyntactic
characteristics of the auxiliaries that can be optionally used in both
affirmative and negative clauses with a non-finite verb (to carry the
inflectional affixes), namely \textit{a} when it is transitive (e.g.,
\ref{ex:tacana-nonfinite-trap}) and \textit{pu} when the SN clause is
intransitive (e.g., \ref{ex:tacana-nonfinite-letgo}). First, in negative
existential predicates, the transitive auxiliary \textit{a} -- or better
said, its etymological source \textit{a} `affect, make, do (tr.), say
(tr.)' -- cannot be used. Second, even though in negative existential
predicates the intransitive auxiliary \textit{pu} -- or better said, its
etymological source \textit{pu} `be\slash exist, be located, do (itr.), say
(itr.)' -- can be used, as illustrated in \REF{ex:tacana-beginning}, it must
be followed by the negator \textit{=mawe} in addition to \textit{aimue}, a
construction which can be analyzed as the embracing SN of clauses with
finite verb constructions [… \textbf{\textit{aimue}} …
V-\textsc{inflections}(=\textbf{\textit{mawe}}) …].
%
\begin{exe}\ex
\label{ex:tacana-beginning}
\gll Upia  =mu  =da  \textbf{aimue}  ejude  ekene
\textbf{pu}{-iti-a}\textbf{=mawe}.\\
 here  =\textsc{contr}  \textsc{=ptc}  nonexistent  village  first
 be\textsc{-pfv-pst=neg}\\
\glt `Aquí no había nada/pueblo / no era pueblo todavía.' hi020\\
`Here (at that time) there was no village at the beginning.'
\end{exe}

Additional work is needed to investigate further whether this second scenario is supported by the data or if other hypotheses need to be sought.

\section{Summary and conclusion}\label{sec:tacana-11}

This paper presented for the first time a synchronic and diachronic study of negation markers and patterns in Tacana as applying to clauses (declarative\slash interrogative and commands) and constituents. The diachronic part focused on two major negation construction: SN and negation of clauses with non-finite verbs.

Starting with SN, I argued that its embracing pattern likely arose out of a
Jespersen Cycle process in which a stand-alone negator `no,' originally
used outside of a negative clause for pragmatic reinforcement, is in the
process of replacing the original postverbal SN marker inside of the
clause. Taking into account the actual polysemy between this stand-alone
word and the negative existential\slash locative predicative adjective
‘nonexistent,' I proposed to ultimately trace the origin of the new SN
marker back to a negative existential predicate, thereby adding Tacana
negation to the list of cases where the Jespersen and negative existential cycles intertwine.

The reconstructed grammaticalization path is also interesting from a
diachronic typological perspective because it goes from the right of the
verb to the left of the verb, unlike the more familiar direction from the
left of the verb to the right of the verb. As such, the Tacana pattern
corresponds to what \citet{AuweraVossen2016} and \citet{Vossen2016} call a
Jespersen Cycle ``in reverse''.

Continuing with the second construction, negation of clauses with
non-finite verbs, it was proposed that they both directly arose, albeit at
different times, out of a negative existential predicate construction where
the original function of \textit{aimue} and \textit{mué=} was to negate the
existence of a nominalized (non-finite) verb. However, if this hypothesis
is correct, how the original
intransitive negative existential predicate construction came to display all the synchronic properties
of negated clauses with non-finite verbs, which are the same as in the SN
construction (split-ergative case marking
and alternation of transitivity-sensitive auxiliaries), remains
largely unexplained and calls for further research.\il{Tacana|)}

\section*{Abbreviations}
\begin{tabularx}{.45\textwidth}{lQ}
    ( ) & morpheme that does not appear on the surface (in morpheme line) \\
    {[} {]} & multiple-word constituent \\
    \textsc{asf} & adjective suffix \\
    \textsc{assc} & associative \\
    \textsc{contr} & contrastive \\
    \textsc{depr} & depreciative \\
    \textsc{dubit} & dubitative \\
    \textsc{hort} & hortative \\
    \end{tabularx}
\begin{tabularx}{.45\textwidth}{lQ}
    \textsc{infl} & inflection \\
    \textsc{intens} & intensifier \\
    \textsc{mid} & middle \\
    \textsc{npf} & noun prefix \\
    \textsc{post} & posture \\
    \textsc{pot} & potential \\
    \textsc{priv} & privative \\
    \textsc{rec.pst} & recent past \\
    \textsc{rem.pst} & remote past \\
    \textsc{rep} & reportative \\
    \textsc{tdm} & temporal distance marker \\
\end{tabularx}


\section*{Acknowledgements}
Part of the research reported in this paper has been presented at the following conferences: Amazonicas VI (Laetitia, Colombia, 24-28 May 2016), Syntax of the World’s Languages VII (Mexico City, Mexico, 17-19 August 2016), Annual Meeting of the Societas Linguistica Europaea 50 (Zurich, Switzerland, 10-13 September 2017) and Syntax of the World's Languages VIII (Paris, 3-5 September 2018). I am grateful to the audiences of these meetings for useful feedback. The paper has also benefited from further valuable comments by Denis Creissels, Olga Krasnoukhova, Françoise Rose, Armin Schwegler, one anynymous reviewer, and Ljuba Veselinova and Arja Hamari as editors of the present volume. Finally, I am grateful to the ASLAN project (ANR-10-LABX-0081) of Université de Lyon for its financial support within the program \textit{“Investissements d'Avenir”} (ANR-11-IDEX-0007) of the French government operated by the National Research Agency (ANR).

{\sloppy\printbibliography[heading=subbibliography,notkeyword=this]}
\end{document}
