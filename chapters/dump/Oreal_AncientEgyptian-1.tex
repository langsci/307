\documentclass[output=paper]{langsci/langscibook}
%\ORCIDs{}

\author{Elsa Or�al\affiliation{French National Centre for Scientific Research}}

\title{The negative existential cycle in Ancient Egyptian}


\abstract{Ancient Egyptian has a very long attested history, which allows us to follow the emergence and evolution of several negative patterns. In spite of the inherent obstacles in a dead language?s documentation, my research ? focusing on negation in Earlier Egyptian (roughly defined as the language of texts written from 3000 to 1300 BCE) but tracing the relevant forms until Coptic (the last phase of the language, written in the Greek alphabet from the 4th to 14th century CE) ? sheds light on a renewal process that appears to belong to the category of the negative existential cycle. This process has long remained misunderstood, but recent progress in the field of linguistic typology regarding linguistic change in the negative domain makes it possible to propose a coherent historical analysis of the data. Starting with a transitional phase (C--A) documented in Old Egyptian, the Egyptian negative existential cycle does not illustrate Croft?s model in an ideal way. However, it offers a concrete case for a better understanding of how structural and functional parameters are intertwined in explaining this type of evolution.

\textbf{Keywords:} Negative existential Cycle. Nominalization. Grammaticalization. Ancient Egyptian

}

\begin{document}
\maketitle

\section{Introduction}\label{s:AE1}
Ancient Egyptian is a dead language whose history is generally divided into the following main phases: Old Egyptian (mid- to late 3rd millennium BCE), Middle Egyptian (ca. 2200?1700 BCE), Late Egyptian (14th?7th cent. BCE), Demotic (7th cent. BCE?5th cent. CE) and Coptic (ca. 4th cent.?ca. 14th cent. CE).\footnote{See \citet{Loprieno1995}, \citet{LoprienoMuller2012}, and \citet{GrossmanRichter} for a general overview.} My contribution involves data coming from the whole history of the language, but focuses on Old and Middle Egyptian (known as Earlier Egyptian in Egyptological linguistics). Negation and negative patterns in Ancient Egyptian have long been one of the thorniest topics in Egyptological linguistics. The reasons for this are twofold. From a philological viewpoint, the very spelling of negative morphemes appears to be problematic. Due to the functional characteristics of hieroglyphic writing, not only the functions but also the forms of negation have been subject to debate. On the other hand, the linguistic approach to Egyptian negation has suffered from a lack of precision in its diachronic dimension. The latter is perfectly understandable, given the state of the art. It is especially noteworthy that the language known as Earlier Egyptian covers a time period spanning from 3000 to 1300 BCE. Such a period may obviously appear to historical linguists as long enough to allow for substantial changes in the language, but for grammarians exploring the enduring history of Egyptian, it has long been considered as a more or less homogeneous unit. A better understanding of the changes happening in the course of this history may be reached with a more fine-grained analysis of the sources. In this context, the evolution of negation represents one of the more fertile fields of investigation. More specifically, the historical attestation of the Egyptian language offers a precious chance to analyse the evolution of a negative system over the long run. Previous studies have already attempted to recognize well-known patterns of change in the Egyptian data. As shown in \cite{vanderAuwera2009}, the very notion of the Jespersen cycle was formulated by Gardiner, a famous Egyptologist, as soon as 1904. An attested history of negation in Earlier Egyptian presents us with an ongoing negative existential cycle (NEC). The understanding of the way negation functions in this phase of the language can thus improve greatly, thanks to recent typological research after Croft?s pioneering formulation in 1991, especially by \citet{Veselinova2013, Veselinova2014, Veselinova2016}. In a reciprocal way, analysing these historical data may bring to light an interesting case that allows an assessment of the postulated motivations for changes at each step of the cyclical model, as elaborated in previous studies.

This article?s contributions thus go as follows: Section \ref{s:AE2} is a general description of the Egyptian NEC, presenting new analysis of the data.\footnote{ For a general introduction of the NEC, see the introduction of this volume.} Section \ref{s:AE3} explores the diachrony of the NEC in a more detailed way, focusing on problems bound with the nature of the documentation, and it highlights the results that can still be gained. Section \ref{s:AE4} examines the interplay of functional motivations, such as analogy or emphasis with structural features in changes within the NEC. A tentative conclusion in Section \ref{s:AE5} widens the perspective opened by the study of the Egyptian NEC, showing its relevance to the larger understanding of grammaticalization paths involved in the renewal of the verbal system in Older Egyptian. At the end of this contribution, a table in Appendix (see \ref{tab:AE-appendix}) presents the main patterns involved in the NEC. It may be useful for the reader to refer to this table in order to better understand the position of each detailed step in the general history.


\section{The Ancient Egyptian NEC: A general view}\label{s:AE2}

In this section, a descriptive analysis of the Egyptian NEC is given according to the chronological order of an ideal cycle, without taking into account the problems raised by transitional phases and their possible overlap. These difficulties are presented in more detail in Section \ref{s:AE3}. However, the following idealized sketch of the general evolution may be useful for the reader:

\begin{enumerate}
	\item The earliest documented stages show a C--A transition. The negator transliterated as \textit{ni} can be shown to have been an earlier negative existential.\footnote{The transliteration of Ancient Egyptian used here, as in Egyptological linguistics, represents written signs (be they hieroglyphics, signs in the various cursive writings, or Greek letters in the case of Coptic) and not the sounds of the language. In some cases, the phonemic reality is still a matter of discussion. }
	\item In Old Egyptian, this negator can occur with verbal nominalizations, leading to its reanalysis as a clausal negation.
	\item The same negator also extends to clauses with different types of stative predicates, showing a convergence with the positive existential predication pattern extending to stative predicates.
	\item This negator \textit{ni} begins to occur with various forms of the verb \textit{wnn} ?to exist? used as an existential copula.
	\item The collocation \textit{ni wn} [neg. ?exist?] fuses into a new negative existential marker, written \textit{nn}.
	\item In several stages, this new negator \textit{nn} spreads again to other contexts, such as those with originally locative adjuncts, leading to reanalysis as a negator plus locative predicate, and to other stative predicates.
	\item The new negator nn comes to be used with verbal predicates, in particular modal ones.
	\item A new stage of the cycle involves the new negator, itself a fusion of the oldest negation plus an existential verb, and a form of the very same verb \textit{wnn}, leading to the collocation \textit{nn wn}. For reasons that are still unclear, \textit{nn} also began to be written \textit{bn}.
	\item The collocation \textit{nn wn} fuses into \textit{mn}, which also extends from existential constructions to constructions with stative predicates and some verbal predicates.
\end{enumerate}


\subsection{A reconstructed Type C and a transitional type C--A: The negation \textit{ni} in Proto-Egyptian}\label{s:AE2-1}

Egyptian documentation begins at a point when a transition between a type C and a type A is already happening in the language. However, it is possible to at least partially reconstruct the phase immediately preceding the most ancient historical data. In Proto-Egyptian, as one may call it, ??there is a special negative existential predicate, which is identical to the verbal negator.
The form of the negator is generally transliterated as \textit{ni}.\footnote{Alternative proposals for the transliteration of the form exist but are not crucial for the historical reconstitution of the Egyptian NEC.}
Its use as a negative existential left traces in the oldest documentation, although \textit{ni} is not limited to this construction in historical times:

\ea Old Egyptian (Egyptian [Afro-Asiatic], Egypt) \label{ex:AE1}\\
  	\gll ni mD\#w D\#i sw m w\#t ppy pn\\
	\textsc{neg.ex}	opponent oppose\textbackslash\textsc{ptcp} \textsc{3m.sg} in way 	PN \textsc{dem}\\
	\glt ?There is no opponent who would oppose himself in the way of this Pepi.? (Pyramid Text � 1237bP)\footnote{Pyramid Texts are a corpus of ritual texts relating to the survival of the dead king in the hereafter. They are inscribed in royal graves from the end of the Old Kingdom but may represent older stages of the language. The extent of the preservation of archaic features in this corpus is still a matter of debate.}
\z

\ea Early Middle Egyptian (Egyptian [Afro-Asiatic], Egypt) \label{ex:AE2}\\
	\gll ni jr-t-n=j r=f\\
	\textsc{neg.ex} do\textbackslash\textsc{rel-f.sg-ant=1sg} against=\textsc{3m.sg}\\
	\glt ?There is nothing that I did against him.? (Letter to the Dead, Hu bowl 3)
\z

The origin of the negation \textit{ni} is likely to be a verb expressing absence or non-existence in a stative construction (sometimes termed ?Old Perfect? in grammars). This is shown by its location at the beginning of the sentence, the very place of the verbal predicate in the construction of the Old Egyptian stative perfect gram that follows VSO order. The use of the form \textit{ni} alone as a predicate appears to be attested in one example belonging to the most ancient documentation in Old Egyptian. It thus occurs in a non-literary papyrus:

\ea Old Egyptian (Egyptian [Afro-Asiatic], Egypt) \label{ex:AE3}\\
	\gll zwnw 1 owt 1 mndm 1 ni wor 2\\
	zunu 1 aut 1 nedjem 1 none/absent war 2\\
	\glt ?zunu-vessel: 1, aut-vessel 1, nedjem-basket 1: lacking, war-vessel: 2? (Papyrus	British Museum 10735)\footnote{This translation reflects the fact that Egyptologists do not know the precise equivalents for these substantives, except that they must refer to a kind of vessel.}
\z

It is only speculative but not irrelevant to mention the verb \textit{nj} ?to rebuff? as a possibly more specific lexical meaning in some proto-stage of the Egyptian language. Graphically, the writing known for this lexeme is interesting, as it shows an arm as a graphemic classifier, possibly a gesture symbolizing refusal, as does the sign of the two open arms that serves to write the negation \textit{ni} itself.\footnote{See \citet[125]{Loprieno1995} for the interpretation of this logogram as symbolizing a gesture of negation. On the function of unpronounced graphemic classifiers, see \citet{GoldwasserGrinevald2012}.}

Already in Old Egyptian, the negation \textit{ni} may be followed by various verbal forms whose common feature is to have their origin in nominalizations, while the primary participant is encoded as a possessor, with no distinctions between S, A, P semantic roles. Verbal paradigms known as recent suffixal conjugations generally share a common origin as former nominalizations. Still under discussion are the criteria allowing consideration of the various relevant constructions as involving grammaticalized verbal forms. I shall focus in the present chapter on negative constructions.\footnote{In a further stage of the renewal of the verbal system, a nominalization showing a <?t> specializes for a passive reading in constructions where the possessor thus appears as P.} Distinct nominal marks are the source for the resulting verbal morphology, as far as it is possible to reconstruct it. The source construction is of the following type:

\begin{enumerate}[label=(\roman*)] %\label{ex:AE4}
	\item Negative existential \textit{ni} $+$ Action nominalization $+$ Suffix pronoun \\
\end{enumerate}

Semantically, the path of change belongs to a well-known type that is attested in many distinct language families:

\begin{enumerate}[label=(\roman*), resume] %\label{ex:AE5}
	\item There is not my V-\textsc{ing} > I did/do not Verb \\
\end{enumerate}

Depending on the nominalization involved in these constructions, different TAM readings arise. When the action nominal is unmarked in the source construction, it is mainly past, but it can also be non-past and generic:

\ea Old Egyptian (Egyptian [Afro-Asiatic], Egypt) \label{ex:AE6}\\
	\gll ni m\#=j mjtj n zrw pn\\
	\textsc{neg} see\textbackslash\textsc{nmlz=1sg} like of goose this\\
	\glt ?I haven?t seen the like of this goose ever.? (lit. ?There is not my seeing the like of this goose?) (Meir III)
\z

With an ending <-\textit{w}>, whose function and meaning have been heavily discussed\footnote{Working on nominalizations as a source for the renewal of the Earlier Egyptian verbal system, I suggest that the Action nominalization marked by this ending <-\textit{w}> in the same way as other non-finite forms of the verbal root (most notably participles) has other uses whose reading can be related to indefinite semantics, the ending <-\textit{w}> prototypically indicating class membership. See \citet{Oreal2017}.}, the same action nominal follows the negation \textit{ni} to form a construction with a modal reading:

\begin{enumerate}[label=(\roman*), resume] %\label{ex:AE7}
	\item Negative Existential + Action nominal with <-\textit{w}> ending + Suffix pronoun\\
\end{enumerate}

Semantically, the path of change may be assumed to be as follows:

\begin{enumerate}[label=(\roman*), resume] %\label{ex:AE8}
	\item There is not any V-\textsc{ing} of me > I shall not Verb\\
\end{enumerate}

It should be noted that the nominalization involved in the source construction is unmarked for voice, so that both an active and a passive reading may arise:

\ea Old Egyptian (Egyptian [Afro-Asiatic], Egypt) \label{ex:AE9}\\
	\gll ni h\#i-w nTr-w r=k m rn=k pw n(j) j\#t\\
	\textsc{neg} go\_down\textbackslash\textsc{nmlz-indef} god-\textsc{pl} to=\textsc{2sg.m} in name=\textsc{2sg.m} \textsc{dem} of Iat\\
		?The gods shall not go down to you in this your name of Iat.? (lit. ?There is not any going down to you?) (Pyramid Text � 1537bP)
\z

\ea Old Egyptian (Egyptian [Afro-Asiatic], Egypt) \label{ex:AE10}\\
	\gll ni rDi-w wnjs n nsr=Tn nTr-w\\
	\textsc{neg} give\textbackslash\textsc{nmlz-indef} Wenis to flame-\textsc{2pl} god-\textsc{pl}\\
	\glt ?Wenis shall not be given to your flame, gods!? (lit. ?There is not any giving of Wenis to your flame?) (Pyramid Text � 323bW)
\z

There remain uncertainties in glossing some examples in Earlier Egyptian. They are connected to a larger question regarding the grammaticalization pathway of verbal paradigms known as recent suffixal conjugation or \textit{sDm=f}. How long should these forms be considered as still being nominalizations in various constructions? In other words, what are the criteria defining a verbal form having a nominalization as a source? This problem is the subject of an ongoing discussion in the field of Egyptological language studies and it cannot be solved in the present contribution. However, the historical mechanisms involved in the grammaticalization of such constructions belong to a typologically well-known phenomenon.\footnote{See, e.g., \citet{Malchukov2013} on similar problems raised by the process of verbalization in Siberian languages.} More peculiar is the notion of indefinite marking on an action nominal giving rise to modality after grammaticalization of the form in a construction.\footnote{See \citet{Oreal2017} on the pathways that allow such a transfer from nominal to verbal morphology.}

The extension of the negation \textit{ni} is not restricted to the negation of verbal predicates according to the asymmetry just described. It also extends to stative predicates, such as locative predication, as shown in the following example:

\ea Old Egyptian (Egyptian [Afro-Asiatic], Egypt) \label{ex:AE11}\\
	\gll ni Tw jm=sn\\
	\textsc{neg} \textsc{2m.sg} among=\textsc{3pl}\\
	\glt ?You are not among them.? (Pyramid Text 251dW)
\z

Such an evolution involves the relaxing of the referential constraint on the subject/topic present in the former locative-presentative when turning into locative predication:\footnote{See \citet[108]{Veselinova2013} on the typologically frequent structural similarity between these constructions and mere existential constructions.}

\begin{enumerate}[label=(\roman*), resume] %\label{ex:AE12}
	\item ni + non referential subject (`there is no man')\\
	> ni + non referential subject + locative predicate (`there is no man in the house')\\
	> ni + referential subject + locative predicate (`the man is not in the house')
\end{enumerate}

This path of change is parallel to the grammaticalization of the initial particle \textit{jw} from locative copula to auxiliary in stative and verbal constructions, as will be illustrated in the next section. The spread of the pattern to a referential subject may indeed not be explained without supposing a convergence with the evolution of the positive existential predication.

\subsection{Towards a type A: The verb of existence \textit{wnn} combined with the negation \textit{ni}}\label{s:AE2-2}

In its most ancient attestation, the existential predication assumes the following form, preserved as an archaizing pattern:

\begin{enumerate}[label=(\roman*), resume] %\label{ex:AE13}
	\item locative particle \textit{jw} + subject
\end{enumerate}

This construction is extremely rare in the documentation and not widely recognized as such in grammars.\footnote{See, however, the analysis along similar lines proposed by Helmut Satzinger at a workshop in 2001: \url{https://homepage.univie.ac.at/helmut.satzinger/Texte/Aspectsjw.pdf}.} \citet[122]{Loprieno1995} cites two occurrences in the Coffin Texts and in a literary text written in Middle Egyptian:

\ea Middle Egyptian (Egyptian [Afro-Asiatic], Egypt) \label{ex:AE14}\\
	\gll jw 	sSp Dd PN jw knH Dd PN\\
	\textsc{loc} light say\textbackslash\textsc{impfv} PN \textsc{loc} darkness say\textbackslash\textsc{impfv} PN\\
	\glt ?There is light, says PN; there is darkness, says PN.? (Coffin Text IV 29e)
\z

\ea Middle Egyptian (Egyptian [Afro-Asiatic], Egypt) \label{ex:AE15}\\
	\gll jw Sw m oq-jb\\
	\textsc{loc} lack in confident\\
	\glt ?There is a lack of close friends.? (Dialogue of a Man with his Ba 123?124)
\z

The latter example is reminiscent of another possible example belonging to the much older corpus of the Pyramid Texts that \citet[333]{Allen2017} translates with an existential reading:\footnote{Although convincing, this analysis cannot exclude a construction involving a zero subject and a stative as predicate (?\textit{jw � Sw m}, It is deprived of?). In the latter case, there would be no existential construction here but rather a stative predication introduced by \textit{jw}, which in synchrony represents a prototypical use of Earlier Egyptian \textit{jw}.}

\ea Old Egyptian (Egyptian [Afro-Asiatic], Egypt) \label{ex:AE16}\\
	\gll jw 	Sw m mtrw\\
	\textsc{loc} lack in testimony\\
	\glt ?There was a lack of testimony.? (Pyramid Text � 317b)
\z

One can perhaps add the following example from a caption in a daily life scene in an Old Kingdom tomb. It might also attest the construction in Old Egyptian, but its meaning is not completely assured:

\ea Old Egyptian (Egyptian [Afro-Asiatic], Egypt) \label{ex:AE17}\\
	\gll jw 	\#pd s\#i-w Hr=f\\
	\textsc{loc} bird satiate\textbackslash\textsc{nmlz-indef} on=\textsc{3m.sg}\\
	\glt ?There is a bird with which one may be satiated.? (Tomb of Ti, pl. 116)
\z

However, the role of \textit{jw} in constructions with a ?locative-presentative? function, in the sense of \citet{hengeveld1992} , fully confirms its role in the most ancient form of existential predication:

\ea Middle Egyptian (Egyptian [Afro-Asiatic], Egypt) \label{ex:AE18}\\
	\gll jw jt jm=f\\
	\textsc{loc} barley in=\textsc{3m.sg}\\
	\glt ?There is barley in it.? (Sinuhe B 84)
\z

Already in Old Egyptian, the ?\textit{jw} + Subject + Predicate? constructions are also used as plain intransitive sentences:

\ea Old Egyptian (Egyptian [Afro-Asiatic], Egypt) \label{ex:AE19}\\
	\gll jw \#pd pn Dd\# wrt\\
	\textsc{loc} bird this fat\textbackslash\textsc{ptcp} very\\
	\glt ?This bird is very fat.? (Tomb of Ibi)
\z

This reading generalizes in Earlier Egyptian. In parallel to this evolution, the existential construction itself is renewed by a construction using a form of the verb of existence \textit{wnn}:

\ea Old Egyptian (Egyptian [Afro-Asiatic], Egypt) \label{ex:AE20}\\
	\gll jn jw wn rm\\
	\textsc{interr} \textsc{loc} exist\textbackslash\textsc{ptcp.stat} fish\\
	\glt ?Is there some fish?? (Tomb of Djau)
\z

The same phenomenon occurs in negative existential statements, where the existential verb \textit{wnn} is introduced in new patterns. Already in Old Egyptian, one thus finds the negation \textit{ni} associated with the same existential verb \textit{wnn} in two distinct constructions:
\begin{enumerate}
    \item The construction \textit{ni wnt}
    \item The construction \textit{ni wn}
\end{enumerate}

Both involve the same form \textit{ni} but with a different status in each case, showing that construction (i) must have emerged before construction (ii), as we shall see now.

\medskip
\noindent
\textbf{(i) The construction \textit{ni wnt}}

This construction involves the negation of existence \textit{ni} followed by a nominalization of the verb \textit{wnn} marked by a -\textit{t} ending:\footnote{The precise semantics of this ending common to the source of passive forms, infinitives of certain classes of verbs and a ?not?yet? construction is still under discussion, but one can safely assume that it is a morphological feature characterizing a nominal form of the verb.}

\ea Old Egyptian (Egyptian [Afro-Asiatic], Egypt) \label{ex:AE21}\\
    \gll Sw\# jwowni wnt Xr=f zS\\
    be\_poor\textbackslash\textsc{ptcp} heir \textsc{neg} exist\textbackslash\textsc{nmlz}.with=\textsc{3msg} writing \\
    \glt ?Poor is the heir who has no writing.? (lit. ?when there exists not a writing with him?) (Pyramid Text � 475aWN)
\z

\ea Old Egyptian (Egyptian [Afro-Asiatic], Egypt) \label{ex:AE22}\\
    \gll ni wnt z\#b T\#tj nb sr nb jm wp-r=j  wo=k \\
    \textsc{neg} exist\textbackslash\textsc{nmlz} senior.vizir any magistrate any there except=\textsc{1csg} be\_alone\textsc{ptcp.resul=1csg}\\
    \glt ?(His majesty let me enter in order to audition alone) no senior vizier or any magistrate being there besides me alone.? (Inscription of Weni, 10?11)
\z


This construction must have emerged at a time when the negation \textit{ni} still had its stative meaning of non-existence. It is thus prior to construction (ii), which is derived from an intransitive construction where \textit{ni} functions as Standard Negation and no longer as a stative predicate, as will be shown in (ii).


\medskip
\noindent
\textbf{(ii) The construction \textit{ni wn}}


The alternative construction \textit{ni wn} is also already attested in an Old Kingdom inscription. It takes the following form:

\ea Old Egyptian (Egyptian [Afro-Asiatic], Egypt) \label{ex:AE23}\\
    \gll sT ni wn mw Hr Tsw\\
    \textsc{conj} \textsc{neg} exist\textbackslash\textsc{ptcp.stat} water on banks\\
    \glt ?When there is no water on the banks? (Inscription of Weni 44?45)
\z

There has been some discussion around the nature of the form taken by the existential verb \textit{wnn} in this construction.\footnote{See \citet{Meltzer1990}. Although \textit{wn} is often analysed as a \textit{sdm-f} form from the verb \textit{wnn}, no such form is ever attested with an actual suffix pronoun (*\textit{ni wn-f}). When the meaning is ?there is none?, the reference to an item that can be deduced from the context is in fact assumed by a zero. } In order to understand it better, one needs to look at the positive existential construction:

\ea Middle Egyptian (Egyptian [Afro-Asiatic], Egypt) \label{ex:AE24}\\
    \gll jw wn nDs Ddj rn=f\\
    \textsc{loc} exist\textbackslash\textsc{ptcp.stat} man Djedi name=\textsc{3m.sg}\\
    \glt ?There is a man named Djedi.? (Tale of pWestcar 6, 26)
\z

In locative predication, usually the subject directly follows the initial particle, be it \textit{jw}?or another one, like the subordinator \textit{sk} in the following example:

\ea Old Egyptian (Egyptian [Afro-Asiatic], Egypt) \label{ex:AE25}\\
    \gll nHm-w Sb(w)=f m-o=f sk sw wn\\
    take\textbackslash\textsc{ptcp-pl} food=\textsc{3m.sg} from=\textsc{3m.sg} as \textsc{3m.sg} exist\textbackslash\textsc{ptcp.pred}\\
    \glt ?Those who wanted to take his food away from him as it was there? (Pyramid Text � 290d?291aT)
\z

What happens in the existential construction is an inversion of the order subject-predicate triggered by information structure. In the existential predication, the subject is not thematic but falls rather under the scope of focus/rheme. In that case, it has to follow the predicate, rather than preceding it. Example (\ref{ex:AE26}) makes this contrast in constituent order obvious:

\ea Middle Egyptian (Egyptian [Afro-Asiatic], Egypt) \label{ex:AE26}\\
    \gll jst wn Hmt=f mrt rn=s\\
    now exist\textbackslash\textsc{ptcp.stat} wife=\textsc{3sg.m} Meret name=\textsc{3sg.f}\\
    \glt ?Now there was his wife named Meret.? (Tale of the Eloquent Peasant R 1, 2)
\z


\subsection{The emergence and extension of the negation \textit{nn}: A type B in Middle Egyptian}\label{s:AE2-3}

The negation \textit{nn} is the form traditionally acknowledged in grammars as the existential negation in Classical Egyptian, be it literary or documentary:

\ea Middle Egyptian (Egyptian [Afro-Asiatic], Egypt) \label{ex:AE27}\\
    \gll nn hrw-jb\\
    \textsc{neg.ex} peaceful\\
    \glt ?There is no peaceful one.? (Dialogue of a Man with his Ba 125?126)
\z

\ea Middle Egyptian (Egyptian [Afro-Asiatic], Egypt) \label{ex:AE28}\\
    \gll nn Hm-t \\
    \textsc{neg.ex} servant-\textsc{f.sg}\\
    \glt ?There is no servant.? (Letter UC 32092C r 11)
\z

The very fact that \textit{nn} functions as existential negation, a role formerly played by \textit{ni}, has sometimes been interpreted by philologists as resulting from a graphic confusion in older sources. Such an analysis is prompted by the absence of a diachronic perspective of Earlier Egyptian, instead conceived of as a homogeneous block. The more recent form \textit{nn} is indeed the successor of \textit{ni} as a form dedicated to existential negation, with \textit{ni wn} and \textit{ni wnt} being intermediary, in accordance with the NEC. Its formal relationship with the general negation \textit{ni} prompted several hypotheses, among them a process of morphological fusion between \textit{ni} and a following particle \textit{jn}. According to \citet[127]{Loprieno1995}, following a suggestion by \citet{Osing1979}, this postulated collocation would attest a kind of Jespersen cycle. However, the attested use of the particle \textit{jn} does not support this historical reconstruction, neither semantically nor syntactically. This particle remains difficult to gloss due to its polyfunctionality, but the crucial point from a historical point of view is that the hypothesized collocation *\textit{ni jn} is not attested except for very rare occurrences, which are not likely to represent a source construction for a morphological fusion. On the contrary, the following proposal is based on a well-attested combination. \citet[359]{Vergote1965formes} already suggested the hypothesis advocated here, namely, that \textit{nn} is the result of the fusion of the negative \textit{ni} with the following?existential verb \textit{wnn}:

\ea \label{ex:AE29}
    \gll nn $<$ ni wn\\
    \textsc{neg.ex} {} \textsc{neg} exist\textbackslash\textsc{ptcp} \\
    \glt ?there is no? ?there exists not?
\z

This note remained mainly unnoticed. However, in the perspective of a NEC, it makes perfect sense.\footnote{\citet[311]{Osing1979} rejects it on the basis of its poor attestation. However, the collocation of the particle \textit{jn} with the older negation \textit{ni} proposed as a source for \textit{nn} is even much more elusive.} In Croft?s \citeyearpar{Croft1991} terms, the process of phonological fusion that gives rise towards the end of the Old Kingdom to the form \textit{nn} dedicated to the negative existential function is characteristic of the transition between a type A and a type B.

The form \textit{nn} does not remain specialized in this function for long. Indeed, the documentation shows a stage where \textit{nn} is already extended to express the meaning ?without? when followed by an infinitive. According to \citet[256?257]{Loprieno-etal2017}, the first examples of the negation \textit{nn} where it does not function as a negative existential but as the negation in locative predication date to the late First Intermediate Period and early Middle Kingdom. It is well established in literary Middle Egyptian:

\ea Middle Egyptian (Egyptian [Afro-Asiatic], Egypt) \label{ex:AE30}\\
    \gll nn sj m jb-j\\
    \textsc{neg} \textsc{3f.sg} in heart-\textsc{1sg}\\
    \glt ?It was not in my heart? (Sinuhe 223?224)
\z

As \citet{Loprieno-etal2017} put it, ``the underlying reason for the change from \textit{n} to \textit{nn}-negations in adverbial sentences is unknown''.\footnote{An analysis based on pragmatic factors was proposed in \citet{Loprieno1991}, but \citet{Uljas2013} points out several aspects that cannot be accounted for according to it.} However, a path of change similar to what happened already in Old Egyptian with the negation \textit{ni} can be postulated within a diachronic perspective. It involves again the relaxing of the referential constraint on the subject/topic from locative-presentative\footnote{See \citet[108]{Veselinova2013} on the typologically frequent structural similarity between these constructions and mere existential constructions.} to locative predication:


\textbf{Step (1)} \textit{nn} + non referential subject (?there is no man?)

\textbf{Step (2)} \textit{nn} + non referential subject + locative predicate (?there is no man in the house?)


\ea Middle Egyptian (Egyptian [Afro-Asiatic], Egypt) \label{ex:AE31}\\
    \gll nn Xn jm n obo pri m r\#=j\\
    \textsc{neg.ex} word thereof exaggeration come\_out\textbackslash\textsc{ptcpg}.in mouth-1sg\\
    \glt ?There is no word of exaggeration in what came out of my mouth.? (Stela of Sesostris III at Semnah = Berlin �M 1157 l. 16?17)
\z

\textbf{Step (3)} \textit{nn} + referential subject + locative predicate (?the man is not in the house?)

The use of \textit{nn} also extends to all kinds of stative predicates (e.g. property predicates) in Middle Egyptian, thus replacing the older \textit{ni} in this function as well:

\ea Middle Egyptian (Egyptian [Afro-Asiatic], Egypt) \label{ex:AE32}\\
    \gll nn nn Sri-w \\
    \textsc{neg} this be\_small\textbackslash\textsc{ptcp.pfv-pred}\\
    \glt ?This is not small.? (Sinuhe, Ashmolean Ostracon 47)
\z

Tenses known as ?Pseudo-verbal? constructions are also concerned, but the spread of \textit{nn} to these followed a different chronology.

It is also to be noted that already in its early attestation \textit{nn} is used with the meaning ?without? before an infinitive:\footnote{\citet[118]{Veselinova2013} shows that in various languages, this use is a frequent extension of the negative existential.}

\ea Middle Egyptian (Egyptian [Afro-Asiatic], Egypt) \label{ex:AE33}\\
    \gll jri-kw m oq nn Dd=f \\
    do\textbackslash\textsc{ptcp-pred.1sg} as enter\textbackslash\textsc{ptcp} \textsc{neg.ex} say\textbackslash\textsc{inf-3sg.m}\\
    \glt ?I was made one who enters without being called.? (Stela of Wepwawetaa = M�nich Gl. WAF 35 line 16)
\z

\subsection{Towards a new type C?: The use of \textit{nn} with a verb}\label{s:AE2-4}

As stated in \citet[258]{Loprieno-etal2017}, ``the use of \textit{nn} seems to have begun in utterances expressing strong denials and refusals by the speaker? However, the spread of \textit{nn} in verbal sentences may also have been motivated by the morphological collapse of specific Prospective form(s) within the so-called \textit{sDm-f} formation.''
The following example from the Middle Kingdom may illustrate such an emphatic use:

\ea Middle Egyptian (Egyptian [Afro-Asiatic], Egypt) \label{ex:AE34}\\
    \gll nn Sno-tw=Tn m st qsnt \\
    \textsc{neg.ex} detain\textbackslash\textsc{mod-pass-2pl} in place dire\\
    \glt ?You shall not be detained in a dire place.? (Stela of Nebipusenwosret = BM EA 101 line 14?15)
\z

As is often the case, the use of the negative existential first extends only to part of the verbal system. Within Middle Egyptian (2200?1700 BCE), \textit{nn} in fact does not become the Standard Negation, but comes to be used in a verbal construction with modal semantics. I will focus here on the beginning of this process of extension to the verbal system that is never fully completed, with part of the verbal system remaining untouched until Coptic. Two historical scenarios appear to compete in the explanation of this process:
\begin{enumerate}
    \item one involves only the transfer of a whole inherited construction, where \textit{nn} stems out of \textit{ni wn}, as is the case in negative existential constructions, and
    \item the other involves a proper expansion of the negative existential \textit{nn} to negate a modal form.
\end{enumerate}

In both cases, there are good reasons to assume that the special paradigm of forms attested in the construction had acquired its modal semantics as a complement of manipulative verbs such as \textit{rDi} ?to let/allow?.\footnote{On the use in main clauses of verb forms whose modal semantics emerge from their primary use in completive clauses, see Evans (2007) and the notion of insubordination, whose applicability to the Egyptian tense sometimes called ?Subjunctive? still needs further research. }

The first scenario (i) involves the transfer of an inherited construction. One may postulate an evolution according to which the collocation \textit{ni wn} that lies at the source of \textit{nn} by fusion was used before a verbal form functioning as its subject:

\ea \label{ex:AE35}
    \gll *ni wn jwt-f > nn jwt-f \\
    \textsc{neg} exist come\textbackslash\textsc{mod-3sg.m} { } \textsc{neg} come\textbackslash\textsc{mod-3sg.m}\\
    \glt ?There is no (possibility) that he shall come > ?he shall not come
\z

However, there is no trace of a construction *\textit{ni wn sDm-f} in the documentation. On the contrary, the \textit{nn sDm-f} construction seems to gradually replace the construction \textit{ni} + Verbal nominalization marked for indefiniteness (mentioned in � 1.1) It appears semantically likely that the need for emphasis prompted a functional renewal, according to which the negative existential as such came to be used with a modal form as subject:

\ea \label{ex:AE36}
    \gll nn jwt-f > nn jwt-f \\
    \textsc{neg.ex} come\textbackslash\textsc{mod-3sg.m} { } \textsc{neg} come\textbackslash\textsc{mod-3sg.m} \\
    \glt ?There is not that he shall come? > ?he shall not come?
\z

This second scenario (ii) thus involves the expansion of the negative existential \textit{nn} to replace the older \textit{ni} in negating a modal form that is also distinct from the older one.\footnote{See \citet{Vernus1990} on this process. } It remains difficult to decide which of these potential paths of change was followed in the Egyptian case.
Less systematically, one also finds the negation \textit{nn} in constructions involving other verbal forms, such as the Anterior \textit{sDm-n-f}, but the reality behind the graphic reading is a matter of debate and this use in any case remains non-standard. It is also important to state that a large part of the Egyptian negative system involves neither locative nor verbal predicates but nominal predication, be it with classifying or identifying semantics.\footnote{On non-verbal predication in Ancient Egyptian, see the rich study by \citet{Loprieno-etal2017}, which includes all known stages of its long history. }
Only towards the end of the period known as Middle Egyptian does the form \textit{nn} also extend to negate nominal predication in the strict sense, taking over another part of the former domain of the negation \textit{ni}.\footnote{See \citet{Uljas2013} on this evolution.}
In Late Egyptian, \textit{nn} takes the graphic form \textit{bn}. As such, it is attested more largely, while the domain of use of \textit{ni}, which becomes graphically \textit{bw}, is drastically reduced. Thus, \textit{bn} can be said to become the standard negation in Later Egyptian, in association with the morph \textit{jwn\#} > Coptic IN. However, at this point in time, \textit{bn} had long ceased to function as a negative existential; thus, no new stage C can be observed. The history of the combination \textit{bn?jwn\#} represents another kind of linguistic cycle typical of the renewal of negation that lies beyond the scope of the present contribution.\footnote{On this Jespersen cycle, see \citet{Winand1997}.}

\subsection{Re-starting a cycle: The rise of \textit{nn wn} and the new type A}\label{s:AE2-5}

As stated by \citet[174]{Veselinova2016}, in various languages a NEC can be observed to re-start many times without being completed. In Earlier Egyptian, the same process seems to happen when the verb \textit{wnn} appears again in a new construction after the negation \textit{nn}, which used to express the existential negation alone:\footnote{Previous Egyptological analysis sometimes considers \textit{nn} as a shortened version of \textit{nn wn}, a view that diachronic data do not support without mentioning the neglected possibility of a NEC happening in Earlier Egyptian. }

\ea Middle Egyptian (Egyptian [Afro-Asiatic], Egypt) \label{ex:AE37}\\
    \gll nn wn oq-w jm  \\
    \textsc{neg} exist\textbackslash\textsc{ptcp.pred.m.sg} ration-\textsc{pl} there\\
    \glt ?(Can I be confident that I did give rations to my household while one writes to me saying) there are no rations there?? (pBM 10549, vso1-2)
\z

The very same relationship as the one between \textit{ni} and \textit{ni wn} obtains between this negative existential construction and the predication involving a thematic subject. The latter respects the usual ?Subject-Predicate? order in locative predications:

\ea Middle Egyptian (Egyptian [Afro-Asiatic], Egypt) \label{ex:AE38}\\
    \gll pf\# Sm Hno=f nn sw wn\\
    \textsc{dem\textbackslash m.sg} go\textbackslash\textsc{ptcp.m.sg} with=\textsc{3m.sg} \textsc{neg} \textsc{3m.sg} exist\textbackslash\textsc{ptcp.pred.m.sg}\\
    \glt ?That one with whom one used to go, he does not exist.? (Dialogue of a Man with his Ba 126-127)
\z

 With a rhematic subject, the existential verb comes before the subject:\footnote{For an alternative analysis of the form \textit{wn} as a \textit{sDm-f} form, see, e.g., \citet{Uljas2013}.}

\ea Middle Egyptian (Egyptian [Afro-Asiatic], Egypt) \label{ex:AE39}\\
    \gll nn wn m\#jr n h\#w=j nn Hqr n rk=j\\
    \textsc{neg} exist\textbackslash\textsc{ptcp.pred.m.sg} miserable of surroundings=\textsc{1sg} \textsc{neg} hungry of time=\textsc{1sg}\\
    \glt ?There was no miserable in my surroundings, no hungry (person) in my time.? (Imeny, Beni Hassan I, 8, 18-19)
\z

In this example, the older dedicated negative existential \textit{nn} also occurs in parallel with \textit{nn wn}. The coexistence of older and newer forms is not surprising, and it is found in other languages as well.\footnote{See \citet{Veselinova2014}.}
The construction also occurs with no subject at all:

\ea Middle Egyptian (Egyptian [Afro-Asiatic], Egypt) \label{ex:AE40}\\
    \gll Hr nb Hr nn wn\\
    face every \textsc{quot} \textsc{neg} exist\textbackslash\textsc{ptcp.pred.m.sg}\\
    \glt ?Everyone is saying: there is nothing.? (Admonitions 6, 3)
\z

The existential negation \textit{nn wn} began to extend to constructions involving a verbal form without ever moving into the domain of standard negation. Already at the beginning of the 12th Dynasty (20th century BCE), one finds the negative existential \textit{nn wn} used in a construction before the Anterior, a verb form marked for anteriority by a suffix -\textit{n}:

\ea Middle Egyptian (Egyptian [Afro-Asiatic], Egypt) \label{ex:AE41}\\
    \gll jw rDi-n=j mw n jb Hbsw n H\#jj nn wn jr-n=j r rmT\\
    \textsc{loc} give-\textsc{ant=1sg} water to thirsty clothes to naked \textsc{negex} do-\textsc{ant=1sg} against people\\
    \glt ?I gave water to the thirsty one, clothes to the naked one, on no occasion did I act against people.? (Stela Louvre C 196)
\z

It is also attested before a nominalization with modal prospective semantics:

\ea Middle Egyptian (Egyptian [Afro-Asiatic], Egypt) \label{ex:AE42}\\
    \gll nn wn mwt-k Hr X\#st\\
    \textsc{nex.ex} die\textbackslash\textsc{nmlz-2sg.m} on desert\\
    \glt ?You shall not die in the desert.? (Sinuhe B 197)
\z

\ea Classical Egyptian (Egyptian [Afro-Asiatic], Egypt) \label{ex:AE43}\\
    \gll nn wn HH-j-krnHH\\
    \textsc{nex.ex} seek\textbackslash\textsc{nmlz-pass-2sg.m} for eternity\\
    \glt ?You shall not be sought in eternity.? (Harpist song, Theban Tomb 50, Text J)
\z

However, this kind of use is rarely attested and did not generalize. It seems to have remained bound to an emphatic context of utterance. Thus, the renewal of the negative existential by way of the existential copula does not necessarily give an impulse to a new NEC, when other structural conditions that favour this evolution are lacking.

The maximal extension of \textit{nn} > \textit{bn} as quasi-Standard Negation and the use of \textit{nn/bn wn} as a usual negative existential construction are likely to have coincided at a point in the history of Ancient Egyptian that is not well documented in the preserved sources (see Section \ref{s:AE3-3}). However, the consecutive situation in Late Egyptian attests that this stage, corresponding to a new type A in an ideal NEC, was indeed realized as a concrete step in its development.\footnote{On the transition between Earlier and Later Egyptian and the discontinuity of sources attesting the various stages in Ancient Egyptian history, see \citet{Zoller-Engelhardt2016}.}

\subsection{The emergence of \textit{mn}: From A to B in Later Egyptian}\label{s:AE2-6}

Phonological fusion happens again in Later Egyptian, the second typologically defined phase in the history of the Egyptian language. It thus characterizes a further transition from a type A to a type B, as was already the case in Earlier Egyptian with the reconstructed emergence of \textit{nn} out of \textit{ni wn}:

Middle Egyptian \textit{nn wn} > Late Egyptian and Demotic \textit{mn} > Coptic \textit{mn}

The following example comes from a documentary source in Late Egyptian, dating approximately from the 12th century BCE:

\ea Late Egyptian (Egyptian [Afro-Asiatic], Egypt) \label{ex:AE44}\\
    \gll Xr jnn mn bHz jmi jni-tw p\#-Hotj Hno p\#-wt \\
    \textsc{ptcl} if \textsc{neg.ex} calf let\textbackslash\textsc{imp} bring-\textsc{pass} \textsc{def}-bed and \textsc{def}-coffin\\
    \glt ?If there is no calf, let the bed and the coffin be brought.? (Ostracon Berlin 12630, verso 1)
\z

The coexistence of \textit{bn} and \textit{mn} in two patterns sharing the function of expressing sentential possession within the same phase of the language has been studied in \citet{depuydt2008}.
According to this study, the construction involving \textit{mn} is an existential sentence while the one involving \textit{bn} is not. Within Late Egyptian synchrony, this is certainly true. However, it does not imply that \textit{bn} never had the function of a negative existential. It seems more relevant to say that the latter was no longer an existential construction.?In the diachronic perspective presented in here, the change in the status of the older construction is only natural and easy to explain, as was already the case in the preceding transition from the reconstructed types B to C in Older Egyptian. One needs only to recognize that twice in the history of Ancient Egyptian, the negative existential (first \textit{ni}, then \textit{nn}) extended its use to negating locative predication, and thus to possessive predication, which follows a well-known locative strategy in this language.

Unlike the former fusion from \textit{ni wn} to \textit{nn}, the fusion from \textit{nn wn} into \textit{mn} has long been recognized as such in the Egyptological literature. The same form remains in use as the negative existential until Coptic, with some variations in form according to the dialect that are not relevant to the general evolution:

\ea Coptic (Egyptian [Afro-Asiatic], Egypt) \label{ex:AE45}\\
    \gll mn-nute nsa-ntok Apoll�n \\
    \textsc{neg}-god except-\textsc{2ms} Apollo\\
    \glt ?There is no god but you, Apollo.? (Till, KHML II 33, 16-17)
\z

One can assume that the Egyptian NEC stops at this point, since the negation \textit{mn} does not seem to extend to cover the domain of standard negation in Coptic, although it does also spread to some modal constructions, including the prohibitive/vetitive and the negative jussive.\footnote{On this topic, see \citet{grossmanpolis2021}.}However, as will be seen in the next section, the whole history of its development from Middle Egyptian onwards is more complex than the presentation of its steps as distinct stages may suggest, and there are reasons to assume than the NEC lost its full force even before its last effects in Late Egyptian and Coptic.


\section{The complex diachrony of the Egyptian NEC}\label{s:AE3}

This section presents a historical analysis of the Egyptian NEC that focuses on data that make the general picture more complicated than what may seem at first sight. As in other language families, synchronic variation plays a key role at different points in this history, thus making transitional stages more important than stable types. However, the nature of the Egyptian documentation needs to be taken into account in order to assess the meaning of variation within the distinct corpuses that constitute it. A methodological preamble in Section \ref{s:AE3-1} thus anticipates a presentation of the particular problems connected to the chronology of the attestation of forms and constructions. Section \ref{s:AE3-2} sheds light on the fact that the existential negation may not be dedicated to one use, even in the first phase of its emergence. Section \ref{s:AE3-3} is devoted to showing some atypical features of the Egyptian NEC.

\subsection{Layering and the problem of diglossia in Ancient Egyptian}\label{s:AE3-1}

Before turning to facts regarding the precise chronology of the Egyptian NEC, an analysis of various factors that make it difficult to establish is in order. It is a well-known fact that prolonged variation between older and emergent forms in a given phase of a language may obscure the historical reconstruction of its evolution. The phenomenon of layering plays the same role in Egyptian as in other languages. Beyond negation, it is observed in many domains of the language. In this respect, there is nothing particular to Ancient Egyptian. However, the nature of the documentation makes the situation of the linguist even more difficult. First, it is not always possible to date a document with the precision needed to reconstruct a historical path. Moreover, even in texts that can be dated with some accuracy and belong to the same historical moment, layering plays different roles and can have different effects, according to its register, genre or discourse category. From the end of Middle Kingdom on, the use of the classical language in texts showing various degrees of formality prompted Egyptologists to create the notion of \emph{Egyptien de tradition}.\footnote{On this situation of diglossia, see \citet{Vernus1996}.}
This label designates a multifaceted reality. Middle Egyptian becomes a literary language of communication in formal register, used much like Latin in the Medieval Period. However, the concrete reality bound with this general definition can vary a lot from one text to another, depending on time but also on various contexts of use. Beyond the extension of the well-known phenomenon of layering, the Egyptian documentation thus shows a preservation of ancient forms in texts dating from times much posterior to their living use in the spoken language. As for negative existential constructions, the conservatism in formal and literary registers may be illustrated through the use of the construction \textit{ni wn}. It is indeed maintained in texts written in \emph{Egyptien de tradition} during the New Kingdom, such as seen in the following?example, which shows its use for expressing sentential possession:

\ea Classical Egyptian (Egyptian [Afro-Asiatic], Egypt) \label{ex:AE46}\\
    \gll ni wn st=sn \\
    \textsc{neg} exist\textbackslash\textsc{ptcp.m.sg} place=\textsc{3pl}\\
    \glt ?Their place does not exist? > ?They have no place? (Theban Tomb 50, Harpist song B 11)
\z

Diglossia is a complex phenomenon, and the Egyptian case for linguistic variation according to discourse registers or contexts of use is still in need of further research, although promising steps have already been taken.\footnote{In particular, see works such as \citet{Stauder2014a}, \citet{Werning2013} and \citet{Polis2017}.}
However, one may assume that there is no progressive transition between the use of two successive written registers in the larger sense, so that the whole history of Ancient Egyptian as a spoken language remains undocumented. As a consequence, there may be a gap in the attestation of linguistic change between Middle Egyptian \emph{stricto sensu} and Late Egyptian texts. This gap affects different linguistic phenomena in various ways, and its impact on the history of existential negation will be examined in Section \ref{s:AE3-3}.

\subsection{When existential negations take over discourse functions in the negative domain}\label{s:AE3-2}

As mentioned in Section \ref{s:AE2-2}, the verb of existence \textit{wnn} is combined in two distinct constructions with the negation \textit{ni} in order to renew the existential construction. However, only one of these follows a typical path of change towards a new type B (\textit{ni wn} > \textit{nn}). The construction \textit{ni wnt} follows another path of evolution, taking over a pragmatically marked function in the negative domain. Already in Old Egyptian, the form \textit{wnt} is grammaticalized as a conjunction in a variety of completive constructions.\footnote{See \citet{Uljas2007} on the use of \textit{wnt} as a complementizer in Old Egyptian. The usual hypothesis concerning its etymology sees it as a perfective feminine/neuter singular participle from the verb \textit{wnn} ?to exist?.}
As a consequence, in the attested documentation, the former existential verb \textit{wnt} is probably also reanalysed as such in the former negative existential construction (within which I propose that it emerged). Semantically, this reanalysis prompted the use of the construction as a denial operator, as is astutely observed by \citet[192?193]{Uljas2007}. Later on, in order to express the negation of existence, a new form of the existential verb \textit{wnn} is even reintroduced as an existential predicate after \textit{ni wnt}:

\ea Middle Egyptian (Egyptian [Afro-Asiatic], Egypt) \label{ex:AE47}\\
    \gll Dr-nttni wnt wn q\#i Xrw r=Tnj o\#\\
    for \textsc{neg} complexist\textbackslash\textsc{ptcp.pred.m.sg} high voice against=\textsc{2du} here\\
    \glt ?For it is not the case that there is/was someone who raises/raised his voice against both of you here.? (Letter to the Dead, Naga ed Deir 3500 K4-5)
\z

\ea Middle Egyptian (Egyptian [Afro-Asiatic], Egypt) \label{ex:AE48}\\
    \gll niwnt wn sX\# st\\
    \textsc{neg} \textsc{conj} exist\textbackslash\textsc{ptcp.pred.m.sg} remember \textsc{3f.sg}\\
    \glt ?There was no one who would remember it.? (Berlin 13272)
\z

In a syntactic context of dependence, the older construction may also be used as an alternative to the use of \textit{nn}, new negative existential having taken over the meaning ?without? in such position. The use of \textit{ni wnt} as a negative existential still appears in a medical treaty from the First Intermediate Period:

\ea Middle Egyptian (Egyptian [Afro-Asiatic], Egypt) \label{ex:AE49}\\
    \gll ni wnt srf nb jm-s\\
    \textsc{neg-ex} heat any in-\textsc{3f.sg}\\
    \glt ?There being no heat at all within it.? (Papyrus Edwin Smith, plate 15, l. 12)
\z

In this subordinate clause, the use of \textit{nn} would have yielded a meaning ?without?. In the very same text, \textit{nn} is indeed attested with this meaning before an infinitive:

\ea Middle Egyptian (Egyptian [Afro-Asiatic], Egypt) \label{ex:AE50}
    \gll pH-wj wgjjt-f pw nn jT-t jn-t\\
    end-\textsc{du} ramus-\textsc{3m.sg} \textsc{cop} without fetch-\textsc{inf} bring-\textsc{inf}\\
    \glt ?It means that the ends of his ramus are without movement back and forth.? (Papyrus Edwin Smith, plate 3, l. 17)
\z

The difference in meaning is slight: while the use of \textit{ni wnt} does not imply anything regarding the expected state of the patient (no heat is a neutral description of what the practitioner might happen to find in his exploration), ?\textit{nn} + Infinitive? indicates that the absence of movement is an atypical state.

\subsection{Atypical features in the Egyptian NEC and its chronology}\label{s:AE3-3}

Despite the inherent limitations presented in Section \ref{s:AE2-1}, it remains possible to retrace a somewhat imprecise path of change along steps with a plausible, if partly speculative, chronology. If one considers the \emph{Pyramid Texts} as a corpus preserving at least partially a phase of the language history anterior to other Old Egyptian sources, the Egyptian attested documentation begins at a point of the cycle where it reaches completion: negation of action and negation of existence are the same (type C). Unfortunately but not unexpectedly, the first attested stage in the history of the Egyptian language is already a layered one. But even this transitional phase seems to be typologically precious, since \citet[18]{Croft1991} states that ?types A and B are far more common than type C? and that type C?A will be extremely rare. It is thus interesting to be able to observe a transitional phase C?A in Old Egyptian. However, this phase and the consecutive phase A are not prototypical in the sense that they do not fit with the original hypothesis. First, as seen in Section ref{s:AE1-2}, two constructions (\textit{ni wn} and \textit{ni wnt}) are successively created, both using the same existential verb. Only one of these (\textit{ni wn}) follows a path of change predicted in the ideal NEC, giving rise to a new negator (\textit{nn}) that is at first dedicated to the existential negation. There is thus a dead end in the development of the transition from type C to type A.

In Later Egyptian, as stated in the preceding section, no new stage C ever happened, for the negation \textit{bn} had ceased to function as an existential negation long before it reached the status of (quasi-)Standard Negation. Even the actual existence of a new stage A cannot definitively be assumed to have taken place. The use of \textit{bn wn} as existential negation is in fact almost not attested in the documentation.\footnote{Many thanks are due to St�phane Polis, who kindly searched the Ramses database for attestations of \textit{bn wn} in Late Egyptian.}
The following example is one of the very rare instances:

\ea Late Egyptian (Egyptian [Afro-Asiatic], Egypt) \label{ex:AE51}\\
    \gll is bn wn m-di=k sSw qnw\\
    \textsc{interr} \textsc{neg} exist\textbackslash\textsc{nmlz} \textsc{prep=2sg.m} scribes many\\
    \glt ?Don?t you have many scribes at your disposal?? (Papyrus Anastasi 5, 11, 4)
\z

The expected existential negation should here be the form \textit{mn} resulting from the fusion of \textit{bn wn} in a text from this stage in the history of the language. However, it belongs to a corpus of texts traditionally called the Late Egyptian Miscellanies, which may show archaizing features. The same phenomenon may occur in the following literary text, where the graphical alternation between Middle Egyptian \textit{nn} and Late Egyptian \textit{bn wn} witnesses a deliberate play of graphical conventions by the scribe who composed the text:

\ea Late Egyptian (Egyptian [Afro-Asiatic], Egypt) \label{ex:AE52}\\
    \gll nn s\# n imj-r\# pr-HD bn wn [lacuna] imj-r\#\\
    \textsc{neg.ex} son to director of the treasury \textsc{neg} exist\textbackslash\textsc{nmlz} director\\
    \glt ?There is no son for the director of the treasury, there exists [no heir] for the director.? (Wisdom of Ani 20, 5-7, O. Gardiner 357)
\z

Thus, it remains possible that the second stage A, which is poorly attested in the preserved documentation, was indeed an actual step in the historical development but remains obscured by the gap in written data between Classical Egyptian and Late Egyptian, as explained in Section \ref{s:AE2-1}.

\section{What motivates type changes within the NEC?}\label{s:AE4}

\citet{Croft1991} proposes distinct characterizations for the change from one type to another, according to the transitional phase considered. This is schematically summarized as follows:

\begin{enumerate}[label=(\roman*)]
    \item from type A to type B: fusion
    \item from type B to type C: emphasis
    \item from type C to type A: syntactic analogy
\end{enumerate}

Thus, each kind of transition would appear to be motivated by a phenomenon belonging to a distinct domain: pragmatics, syntax and morphology. The process of phonological fusion as it is analysed in Croft (1991) appears rather less problematic than the precise role of emphasis and analogy in other transitional phases postulated in the ideal model of the cycle. It only raises a question as to the chronological relationship between the morphological process, according to which the former Standard Negation fuses with the existential verb and the ?spread? of the dedicated existential copula to other verbal and non-verbal predications, as seen in Section \ref{s:AE2}. In the next sections, I shall try to reconsider the qualification of changes (i) and (ii) in light of the Ancient Egyptian data.

\subsection{Re-assessing the role of syntactic analogy}\label{s:AE4-1}

Syntactic analogy is supposed to characterize transitions from type C to type A. In the history of Earlier Egyptian, this process seems to happen three times:

\begin{description}
    \item[(Phase 1)] From \textit{ni} to \textit{ni wnt}
    \item[(Phase 2)] From \textit{ni} to \textit{ni wn}
    \item[(Phase 3)] From \textit{nn} to \textit{nn wn}
\end{description}

Each time, a form of the existential verb \textit{wnn} is added to the former existential construction. In each case, however, there is no precise analogy to a standard verbal pattern. In phases (2) and (3), it is the stative predication that gives rise to the negative existential construction via a word order change that befits its information structure features. The only syntactic analogy that may be invoked is the one with the positive existential statement, which is itself renewed by adding the existential copula:

\ea
    \ea \label{ex:AE53a}
    \gll jw S > jw wn S\\
    \textsc{loc} S { } \textsc{loc} exist\textbackslash\textsc{ptcp.stat} S\\
    \glt ?There is S.?
    \ex \label{ex:AE53b}
    \gll ni S > ni wn S\\
    \textsc{neg.ex} S { } \textsc{neg} exist\textbackslash\textsc{ptcp.stat} S\\
    \glt ?There is no S.?
    \ex \label{ex:AE53ac}
    \glt nn S > nn wn S\\
    \textsc{neg.ex} S { } \textsc{neg} exist\textbackslash\textsc{ptcp.stat} S\\
    \glt ?There is no S.?
\z \z


One may now ask whether the evolution in the positive existential construction is itself motivated by some kind of syntactic analogy. This is not the case, for this construction was shown in Section \ref{s:AE2-2} to derive from the stative pattern according to a word order change whereby the indefinite subject, being rhematic, comes to be placed after the existential copula. Its emergence is likely linked to the fact that the former existential pattern had been extended to the stative predication in general,?according to a path that can be schematized as follows:

\ea
    \ea \textit{jw} + indefinite S\\
    \glt ?There is bread.?
    \ex \textit{jw} + indefinite S + locative predicate\\
    \glt ?There is bread in the house.?
\z \z

This pattern is extended to referential subjects via a bridging use involving a change of scope: the presentative reading is interpreted by the listener as concerning the locative predication as a whole with a thetic information structure, much like the scope of French \textit{il y a} can extend over an existential subject or a whole presentative clause:

\ea \textit{jw} + referential S + locative predicate\\
    \glt ?There is your bread in the house.?
\z

One step beyond, the particle \textit{jw} loses its locative semantics and becomes an auxiliary with a grounding discourse function?as a textual marker:

\ea  \textit{jw} + referential S + stative predicate\\
    \glt ?Your bread is in the house.?
\z

By the same token, a need arises for a renewed expressive mean. The new existential predication introduces the existential verb as a stative predicate, thus making existential semantics explicit again, while displacing the subject to the rhematic position, befitting its informational status:

\ea  \textit{jw} + referential S + \textit{wn} $?$ \textit{jw} + \textit{wn} + indefinite S\\
    \glt ?The bread is existing.??there is bread.?
\z

Only in phase (1) is there a possible (though not directly attested) source construction common to Standard Negation and the new negative existential. The choice of the nominalization marked with an ending -\textit{t} would parallel the grammaticalization of the so-called t-passive form from an action nominal marked for completive aspect following the former negative verb \textit{ni}:\footnote{For data concerning the evolution of passive forms and constructions in Earlier Egyptian, see Stauder (2014b). However, his analysis and the diachronic path of change proposed in this paper differ in more than one respect.}

\ea
    \ea
    \gll ni jri-t P\\
    \textsc{neg} do\textbackslash\textsc{nmlz-compl} P\\
    \glt ?There is no action of P? > ?P was not done.?
    \ex
    \gll ni wn-t S\\
    \textsc{neg} exist\textbackslash\textsc{nmlz-compl} S\\
    \glt ?There is no completed existence of S? > ?there is no S.?
\z \z

Even in this case, however, the analogy appears to be a superficial description of what happens in the language. From a functional point of view, the introduction of the verb of existence \textit{wnn} in each of the patterns under consideration can be explained as arising from the need for the renewed expression of existence. After the relaxation of the conditions of use bearing on the former existential construction, which has now become a stative construction, existential semantics would otherwise be absent. Thus, it is only from a partial point of view that this change may be qualified as formal and motivated by analogy as such. It cannot be demonstrated based on the preserved documentation, but one can even consider that this functional need for a renewal of expressive means has more to do with pragmatics than with syntax. Along this line of analysis, emphasis thus does not seem to be characteristic only of the transition from a type B to a type C, an evolution that is the subject of the next section.

\subsection{Emphasis as a functional motivation for change: NEC as another kind of Jespersen cycle}\label{s:AE4-2}

According to \citet{Croft1991}, the communicative need for more emphasis would be the functional input for the change from type B to type C, when the existential negation is extended to negate verbal clauses. Let us look back to the Egyptian case and what the documentation allows us to observe in this regard. This transition is potentially relevant to two distinct phases in the history of the Egyptian language:

\begin{enumerate}
    \item (i) The reconstructed phase C with \textit{ni} serving both as Standard Negation and as existential negation.
    \item (ii) The transitional phase B? to C? (incomplete), during which \textit{nn}/\textsc{bn} gradually extends to negate almost every kind of predication.
\end{enumerate}

These can be further explained as follows:

(i) As seen in the preceding section, it is not clear whether the use of the existential negation \textit{ni} with nominalizations according to an asymmetric negative strategy correlates with a pragmatic need for more emphasis. Since this extension is already completed in the most ancient documented stage of the language, one can only assume that such a need was an input towards change. In a way, the very fact that nominalizations in general play a crucial role in the renewal of the verbal system in the stage prior to Old Egyptian represents an argument against such a pragmatic motivation in the negative domain. However, there is also the possibility that a change in the negative domain, motivated in its incipient stage by a pragmatic need, might have been later extended to the rest of the verbal system. Along this line of historical explanation, the NEC would indeed help us to understand what prompted a more general evolution in the structure of the Egyptian verbal system.

(ii) As was already mentioned in the preceding Section \ref{s:AE2}, a new type C never obtains in Middle Egyptian or in Late Egyptian, for \textit{bn} no longer functions as a negative existential when its development as a negator reaches its maximal extension and becomes something close to the Standard Negation in the language. Moreover, its pathway of change was even shown to be the same as was the case for \textit{ni} in the preceding evolution from type C to type A: the relaxing of the constraint of indefinite reference on the subject made \textit{nn} capable of expressing the negation in all sorts of stative predications. From there, its spread to the nominal predication appears to fall within the scope of analogy more than anything else. On the other hand, its use with a modal verbal form (sometimes called subjunctive \textit{sDm-f}) also used in positive modal main clauses by insubordination may indeed be considered to derive from a need for emphasis. Such an explanation would account for its gradual replacing of the more ancient modal construction based on the negation \textit{ni} and a nominalization. However, as seen in Section \ref{s:AE2}, there is a gap in the attestation of the spoken register in Earlier Egyptian that allows for an unattested intermediary step *\textit{ni wn} subjunctive \textit{sDm-f}. In that case, the construction using the dedicated negative existential \textit{nn} would be inherited, deriving from a former construction involving the existential verb and the Standard Negation. As a consequence, it would seem questionable to speak of the negative existential extending to Standard Negation.\footnote{In such cases, \citet{Veselinova2014} states that the cycle is not operational.}
However, such a situation does not exclude emphasis as a motivation for the emergent construction in its incipient stage. An especially interesting point for a better understanding of the NEC might lie in the similar development attested in a modern language, such as Kannada:

\begin{displayquote}
 Kannada is a good example of diglossia as there are substantive differences between the literary and the spoken language on many levels. With regard to negation, it has to be said that one and the same strategy seems to be used to negate verbs with past time reference in both Literary and Spoken Kannada. Clear differences for the expressions of SN are observed for verbs/simple verbal sentences with present/future time reference. \citep[168]{Veselinova2016}
\end{displayquote}

In Late Egyptian, the negation for verbs with past time reference, \textit{bw}, is the direct successor of the negation \textit{ni}. But constructions used to negate sentences with future/modal reference differ in more than one aspect from earlier patterns. It thus confirms that the expansion of the negative existential into the verbal domain may obtain preferably both in the spoken register and in non-past contexts.


\section{Conclusion and avenues for further research} \label{s:AE5}

From the point of view of Ancient Egyptian grammatical studies, it thus appears that the reconstruction of a NEC allows a solution to problems about negation that had remained pending for a long time. In particular, the coexistence of distinct forms expressing existential negation in Earlier Egyptian can be better understood as the result of layering in documentation that contains more diachronic change than has been generally assessed in philological studies. On the flip side, the Ancient Egyptian data bring to linguistic research on negation an interesting case for an evolution attested over a long period of time. The study of these data confirms that transitional stages tend to be stable, as was already established in Veselinova?s pioneering studies on other language families. It also shows ideal stages that are not strictly consecutive: Ancient Egyptian illustrates the fact that overlaps are likely to occur between stages. As for the diachronic dimension, the attested history of Ancient Egyptian would appear to provide a case for the study of a potential NEC without reconstruction. Somewhat disappointingly, Middle Egyptian itself shows just another case of a long period of time where the negative existential is used for a specific sub-domain (modal/future negation, as in Bulgarian and Macedonian, with an inherited construction rather than an actual instantiation of the NEC, according to \citet{Veselinova2014}). However, the very notion of Standard Negation in a language such as Ancient Egyptian is obscured by the presence of other types of predication as prominent as verbal predication. Taking into account this specific structural feature, it appears that the NEC is almost completed in Late Egyptian (with \textit{bn} serving as quasi-Standard Negation), even if it does not reach a new stage C. Since the negative existential itself had been re-created long before (emergence of \textit{mn}), the ideal type C postulated in \citet{Croft1991} never happened to exist as such. A new cycle was thus re-started before the previous one completed, much in line with other situations studied in \citet{Veselinova2016}. As seen in the preceding sections, establishing the precise chronology in the Egyptian case is not easy. However, the extant data appear to suggest a period stretching from 1000 years as the time from a phase A to the next transitional phase C-A.\footnote{As for the time dimension of the NEC in general, \citet[1327]{Veselinova2014} states that ?modeling this cycle, as its completion, i.e. the negative existential turning into a full-fledged marker of standard negation, appears to take longer than 2000 years?. }

Regarding the path of change involving nominalizations and negative existentials, the Egyptian data bring to light a further question as to the historical intricacies of the NEC and the grammaticalization process of new verbal constructions based on nominalizations. Veselinova (2013: 139) argues that ?negative existentials state the absence of an entity rather than negating its presence?. Consequently, the action, when negated by a negative existential in a given language, is conceptualized and encoded as an entity. This approach requires particular consideration in the Egyptian case. In the phase of its history just before Old Egyptian, it appears most likely that the entire verbal system was renewed according to a process that involved the grammaticalization of constructions with nominalizations. This phenomenon is not restricted to the negative polarity, but extends to positive constructions as well. Thus, a question may be asked as to the relevance of the renewal of negative patterns for the evolution of the whole verbal system. Following \citet{Veselinova2016}, the encoding of actions in nominalizations appears functionally correlated with the use of the negative existential as a more emphatic construction in its incipient stage.\footnote{See \citet[160]{Veselinova2016} on Turkish.}
Such a process would suggest that the impetus for the global renewal of verbal predication in pre-Old Egyptian may have come from the new negative pattern using the negative existential in combination with verbal nominalizations.

\section*{Abbreviations}
\begin{tabularx}{.45\textwidth}{lQ}
\textsc{ant} & Anterior\\
\textsc{c} & common gender\\
\textsc{compl} & completive\\
\textsc{conj} & conjunction\\
\textsc{cop} & copula\\
\textsc{def}	 & definite\\
\textsc{dem} & demonstrative\\
\textsc{du} & dual\\
\textsc{f} & feminine\\
\textsc{imp} & imperative\\
\textsc{impfv} & imperfective\\
\textsc{indef} & indefinite\\
\textsc{inf} & infinitive\\
\textsc{interr} & interrogative\\
\textsc{loc} & locative particle\\
\textsc{m} & masculine\\
\textsc{mod} & modal\\
\textsc{neg} & negation\\
\textsc{neg.ex} & negative existential\\
\textsc{nmlz} &	nominalization\\
\textsc{pass	} & passive\\
\textsc{pl} & plural\\
\textsc{prep	} & preposition\\
\textsc{ptcl	} & particle\\
\textsc{ptcp} & participle\\
\textsc{quot} & 	quotative\\
\textsc{rel} & relative\\
\textsc{resul} & resultative\\
\textsc{sg} & singular\\
\textsc{stat} & stative\\
\end{tabularx}
%\begin{tabularx}{.45\textwidth}{lQ}
%... & \\
%... & \\
%\end{tabularx}

\section*{Acknowledgements}
My thanks are due to St�phane Polis for searching the Ramses database and collecting information for me about the existential negations in Late Egyptian. All errors and approximations remain mine. I am also deeply indebted to Eitan Grossman for his perceptive reading and many suggestions.


%{\sloppy\printbibliography[heading=subbibliography,notkeyword=this]}

\section*{Appendix}

The following table lists the various patterns and stages of the NEC. It is to be understood as a postulated reconstruction based on documentation that does not allow us to establish a precise chronology of changes (for reasons discussed in Section \ref{s:AE3-1}). Thus, while the order of succession and the approximate time of emergence of the relevant forms and constructions can be assumed with relative confidence, a strict dating remains out of reach and the proposed dates remain suggestive and hypothetical. The asterisk before a type indicates a historical phase that may be assumed to have happened when the gap between the written register and the spoken language was maximal, so that the documentation does not attest its realization. Such periods occur in particular during the transition between two language norms and are designated here as Proto-Middle Egyptian and Proto-Late Egyptian.

\begin{table}[]
    %\caption{}
    \label{tab:AE-appendix}
    \resizebox{\textwidth}{!}{
    \begin{tabular}{@{}lllll@{}}
    \lsptoprule
    Type & \begin{tabular}[c]{@{}l@{}}Reconstructed\\ chronology\end{tabular} & \begin{tabular}[c]{@{}l@{}}Incipient\\ Use as\\ NegEx\end{tabular} & Standard NegEx & \begin{tabular}[c]{@{}l@{}}Standard or\\ Close-to-standard\\ Negation\end{tabular} \\ \midrule
    *C & \begin{tabular}[c]{@{}l@{}}Proto-Egyptian\\ (unattested)\end{tabular} & \textit{} & \textit{ni} & \textit{ni} \\
    A & \begin{tabular}[c]{@{}l@{}}Old Egyptian	\\ (ca. 2500?2200 BCE)\end{tabular} & \textit{} & \textit{ni wn(t)} & \textit{ni} \\
    *A-B & \begin{tabular}[c]{@{}l@{}}Proto-Middle Egyptian\\ (unattested)\end{tabular} & \textit{nn} & \textit{ni wn(t)} & \textit{ni} \\
    B-C & \begin{tabular}[c]{@{}l@{}}Early Middle Egyptian\\ (ca. 2200?2000 BCE)\end{tabular} & \textit{} & \textit{nn} & \begin{tabular}[c]{@{}l@{}}\textit{ni}\\ \textit{nn} + Verb (modal)\end{tabular} \\
    C-A & \begin{tabular}[c]{@{}l@{}}Middle Egyptian\\ (ca. 2200?1700 BCE)\end{tabular} & \textit{nn wn} & \textit{nn} & \begin{tabular}[c]{@{}l@{}}\textit{ni}\\ \textit{nn} + Verb (modal)\end{tabular} \\
    *A-B & \begin{tabular}[c]{@{}l@{}}Proto-Late Egyptian\\ (unattested)\end{tabular} & \textit{mn} & \textit{bn wn (= nn wn)} & \textit{\begin{tabular}[c]{@{}l@{}}bn (= nn)\\ bw (= ni)\end{tabular}} \\
    B & \begin{tabular}[c]{@{}l@{}}Late Egyptian\\ (ca. 1400?700 BCE)\end{tabular} &  & \begin{tabular}[c]{@{}l@{}}\textit{mn}		\\ \textit{bn wn} (rare)\end{tabular} & \textit{\begin{tabular}[c]{@{}l@{}}bn\\ bw\end{tabular}} \\ \lspbottomrule
    \end{tabular}
    }
    \end{table}

{\sloppy\printbibliography[heading=subbibliography,notkeyword=this]}
\end{document}
